\documentclass{article}

\input{preamble}
\input{letterfont}
\input{macros}

\fancyhead[L]{\bd{Josh Park \\ Prof. Ilya Shkredov}}
\fancyhead[C]{\bd{MA 45401-H01 -- Galois Theory Honors \\ Homework \thesection~(Apr 11)}}
\fancyhead[R]{\bd{Spring 2025 \\ Page \thepage}}

\begin{document}
\setcounter{section}{10} % HW NUMBER
\begin{exercise}
  Let \( K,E,F\sseq L \) be fields, \( E:K,F:K \) be finite extensions. Prove \begin{enumerate}[label=(\alph*)]
    \item if \( E:K \) is separable, then \( EF:F \) is separable;
    \item if \( E:K\tand F:K \) are both separable, then \( EF:K \tand E\cap F:K \) are both separable;
    \item if \( E:K \) is Galois, then \( EF:F \) is Galois;
    \item if \( E:K\tand F:K \) are both Galois, then \( EF:K\tand E\cap F:K \) are both Galois.
  \end{enumerate}
\end{exercise}
\begin{enumerate}[label=(\alph*)]
  \item \label{qs:one-a} \begin{solution}
    Suppose \( E:K \) is separable.
    We are given that \( E:K\tand F:K \) are finite, so we can write \( E = K(\llist{\alpha}{1}{n}) \tand F = K(\llist{\beta}{1}{m}) \) for \( \alpha_i\in E \tand \beta_j\in F \).
    Then the composite field \( EF \) becomes \begin{align*}
      EF &= K(\llist{\alpha}{1}{n},\llist{\beta}{1}{m}) \\
      &= F(\llist{\alpha}{1}{n}).
    \end{align*}
    Since \( E:K \) is finite it is also algebraic, hence the minimum polynomial for each element of \( E \) is well defined over \( K \), and similarly for \( EF:F \).
    For any \( b\in F \), the minimal polynomial over \( F \) is \( x - b \), which has distinct roots, so \( b \) is separable over \( F \).
    Hence it is enough to show that \( \llist{\alpha}{1}{n} \) is separable over \( F \).

    We have that \( \mak \) is separable by hypothesis for all \( \alpha\in \left\{ \llist{\alpha}{1}{n} \right\} \).
    Then \( \mak(x)\in K[x]\subseteq F[x] \) so \( \mu_\alpha^F \) divides \( \mak \) and thus \( \mu_\alpha^F \) is thus also separable, whence \( EF:F \) is separable.
  \end{solution}

  \item \label{qs:one-b} \begin{solution}
    Suppose \( E:K\tand F:K \) are both separable.
    Similarly to part \ref{qs:one-a}, we can write \begin{align*}
      EF &= K(\llist{\alpha}{1}{n},\llist{\beta}{1}{m}),
    \end{align*}
    for \( \alpha_i\in E \tand \beta_j\in F \).
    By definition, \( a \) is separable over \( K \) for all \( a\in E \), and similarly for \( b\in F \).
    Then each \( \llist{\alpha}{1}{n}\in E,\ \llist{\beta}{1}{m}\in F \) is separable over \( K \).
    By theorem an extension \( K(\llist{\gamma}{1}{k}):K \) is separable iff each \( \gamma_i \) is separable over \( K \).
    Thus \( EF:K \) is separable.
    Furthermore, we know \( E:K \) is separable and \( E\cap F\sseq E \), so \( E\cap F:K \) is separable by definition.
  \end{solution}

  \item \begin{solution}
    Suppose \( E:K \) is Galois.
    Then \( E:K \) is normal and separable by definition.
    Since \( E:K\tand F:K \) are both finite and \( E:K \) is normal, we have by lemma that \( EF:F \) is normal and by part \ref{qs:one-a}, \( EF:F \) is separable.
    Thus \( EF:F \) is Galois.
  \end{solution}

  \item \begin{solution}
    Suppose \( E:K\tand F:K \) are both Galois.
    Then \( E:K\tand F:K \) are both normal and separable by definition.
    Since \( E:K\tand F:K \) are both finite and normal, we have by lemma that \( EF:K \tand  E\cap F:K \) are both normal and by part \ref{qs:one-b}, \( EF:K\tand E\cap F:K \) are both separable.
    Thus \( EF:K\tand E\cap F:K \) are both Galois.
  \end{solution}
\end{enumerate}

\begin{exercise}
  \begin{enumerate}[label=(\alph*)]
    \item Find the splitting field \( L \) of the polynomial \( f(t) = t^4 - 4t\sq + 5 \).
    \item Prove that \( \fdeg{L}{\Q} \) is either 4 or 8.
    \item Find 10 intermediate fields of the extension \( L:\Q \) and their degrees.
    \item (for enthusiasts) Draw the lattice of subfields and corresponding lattice of subgroups of \( \Gal_\Q(f) \).
  \end{enumerate}
\end{exercise}
\begin{enumerate}[label=(\alph*)]
  \item \begin{solution}
    Notice if we set \( t^4 - 4t\sq + 5 = 0 \), then we can subtract 1 to see \( t^4 - 4t\sq + 4 = (t^2 - 2)^2 = -1 \).
    Hence \( t^2 - 2 = \pm i \) and \( t \in \left\{ \pm\sqrt{2\pm i} \right\} \).
    We note that if \( w = \sqrt{a+bi} \) then \( w^2 = a+bi \) and \( \bar{w^2} = \bar{w}^2 = a-bi \), whence \( \bar w = \sqrt{a-bi} \).
    That is, the square roots of complex conjugates are themselves complex conjugates.
    So it is enough to construct \( L \) by adjoining \( \sqrt{2+i} \) to \( \Q \) and thus \( L = \Q(\sqrt{2+i}) \).
  \end{solution}

  \item \begin{solution}
    Set \( x = \sqrt{2+i} \).
    Then \begin{align*}
      x^2 &= 2+i \\
      x^2 - 2 &= i \\
      x^4-4x+4 &= -1 \\
      x^4-4x+5 &= 0
    \end{align*}
    Hence the minimum polynomial for \( \sqrt{2+i} \) is \( \mu_{\sqrt{2+i}}^\Q(x) = x^4-4x+5 = f(x) \) and \( \fdeg{L}{\Q} = 4 \).
  \end{solution}
  \item \begin{solution}

  \end{solution}
  \item \begin{solution}

  \end{solution}
\end{enumerate}

\begin{exercise}
  Draw the lattice of subfields and corresponding lattice of subgroups of \( Gal_\Q(t^6 + 3) \). \it{Hint}: Use the calculations (and the notation, if you like) from Lecture 18.
\end{exercise}
\begin{solution}

\end{solution}
\end{document}
