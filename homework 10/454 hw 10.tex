\documentclass{article}

\input{preamble}
\input{letterfont}
\input{macros}

\fancyhead[L]{\bd{Josh Park \\ Prof. Ilya Shkredov}}
\fancyhead[C]{\bd{MA 45401-H01 -- Galois Theory Honors \\ Homework \thesection~(Apr 18)}}
\fancyhead[R]{\bd{Spring 2025 \\ Page \thepage}}

\begin{document}
\setcounter{section}{10} % HW NUMBER
\begin{exercise}
  Let \( K,E,F\sseq L \) be fields, \( E:K,F:K \) be finite extensions. Prove \begin{enumerate}[label=(\alph*)]
    \item if \( E:K \) is separable, then \( EF:F \) is separable;
    \item if \( E:K\tand F:K \) are both separable, then \( EF:K \tand E\cap F:K \) are both separable;
    \item if \( E:K \) is Galois, then \( EF:F \) is Galois;
    \item if \( E:K\tand F:K \) are both Galois, then \( EF:K\tand E\cap F:K \) are both Galois.
  \end{enumerate}
\end{exercise}
\begin{enumerate}[label=(\alph*)]
  \item \label{qs:one-a} \begin{solution}
    Suppose \( E:K \) is separable.
    We are given that \( E:K\tand F:K \) are finite, so we can write \( E = K(\llist{\alpha}{1}{n}) \tand F = K(\llist{\beta}{1}{m}) \) for \( \alpha_i\in E \tand \beta_j\in F \).
    Then the composite field \( EF \) becomes \begin{align*}
      EF &= K(\llist{\alpha}{1}{n},\llist{\beta}{1}{m}) \\
      &= F(\llist{\alpha}{1}{n}).
    \end{align*}
    Since \( E:K \) is finite it is also algebraic, hence the minimum polynomial for each element of \( E \) is well defined over \( K \), and similarly for \( EF:F \).
    For any \( b\in F \), the minimal polynomial over \( F \) is \( x - b \), which has distinct roots, so \( b \) is separable over \( F \).
    Hence it is enough to show that \( \llist{\alpha}{1}{n} \) is separable over \( F \).

    We have that \( \mak \) is separable by hypothesis for all \( \alpha\in \left\{ \llist{\alpha}{1}{n} \right\} \).
    Then \( \mak(x)\in K[x]\subseteq F[x] \) so \( \mu_\alpha^F \) divides \( \mak \) and thus \( \mu_\alpha^F \) is thus also separable, whence \( EF:F \) is separable.
  \end{solution}

  \item \label{qs:one-b} \begin{solution}
    Suppose \( E:K\tand F:K \) are both separable.
    Similarly to part \ref{qs:one-a}, we can write \begin{align*}
      EF &= K(\llist{\alpha}{1}{n},\llist{\beta}{1}{m}),
    \end{align*}
    for \( \alpha_i\in E \tand \beta_j\in F \).
    By definition, \( a \) is separable over \( K \) for all \( a\in E \), and similarly for \( b\in F \).
    Then each \( \llist{\alpha}{1}{n}\in E,\ \llist{\beta}{1}{m}\in F \) is separable over \( K \).
    By theorem an extension \( K(\llist{\gamma}{1}{k}):K \) is separable iff each \( \gamma_i \) is separable over \( K \).
    Thus \( EF:K \) is separable.
    Furthermore, we know \( E:K \) is separable and \( E\cap F\sseq E \), so \( E\cap F:K \) is separable by definition.
  \end{solution}

  \item \begin{solution}
    Suppose \( E:K \) is Galois.
    Then \( E:K \) is normal and separable by definition.
    Since \( E:K\tand F:K \) are both finite and \( E:K \) is normal, we have by lemma that \( EF:F \) is normal and by part \ref{qs:one-a}, \( EF:F \) is separable.
    Thus \( EF:F \) is Galois.
  \end{solution}

  \item \begin{solution}
    Suppose \( E:K\tand F:K \) are both Galois.
    Then \( E:K\tand F:K \) are both normal and separable by definition.
    Since \( E:K\tand F:K \) are both finite and normal, we have by lemma that \( EF:K \tand  E\cap F:K \) are both normal and by part \ref{qs:one-b}, \( EF:K\tand E\cap F:K \) are both separable.
    Thus \( EF:K\tand E\cap F:K \) are both Galois.
  \end{solution}
\end{enumerate}

\begin{exercise}
  \begin{enumerate}[label=(\alph*)]
    \item Find the splitting field \( L \) of the polynomial \( f(t) = t^4 - 4t\sq + 5 \).
    \item Prove that \( \fdeg{L}{\Q} \) is either 4 or 8.
    \item Find 10 intermediate fields of the extension \( L:\Q \) and their degrees.
    \item (for enthusiasts) Draw the lattice of subfields and corresponding lattice of subgroups of \( \Gal_\Q(f) \).
  \end{enumerate}
\end{exercise}
\begin{enumerate}[label=(\alph*)]
  \item \begin{solution}
    Notice that \begin{align*}
      t^4 - 4t\sq + 5 = 0 \qimp t^4 - 4t\sq + 4 = \left( t\sq - 2 \right)\sq = -1.
    \end{align*}
    Hence \( t\sq - 2 = \pm i \) and we have roots \( t \in \left\{ \sqrt{2+i},-\sqrt{2+i},\sqrt{2-i},-\sqrt{2-i} \right\} \).
    Thus \begin{align*}
      L &= \Q\left( \sqrt{2+i},\sqrt{2-i} \right)
    \end{align*}
  \end{solution}

  \item \begin{solution}
    Clearly for \( E:= \Q\left(\sqrt{2+i}\right) \), we have that \( E:\Q \) is a degree 4 extension.
    We note here that \( i \in E \), which follows from the fact that \( \left( \sqrt{2+i} \right)\sq-2 = i \).
    So the minimum polynomial for \( \sqrt{2-i} \) over \( E \) is \( x\sq-(2-i) \).
    Hence if \( \sqrt{2-i}\in E \), then \( \fdeg{L}{\Q}=4 \) but if not, then \( \fdeg{L}{E}=2 \) whence \( \fdeg{L}{\Q}=8 \) by the tower law.

    Notice that \( F:= \Q\left( \sqrt{2+i}+\sqrt{2-i} \right) \) is a proper subset of \( L \).
    It is easy to see that \( \fdeg{F}{\Q}=4 \), so we have \( \fdeg{L}{\Q}>4 \).
    Thus \( \sqrt{2-i}\not\in E \) whence \( L:E \) must have degree 2 and by the tower law, \( \fdeg{L}{\Q} = 8 \).
  \end{solution}

  \item \begin{solution}
    Notice \begin{align*}
      \bigg[ \bar{\sqrt{2+i}} \bigg]\sq = \bar{\left[ \left( \sqrt{2+i} \right)\sq \right]} = \bar{2+i} = 2-i \qimp \bar{\sqrt{2+i}} = \sqrt{2-i}.
    \end{align*}
    That is, the square roots of complex conjugates are themselves complex conjugates.
    Define \( \sigma \) such that \( \sqrt{2+i}\mapsto \sqrt{2-i} \tand \sqrt{2-i}\mapsto-\sqrt{2+i} \), and let \( \tau \) be complex conjugation.
    Obviously \( \tau\sq = \sigma^4 = \id \).
    Notice \begin{align*}
      \tau\sigma\tau\left(\sqrt{2+i}\right) = \tau\sigma\left(\sqrt{2-i}\right) = \tau\left(-\sqrt{2+i}\right) = -\sqrt{2-i} = \sigma\inv\left(\sqrt{2+i}\right).
    \end{align*}
    That is, \( \tau\sigma\tau = \sigma\inv \)
    These are the defining features of \( D_4 \), the dihedral group of 4 points.
    Hence \( \Gal_\Q\left(t^4-4t\sq+5\right) \iso D_4 \) has exactly ten subgroups, and by the Galois correspondence there are ten intermediate fields.
    We can idenitfy these subfields of \( L \) by finding the fixed field \( L^H \) for each subgroup \( H \) of \( D_4 \).
    Letting \( \alpha=\sqrt{2+i}\tand \beta=\sqrt{2-i} \), we have:
    \begin{align*}
    1 &= \fdeg{\Q}{\Q},\\
    2 &= \fdeg{\Q(i)}{\Q} = \fdeg{\Q\left(\sqrt{5}\right)}{\Q} = \fdeg{\Q(\alpha/\beta)}{\Q},\\
    4 &= \fdeg{\Q(\alpha)}{\Q} = \fdeg{\Q(\beta)}{\Q} = \fdeg{\Q\left( i,\sqrt{5} \right)}{\Q} = \fdeg{\Q(\alpha+\beta)}{\Q} = \fdeg{\Q(\alpha-\beta)}{\Q}, \\
    8 &= \fdeg{L}{\Q}
    \end{align*}
  \end{solution}
\end{enumerate}

\begin{exercise}
  Draw the lattice of subfields and corresponding lattice of subgroups of \( \Gal_\Q\left(t^6 + 3\right) \). \it{Hint}: Use the calculations (and the notation, if you like) from Lecture 18.
\end{exercise}
\begin{solution}
  From Lecture 18, we have that the splitting field is \( L=\Q\left( \sqrt[6]{-3} \right) \) and \( \Gal_{\Q}\left(t^6+3\right) \iso D_3 \iso S_3 \).
  Cubing the generator yields \( \sqrt[3]{-3} \), whence we have the subfield \( \Q\left(\sqrt[3]{-3}\right)\sseq L \).
  Moreover, we know \( \veps_6\in L \) from lecture so we have \( \veps_3 = \veps_6\sq \in L \) and we can generate subfields \( \Q\left(\veps_3\right) \), \( \Q\left( \veps_3\sqrt[3]{-3} \right),\tand\Q\left( \veps\sq_3\sqrt[3]{-3} \right) \).
  We note here that \( \sqrt[6]{-3}^3 = i\sqrt{3} \) and \( \Q\left(\veps_3\right)=\Q\left(\veps_6\right)=\Q\left(i\sqrt{3}\right) \), which can easily be seen by decomposing \( \veps_3\tand \veps_6 \) by Euler's formula.
  Thus we have identified all the unique subfields of \( \Q\left(\sqrt[6]{-3}\right) \).
  \img{ma454 hw10 q3.png}{1}
\end{solution}
\end{document}
