% %%%%%%%%%%%%%%%%%%%%%%%%%%%%%%%%%%%%%%%%%%%%%%%%%%%%%%%%%%%%%%%%%%%%%%% %
%                                                                         %
% The Project Gutenberg EBook of Oeuvres mathématiques d'Évariste Galois, by
% Évariste Galois                                                         %
%                                                                         %
% This eBook is for the use of anyone anywhere at no cost and with        %
% almost no restrictions whatsoever.  You may copy it, give it away or    %
% re-use it under the terms of the Project Gutenberg License included     %
% with this eBook or online at www.gutenberg.org                          %
%                                                                         %
%                                                                         %
% Title: Oeuvres mathématiques d'Évariste Galois                          %
%                                                                         %
% Author: Évariste Galois                                                 %
%                                                                         %
% Editor: Société mathématique de France                                  %
%         Émile Picard                                                    %
%                                                                         %
% Release Date: July 11, 2012 [EBook #40213]                              %
% Most recently updated: June 11, 2021                         %
%                                                                         %
% Language: French                                                        %
%                                                                         %
% Character set encoding: UTF-8                                           %
%                                                                         %
% *** START OF THIS PROJECT GUTENBERG EBOOK OEUVRES MATHÉMATIQUES ***     %
%                                                                         %
% %%%%%%%%%%%%%%%%%%%%%%%%%%%%%%%%%%%%%%%%%%%%%%%%%%%%%%%%%%%%%%%%%%%%%%% %

\def\ebook{40213}
%%%%%%%%%%%%%%%%%%%%%%%%%%%%%%%%%%%%%%%%%%%%%%%%%%%%%%%%%%%%%%%%%%%%%%
%%                                                                  %%
%% Packages and substitutions:                                      %%
%%                                                                  %%
%% book:     Required.                                              %%
%% inputenc: Standard DP encoding. Required.                        %%
%%                                                                  %%
%% fontenc: For French hyphenation. Required.                       %%
%% babel:    French language features. Required.                    %%
%%                                                                  %%
%% ifthen:   Logical conditionals. Required.                        %%
%%                                                                  %%
%% amsmath:  AMS mathematics enhancements. Required.                %%
%% amssymb:  Additional mathematical symbols. Required.             %%
%%                                                                  %%
%% alltt:    Fixed-width font environment. Required.                %%
%%                                                                  %%
%% indentfirst: Indent first word of each sectional unit. Required. %%
%% footmisc: Renumber footnotes on each page. Required.             %%
%%                                                                  %%
%% calc:     Length calculations. Required.                         %%
%%                                                                  %%
%% fancyhdr: Enhanced running headers and footers. Required.        %%
%%                                                                  %%
%% geometry: Enhanced page layout package. Required.                %%
%% hyperref: Hypertext embellishments for pdf output. Required.     %%
%%                                                                  %%
%%                                                                  %%
%% Producer's Comments:                                             %%
%%                                                                  %%
%%   Changes are noted in this file in two ways.                    %%
%%   1. \DPtypo{}{} for typographical corrections, showing original %%
%%      and replacement text side-by-side.                          %%
%%   2. [** TN: Note]s for minor typographical comments.            %%
%%                                                                  %%
%%                                                                  %%
%% Compilation Flags:                                               %%
%%                                                                  %%
%%   The following behavior may be controlled by boolean flags.     %%
%%                                                                  %%
%%   ForPrinting (true by default):                                 %%
%%   Compile a print-optimized PDF file. Set to false for screen-   %%
%%   optimized file (pages cropped, one-sided, blue hyperlinks).    %%
%%                                                                  %%
%%                                                                  %%
%% PDF pages: 78 (if ForPrinting set to false)                      %%
%% PDF page size: 5.5 x 8in (non-standard)                          %%
%%                                                                  %%
%% Summary of log file:                                             %%
%% * One overfull hbox (4.1pt too wide), two underfull hboxes.      %%
%% * Five underfull vboxes.                                         %%
%%                                                                  %%
%%                                                                  %%
%% Compile History:                                                 %%
%%                                                                  %%
%% July, 2012: adhere (Andrew D. Hwang)                             %%
%%             texlive2007, GNU/Linux                               %%
%%                                                                  %%
%% Command block:                                                   %%
%%                                                                  %%
%%     pdflatex x2                                                  %%
%%                                                                  %%
%%                                                                  %%
%% July 2012: pglatex.                                              %%
%%   Compile this project with:                                     %%
%%   pdflatex 40213-t.tex ..... TWO times                           %%
%%                                                                  %%
%%   pdfTeX, Version 3.1415926-1.40.10 (TeX Live 2009/Debian)       %%
%%                                                                  %%
%%%%%%%%%%%%%%%%%%%%%%%%%%%%%%%%%%%%%%%%%%%%%%%%%%%%%%%%%%%%%%%%%%%%%%
\listfiles
\documentclass[leqno,12pt]{book}[2005/09/16]

%%%%%%%%%%%%%%%%%%%%%%%%%%%%% PACKAGES %%%%%%%%%%%%%%%%%%%%%%%%%%%%%%%
\usepackage[utf8]{inputenc}[2006/05/05]

\usepackage[T1]{fontenc}[2005/09/27]
\usepackage[french]{babel}[2005/11/23]

\usepackage{ifthen}[2001/05/26]  %% Logical conditionals

\usepackage{amsmath}[2000/07/18] %% Displayed equations
\usepackage{amssymb}[2002/01/22] %% and additional symbols

\usepackage{alltt}[1997/06/16]   %% boilerplate, credits, license

\usepackage{indentfirst}[1995/11/23]
\usepackage[perpage]{footmisc}[2005/03/17]

\usepackage{calc}[2005/08/06]

\usepackage{fancyhdr}

%%%%%%%%%%%%%%%%%%%%%%%%%%%%%%%%%%%%%%%%%%%%%%%%%%%%%%%%%%%%%%%%%
%%%% Interlude:  Set up PRINTING (default) or SCREEN VIEWING %%%%
%%%%%%%%%%%%%%%%%%%%%%%%%%%%%%%%%%%%%%%%%%%%%%%%%%%%%%%%%%%%%%%%%

% ForPrinting=true (default)           false
% Asymmetric margins                   Symmetric margins
% Black hyperlinks                     Blue hyperlinks
% Start Preface, ToC, etc. recto       No blank verso pages
%
\newboolean{ForPrinting}
%% UNCOMMENT the next line for a PRINT-OPTIMIZED VERSION of the text %%
%\setboolean{ForPrinting}{true}

%% Initialize values to ForPrinting=false
\newcommand{\Margins}{hmarginratio=1:1} % Symmetric margins
\newcommand{\HLinkColor}{blue} % Hyperlink color
\newcommand{\PDFPageLayout}{SinglePage}
\newcommand{\TransNote}{Note sur la transcription}
\newcommand{\TransNoteCommon}{%
%  Ce livre a été préparé à l'aide d'images fournies par la
%  Département des Mathématiques, Université de Glasgow.
  Des modifications mineures ont été apportées à la présentation,
  l'orthographe, la ponctuation et aux notations mathématiques. Le
  fichier \LaTeX\ source, qui on peut trouve à
  \begin{center}
    \texttt{www.gutenberg.org/ebooks/\ebook},
  \end{center}
  contient les notes sur ces corrections.
  \bigskip
}
\newcommand{\TransNoteText}{%
  \TransNoteCommon

  Ce fichier est optimisée pour être lu sur un écran, mais peut être
  aisément reformater pour être imprimer. Veuillez consulter le
  préambule du fichier \LaTeX\ source pour les instructions.
}

%% Re-set if ForPrinting=true
\ifthenelse{\boolean{ForPrinting}}{%
  \renewcommand{\Margins}{hmarginratio=2:3} % Asymmetric margins
  \renewcommand{\HLinkColor}{black}         % Hyperlink color
  \renewcommand{\PDFPageLayout}{TwoPageRight}
  \renewcommand{\TransNote}{Notes sur la transcription}
  \renewcommand{\TransNoteText}{%
    \TransNoteCommon
    \bigskip

    Ce fichier est optimisée pour imprimer, mais peut être aisément
    reformater pour être lu sur un écran. Veuillez consulter le
    préambule du fichier \LaTeX\ source pour les instructions.
  }
}{% If ForPrinting=false, don't skip to recto
  \renewcommand{\cleardoublepage}{\clearpage}
%  \raggedbottom
}
%%%%%%%%%%%%%%%%%%%%%%%%%%%%%%%%%%%%%%%%%%%%%%%%%%%%%%%%%%%%%%%%%
%%%%  End of PRINTING/SCREEN VIEWING code; back to packages  %%%%
%%%%%%%%%%%%%%%%%%%%%%%%%%%%%%%%%%%%%%%%%%%%%%%%%%%%%%%%%%%%%%%%%

\ifthenelse{\boolean{ForPrinting}}{%
  \setlength{\paperwidth}{8.5in}%
  \setlength{\paperheight}{11in}%
  \usepackage[body={5.25in,8.5in},\Margins]{geometry}[2002/07/08]
}{%
  \setlength{\paperwidth}{5.5in}%
  \setlength{\paperheight}{8in}%
  \usepackage[body={5.25in,6.9in},\Margins,includeheadfoot]{geometry}[2002/07/08]
}

\providecommand{\ebook}{00000}    % Overridden during white-washing
\usepackage[pdftex,
  hyperfootnotes=false,
  pdfkeywords={Andrew D. Hwang, K. F. Greiner, Paul Murray,
               Project Gutenberg Online Distributed Proofreading Team},
  pdfstartview=Fit,    % default value
  pdfstartpage=1,      % default value
  pdfpagemode=UseNone, % default value
  bookmarks=true,      % default value
  linktocpage=false,   % default value
  pdfpagelayout=\PDFPageLayout,
  pdfdisplaydoctitle,
  pdfpagelabels=true,
  bookmarksopen=true,
  bookmarksopenlevel=1,
  colorlinks=true,
  linkcolor=\HLinkColor]{hyperref}[2007/02/07]

\hypersetup{pdftitle={The Project Gutenberg eBook \#\ebook:%
    \texorpdfstring{{\OE}uvres Mathématiques d'Évariste Galois}{Oeuvres Mathematiques d'Evariste Galois}},
  pdfauthor={\texorpdfstring{Évariste Galois}{Evariste Galois}}}

%%%% Fixed-width environment to format PG boilerplate %%%%
\newenvironment{PGtext}{%
\begin{alltt}
\fontsize{9.2}{10.5}\ttfamily\selectfont}%
{\end{alltt}}

%%%% Global style parameters %%%%
% No hrule in page header
\renewcommand{\headrulewidth}{0pt}
\setlength{\headheight}{15pt}

% Loosen horizontal spacing
\setlength{\emergencystretch}{1.3em}

% "Scratch pad" for length calculations
\newlength{\TmpLen}

%%%% Running heads %%%%
\newcommand{\FlushRunningHeads}{%
  \clearpage
  \pagestyle{fancy}
  \fancyhf{}
  \cleardoublepage
  \fancyhead[C]{--- {\footnotesize\thepage}\ ---}
}

\newcommand{\BookMark}[3][]{%
  \phantomsection%
  \ifthenelse{\equal{#1}{}}{%
    \pdfbookmark[#2]{#3}{#3}%
  }{%
    \pdfbookmark[#2]{#3}{#1}%
  }%
}

%%%% Major document divisions %%%%
\newcommand{\FrontMatter}{%
  \cleardoublepage
  \frontmatter
  \BookMark{-1}{Front Matter}
}
\newcommand{\PGBoilerPlate}{%
  \pagenumbering{alph}
  \pagestyle{empty}
  \BookMark{0}{PG Boilerplate}
}
\newcommand{\MainMatter}{%
  \FlushRunningHeads
  \mainmatter
  \BookMark{-1}{Main Matter}
}
\newcommand{\BackMatter}{%
  \FlushRunningHeads
  \backmatter
  \BookMark{-1}{Back Matter}
}
\newcommand{\PGLicense}{%
  \FlushRunningHeads
  \pagenumbering{Alph}
  \BookMark{-1}{PG License}
  \fancyhead[C]{LICENSE}
  \ifthenelse{\boolean{ForPrinting}}
             {\fancyhead[RO,LE]{\thepage}}
             {\fancyhead[R]{\thepage}}
}

\newcommand{\TranscribersNote}[2][]{%
  \begin{minipage}{0.85\textwidth}
    \small
    \BookMark[#1]{0}{Note sur la transcription}
    \subsection*{\centering\normalfont\scshape\normalsize\TransNote}
    #2
  \end{minipage}
}

%%%% Table of Contents %%%%
% Contents heading
\newcommand{\Contents}{%
  \FlushRunningHeads
  \BookMark{0}{Table des Matières}
  \thispagestyle{empty}
  \section*{\centering TABLE DES MATIÈRES}
  \noindent\hfill{\footnotesize Pages.}
}

% Chapter entries
\newcommand{\ToCChap}[2]{%
  \subsection*{\centering\normalfont #1\ --- #2}
}

% \Introduction, \Article, and \Letter numbered consecutively for ToC use
\newcounter{ToCArtNo}
% \ToCArt{Title}{original page number (discarded)}
\newcommand{\ToCArt}[2]{%
  \stepcounter{ToCArtNo}%
  % ** Width (2em) must match \ToCPage width below
  \noindent\parbox[b]{\textwidth-2em}{\Strut\small\hangindent4em%
    #1\ \dotfill}\ToCPage{art:\theToCArtNo}%
}

% Page numbers
\newcommand{\ToCPage}[1]{\makebox[2em][r]{\small\pageref{#1}}}


%%%% Document Sectioning %%%%

% \Chapter{Number}{Heading title}
\newcommand{\Chapter}[2]{%
  \FlushRunningHeads
  \Label{chapter:#1}
  \addtocontents{toc}{\protect\ToCChap{#1}{#2}}
  \BookMark{0}{Chapter #1}%
  \ifthenelse{\equal{#1}{I.}}{%
    \pagenumbering{arabic}%
    \begin{center}
      \OEUVRES \\
      \rule{\textwidth}{0.5pt}
    \end{center}
  }{}%

  \section*{\centering #1\ --- #2}

  \vspace*{3\baselineskip}
}

\newcounter{ArtNo}
\newcommand{\Article}[3]{%
% [** TN: First article of Chapter 1 does not get a page break]
  \ifthenelse{\not\equal{\theArtNo}{1}}{\FlushRunningHeads}{}
  \refstepcounter{ArtNo}\Label{art:\theArtNo}%
% [** TN: Heading of "Des équations primitives..." gets special handling]
  \ifthenelse{\not\equal{#2}{}}{%
    \subsection*{\centering \normalsize #1 \\[6pt] \scriptsize #2 \\[6pt] \footnotesize #3}
  }{%
    \subsection*{\centering \normalsize #1 \\[6pt] \footnotesize #3}
  }
{\centering\TB\par}
\vspace*{\baselineskip}
}

\newcommand{\Letter}[1]{%
  \refstepcounter{ArtNo}\Label{art:\theArtNo}%
  \subsection*{\centering \normalsize #1}
}

\newcommand{\Section}[2]{%
%[** TN: Size-dependent vertical space tweaking when \Section follows \Article]
  \ifthenelse{\equal{#1}{§~I.}}{%
    \vspace{-1.5\baselineskip}%
  }{}

  \ifthenelse{\equal{#1}{}}{%
    \subsection*{\centering\normalsize #2}
  }{
    \subsection*{\centering\normalsize #1\ --- #2}
  }
}
\newcommand{\Subsection}[1]{\subsubsection*{\centering\small\normalfont #1}}

\newcommand{\Introduction}{%
  \FlushRunningHeads
  \refstepcounter{ArtNo}\Label{art:\theArtNo}%
  \section*{\centering INTRODUCTION.}
}

%%%% Other semantic units %%%%

% Template for definitions, theorems, corollaries
\newenvironment{MyEnvt}[2]{%
  \medskip\par
  \ifthenelse{\equal{#1}{}}{%
    \textsc{#2.}\ --- %
  }{%
    \textsc{#2 #1}\ --- %
  }%
  \itshape\ignorespaces
}{\medskip\normalfont}

% Document-level semantic formatting
\newenvironment{Theoreme}[1][]{\begin{MyEnvt}{#1}{Theorème}}{\end{MyEnvt}}
\newenvironment{Probleme}[1][]{\begin{MyEnvt}{#1}{Problème}}{\end{MyEnvt}}
\newenvironment{Lemme}[1][]{\begin{MyEnvt}{#1}{Lemme}}{\end{MyEnvt}}

\newenvironment{Thm}{\itshape\ignorespaces}{\normalfont}

\newenvironment{Demonstration}{%
  \medskip\par
  \textit{Démonstration.}\ --- \ignorespaces
}{\medskip\normalfont}

\newcommand{\Par}[1]{\medskip\par\textit{#1}\ ---\ \ignorespaces}

\newcommand{\Date}[2]{%
  \pagebreak[0]\par\hfill\textsc{#1}\mbox{\quad}\\
  \mbox{\quad}{\footnotesize #2}}

\newcommand{\Signature}[2]{%
  \settowidth{\TmpLen}{\footnotesize #2}%
\begin{flushright}
  \parbox{\TmpLen}{\centering \textsc{#1} \\
    \footnotesize #2}
\end{flushright}
}

%%%% Misc. textual macros %%%%
\newcommand{\OEUVRES}[1][.]{%
  \textbf{\LARGE {\OE}UVRES} \\[\baselineskip]
  \footnotesize MATHÉMATIQUES \\[\baselineskip]
  \textbf{\Huge D'ÉVARISTE GALOIS#1}
}

\newcommand{\Label}[1]{\phantomsection\label{#1}}
\newcommand{\Eq}[1]{\ensuremath{#1}}
\newcommand{\Tag}[1]{\tag*{\ensuremath{#1}}}

\newcommand{\Title}[1]{\textit{#1}}

\newcommand{\Annot}[1]{%
  \null\hfill\nobreak\textsc{#1}\mbox{\quad}%
}

%%%% Miscellaneous formatting %%%%

% Surround footnote markers with upright parentheses
\makeatletter
\renewcommand\@makefnmark%
  {\mbox{\;\upshape(\@textsuperscript{\normalfont\@thefnmark})}}

\renewcommand\@makefntext[1]%
  {\setlength{\parindent}{1em}\noindent\makebox[2.4em][r]{\@makefnmark\;}#1}
\makeatother

\newcommand{\Strut}[1][16pt]{\rule{0pt}{#1}}

\newcommand{\TB}[1][1.5cm]{\rule{#1}{0.5pt}}

\makeatletter
\def\Cfrac{\global\@tempdimb\z@\cfrac}
\def\hlap#1{{\setbox\@tempboxa\hbox{#1}\@tempdima=\wd\@tempboxa
    \global\advance\@tempdimb.45\@tempdima
    \hbox to.55\@tempdima{\unhbox\@tempboxa\hss}}}
\def\CfracCorrect{\kern\@tempdimb\relax}
\makeatother

\renewcommand{\phi}{\varphi}
%[** TN: Original uses "mod." and "mod" inconsistently]
\DeclareMathOperator{\Mod}{mod\;}
\renewcommand{\mod}{\Mod}
\newcommand{\efrac}[2]{\tfrac{#1}{#2}}

% Make upright regardless of surrounding font
\let\Primo=\primo
\let\Secundo=\secundo
\let\Tertio=\tertio
\let\Quarto=\quarto

\renewcommand{\primo}{{\upshape\Primo}}
\renewcommand{\secundo}{{\upshape\Secundo}}
\renewcommand{\tertio}{{\upshape\Tertio}}
\renewcommand{\quarto}{{\upshape\Quarto}}

% Use upright capitals, tweak math size
\DeclareMathSizes{12}{11}{9}{7}
\DeclareMathSymbol{A}{\mathalpha}{operators}{`A}
\DeclareMathSymbol{B}{\mathalpha}{operators}{`B}
\DeclareMathSymbol{C}{\mathalpha}{operators}{`C}
\DeclareMathSymbol{D}{\mathalpha}{operators}{`D}
\DeclareMathSymbol{E}{\mathalpha}{operators}{`E}
\DeclareMathSymbol{F}{\mathalpha}{operators}{`F}
\DeclareMathSymbol{G}{\mathalpha}{operators}{`G}
\DeclareMathSymbol{H}{\mathalpha}{operators}{`H}
\DeclareMathSymbol{I}{\mathalpha}{operators}{`I}
\DeclareMathSymbol{J}{\mathalpha}{operators}{`J}
\DeclareMathSymbol{K}{\mathalpha}{operators}{`K}
\DeclareMathSymbol{L}{\mathalpha}{operators}{`L}
\DeclareMathSymbol{M}{\mathalpha}{operators}{`M}
\DeclareMathSymbol{N}{\mathalpha}{operators}{`N}
\DeclareMathSymbol{O}{\mathalpha}{operators}{`O}
\DeclareMathSymbol{P}{\mathalpha}{operators}{`P}
\DeclareMathSymbol{Q}{\mathalpha}{operators}{`Q}
\DeclareMathSymbol{R}{\mathalpha}{operators}{`R}
\DeclareMathSymbol{S}{\mathalpha}{operators}{`S}
\DeclareMathSymbol{T}{\mathalpha}{operators}{`T}
\DeclareMathSymbol{U}{\mathalpha}{operators}{`U}
\DeclareMathSymbol{V}{\mathalpha}{operators}{`V}
\DeclareMathSymbol{W}{\mathalpha}{operators}{`W}
\DeclareMathSymbol{X}{\mathalpha}{operators}{`X}
\DeclareMathSymbol{Y}{\mathalpha}{operators}{`Y}
\DeclareMathSymbol{Z}{\mathalpha}{operators}{`Z}

% Handle degree symbols and centered dots as Latin-1 characters
\DeclareUnicodeCharacter{00A3}{\pounds}
\DeclareInputText{183}{\ifmmode\mathbin{.}\else\textperiodcentered\fi}

% Corrections
\newcommand{\DPtypo}[2]{#2}

%%%%%%%%%%%%%%%%%%%%%%%% START OF DOCUMENT %%%%%%%%%%%%%%%%%%%%%%%%%%
\begin{document}
\FrontMatter
%%%% PG BOILERPLATE %%%%
\PGBoilerPlate
\begin{center}
\begin{minipage}{\textwidth}
\small
\begin{PGtext}
The Project Gutenberg EBook of Oeuvres mathématiques d'Évariste Galois, by
Évariste Galois

This eBook is for the use of anyone anywhere at no cost and with
almost no restrictions whatsoever.  You may copy it, give it away or
re-use it under the terms of the Project Gutenberg License included
with this eBook or online at www.gutenberg.org


Title: Oeuvres mathématiques d'Évariste Galois

Author: Évariste Galois

Editor: Société mathématique de France
        Émile Picard

Release Date: July 11, 2012 [EBook #40213]
Most recently updated: June 11, 2021

Language: French

Character set encoding: UTF-8

*** START OF THIS PROJECT GUTENBERG EBOOK OEUVRES MATHÉMATIQUES ***
\end{PGtext}
\end{minipage}
\end{center}
\clearpage

%%%% Credits and transcriber's note %%%%
\begin{center}
\begin{minipage}{\textwidth}
\begin{PGtext}
Produced by Andrew D. Hwang, K. F. Greiner, Paul Murray
and the Online Distributed Proofreading Team at
http://www.pgdp.net
\end{PGtext}
\end{minipage}
\vfill
\TranscribersNote{\TransNoteText}
\end{center}
%%%%%%%%%%%%%%%%%%%%%%%%%%% FRONT MATTER %%%%%%%%%%%%%%%%%%%%%%%%%%
%% -----File: 001.png---Folio iii-------
\cleardoublepage
\pagenumbering{Roman}
\null\vfil
\begin{center}
% ** Macro (see also top of next page) prints
% {\OE}UVRES
% MATHÉMATIQUES
% D'ÉVARISTE GALOIS.
\OEUVRES
\end{center}
\vfil
%% -----File: 002.png---Folio iv-------
\newpage
\begin{center}
\OEUVRES[,]
\vfill

\footnotesize PUBLIÉES \\[6pt]
\small SOUS LES AUSPICES DE LA SOCIÉTÉ MATHÉMATIQUE DE FRANCE,
\vfill

\large AVEC UNE INTRODUCTION \\[6pt]
\footnotesize PAR \\[6pt]
\large M. \textsc{Émile PICARD}, \\[6pt]
\footnotesize MEMBRE DE L'INSTITUT.
\vfill

%Illustration

\large PARIS, \\[6pt]
\normalsize GAUTHIER-VILLARS ET FILS, IMPRIMEURS-LIBRAIRES \\[6pt]
\footnotesize DU BUREAU DES LONGITUDES, DE L'ÉCOLE POLYTECHNIQUE, \\[6pt]
Quai des Grands-Augustins, 55. \\[6pt]
\TB[0.5cm] \\[6pt]
\normalsize 1897 \\[6pt]
\footnotesize (Tous droits réservés.)
\end{center}
%% -----File: 003.png---Folio v-------

\Introduction

Les {\OE}uvres de Galois ont, comme on sait, été publiées
en 1846 par Liouville, dans le \Title{Journal de Mathématiques}.
Il était regrettable que l'on ne pût posséder à part les
{\OE}uvres du grand géomètre; aussi la Société mathématique
a-t-elle décidé de faire réimprimer les Mémoires de Galois.
Cette édition est conforme à la précédente; on a seulement
supprimé l'avertissement placé par Liouville au début de la
publication.

Un travail, qui paraît définitif, sur la vie de Galois vient
d'être publié par M.~Paul Dupuy, dans les \Title{Annales de
l'École Normale supérieure} (1896). Comme documents
antérieurs relatifs à la vie de Galois, il faut citer la Notice
nécrologique que lui consacra son ami Auguste Chevalier,
dans la \Title{Revue encyclopédique} (septembre 1832), et un
article paru dans le \Title{Magasin pittoresque}, en 1848. Évariste
Galois est né à Bourg-la-Reine, près de Paris, le
25~octobre 1811; il quitta la maison paternelle en 1823,
pour entrer en quatrième au collège Louis-le-Grand. Dès
l'âge de quinze ans, ses dispositions extraordinaires pour les
Sciences mathématiques commencent à se manifester; les
livres élémentaires d'Algèbre ne le satisfont pas, et c'est dans
les Ouvrages classiques de Lagrange qu'il fait son éducation
algébrique. Il semble qu'à dix-sept ans Galois avait déjà
obtenu des résultats de la plus haute importance concernant
la théorie des équations algébriques. On ne peut faire que
%% -----File: 004.png---Folio xx-------
des conjectures sur la marche de ses idées, les deux Mémoires
qu'il présenta à l'Académie des Sciences ayant été
perdus; une chose toutefois est certaine: il était, au commencement
de 1830, en possession de ses principes généraux,
comme le montre l'analyse d'un Mémoire sur la résolution
algébrique des équations dans le \Title{Bulletin de
Férussac}, où sont énoncés une série de résultats qui ne sont
manifestement que des applications d'une théorie générale.
Ce court article est le plus important qui ait été publié par
Galois lui-même, le Mémoire fondamental sur l'Algèbre retrouvé
dans ses papiers n'ayant été imprimé qu'en 1846.

On trouvera, dans l'article de M.~Dupuy, des renseignements
d'un grand intérêt sur la vie de Galois. Il est peu probable
que de nouveaux documents viennent désormais
s'ajouter à ceux que nous possédons maintenant. Après
deux échecs à l'École Polytechnique, Galois entra à l'École
Normale en 1829 et fut obligé de la quitter l'année suivante.
Dans la dernière année de sa vie, il se donna tout entier à la
politique, passa plusieurs mois sous les verrous de Sainte-Pélagie
et, blessé mortellement en duel, mourut le 31~mai
1832. En présence d'une vie si courte et si tourmentée, l'admiration
redouble pour le génie prodigieux qui a laissé dans
la Science une trace aussi profonde; les exemples de productions
précoces ne sont pas rares chez les grands géomètres,
mais celui de Galois est remarquable entre tous.
Il semble, hélas! que le malheureux jeune homme ait tristement
payé la rançon de son génie. A mesure que se \DPtypo{déveveloppent}{développent}
ses brillantes facultés mathématiques, on voit
s'assombrir son caractère, autrefois gai et ouvert, et le
sentiment de son immense supériorité développe chez lui un
orgueil excessif. Ce fut la cause des déceptions qui eurent
tant d'influence sur sa carrière, et dont la première fut son
échec à l'École Polytechnique. Son examen dans cette
%% -----File: 005.png---Folio xx-------
École a laissé des souvenirs; sans aller aussi loin que le
veut la légende, disons seulement que Galois refusa de répondre
à une question, qu'il jugeait ridicule, sur la théorie
arithmétique des logarithmes. On ne peut douter aussi qu'il
ne se soit pas prêté à fournir sur ses travaux les explications
que lui demandaient les mathématiciens avec qui il s'est
trouvé en relations, explications que rendait nécessaires la
rédaction rapide de ses Mémoires; aussi comprend-on facilement
que son mérite n'ait pas été reconnu de ses contemporains.
Ce n'est pas sans peine que Liouville réussit à saisir
l'enchaînement des idées de Galois, et il fallut encore de
nombreux commentateurs pour combler les lacunes qui subsistaient
dans plus d'une démonstration, et amener les théories
du grand géomètre au degré de simplicité qu'elles sont
susceptibles de revêtir aujourd'hui.

La théorie des équations doit à Lagrange, Gauss et Abel
des progrès considérables, mais aucun d'eux n'arriva à
mettre en évidence l'élément fondamental dont dépendent
toutes les propriétés de l'équation; cette gloire était réservée
à Galois, qui montra qu'à chaque équation algébrique correspond
un groupe de substitutions dans lequel se reflètent
les caractères essentiels de l'équation. En Algèbre, la théorie
des groupes avait fait auparavant l'objet de nombreuses recherches
dues, pour la plupart, à Cauchy, qui avait introduit
déjà certains éléments de classification; les études de
Galois sur la Théorie des équations lui montrèrent l'importance
de la notion de sous-groupe invariant d'un groupe
donné, et il fut ainsi conduit à partager les groupes en
groupes simples et groupes composés, distinction fondamentale
qui dépasse de beaucoup, en réalité, le domaine de
l'Algèbre et s'étend au concept de groupes d'opérations dans
son acception la plus étendue.

Les théories générales, pour prendre dans la Science un
%% -----File: 006.png---Folio xx-------
droit de cité définitif, ont le plus souvent besoin de s'illustrer
par des applications particulières. Dans plusieurs domaines,
celles-ci ne sont pas toujours faciles à trouver, et l'on pourrait
citer, dans les Mathématiques modernes, plus d'une théorie
confinée, si j'ose le dire, dans sa trop grande généralité;
au point de vue artistique, qui joue un rôle capital dans les
Mathématiques pures, rien n'est plus satisfaisant que le développement
de ces grandes théories, cependant de bons
esprits regrettent cette tendance, qui a peut-être ses dangers.
On ne peut, pour Galois, émettre un pareil regret; la résolution
algébrique des équations lui a fourni, dès le début, un
champ particulier d'applications où le suivirent depuis de
nombreux géomètres, parmi lesquels il faut citer au premier
rang M.~Camille Jordan.

Les travaux de Galois, sur les équations algébriques, ont
rendu son nom célèbre, mais il semble qu'il avait fait, en
Analyse, des découvertes au moins aussi importantes. Dans
sa lettre à Auguste Chevalier, écrite la veille de sa mort, et
qui est une sorte de testament scientifique, Galois parle d'un
Mémoire qu'on pourrait composer avec ses recherches sur
les intégrales. Nous ne connaissons de ces recherches que ce
qu'il en dit dans cette lettre; plusieurs points restent obscurs
dans quelques énoncés de Galois, mais on peut cependant se
faire une idée précise de quelques-uns des résultats auxquels
il était arrivé dans la théorie des intégrales de fonctions algébriques.
On acquiert ainsi la conviction qu'il était en possession
des résultats les plus essentiels sur les intégrales abéliennes
que Riemann devait obtenir vingt-cinq ans plus tard.
Nous ne voyons pas sans étonnement Galois parler des périodes
d'une intégrale abélienne relative à une fonction
algébrique quelconque; pour les intégrales hyperelliptiques,
nous n'avons aucune difficulté à comprendre ce qu'il entend
par \emph{période}, mais il en est autrement dans le cas général, et
%% -----File: 007.png---Folio xx-------
l'on est presque tenté de supposer que Galois avait tout au
moins pressenti certaines notions sur les fonctions d'une variable
complexe, qui ne devaient être développées que plusieurs
années après sa mort. Les énoncés sont précis;
l'illustre auteur fait la classification en trois espèces des intégrales
abéliennes, et affirme que, si $n$~désigne le nombre des
intégrales de première espèce linéairement indépendantes,
les périodes seront en nombre~$2n$. Le théorème relatif à
l'inversion du paramètre et de l'argument dans les intégrales
de troisième espèce est nettement indiqué, ainsi que
les relations entre les périodes des intégrales abéliennes;
Galois parle aussi d'une généralisation de l'équation classique
de Legendre, où figurent les périodes des intégrales
elliptiques, généralisation qui l'avait probablement conduit
aux importantes relations découvertes depuis par
Weierstrass et par M.~Fuchs. Nous en avons dit assez pour
montrer l'étendue des découvertes de Galois en Analyse;
si quelques années de plus lui avaient été données pour développer
ses idées dans cette direction, il aurait été le glorieux
continuateur d'Abel et aurait édifié, dans ses parties
essentielles, la théorie des fonctions algébriques d'une
variable telle que nous la connaissons aujourd'hui. Les
méditations de Galois portèrent encore plus loin; il termine
sa lettre en parlant de l'application à l'Analyse transcendante
de la théorie de l'ambiguïté. On devine à peu près
ce qu'il entend par là, et sur ce terrain qui, comme il le dit,
est immense, il reste encore aujourd'hui bien des découvertes
à faire.

Ce n'est pas sans émotion que l'on achève la lecture du
testament scientifique de ce jeune homme de vingt ans, écrit
la veille du jour où il devait disparaître dans une obscure
querelle. Sa mort fut pour la Science une perte immense;
l'influence de Galois, s'il eût vécu, aurait grandement modifié
%% -----File: 008.png---Folio xx-------
l'orientation des recherches mathématiques dans notre
pays. Je ne me risquerai pas à des comparaisons périlleuses:
Galois a sans doute des égaux parmi les grands mathématiciens
de ce siècle; aucun ne le surpasse par l'originalité et la
profondeur de ses conceptions.

\Signature{Émile Picard,}{Président de la Société mathématique de France.}
%% -----File: 009.png---Folio 1-------
\MainMatter
% ** Text is printed by the \Chapter macro
% {\OE}UVRES
% MATHÉMATIQUES
% D'ÉVARISTE GALOIS.

\Chapter{I.}{ARTICLES PUBLIÉS PAR GALOIS.}

\Article{DÉMONSTRATION}{D'UN}{THÉORÈME SUR LES FRACTIONS CONTINUES PÉRIODIQUES\footnotemark.}

\footnotetext{Tome~XIX des \Title{Annales de Mathématiques} de M.~Gergonne, page~294
  (1828--1829). Galois était alors élève au collège Louis-le-Grand. \Annot{(J.~Liouville.)}}

On sait que si, par la méthode du Lagrange, on développe en
fraction continue une des racines d'une équation du second degré,
cette fraction continue sera périodique, et qu'il en sera encore de
même de l'une des racines d'une équation de degré quelconque,
si cette racine est racine d'un facteur rationnel du second degré
du premier membre de la proposée, auquel cas cette équation
aura, tout au moins, une autre racine qui sera également périodique.
Dans l'un et dans l'autre cas, la fraction continue pourra
d'ailleurs être immédiatement périodique ou ne l'être pas immédiatement;
%% -----File: 010.png---Folio 2-------
mais, lorsque cette dernière circonstance aura lieu, il
y aura du moins une des transformées dont une des racines sera
immédiatement périodique.

Or, lorsqu'une équation a deux racines périodiques répondant
à un même facteur rationnel du second degré, et que l'une d'elles
est immédiatement périodique, il existe entre ces deux racines
une relation assez singulière qui paraît n'avoir pas encore été remarquée,
et qui peut être exprimée par le théorème suivant:

\begin{Theoreme}
Si une des racines d'une équation de degré
quelconque est une fraction continue immédiatement périodique,
cette équation aura nécessairement une autre racine
également périodique que l'on obtiendra en divisant l'unité
négative par cette même fraction continue périodique, écrite
dans un ordre inverse.
\end{Theoreme}

\begin{Demonstration}
Pour fixer les idées, ne prenons que des
périodes de quatre termes; car la marche uniforme du calcul
prouve qu'il en serait de même si nous en admettions un plus
grand nombre. Soit une des racines d'une équation de degré quelconque
exprimée comme il suit;
\[
a - x = a + \Cfrac{1}{
      \,b + \hlap{$\cfrac{1}{
      \,c + \hlap{$\cfrac{1}{
      \,d + \hlap{$\cfrac{1}{
      \,a + \hlap{$\cfrac{1}{
      \,b + \hlap{$\cfrac{1}{
      \,c + \hlap{$\cfrac{1}{\Strut
      \,d + \raisebox{-2ex}{$\ddots$\rlap{\,;}} }$}}\;$}}\;$}}\;$}}\;$}}\;$}}\CfracCorrect
\]
l'équation du second degré, à laquelle appartiendra cette racine,
et qui contiendra conséquemment sa corrélative, sera
\[
x = a + \Cfrac{1}{
  \,b + \hlap{$\cfrac{1}{
  \,c + \hlap{$\cfrac{1}{
  \,d + \hlap{$\cfrac{1}{\quad x \quad } \;\rlap{;}$}}\;$}}\;$}}\CfracCorrect
\]
%% -----File: 011.png---Folio 3-------
or, on tire de là successivement
\begin{align*}
&a - x = - \Cfrac{1}{
      \,b + \hlap{$\cfrac{1}{
      \,c + \hlap{$\cfrac{1}{
      \,d + \hlap{$\cfrac{1}{\quad x \quad } \;\rlap{,}$}}\;$}}\;$}}\CfracCorrect \qquad
&&\frac{1}{a-x} =
      - b + \Cfrac{1}{
      \,c + \hlap{$\cfrac{1}{
      \,d + \hlap{$\cfrac{1}{\quad x \quad } \;\rlap{,}$}}\;$}}\CfracCorrect
\displaybreak[0] \\[\baselineskip]
& b + \frac{1}{a-x} =
          - \Cfrac{1}{
      \,c + \hlap{$\cfrac{1}{
      \,d + \hlap{$\cfrac{1}{\quad x \quad } \;\rlap{,}$}}\;$}}\CfracCorrect
&&\Cfrac{1}{
      \,b + \hlap{$\cfrac{1}{
      \,a - x \, }\;$}}\CfracCorrect
=     - c + \Cfrac{1}{
      \,d + \hlap{$\cfrac{1}{\quad x \quad } \;\rlap{,}$}}\CfracCorrect
\displaybreak[0] \\[\baselineskip]
&c + \Cfrac{1}{
      \,b + \hlap{$\cfrac{1}{
      \,a - x \, }\;$}}\CfracCorrect
=         - \Cfrac{1}{
      \,d + \hlap{$\cfrac{1}{\quad x \quad } \;\rlap{,}$}}\CfracCorrect
&&\Cfrac{1}{
      \,c + \hlap{$\cfrac{1}{
      \,b + \hlap{$\cfrac{1}{
      \,a - x \, }\;$}}\;$}}\CfracCorrect
=     -d + \frac{1}{x} \,,
\displaybreak[0] \\[\baselineskip]
&       d + \Cfrac{1}{
      \,c + \hlap{$\cfrac{1}{
      \,b + \hlap{$\cfrac{1}{
      \,a - x \, }$}}\;$}}\CfracCorrect
=         - \frac{1}{x} \,;
&&\Cfrac{1}{
      \,d + \hlap{$\cfrac{1}{
      \,c + \hlap{$\cfrac{1}{
      \,b + \hlap{$\cfrac{1}{
      \,a - x \, }$}}\;$}}\;$}}\CfracCorrect
=     -x\,,
\end{align*}
c'est-à-dire
\[
  x =     - \Cfrac{1}{
      \,d + \hlap{$\cfrac{1}{
      \,c + \hlap{$\cfrac{1}{
      \,b + \hlap{$\cfrac{1}{
      \,a - x\, } \;\rlap{;}$}}\;$}}$}}\CfracCorrect
\]
c'est donc toujours là l'équation du second degré qui donne les
deux racines dont il s'agit; mais, en mettant continuellement
pour $x$, dans son second membre, ce même second membre, qui
en est, en effet, la valeur, elle donne
\[
  x =     - \Cfrac{1}{
      \,d + \hlap{$\cfrac{1}{
      \,c + \hlap{$\cfrac{1}{
      \,b + \hlap{$\cfrac{1}{
      \,a + \hlap{$\cfrac{1}{
      \,d + \hlap{$\cfrac{1}{
      \,c + \hlap{$\cfrac{1}{
      \,b + \hlap{$\cfrac{1}{
      \,a + \raisebox{-2ex}{$\ddots$\rlap{\,;}} }\;$}}\;$}}\;$}}\;$}}\;$}}\;$}}\;$}}
\CfracCorrect
\]
c'est donc là l'autre valeur de $x$, donnée par cette équation, valeur
qui, comme l'on voit, est égale à $-1$ divisé par la première écrite
dans un ordre inverse.
%% -----File: 012.png---Folio 4-------

Dans ce qui précède, nous avons supposé que la racine proposée
était plus grande que l'unité; mais, si l'on avait
\[
  x =       \Cfrac{1}{
      \,a + \hlap{$\cfrac{1}{
      \,b + \hlap{$\cfrac{1}{
      \,c + \hlap{$\cfrac{1}{
      \,d + \hlap{$\cfrac{1}{
      \,a + \hlap{$\cfrac{1}{
      \,b + \hlap{$\cfrac{1}{
      \,c + \hlap{$\cfrac{1}{
      \,d + \raisebox{-2ex}{$\ddots$\rlap{\,;}} }$}}\;$}}\;$}}\;$}}\;$}}\;$}}\;$}}
\CfracCorrect
\]
on en conclurait, pour une des valeurs de~$\dfrac{1}{x}$,
\[
  \frac{1}{x} =   a + \Cfrac{1}{
      \,b + \hlap{$\cfrac{1}{
      \,c + \hlap{$\cfrac{1}{
      \,d + \hlap{$\cfrac{1}{
      \,a + \hlap{$\cfrac{1}{
      \,b + \hlap{$\cfrac{1}{
      \,c + \hlap{$\cfrac{1}{
      \,d + \raisebox{-2ex}{$\ddots$\rlap{\,;}} }$}}\;$}}\;$}}\;$}}\;$}}\;$}}
\CfracCorrect
\]
l'autre valeur de $\dfrac{1}{x}$ serait donc, par ce qui précède,
\[
  \frac{1}{x} =   - \Cfrac{1}{
      \,d + \hlap{$\cfrac{1}{
      \,c + \hlap{$\cfrac{1}{
      \,b + \hlap{$\cfrac{1}{
      \,a + \hlap{$\cfrac{1}{
      \,d + \hlap{$\cfrac{1}{
      \,c + \hlap{$\cfrac{1}{
      \,b + \hlap{$\cfrac{1}{
      \,a + \raisebox{-2ex}{$\ddots$\rlap{\,,}} }$}}\;$}}\;$}}\;$}}\;$}}\;$}}\;$}}
\CfracCorrect
\]
d'où l'on conclurait, pour l'autre valeur de $x$,
\[
  x = - d + \Cfrac{1}{
      \,c + \hlap{$\cfrac{1}{
      \,b + \hlap{$\cfrac{1}{
      \,a + \hlap{$\cfrac{1}{
      \,d + \hlap{$\cfrac{1}{
      \,c + \hlap{$\cfrac{1}{
      \,b + \hlap{$\cfrac{1}{
      \,a + \raisebox{-2ex}{$\ddots$} }$}}\;$}}\;$}}\;$}}\;$}}\;$}}
\CfracCorrect
\]
%% -----File: 013.png---Folio 5-------
ou
\[
  x = - 1 : \Cfrac{1}{
      \,d + \hlap{$\cfrac{1}{
      \,c + \hlap{$\cfrac{1}{
      \,b + \hlap{$\cfrac{1}{
      \,a + \hlap{$\cfrac{1}{
      \,d + \hlap{$\cfrac{1}{
      \,c + \hlap{$\cfrac{1}{
      \,b + \hlap{$\cfrac{1}{
      \,a + \raisebox{-2ex}{$\ddots$\rlap{\,;}}}$}}\;$}}\;$}}\;$}}\;$}}\;$}}\;$}}
\CfracCorrect
\]
ce qui rentre exactement dans notre théorème.
\end{Demonstration}

Soit $A$ une fraction continue immédiatement périodique quelconque,
et soit $B$ la fraction continue qu'on en déduit en renversant
la période; on voit que, si l'une des racines d'une équation
est $x = A$, elle aura nécessairement une autre racine $x = -\dfrac{1}{B}$;
or, si $A$ est un nombre positif plus grand que l'unité, $-\dfrac{1}{B}$~sera
négatif et compris entre $0$ et~$-1$; et, à l'inverse, si $A$ est un
nombre négatif compris entre $0$ et~$-1$, $-\dfrac{1}{B}$~sera un nombre positif
plus grand que l'unité. Ainsi, lorsque l'une des racines d'une
équation du second degré est une fraction continue immédiatement
périodique, plus grande que l'unité, l'autre est nécessairement
comprise entre $0$ et $-1$; et réciproquement, si l'une d'elles
est comprise entre $0$ et $-1$, l'autre sera nécessairement positive
et plus grande que l'unité.

On peut prouver que, réciproquement, si l'une des deux racines
d'une équation du second degré est positive et plus grande
que l'unité, et que l'autre soit comprise entre $0$ et $-1$, ces racines
seront exprimables en fractions continues immédiatement périodiques.
En effet, soit toujours $A$ une fraction continue immédiatement
périodique quelconque, positive et plus grande que
l'unité, et $B$ la fraction continue immédiatement périodique qu'on
en déduit, en renversant la période, laquelle sera aussi, comme
elle, positive et plus grande que l'unité. La première des racines
de la proposée ne pourra être de la forme
\[
  x = p + \frac{1}{A},
\]
%% -----File: 014.png---Folio 6-------
car alors, en vertu de notre théorème, la seconde devrait être
\[
  x = a + \frac{1}{-\dfrac{1}{B}} = a - B ;
\]
or $a - B$ ne saurait être compris entre 0 et 1 qu'autant que la
partie entière de $B$ serait égale à $p$, auquel cas la première valeur
serait immédiatement périodique. On ne pourrait avoir davantage,
pour la première valeur de $x$, $x= p + \dfrac{1}{q + \dfrac{1}{A}}$, car alors l'autre
serait
\[
  x = p + \frac{1}{q - B} \qquad \text{ou} \qquad
  x = p - \frac{1}{B - q};
\]
or, pour que cette valeur fût comprise entre 0 et $-1$, il faudrait
d'abord que $\dfrac{1}{B-q}$ fût égal à $p$, plus une fraction. Il faudrait
donc que $B-q$ fût plus petit que l'unité, ce qui exigerait que $B$
fût égal à $q$, plus une fraction; d'où l'on voit que $q$ et $p$ devraient
être respectivement égaux aux deux premiers termes de la période
qui répond à $B$, ou aux deux derniers de la période qui répond à $A$;
de sorte que, contrairement à l'hypothèse, la valeur $x = p + \dfrac{1}{q + \dfrac{1}{A}}$
serait immédiatement périodique. On prouverait, par un raisonnement
analogue, que les périodes ne sauraient être précédées
d'un plus grand nombre de termes n'en faisant pas partie.

Lors donc que l'on traitera une équation numérique par la méthode
de Lagrange, on sera sûr qu'il n'y a point de racines périodiques
à espérer tant qu'on ne rencontrera pas une transformée
ayant au moins une racine positive plus grande que l'unité, et une
autre comprise entre 0 et $-1$; et si, en effet, la racine que l'on
poursuit doit être périodique, ce sera tout au plus à cette transformée
que les périodes commenceront.

Si l'une des racines d'une équation du second degré est non
seulement immédiatement périodique, mais encore symétrique,
c'est-à-dire si les termes de la période sont égaux à égale distance
des extrêmes, on aura $B = A$; de sorte que ces deux racines
%% -----File: 015.png---Folio 7-------
seront $A$ et $-\dfrac{1}{A}$; l'équation sera donc
\[
Ax^{2} - (A^{2}-1)x - A = 0.
\]
Réciproquement, toute équation du second degré de la forme
\[
ax^{2} - bx - a = 0
\]
aura ses racines à la fois immédiatement périodiques et symétriques.
En effet, on mettant tour à tour pour $x$ l'infini et $-1$, on
obtient des résultats positifs, tandis qu'en faisant $x=1$ et $x=0$,
on obtient des résultats négatifs; d'où l'on voit d'abord que cette
équation a une racine positive plus grande que l'unité et une racine
négative comprise entre $0$ et $-1$, et qu'ainsi ces racines sont
immédiatement périodiques; de plus, cette équation ne change
pas en y changeant $x$ en $-\dfrac{1}{x}$; d'où il suit que, si $A$ est une de ses
racines, l'autre sera $-\dfrac{1}{A}$, et qu'ainsi, dans ce cas, $B=A$.

Appliquons ces généralités à l'équation du second degré
\[
3x^{2} - 16x + 18 = 0;
\]
on lui trouve d'abord une racine positive comprise entre $3$ et $4$;
en posant
\[
x = 3 + \frac{1}{y},
\]
on obtient la transformée
\[
3y^{2} - 2y - 3 = 0,
\]
dont la forme nous apprend que les valeurs de $y$ sont à la fois
immédiatement périodiques et symétriques; en effet, en posant
tour à tour
\[
y = 1 + \frac{1}{z}, \qquad
z = 2 + \frac{1}{t}, \qquad
t = 1 + \frac{1}{u},
\]
on obtient les transformées
\[
2z^{2} - 4z - 3 = 0, \qquad
3t^{2} - 4t - 2 = 0, \qquad
3u^{2} - 2u - 3 = 0.
\]
L'identité entre les équations en $u$ et en $y$ prouve que la valeur
%% -----File: 016.png---Folio 8-------
positive de $y$ est
\[
  y =   1 + \Cfrac{1}{
      \,2 + \hlap{$\cfrac{1}{
      \,1 + \hlap{$\cfrac{1}{
      \,1 + \hlap{$\cfrac{1}{
      \,2 + \hlap{$\cfrac{1}{
      \,1 + \raisebox{-2ex}{$\ddots$\rlap{\,;}} }$}}\;$}}\;$}}\;$}}
\CfracCorrect
\]
sa valeur négative sera donc
\[
  y =     - \Cfrac{1}{
      \,1 + \hlap{$\cfrac{1}{
      \,2 + \hlap{$\cfrac{1}{
      \,1 + \hlap{$\cfrac{1}{
      \,1 + \hlap{$\cfrac{1}{
      \,2 + \hlap{$\cfrac{1}{
      \,1 + \raisebox{-2ex}{$\ddots$\rlap{\,;}} }$}}\;$}}\;$}}\;$}}\;$}}
\CfracCorrect
\]
les deux valeurs de $x$ seront donc
%[** TN: Slightly wide]
\[
  x =   3 + \Cfrac{1}{
      \,1 + \hlap{$\cfrac{1}{
      \,2 + \hlap{$\cfrac{1}{
      \,1 + \hlap{$\cfrac{1}{
      \,1 + \hlap{$\cfrac{1}{
      \,2 + \hlap{$\cfrac{1}{
      \,1 + \raisebox{-2ex}{$\ddots$\rlap{\,,}} }$}}\;$}}\;$}}\;$}}\;$}}
\CfracCorrect
  x =   3 - \Cfrac{1}{
      \,1 + \hlap{$\cfrac{1}{
      \,2 + \hlap{$\cfrac{1}{
      \,1 + \hlap{$\cfrac{1}{
      \,1 + \hlap{$\cfrac{1}{
      \,2 + \hlap{$\cfrac{1}{
      \,1 + \raisebox{-2ex}{$\ddots$\rlap{\,,}} }$}}\;$}}\;$}}\;$}}\;$}}
\CfracCorrect
\]
dont la dernière, en vertu de la formule connue
\[
  p - \frac{1}{q}
= p -   1 + \Cfrac{1}{
      \,1 + \hlap{$\cfrac{1}{q-1}\, ,$}}
\CfracCorrect
\]
devient
\[
  x =   1 + \Cfrac{1}{
      \,1 + \hlap{$\cfrac{1}{
      \,1 + \hlap{$\cfrac{1}{
      \,1 + \hlap{$\cfrac{1}{
      \,1 + \hlap{$\cfrac{1}{
      \,3 + \hlap{$\cfrac{1}{
      \,1 + \hlap{$\cfrac{1}{
      \,1 + \hlap{$\cfrac{1}{
      \,2 + \hlap{$\cfrac{1}{
      \,1 + \raisebox{-2ex}{$\ddots$\rlap{\ .}} }$}}\;$}}\;$}}\;$}}\;$}}\;$}}\;$}}\;$}}
\CfracCorrect
\]
%% -----File: 017.png---Folio 9-------

\Article{NOTES}{SUR}{QUELQUES POINTS D'ANALYSE\footnotemark.}

\footnotetext{\Title{Annales de Mathématiques} de M.~Gergonne, tome~XXI, page~182 (1830--1831).
  C'est par suite d'une faute d'impression qu'on y lit: \Title{Galais, élève à l'École
  normale}, au lieu de \textit{Galois}. \Annot{(J.~Liouville.)}}


\Section{§~I.}{Démonstration d'un théorème d'Analyse.}

\begin{Theoreme}
Soient $Fx$ et $fx$ deux fonctions quelconques
données; on aura, quels que soient $x$ et~$h$,
\[
\frac{F(x + h) - Fx}{f(x + h) - fx} = \phi(k),
\]
$\phi$ étant une fonction déterminée, et $k$ une quantité intermédiaire
entre $x$ et~$x + h$.
\end{Theoreme}

\begin{Demonstration}
Posons, en effet,
\[
\frac{F(x + h) - Fx}{f(x + h) - fx} = P;
\]
on en déduira
\[
F(x + h) - Pf(x + h) = Fx - Pfx;
\]
d'où l'on voit que la fonction $Fx - Pfx$ ne change pas quand on
y change $x$ en $x+h$; d'où il suit qu'à moins qu'elle ne reste
constante entre ces limites, ce qui ne pourrait avoir lieu que dans
des cas particuliers, cette fonction aura, entre $x$ et $x+h$, un ou
plusieurs maxima et minima. Soit $k$ la valeur de $x$ répondant à
l'un d'eux; on aura évidemment $k = \psi(P)$, $\psi$ étant une fonction
déterminée; donc on doit avoir aussi $P = \phi(k)$, $\phi$~étant une autre
fonction également déterminée; ce qui démontre le théorème.
\end{Demonstration}

De là on peut conclure, comme corollaire, que la quantité
\[
%[** TN: Original uses "lim."]
\lim \frac{F(x + h) - Fx}{f(x + h) - fx} = \psi(x),
\]
pour $h = 0$, est nécessairement une fonction de~$x$, ce qui démontre,
\textit{a~priori}, l'existence des fonctions dérivées.
%% -----File: 018.png---Folio 10-------

\Section{§~II.}{Rayon de courbure des courbes de l'espace.}

Le rayon de courbure d'une courbe en l'un quelconque de ses
points~$M$ est la perpendiculaire abaissée de ce point sur l'intersection
du plan normal au point~$M$ avec le plan normal consécutif,
comme il est aisé de s'en assurer par des considérations
géométriques.

Cela posé, soit $(x, y, z)$ un point de la courbe; on sait que le
plan normal en ce point aura pour équation
\[
\Tag{(N)}
(X - x) \frac{dx}{ds} + (Y - y) \frac{dy}{ds} + (Z - z) \frac{dz}{ds} = 0,
\]
$X$, $Y$, $Z$ étant les symboles des coordonnées courantes. L'intersection
de ce plan normal avec le plan normal consécutif sera
donnée par le système de cette équation et de la suivante
\[
\Tag{(I)}
(X - x) \frac{d · \left(\dfrac{dx}{ds}\right)}{ds} +
(Y - y) \frac{d · \left(\dfrac{dy}{ds}\right)}{ds} +
(Z - z) \frac{d · \left(\dfrac{dz}{ds}\right)}{ds}
= 1,
\]
attendu que
\[
\left(\frac{dx}{ds}\right)^{2} +
\left(\frac{dy}{ds}\right)^{2} +
\left(\frac{dz}{ds}\right)^{2}
= 1.
\]
Or, il est aisé de voir que le plan~\Eq{(I)} est perpendiculaire an
plan~\Eq{(N)}, car on a
\[
\frac{dx}{ds}\, d· \left(\frac{dx}{ds}\right) +
\frac{dy}{ds}\, d· \left(\frac{dy}{ds}\right) +
\frac{dz}{ds}\, d· \left(\frac{dz}{ds}\right)
= 0;
\]
donc la perpendiculaire abaissée du point $(x, y, z)$ sur l'intersection
des deux plans \Eq{(N)}~et~\Eq{(I)} n'est autre chose que la perpendiculaire
abaissée du même point sur le plan~\Eq{(I)}. Le rayon de
courbure est donc la perpendiculaire abaissée du point $(x, y, z)$
sur le plan~\Eq{(I)}. Cette considération donne, très simplement, les
théorèmes connus sur les rayons de courbure des courbes dans
l'espace.
%% -----File: 019.png---Folio 11-------


\Article{ANALYSE}{D'UN}{MÉMOIRE SUR LA RÉSOLUTION ALGÉBRIQUE DES ÉQUATIONS\footnotemark.}

\footnotetext{\Title{Bulletin des Sciences mathématiques} de M.~Férussac, t.~XIII, p.~271
  (année 1830, cahier d'avril). \Annot{(J.~Liouville.)}}

On appelle équations non primitives les équations qui, étant,
par exemple, du degré~$mn$, se décomposent en $m$~facteurs du
degré~$n$, au moyen d'une seule équation du degré~$m$. Ce sont les
équations de M.~Gauss. Les équations primitives sont celles qui
ne jouissent pas d'une pareille simplification. Je suis, à l'égard
des équations primitives, parvenu aux résultats suivants:

\primo Pour qu'une équation de degré premier soit résoluble par
radicaux, il faut et il suffit que, deux quelconques de ses racines
étant connues, les autres s'en déduisent rationnellement.

\secundo Pour qu'une équation primitive du degré~$m$ soit résoluble
par radicaux, il faut que $m = p^{\nu}$, $p$~étant un nombre premier.

\tertio A part les cas mentionnés ci-dessous, pour qu'une équation
primitive du degré $p^{\nu}$ soit résoluble par radicaux, il faut que,
deux quelconques de ses racines étant connues, les autres s'en
déduisent rationnellement.
\medskip

A la règle précédente échappent les cas très particuliers qui
suivent:

\primo Le cas de $m = p^{\nu} = 9$, $= 25$;

\secundo Le cas de $m = p^{\nu} = 4$ et généralement celui où, $a^{\alpha}$~étant un
diviseur de $\dfrac{p^{\nu} - 1}{p - 1}$, on aurait $a$ premier, et
\[
\frac{p^{\nu} - 1}{a^{\alpha} (p - 1)} \nu = p\ (\mod a^{\alpha}).
\]
Ces cas s'écartent toutefois fort peu de la règle générale.

Quand $m = 9$, $= 25$, l'équation devra être du genre de celles
qui déterminent la trisection et la quintisection des fonctions
elliptiques.
%% -----File: 020.png---Folio 12-------

Dans le second cas, il faudra toujours que, deux des racines
étant connues, les autres s'en déduisent, du moins au moyen d'un
nombre de radicaux, du degré~$p$, égal au nombre des diviseurs~$a^{\alpha}$
de~$\dfrac{p^{\nu} - 1}{p - 1}$, qui sont tels que
\[
\frac{p^{\nu} - 1}{a^{\alpha}(p - 1)} \nu = p\ (\mod a^{\alpha}), \quad\text{$\alpha$ premier}.
\]

Toutes ces propositions ont été déduites de la théorie des permutations.

Voici d'autres résultats qui découlent de ma théorie.

\primo Soit $k$ le module d'une fonction elliptique, $p$~un nombre
premier donné~$> 3$; pour que l'équation du degré~$p + 1$, qui
donne les divers modules des fonctions transformées relativement
au nombre $p$, soit résoluble par radicaux, \emph{il faut} de deux choses
l'une: ou bien qu'une des racines soit rationnellement connue,
ou bien que toutes soient des fonctions rationnelles les unes des
autres. Il ne s'agit ici, bien entendu, que des valeurs particulières
du module $k$. Il est évident que la chose n'a pas lieu en général.
Cette règle n'a pas lieu pour $p = 5$.

\secundo Il est remarquable que l'équation modulaire générale du
sixième degré, correspondant au nombre 5, peut s'abaisser à une
du cinquième degré dont elle est la réduite. Au contraire, pour
des degrés supérieurs, les équations modulaires ne peuvent s'abaisser\footnotemark.

\footnotetext{Cette assertion n'est pas tout à fait exacte, comme Galois en avertit lui-même
  dans sa Lettre à M.~Auguste Chevalier, qu'on trouve plus bas. Il dit en général,
  au sujet de l'article que nous reproduisons ici: «La condition que j'ai indiquée
  dans le \Title{Bulletin de Férussac}, pour la solubilité par radicaux, est trop
  restreinte; il y a peu d'exceptions, mais il y en a.» Quant aux équations modulaires
  en particulier, il déclare l'abaissement du degré~$p + 1$ au degré~$p$ possible, non
  seulement pour $p = 5$, mais encore pour $p = 7$ et $p = 11$; mais il en maintient
  l'impossibilité pour $p > 11$. \Annot{(J.~Liouville.)}}
%% -----File: 021.png---Folio 13-------


\Article{NOTE}{SUR LA}{RÉSOLUTION DES ÉQUATIONS NUMÉRIQUES\footnotemark.}

\footnotetext{\Title{Bulletin des Sciences mathématiques} de M.~Férussac, t.~XIII, p.~413
  (année 1830, cahier de juin). \Annot{(J.~Liouville.)}}

M.~Legendre a le premier remarqué que, lorsqu'une équation
algébrique était mise sous la forme
\[
\phi x = x,
\]
où $\phi x$ est une fonction de~$x$ qui croît constamment en même
temps que~$x$, il était facile de trouver la racine de cette équation
immédiatement plus petite qu'un nombre donné~$a$, si $\phi a < a$, et
la racine immédiatement plus grande que~$a$, si $\phi a > a$.

Pour le démontrer, on construit la courbe $y = \phi x$ et la droite
$y = x$. Soit prise une abscisse $= a$, et supposons, pour fixer les
idées, $\phi a > a$, je dis qu'il sera aisé d'obtenir la racine immédiatement
supérieure à $a$. En effet, les racines de l'équation $\phi x = x$
ne sont que les abscisses des points d'intersection de la droite et
de la courbe, et il est clair que l'on s'approchera du point le plus
voisin d'intersection en substituant à l'abscisse $a$ l'abscisse $\phi a$.
On aura une valeur plus approchée encore en prenant $\phi\phi a$,
puis $\phi\phi\phi a$, et ainsi de suite.

Soit $Fx = 0$ une équation donnée du degré $n$, et $Fx = X - Y$,
$X$~et~$Y$ n'ayant que des termes positifs. Legendre met successivement
l'équation sous ces deux formes:
\[
x = \phi x = \sqrt[n]{\frac{X}{\left(\dfrac{Y}{x^{n}}\right)}}, \qquad
x = \psi x = \sqrt[n]{\frac{X}{\left(\dfrac{x^{n}}{Y}\right)}};
\]
les deux fonctions $\phi x$ et $\psi x$ sont toujours, comme on voit, l'une
plus grande, l'autre plus petite que $x$. Ainsi, à l'aide de ces deux
fonctions, on pourra avoir les deux racines de l'équation les plus
approchées d'un nombre donné $a$, l'une en plus et l'autre en
moins.
%% -----File: 022.png---Folio 14-------

Mais cette méthode a l'inconvénient d'exiger, à chaque opération,
l'extraction d'une racine~$n^{\text{ième}}$. Voici deux formes plus commodes.
Cherchons un nombre~$k$ tel que la fonction
\[
x + \frac{Fx}{kx^{n}}
\]
croisse avec~$x$, quand $x > 1$. (Il suffit, en effet, de savoir trouver
les racines d'une équation qui sont plus grandes que l'unité.)

Nous aurons, pour la condition proposée,
\[
1 + \frac{d\dfrac{X - Y}{kx^{n}}}{dx} > 0, \quad\text{ou bien}\quad
1 - \frac{nX - xX'}{kx^{n+1}} + \frac{nY - xY'}{kx^{n+1}} > 0;
\]
or on a identiquement
\[
nX - xX' > 0, \qquad nY - xY' >0;
\]
il suffit donc de poser
\[
\frac{nX - xX'}{kx^{n+1}} < 1 \quad\text{pour}\quad x > 1,
\]
et il suffit, pour cela, de prendre pour $k$ la valeur de la fonction
$nX - xX'$, relative à $x = 1$.

On trouvera de même un nombre~$h$ tel que la fonction
\[
x - \frac{Fx}{hx^{n}}
\]
croîtra avec~$x$, quand $x$~sera $> 1$, en changeant $Y$~en~$X$.

Ainsi, l'équation donnée pourra se mettre sous l'une des formes
\[
x = x + \frac{Fx}{kx^{n}}, \qquad
x = x - \frac{Fx}{hx^{n}},
\]
qui sont toutes deux rationnelles et donnent pour la résolution
une méthode facile.
%% -----File: 023.png---Folio 15-------


\Article{}{SUR}{LA THÉORIE DES NOMBRES\footnotemark.}

\footnotetext{\Title{Bulletin des Sciences mathématiques} de M.~Férussac, t.~XIII, p.~438
  (année 1830, cahier de juin); avec la note suivante: «Ce Mémoire fait partie
  des recherches de M.~Galois sur la théorie des permutations et des équations
  algébriques.» \Annot{(J.~Liouville.)}}

Quand on convient de regarder comme nulles toutes les quantités
qui, dans les calculs algébriques, se trouvent multipliées par
un nombre premier donné~$p$, et qu'on cherche, dans cette convention,
les solutions d'une équation algébrique $Fx = 0$, ce que
M.~Gauss désigne par la notation $Fx \equiv 0$, on n'a coutume de
considérer que les solutions entières de ces sortes de questions.
Ayant été conduit par des recherches particulières à considérer
les solutions incommensurables, je suis parvenu à quelques résultats
que je crois nouveaux.

Soit une pareille équation ou congruence, $Fx = 0$, et $p$~le module.
Supposons d'abord, pour plus de simplicité, que la congruence
en question n'admette aucun facteur commensurable,
c'est-à-dire qu'on ne puisse pas trouver trois fonctions $\phi x$, $\psi x$,
$\chi x$ telles que
\[
\phi x · \psi x = Fx + p · \chi x.
\]
Dans ce cas, la congruence n'admettra donc aucune racine entière,
ni même aucune racine incommensurable de degré inférieur.
Il faut donc regarder les racines de cette congruence comme
des espèces de symboles imaginaires, puisqu'elles ne satisfont pas
aux questions des nombres entiers, symboles dont l'emploi, dans
le calcul, sera souvent aussi utile que celui de l'imaginaire~$\sqrt{-1}$
dans l'analyse ordinaire.

C'est la classification de ces imaginaires, et leur réduction au
plus petit nombre possible, qui va nous occuper.

Appelons $i$ l'une des racines de la congruence $Fx = 0$, que
nous supposerons du degré~$\nu$.

Considérons l'expression générale
\[
\Tag{(A)}
a + a_{1} i + a_{2} i^{2} + \dots + a_{\nu-1} i^{\nu-1},
\]
%% -----File: 024.png---Folio 16-------
où $a$,~$a_{1}$, $a_{2}$,~\dots, $a_{\nu-1}$ représentent des nombres entiers. En donnant
à ces nombres toutes les valeurs, l'expression~\Eq{(A)} en
acquiert $p^{\nu}$, qui jouissent, ainsi que je vais le faire voir, des
mêmes propriétés que les nombres naturels dans la \emph{théorie des
résidus des puissances}.

Ne prenons des expressions~\Eq{(A)} que les $p^{\nu} - 1$ valeurs où $a$,
$a_{1}$, $a_{2}$,~\dots, $a_{\nu-1}$ ne sont pas toutes nulles: soit $\alpha$ l'une de ces
expressions.

Si l'on élève successivement $\alpha$ aux puissances $2$\ieme, $3$\ieme,~\dots, on
aura une suite de quantités de même forme [parce que toute
fonction de~$i$ peut se réduire au $(\nu - 1)^\text{ième}$~degré]. Donc on devra
avoir $\alpha^{n} = 1$, $n$~étant un certain nombre; soit $n$ le plus petit
nombre qui soit tel que l'on ait $\alpha^{n} = 1$. On aura un ensemble
de $n$~expressions, toutes différentes entre elles,
\[
1,\ \alpha,\ \alpha^{2},\ \alpha^{3},\ \dots,\ \alpha^{n-1}.
\]
Multiplions ces $n$~quantités par une autre expression~$\beta$ de la même
forme. Nous obtiendrons encore un nouveau groupe de quantités
toutes différentes des premières et différentes entre elles. Si les
quantités~\Eq{(A)} ne sont pas épuisées, on multipliera encore les
puissances de~$\alpha$ par une nouvelle expression~$\gamma$, et ainsi de suite.
On voit donc que le nombre $n$ divisera nécessairement le nombre
total des quantités~\Eq{(A)}. Ce nombre étant $p^{\nu} - 1$, on voit que $n$
divise $p^{\nu} - 1$. De là suit encore que l'on aura
\[
\alpha^{p^{\nu}-1}\! = 1, \quad\text{ou bien}\quad
\alpha^{p^{\nu}}\! = \alpha.
\]

Ensuite on prouvera, comme dans la théorie des nombres, qu'il
y a des racines primitives $\alpha$ pour lesquelles on ait précisément
$p^{\nu} - 1 = n$, et qui reproduisent par conséquent, par l'élévation
aux puissances, toute la suite des autres racines.

Et l'une quelconque de ces racines primitives ne dépendra que
d'une congruence du degré~$\nu$, congruence \emph{irréductible}, sans quoi
l'équation en~$i$ ne le serait pas non plus, parce que les racines de
la congruence en $i$ sont toutes des puissances de la racine primitive.

On voit ici cette conséquence remarquable que toutes les quantités
algébriques qui peuvent se présenter dans la théorie sont
racines d'équations de la forme
\[
x^{p^{\nu}}\! = x.
\]
%% -----File: 025.png---Folio 17-------

Cette proposition, énoncée algébriquement, est celle-ci: Étant
donnés une fonction~$Fx$ et un nombre premier~$p$, on peut poser
\[
fx · Fx = x^{p^{\nu}}\! - x + p \phi x,
\]
$fx$ et $\phi x$ étant des fonctions entières, toutes les fois que la congruence
$Fx = 0\ (\mod p)$ sera irréductible.

Si l'on veut avoir toutes les racines d'une pareille congruence
au moyen d'une seule, il suffit d'observer que l'on a généralement
\[
(Fx)^{p^{n}}\! = F(x^{p^{n}})
\]
et que, par conséquent, l'une des racines étant~$x$, les autres
seront
\[
x^{p},\ x^{p^{2}}\!,\ \dots,\ x^{p^{\nu-1}}\footnotemark.
\]
\footnotetext{De ce que les racines de la congruence irréductible de degré~$\nu$
\[
Fx = 0
\]
sont exprimées par la suite
\[
x,\ x^{p},\ x^{p^{2}}\!,\ \dots, x^{p^{\nu-1}}\!,
\]
on aurait tort de conclure que ces racines soient toujours des quantités exprimables
par radicaux. Voici un exemple du contraire:

La congruence irréductible
\[
x^{2} + x + 1 = 0\ (\mod 2)
\]
donne
\[
x = \frac{-1 + \sqrt{-3}}{2},
\]
qui se réduit à
\[
\frac{0}{0}\ (\mod 2),
\]
formule qui n'apprend rien.}

Il s'agit maintenant de faire voir que, réciproquement à ce que
nous venons de dire, les racines de l'équation ou de la congruence
$x^{p^{\nu}}\! = x$ dépendront toutes d'une seule congruence du degré~$\nu$.

Soit en effet $i$ une racine d'une congruence irréductible et telle
que toutes les racines de la congruence $x^{p^{\nu}}\! = x$ soient fonctions
rationnelles de~$i$ (Il est clair qu'ici, comme dans les équations
ordinaires, cette propriété a lieu)\footnotemark.
\footnotetext{La proposition générale dont il s'agit ici peut s'énoncer ainsi: Étant donnée
  une équation algébrique, on pourra trouver une fonction rationnelle~$\theta$ de
  toutes ses racines, de telle sorte que, réciproquement, chacune des racines s'exprime
  rationnellement en~$\theta$. Ce théorème était connu d'Abel, ainsi qu'on peut le
  voir par la première Partie du Mémoire que ce célèbre géomètre a laissé sur les
fonctions elliptiques.}
%% -----File: 026.png---Folio 18-------

Il est d'abord évident que le degré~$\mu$ de la congruence en~$i$ ne
saurait être plus petit que~$\nu$, sans quoi la congruence
\[
\Tag{(\nu)}
x^{p^{\nu-1}}\! - 1 = 0
\]
aurait toutes ses racines communes avec la congruence
\[
x^{p^{\mu-1}}\! - 1 = 0,
\]
ce qui est absurde, puisque la congruence~\Eq{(\nu)} n'a pas de racines
égales, comme on le voit en prenant la dérivée du premier
membre. Je dis maintenant que $\mu$ ne peut non plus être~$> \nu$.

En effet, s'il en était ainsi, toutes les racines de la congruence
\[
x^{p^{\mu}}\! = x
\]
devraient dépendre rationnellement de celles de la congruence
\[
x^{p^{\nu}}\! = x.
\]
Mais il est aisé de voir que, si l'on a
\[
i^{p^{\nu}}\! = i,
\]
toute fonction rationnelle $h = fi$ donnera encore
\[
(fi)^{p^{\nu}}\! = f (i^{p^{\nu}}) = fi, \quad\text{d'où}\quad
h^{p^{\nu}}\! = h.
\]

Donc toutes les racines de la congruence $x^{p^{\mu}}\! = x$ lui seraient
communes avec l'équation $x^{p^{\nu}}\! = x$, ce qui est absurde.

Nous savons donc enfin que toutes les racines de l'équation ou
congruence $x^{p^{\nu}}\! = x$ dépendent nécessairement d'une \emph{seule} congruence
\emph{irréductible} de degré~$\nu$.

Maintenant, pour avoir cette congruence irréductible d'où
dépendent les racines de la congruence $x^{p^{\nu}}\! = x$, la méthode la
plus générale sera de délivrer d'abord cette congruence de tous
les facteurs communs qu'elle pourrait avoir avec des congruences
de degré inférieur et de la forme
\[
x^{p^{\mu}}\! = x.
\]

On obtiendra ainsi une congruence qui devra se partager en
congruences irréductibles de degré~$\nu$. Et, comme on sait exprimer
toutes les racines de chacune de ces congruences irréductibles au
%% -----File: 027.png---Folio 19-------
moyen d'une seule, il sera aisé de les obtenir toutes par la méthode
de M.~Gauss.

Le plus souvent, cependant, il sera aisé de trouver par le tâtonnement
une congruence irréductible d'un degré donné~$\nu$, et l'on
doit en déduire toutes les autres.

Soient, pour exemple, $p = 7$, $\nu = 3$. Cherchons les racines de
la congruence
\[
\Tag{(1)}
x^{7^{3}} = x\ (\mod 7).
\]
J'observe que la congruence
\[
\Tag{(2)}
i^{3} = 2\ (\mod 7)
\]
étant irréductible, et du degré~$3$, toutes les racines de la congruence~\Eq{(1)}
dépendent rationnellement de celles de la congruence~\Eq{(2)},
en sorte que toutes les racines de~\Eq{(1)} sont de la
forme
\[
\Tag{(3)}
a + a_{1} i + a_{2} i^{2} \quad\text{ou bien}\quad
a + a_{1} \sqrt[3]{2} + a_{2} \sqrt[3]{4}.
\]

Il faut maintenant trouver une racine primitive, c'est-à-dire une
forme de l'expression~\Eq{(3)} qui, élevée à toutes les puissances,
donne toutes les racines de la congruence
\[
x^{7^{3}-1} = 1, \quad\text{savoir}\quad
x^{2^{1} · 3^{2} · 19} = 1\ (\mod 7),
\]
et nous n'avons besoin pour cela, que d'avoir une racine primitive
de chaque congruence
\[
x^{2} = 1,\qquad
x^{3^{2}} = 1,\qquad
x^{19} = 1.
\]

La racine primitive de la première est~$-1$; celles de $x^{3^{2}} - 1 = 0$
sont données par les équations
\[
x^{3} = 2,\qquad
x^{3} = 4,
\]
en sorte que $i$ est une racine primitive de $x^{3^{2}} = 1$.

Il ne reste qu'à trouver une racine de $x^{19} - 1 = 0$, ou plutôt de
\[
\frac{x^{19} - 1}{x - 1} = 0,
\]
et essayons pour cela si l'on ne peut pas satisfaire à la question
%% -----File: 028.png---Folio 20-------
en posant simplement $x = a + a_{1} i$, au lieu de $a + a_{1} i + a_{2} i^{2}$;
nous devrons avoir
\[
(a + a_{1} i)^{19} = 1,
\]
ce qui, en développant par la formule de Newton, et réduisant les
puissances de~$a$, de~$a_{1}$ et de~$i$, par les formules
\[
a^{m(p-1)} = 1, \qquad
a_{1}^{m(p-1)} = 1, \qquad
i^{3} = 2,
\]
se réduit à
\[
3 \bigl[a - a^{4} a_{1}^{3} + (a^{5} a_{1}^{2} + a^{2} a_{1}^{5}) i^{2} \bigr] = 1,
\]
d'où, en séparant,
\[
3a - 3a^{4} a_{1}^{3} = 1, \qquad
a^{5} a_{1}^{2} + a^{2} a_{1}^{5} = 0.
\]

Ces deux dernières équations sont satisfaites en posant $a = -1$,
$a_{1} = 1$. Donc
\[
-1 + i
\]
est une racine primitive de $x^{19} = 1$. Nous avons trouvé plus haut,
pour racines primitives de $x^{2} = 1$ et de $x^{3^{2}} = 1$, les valeurs $-1$
et~$i$; il ne reste plus qu'à multiplier entre elles les trois quantités
\[
-1,\quad i,\quad -1 + i,
\]
et le produit $i - i^{2}$ sera une racine primitive de la congruence
\[
x^{7^{3} - 1} = 1.
\]

Donc ici l'expression $i - i^{2}$ jouit de la propriété que, en l'élevant
à toutes les puissances, on obtiendra $7^{3} - 1$ expressions différentes
et de la forme
\[
a + a_{1} i + a_{2} i^{2}.
\]

Si nous voulons avoir la congruence de moindre degré d'où
dépend notre racine primitive, il faut éliminer~$i$ entre les deux
équations
\[
i^{3} = 2, \qquad
\alpha = i - i^{2}.
\]
On obtient ainsi
\[
\alpha^{3} - \alpha + 2 = 0.
\]

Il sera convenable de prendre pour base des imaginaires et de
représenter par~$i$ la racine de cette équation, en sorte que
\[
\Tag{(i)}
i^{3} - i + 2 = 0,
\]
%% -----File: 029.png---Folio 21-------
et l'on aura toutes les imaginaires de la forme
\[
a + a_{1} i + a_{2} i^{2},
\]
en élevant $i$ à toutes les puissances et réduisant par l'équation~\Eq{(i)}.

Le principal avantage de la nouvelle théorie que nous venons
d'exposer est de ramener les congruences à la propriété (si utile
dans les équations ordinaires), d'admettre précisément autant de
racines qu'il y a d'unités dans l'ordre de leur degré.

La méthode pour avoir toutes ces racines sera très simple. Premièrement,
on pourra toujours préparer la congruence donnée
$Fx = 0$ de manière qu'elle n'ait plus de racines égales, ou,
en d'autres termes, qu'elle n'ait plus de facteur commun avec
$F'x = 0$, et le moyen de le faire est évidemment le même que
pour les équations ordinaires.

Ensuite, pour avoir les solutions entières, il suffira, ainsi que
M.~Libri paraît en avoir fait le premier la remarque, de chercher
le plus grand facteur commun à $Fx = 0$ et à $x^{p-1} = 1$.

Si maintenant on veut avoir les solutions imaginaires du second
degré, on cherchera le plus grand facteur commun à $Fx = 0$
et à $x^{p^{2}-1}\! = 1$, et, en général, les solutions de l'ordre~$\nu$ seront
données par le plus grand commun diviseur à $Fx = 0$ et à
$x^{p^{\nu}-1}\! = 1$.

C'est surtout dans la théorie des permutations, où l'on a sans
cesse besoin de varier la forme des indices, que la considération
des racines imaginaires des congruences paraît indispensable.
Elle donne un moyen simple et facile de reconnaître dans quel
cas une équation primitive est soluble par radicaux, comme je
vais essayer d'en donner en deux mots une idée.

Soit une équation algébrique $fx = 0$, de degré~$p^{\nu}$; supposons
que les $p^{\nu}$~racines soient désignées par~$x_{k}$, en donnant à l'indice~$k$
les $p^{\nu}$~valeurs déterminées par la congruence $k^{p^{\nu}}\! =k\ (\mod p)$.

Prenons une fonction quelconque rationnelle $V$ des $p^{\nu}$ racines $x_{k}$.
Transformons cette fonction en substituant partout à l'indice~$k$
l'indice $(ak + b)^{p^{r}}\!$, $a$,~$b$,~$r$ étant des constantes arbitraires satisfaisant
aux conditions de $a^{p^{\nu}-1}\! = 1$, $b^{p^{\nu}}\! = b\ (\mod p)$ et de $r$~entier.

En donnant aux constantes $a$,~$b$,~$r$ toutes les valeurs dont elles
sont susceptibles, on obtiendra en tout $p^{\nu}(p^{\nu} - 1)\nu$ manières de
%% -----File: 030.png---Folio 22-------
permuter les racines entre elles par des substitutions de la forme
$[x_{k}, x_{(ak+b)^{p^{r}}}]$, et la fonction~$V$ admettra en général par ces substitutions
$p^{\nu}(p^{\nu} - 1)\nu$ formes différentes.

Admettons maintenant que l'équation proposée $fx = 0$ soit
telle que toute fonction des racines, invariable par les $p^{\nu}(p^{\nu} - 1)\nu$
permutations que nous venons de construire, ait pour cela même
une valeur numérique rationnelle.

On remarque que, dans ces circonstances, l'équation $fx = 0$
sera soluble par radicaux, et, pour parvenir à cette conséquence,
il suffit d'observer que la valeur substituée à~$k$, dans chaque
indice, peut, se mettre sous les trois formes
\[
(ak + b)^{p^{r}}\!
  = \bigl[a(k + b^{1})\bigr]^{p^{r}}\!
  = a' k^{p^{r}} + b''
  = a' (k+b')^{p^{r}}.
\]

Les personnes habituées à la théorie des équations le verront
sans peine.

Cette remarque aurait peu d'importance si je n'étais parvenu à
démontrer que, réciproquement, une équation primitive ne saurait
être soluble par radicaux, à moins de satisfaire aux conditions
que je viens d'énoncer. (J'excepte les équations du neuvième et
du vingt-cinquième degré.)

Ainsi, pour chaque nombre de la forme~$p^{\nu}$, on pourra former
un groupe de permutations tel, que toute fonction des racines
invariable par ces permutations devra admettre une valeur rationnelle
quand l'équation de degré~$p^{\nu}$ sera primitive et soluble par
radicaux.

D'ailleurs, il n'y a que les équations d'un pareil degré~$p^{\nu}$ qui
soient à la fois primitives et solubles par radicaux.

Le théorème général que je viens d'énoncer précise et développe
les conditions que j'avais données dans le \Title{Bulletin} du mois
d'avril. Il indique le moyen de former une fonction des racines
dont la valeur sera rationnelle, toutes les fois que l'équation primitive
de degré~$p^{\nu}$ sera soluble par radicaux, et mène, par conséquent,
aux caractères de résolubilité de ces équations, par
des calculs sinon praticables, du moins qui sont possibles en
théorie.

Il est à remarquer que, dans le cas où $\nu = 1$, les diverses valeurs
de~$k$ ne sont autre chose que la suite des nombres entiers. Les
%% -----File: 031.png---Folio 23-------
substitutions de la forme $(x_{k}, x_{ak+b})$ seront au nombre de~$p(p - 1)$.

La fonction qui, dans le cas des équations solubles par radicaux,
doit avoir une valeur rationnelle, dépendra, en général,
d'une équation de degré $1 · 2 · 3 \dots (p - 2)$, à laquelle il faudra,
par conséquent, appliquer la méthode des racines rationnelles.
%% -----File: 032.png---Folio 24-------
%[Blank Page]
%% -----File: 033.png---Folio 25-------


\Chapter{II.}{{\OE}UVRES POSTHUMES.}

\Letter{Lettre a Auguste Chevalier\footnotemark.}

\footnotetext{Écrite la veille de la mort de l'auteur. (Insérée en 1832 dans la \Title{Revue encyclopédique},
  numéro de septembre, page~568.) \Annot{(J.~Liouville.)}}

\qquad Mon cher ami,
\medskip

J'ai fait en Analyse plusieurs choses nouvelles.

Les unes concernent la théorie des équations; les autres, les
fonctions intégrales.

Dans la théorie des équations, j'ai recherché dans quels cas les
équations étaient résolubles par des radicaux, ce qui m'a donné
occasion d'approfondir cette théorie et de décrire toutes les transformations
possibles sur une équation, lors même qu'elle n'est
pas soluble par radicaux.

On pourra faire avec tout cela trois Mémoires.

Le premier est écrit, et, malgré ce qu'en a dit Poisson, je le
maintiens, avec les corrections que j'y ai faites.
\bigskip

Le second contient des applications assez curieuses de la
théorie des équations. Voici le résumé des choses les plus importantes:

\primo D'après les propositions II~et~III du premier Mémoire, on
voit une grande différence entre adjoindre à une équation une des
racines d'une équation auxiliaire ou les adjoindre toutes.

Dans les deux cas, le groupe de l'équation se partage par l'adjonction
en groupes tels, que l'on passe de l'un à l'autre par une
même substitution; mais la condition que ces groupes aient les
mêmes substitutions n'a lieu certainement que dans le second
cas. Cela s'appelle la \emph{décomposition propre}.

En d'autres termes, quand un groupe~$G$ en contient un autre~$H$,
%% -----File: 034.png---Folio 26-------
le groupe~$G$ peut se partager en groupes, que l'on obtient chacun
en opérant sur les permutations de~$H$ une même substitution; en
sorte que
\[
G = H + HS + HS' + \dots.
\]
Et aussi il peut se diviser en groupes qui ont tous les mêmes substitutions,
en sorte que
\[
G = H + TH + T'H + \dots.
\]

Ces deux genres de décompositions ne coïncident pas ordinairement.
Quand ils coïncident, la décomposition est dite \emph{propre}.

Il est aisé de voir que, quand le groupe d'une équation n'est
susceptible d'aucune décomposition propre, on aura beau transformer
cette équation, les groupes des équations transformées
auront toujours le même nombre de permutations.

Au contraire, quand le groupe d'une équation est susceptible
d'une décomposition propre, en sorte qu'il se partage en $M$~groupes
de $N$~permutations, on pourra résoudre l'équation donnée
au moyen de deux équations: l'une aura un groupe de $M$~permutations,
l'autre un de $N$~permutations.

Lors donc qu'on aura épuisé sur le groupe d'une équation tout
ce qu'il y a de décompositions propres possibles sur ce groupe,
on arrivera à des groupes qu'on pourra transformer, mais dont
les permutations seront toujours en même nombre.

Si ces groupes ont chacun un nombre premier de permutations,
l'équation sera soluble par radicaux; sinon, non.

Le plus petit nombre de permutations que puisse avoir un
groupe indécomposable, quand ce nombre n'est pas premier,
est $5 · 4 · 3$.

\secundo Les décompositions les plus simples sont celles qui ont lieu
par la méthode de M.~Gauss.

Comme ces décompositions sont évidentes, même dans la forme
actuelle du groupe de l'équation, il est inutile de s'arrêter longtemps
sur cet objet.

Quelles décompositions sont praticables sur une équation qui
ne se simplifie pas par la méthode de M.~Gauss?

J'ai appelé \emph{primitives} les équations qui ne peuvent se simplifier
par la méthode de M.~Gauss; non que ces équations soient
%% -----File: 035.png---Folio 27-------
réellement indécomposables, puisqu'elles peuvent même se
résoudre par radicaux.

Comme lemme à la théorie des équations primitives solubles
par radicaux, j'ai mis en juin~1830, dans le \Title{Bulletin de Férussac},
une analyse sur les imaginaires de la théorie des nombres.

On trouvera ci-jointe\footnote
  {Galois parle des manuscrits, jusqu'ici inédits, que nous publions. \\
  \Annot{(J.~Liouville.)}}
la démonstration des théorèmes suivants:

\primo Pour qu'une équation primitive soit soluble par radicaux,
elle doit être du degré~$p^{\nu}$, $p$~étant premier.

\secundo Toutes les permutations d'une pareille équation sont de la
forme
\[
x_{k, l, m\dots} \mid x_{ak+bl+cm+\dots+h,\: a'k+b'l+c'm+\dots+h',\: a''k+\dots\dots},
\]
$k$,~$l$,~$m$,~\dots\ étant $\nu$~indices, qui, prenant chacun $p$~valeurs, indiquent
toutes les racines. Les indices sont pris suivant le module~$p$;
c'est-à-dire que la racine sera la même quand on ajoutera à l'un
des indices un multiple de~$p$.

Le groupe qu'on obtient en opérant toutes les substitutions de
cette forme linéaire contient, en tout,
\[
p^{\nu} (p^{\nu} - 1) (p^{\nu} - p) \dots (p^{\nu} - p^{\nu-1}) \text{ permutations.}
\]

Il s'en faut que, dans cette généralité, les équations qui lui
répondent soient solubles par radicaux.

La condition que j'ai indiquée dans le \Title{Bulletin de Férussac}
pour que l'équation soit soluble par radicaux est trop restreinte;
il y a peu d'exceptions, mais il y en a.

La dernière application de la théorie des équations est relative
aux équations modulaires des fonctions elliptiques.

On sait que le groupe de l'équation qui a pour racines les sinus
de l'amplitude des $p^{2} - 1$ divisions d'une période est celui-ci:
\[
x_{k, l}, \quad x_{ak+bl,\: ck+dl};
\]
par conséquent l'équation modulaire correspondante aura pour
groupe
\[
x_{\efrac{k}{l}}, \quad x_{\efrac{ak+bl}{ck+dl}},
\]
%% -----File: 036.png---Folio 28-------
dans laquelle $\dfrac{k}{l}$ peut avoir les $p + 1$~valeurs
\[
\infty,\ 0,\ 1,\ 2,\ \dots,\ p - 1.
\]
Ainsi, en convenant que $k$ peut être infini, on peut écrire simplement
\[
x_{k},\quad x_{\efrac{ak+b}{ck+d}}.
\]

En donnant à $a$, $b$, $c$, $d$ toutes les valeurs, on obtient
\[
(p+1) p (p-1) \text{ permutations}.
\]

Or ce groupe se décompose \emph{proprement} en deux groupes,
dont les substitutions sont
\[
x_{k},\quad x_{\efrac{ak+b}{ck+d}} ,
\]
$ad - bc$ étant un résidu quadratique de~$p$.

Le groupe ainsi simplifié est de
\[
(p + 1) p\, \frac{p - 1}{2} \text{ permutations}.
\]

Mais il est aisé de voir qu'il n'est plus décomposable proprement,
à moins que $p = 2$, ou $p = 3$.

Ainsi, de quelle manière que l'on transforme l'équation, son
groupe aura toujours le même nombre de permutations.

Mais il est curieux de savoir si le degré peut s'abaisser.

Et d'abord il ne peut s'abaisser plus bas que~$p$, puisqu'une
équation de degré moindre que $p$ ne peut avoir~$p$ pour facteur
dans le nombre des permutations de son groupe.

Voyons donc si l'équation de degré~$p + 1$, dont les racines~$x_{k}$
s'indiquent en donnant à~$k$ toutes les valeurs, y compris l'infini,
et dont le groupe a pour substitutions
\[
x_{k},\quad x_{\efrac{ak+b}{ck+d}},
\]
$ad - bc$ étant un carré, peut s'abaisser au degré~$p$.

Or il faut pour cela que le groupe se décompose (improprement,
s'entend) en $p$~groupes de $(p + 1)\dfrac{p - 1}{2}$ permutations
chacun.
%% -----File: 037.png---Folio 29-------

Soient $0$ et $\infty$ deux lettres conjointes dans l'un de ces groupes.
Les substitutions qui ne font pas changer $0$~et~$\infty$ de place seront
de la forme
\[
x_{k},\quad x_{m^{2} k}.
\]

Donc si $M$ est la lettre conjointe de~$1$, la lettre conjointe de $m^{2}$
sera~$m^{2} M$. Quand $M$~est un carré, on aura donc $M^{2} = 1$. Mais
cette simplification ne peut avoir lieu que pour $p = 5$.

Pour $p = 7$ on trouve un groupe de $(p + 1)\dfrac{p - 1}{2}$ permutations,
où
\[
\infty,\ 1,\ 2,\ 4
\]
ont respectivement pour lettres conjointes
\[
0,\ 3,\ 6,\ 5.
\]

Ce groupe a ses substitutions de la forme
\[
x_{k}, \quad x_{a\efrac{k - b}{k - c}},
\]
$b$ étant la lettre conjointe de~$c$, et $a$~une lettre qui est résidu ou
non résidu en même temps que~$c$.

Pour $p = 11$, les mêmes substitutions auront lieu avec les
mêmes notations,
\[
\infty,\ 1,\ 3,\ 5,\ 5,\ 9
\]
ayant respectivement pour conjointes
\[
0,\ 2,\ 6,\ 8,\ 10,\ 7.
\]

Ainsi, pour le cas de $p = 5$, $7$, $11$, l'équation modulaire
s'abaisse au degré~$p$.

En toute rigueur, cette équation n'est pas possible dans les cas
plus élevés.

Le troisième Mémoire concerne les intégrales.

On sait qu'une somme de termes d'une même fonction elliptique
se réduit toujours à un seul terme, plus des quantités algébriques
ou logarithmiques.

Il n'y a pas d'autres fonctions pour lesquelles cette propriété
ait lieu.

Mais des propriétés absolument semblables y suppléent dans
toutes les intégrales de fonctions algébriques.
%% -----File: 038.png---Folio 30-------

On traite à la fois toutes les intégrales dont la différentielle est
une fonction de la variable et d'une même fonction irrationnelle
de la variable, que cette irrationnelle soit ou ne soit pas un radical,
qu'elle s'exprime ou ne s'exprime pas par des radicaux.

On trouve que le nombre des périodes distinctes de l'intégrale
la plus générale relative à une irrationnelle donnée est toujours
un nombre pair.

Soit $2n$ ce nombre, on aura le théorème suivant:

Une somme quelconque de termes se réduit à $n$~termes, plus
des quantités algébriques et logarithmiques.

Les fonctions de première espèce sont celles pour lesquelles la
partie algébrique et logarithmique est nulle.

Il y en a $n$~distinctes.

Les fonctions de seconde espèce sont celles pour lesquelles la
partie complémentaire est purement algébrique.

Il y en a $n$~distinctes.

On peut supposer que les différentielles des autres fonctions
ne soient jamais infinies qu'une fois pour $x = a$, et, de plus, que
leur partie complémentaire se réduise à un seul logarithme, $\log P$,
$P$~étant une quantité algébrique. En désignant par $\prod(x, a)$ ces
fonctions, on aura le théorème
\[
\textstyle \prod(x, a) - \prod(a, x) = \sum \phi a · \psi x,
\]
$\phi a$ et $\psi x$ étant des fonctions de première et de seconde espèce.

On en déduit, en appelant $\prod(a)$ et $\psi$ les périodes de $\prod(x, a)$
et $\psi x$ relatives à une même révolution de~$x$,
\[
\textstyle \prod(a) = \sum \psi × \phi a.
\]
Ainsi les périodes des fonctions de troisième espèce s'expriment
toujours en fonction de première et de seconde espèce.

On peut en déduire aussi des théorèmes analogues au théorème
de Legendre
\[
FE' + EF' - FF' = \frac{\pi}{2}.
\]

La réduction des fonctions de troisième espèce à des intégrales
définies, qui est la plus belle découverte de M.~Jacobi, n'est pas
praticable, hors le cas des fonctions elliptiques.

La multiplication des fonctions intégrales par un nombre entier
%% -----File: 039.png---Folio 31-------
est toujours possible, comme l'addition, au moyen d'une équation
de degré~$n$ dont les racines sont les valeurs à substituer dans
l'intégrale pour avoir les termes réduits.

L'équation qui donne la division des périodes en $p$~parties
égales est du degré~$p^{2n} - 1$. Son groupe a en tout
\[
(p^{2n} - 1)(p^{2n} - p) \dots (p^{2n} - p^{2n-1}) \text{ permutations}.
\]

L'équation qui donne la division d'une somme de $n$~termes
en $p$~parties égales est du degré~$p^{2n}$. Elle est soluble par radicaux.

\Par{De la transformation.} On peut d'abord, en suivant des raisonnements
analogues à ceux qu'Abel a consignés dans son dernier
Mémoire, démontrer que si, dans une même relation entre
des intégrales, on a les deux fonctions
\[
\int \Phi(x, X)\, dx, \quad
\int \Psi(y, Y)\, dy,
\]
la dernière intégrale ayant $2n$~périodes, il sera permis de supposer
que $y$~et~$Y$ s'expriment moyennant une seule équation de
degré~$n$ en fonction de~$x$ et de~$X$.

D'après cela on peut supposer que les transformations aient
lieu constamment entre deux intégrales seulement, puisqu'on
aura évidemment, en prenant une fonction quelconque rationnelle
de~$y$ et de~$Y$,
\[
{\textstyle\sum} \int f(y, Y)\, dy
  = \int F(x, X)\, dx + \text{ une quant.\ alg.\ et log.}
\]

Il y aurait sur cette équation des réductions évidentes dans le
cas où les intégrales de l'un et de l'autre membre n'auraient pas
toutes deux le même nombre de périodes.

Ainsi nous n'avons à comparer que des intégrales qui aient
toutes deux le même nombre de périodes.

On démontrera que le plus petit degré d'irrationnalité de deux
pareilles intégrales ne peut être plus grand pour l'une que pour
l'autre.

On fera voir ensuite qu'on peut toujours transformer une intégrale
donnée en une autre dans laquelle une période de la première
soit divisée par le nombre premier~$p$, et les $2n - 1$~autres
restent les mêmes.
%% -----File: 040.png---Folio 32-------

Il ne restera donc à comparer que des intégrales où les périodes
seront les mêmes de part et d'autre, et telles, par conséquent,
que $n$~termes de l'une s'expriment sans autre équation qu'une
seule du degré~$n$, au moyen de ceux de l'autre, et réciproquement.
Ici nous ne savons rien.
\bigskip

Tu sais, mon cher Auguste, que ces sujets ne sont pas les seuls
que j'aie explorés. Mes principales méditations, depuis quelque
temps étaient dirigées sur l'application à l'analyse transcendante
de la théorie de l'ambiguïté. Il s'agissait de voir \textit{a~priori}, dans
une relation entre des quantités ou fonctions transcendantes,
quels échanges on pouvait faire, quelles quantités on pouvait
substituer aux quantités données, sans que la relation pût cesser
d'avoir lieu. Cela fait reconnaître de suite l'impossibilité de beaucoup
d'expressions que l'on pourrait chercher. Mais je n'ai pas le
temps, et mes idées ne sont pas encore bien développées sur ce
terrain, qui est immense.

Tu feras imprimer cette Lettre dans la \Title{Revue encyclopédique}.

Je me suis souvent hasardé dans ma vie à avancer des propositions
dont je n'étais pas sûr; mais tout ce que j'ai écrit là est depuis
bientôt un an dans ma tête, et il est trop de mon intérêt de
ne pas me tromper pour qu'on me soupçonne d'énoncer des théorèmes
dont je n'aurais pas la démonstration complète.

Tu prieras publiquement Jacobi ou Gauss de donner leur avis,
non sur la vérité, mais sur l'importance des théorèmes.

Après cela, il y aura, j'espère, des gens qui trouveront leur
profit à déchiffrer tout ce gâchis.

Je t'embrasse avec effusion.
\Date{E.~Galois.}{Le 29 mai 1832.}
%% -----File: 041.png---Folio 33-------


\Article{MÉMOIRE}{SUR LES}{CONDITIONS DE RÉSOLUBILITÉ DES ÉQUATIONS PAR RADICAUX\footnotemark.}

\footnotetext{Ce Mémoire et le suivant ont été retrouvés dans les papiers de Galois et
  publiés pour la première fois en 1846 par Liouville, qui les avait fait précéder de
  la note suivante:
\Label{page33}

  «En insérant dans leur Recueil la lettre qu'on vient de lire, les éditeurs de la
  \Title{Revue encyclopédique} annonçaient qu'ils publieraient prochainement les manuscrits
  laissés par Galois. Mais celle promesse n'a pas été tenue. M.~Auguste Chevalier
  avait cependant préparé le travail. Il nous a remis et l'on trouvera dans
  les feuilles qui vont suivre:

  »\primo Un Mémoire entier sur les conditions de résolubilité des équations par
radicaux, avec l'application aux équations de degré premier;

  »\secundo Un fragment d'un second Mémoire où Galois traite de la théorie générale
  des équations qu'il nomme \emph{primitives}.

  »Nous avons conservé la plupart des notes que M.~Auguste Chevalier avait
  jointes aux Mémoires dont nous venons de parler. Ces notes sont toutes marquées
  des initiales A.~Ch. Les notes non signées sont de Galois lui-même.

  »Nous compléterons cette publication par quelques autres morceaux extraits
  des papiers de Galois, et qui, sans avoir une grande importance, pourront cependant
  encore être lus avec intérêt par les géomètres.»

  Les extraits dont parle Liouville dans la dernière phrase de cette note n'ont
  jamais été publiés.}

Le Mémoire ci-joint\footnote
  {J'ai jugé convenable de placer en tête de ce Mémoire la préface qu'on va
  lire, bien que je l'aie trouvée biffée dans le manuscrit. \Annot{(A.~Ch.)}}
est extrait d'un Ouvrage que j'ai eu
l'honneur de présenter à l'Académie il y a un an. Cet Ouvrage
n'ayant pas été compris, les propositions qu'il renferme ayant été
révoquées en doute, j'ai dû me contenter de donner, sous forme
synthétique, les principes généraux et une \emph{seule} application de
ma théorie. Je supplie mes juges de lire du moins avec attention
ce peu de pages.

On trouvera ici une \emph{condition} générale à laquelle \emph{satisfait
toute équation soluble par radicaux}, et qui réciproquement
assure leur résolubilité. On en fait l'application seulement aux
%% -----File: 042.png---Folio 34-------
équations dont le degré est un nombre premier. Voici le théorème
donné par notre analyse:
\medskip

\begin{Thm}
Pour qu'une équation de degré premier, qui n'a pas de diviseurs
commensurables, soit soluble par radicaux, il \emph{faut} et
il \emph{suffit} que toutes les racines soient des fonctions rationnelles
de deux quelconques d'entre elles.
\end{Thm}
\medskip

Les autres applications de la théorie sont elles-mêmes autant
de théories particulières. Elles nécessitent d'ailleurs l'emploi de
la théorie des nombres, et d'un algorithme particulier: nous les
réservons pour une autre occasion. Elles sont en partie relatives
aux équations modulaires de la théorie des fonctions elliptiques,
que nous démontrons ne pouvoir se résoudre par radicaux.
%[** TN: Name printed below date in the original]
\Date{E.~Galois.}{Ce 16 janvier 1831.}


\Subsection{PRINCIPES.}

Je commencerai par établir quelques définitions et une suite de
lemmes qui sont tous connus.

\Par{Définitions.} Une équation est dite \emph{réductible} quand elle
admet des diviseurs rationnels; \emph{irréductible} dans le cas contraire.

Il faut ici expliquer ce qu'on doit entendre par le mot \emph{rationnel},
car il se représentera souvent.

Quand l'équation a \emph{tous} ses coefficients numériques et rationnels,
cela veut dire simplement que l'équation peut se décomposer
en facteurs qui aient leurs coefficients numériques et rationnels.

Mais quand les coefficients d'une équation ne seront \emph{pas tous}
numériques et rationnels, alors il faudra entendre par diviseur
rationnel un diviseur dont les coefficients s'exprimeraient en
fonction rationnelle des coefficients de la proposée.

Il y a plus: on pourra convenir de regarder comme rationnelle
toute fonction rationnelle d'un certain nombre de quantités déterminées,
supposées connues \textit{a~priori}. Par exemple, on pourra
%% -----File: 043.png---Folio 35-------
choisir une certaine racine d'un nombre entier, et regarder comme
rationnelle toute fonction rationnelle de ce radical.

Lorsque nous conviendrons de regarder ainsi comme connues
de certaines quantités, nous dirons que nous les \emph{adjoignons} à
l'équation qu'il s'agit de résoudre. Nous dirons que ces quantités
sont \emph{adjointes} à l'équation.

Cela posé, nous appellerons \emph{rationnelle} toute quantité qui
s'exprimera en fonction rationnelle des coefficients de l'équation
et d'un certain nombre de quantités \emph{adjointes} à l'équation et
convenues arbitrairement.

Quand nous nous servirons d'équations auxiliaires, elles seront
rationnelles, si leurs coefficients sont rationnels en notre sens.

On voit, au surplus, que les propriétés et les difficultés d'une
équation peuvent être tout à fait différentes suivant les quantités
qui lui sont adjointes. Par exemple, l'adjonction d'une quantité
peut rendre réductible une équation irréductible.

Ainsi, quand on adjoint à l'équation
\[
\frac{x^{n} - 1}{x - 1} = 0\DPtypo{.}{,}
\]
où $n$ est premier, une racine d'une des équations auxiliaires de
M.~Gauss, cette équation se décompose en facteurs et devient, par
conséquent, réductible.

Les substitutions sont le passage d'une permutation à l'autre.

La permutation d'où l'on part pour indiquer les substitutions
est toute arbitraire, quand il s'agit de fonctions; car il n'y a aucune
raison pour que, dans une fonction de plusieurs lettres, une
lettre occupe un rang plutôt qu'un autre.

Cependant, comme on ne peut guère se former l'idée d'une
substitution sans se former celle d'une permutation, nous ferons,
dans le langage, un emploi fréquent des permutations, et nous ne
considérerons les substitutions que comme le passage d'une permutation
à une autre.

Quand nous voudrons grouper des substitutions, nous les ferons
toutes provenir d'une même permutation.

Comme il s'agit toujours de questions où la disposition primitive
des lettres n'influe en rien dans les groupes que nous considérerons,
on devra avoir les mêmes substitutions, quelle que soit
%% -----File: 044.png---Folio 36-------
la permutation d'où l'on sera parti. Donc, si dans un pareil
groupe on a les substitutions $S$~et~$T$, on est sûr d'avoir la substitution~$ST$.

Telles sont les définitions que nous avons cru devoir rappeler.

\begin{Lemme}[I.]
Une équation irréductible ne peut avoir aucune
racine commune avec une équation rationnelle, sans la diviser.
\end{Lemme}

Car le plus grand commun diviseur entre l'équation irréductible
et l'autre équation sera encore rationnel; donc, etc.

\begin{Lemme}[II.]
Étant donnée une équation quelconque, qui
n'a pas de racines égales, dont les racines sont $a$,~$b$, $c$,~\dots, on
peut toujours former une fonction~$V$ des racines, telle qu'aucune
des valeurs que l'on obtient en permutant dans celle fonction
les racines de toutes manières, ne soit égale à une autre.
\end{Lemme}

Par exemple, on peut prendre
\[
V = Aa + Bb + Cc + \dots,
\]
$A$, $B$, $C$ étant des nombres entiers convenablement choisis.

\begin{Lemme}[III.]
La fonction~$V$ étant choisie comme il est indiqué
dans l'article précèdent, elle jouira de cette propriété,
que toutes les racines de l'équation proposée s'exprimeront rationnellement
en fonction de~$V$.
\end{Lemme}

En effet, soit
\[
V = \phi (a, b, c, d, \dots),
\]
ou bien
\[
V - \phi (a, b, c, d, \dots) = 0.
\]
Multiplions entre elles toutes les équations semblables, que l'on
obtient en permutant dans celles-ci toutes les lettres, la première
seulement restant fixe; il viendra une expression suivante:
\[
\bigl[V - \phi (a, b, c, d, \dots)\bigr]
\bigl[V - \phi (a, c, b, d, \dots)\bigr]
\bigl[V - \phi (a, b, d, c, \dots)\bigr] \dots,
\]
symétrique en $b$, $c$, $d$,~\dots, laquelle pourra, par conséquent,
s'écrire en fonction de~$a$. Nous aurons donc une équation de la
forme
\[
F(V, a) = 0.
\]
%% -----File: 045.png---Folio 37-------
Or je dis que de là on peut tirer la valeur de~$a$. Il suffit pour cela
de chercher la solution commune à cette équation et à la proposée.
Cette solution est la seule commune, car on ne peut avoir,
par exemple,
\[
F(V, b) = 0,
\]
cette équation ayant un facteur commun avec l'équation semblable,
sans quoi l'une des fonctions $\phi(a, \dots)$ serait égale à
l'une des fonctions $\phi(b, \dots)$; ce qui est contre l'hypothèse.

Il suit de là que $a$ s'exprime en fonction rationnelle de~$V$, et il
en est de même des autres racines.

Cette proposition\footnote
  {Il est remarquable que de cette proposition on peut conclure que toute
  équation dépend d'une équation auxiliaire telle, que toutes les racines de cette
  nouvelle équation soient des fonctions rationnelles les unes des autres: car
  l'équation auxiliaire en~$V$ est dans ce cas.

  Au surplus, cette remarque est purement curieuse. En effet, une équation qui
  a cette propriété n'est pas, en général, plus facile à résoudre qu'une autre.}
est citée sans démonstration par Abel,
dans le Mémoire posthume sur les fonctions elliptiques.

\begin{Lemme}[IV]
Supposons que l'on ait formé l'équation en~$V$,
et que l'on ait pris l'un de ses facteurs irréductibles, en sorte
que~$V$ soit racine d'une équation irréductible. Soient $V$,~$V'$,
$V''$,~\dots\ les racines de cette équation irréductible. Si $a = f(V)$
est une des racines de la proposée, $f(V')$~de même sera une racine
de la proposée.
\end{Lemme}

En effet, en multipliant entre eux tous les facteurs de la forme
$V - \phi (a, b, c, \dots, d)$, où l'on aura opéré sur les lettres toutes
les permutations possibles, on aura une équation rationnelle en~$V$,
laquelle se trouvera nécessairement divisible par l'équation en
question; donc $V'$~doit s'obtenir par l'échange des lettres dans la
fonction~$V$. Soit $F(V, a) = 0$ l'équation qu'on obtient en permutant
dans $V$ toutes les lettres, hors la première. On aura donc
$F(V', b) = 0$, $b$ pouvant être égal à~$a$, mais étant certainement
l'une des racines de l'équation proposée; par conséquent, de
même que de la proposée et de $F(V, a) = 0$ est résulté $a = f(V)$,
de même il résultera de la proposée et de $F(V', b) = 0$ combinées,
la suivante $b = f(V')$.
%% -----File: 046.png---Folio 38-------


\Subsection{PROPOSITION I.}

\begin{Theoreme}
Soit une équation donnée, dont $a$,~$b$, $c$,~\dots\
sont les $m$~racines. Il y aura toujours un groupe de permutations
des lettres $a$,~$b$, $c$,~\dots\ qui jouira de la propriété suivante:

\primo Que toute fonction des racines, invariable\footnote
  {Nous appelons ici invariable non seulement une fonction dont la forme est
  invariable par les substitutions des racines entre elles, mais encore celle dont la
  valeur numérique ne varierait pas par ces substitutions. Par exemple, si $Fx = 0$
  est une équation, $Fx$~est une fonction des racines qui ne varie par aucune permutation.

  Quand nous disons qu'une fonction est rationnellement connue, nous voulons
  dire que sa valeur numérique est exprimable en fonction rationnelle des coefficients
  de l'équation et des quantités adjointes.}
par les substitutions
de ce groupe, soit rationnellement connue;

\secundo Réciproquement, que toute fonction des racines, déterminable
rationnellement, soit invariable par les substitutions.
\end{Theoreme}

(Dans le cas des équations algébriques, ce groupe n'est autre
chose que l'ensemble des $1 · 2 · 3 \dots\ m$ permutations possibles
sur les $m$~lettres, puisque, dans ce cas, les fonctions symétriques
sont seules déterminables rationnellement.)

(Dans le cas de l'équation $\dfrac{x^{n} - 1}{x - 1} = 0$, si l'on suppose $a = r$,
$b = r^{g}$, $c = r^{g^{2}}$,~\dots, $g$~étant une racine primitive, le groupe de
permutations sera simplement celui-ci:
\[
\begin{array}{c}
  abcd \dots\dots k, \\
  bcd  \dotfill  ka, \\
  cd   \dotfill kab, \\
       \dotfill      \\
  kabc \dotfill   i;
\end{array}
\]
dans ce cas particulier, le nombre des permutations est égal au
degré de l'équation, et la même chose aurait lieu dans les équations
dont toutes les racines seraient des fonctions rationnelles
les unes des autres.)

\Par{Démonstration.} Quelle que soit l'équation donnée, on pourra
trouver une fonction rationnelle $V$ des racines, telle que toutes
%% -----File: 047.png---Folio 39-------
les racines soient fonctions rationnelles de~$V$. Cela posé, considérons
l'équation irréductible dont $V$~est racine (lemmes III~et~IV)\@.
Soient $V$,~$V'$, $V''$,~\dots, $V^{(n-1)}$ les racines de cette équation.

Soient $\phi V$, $\phi_{1} V$, $\phi_{2} V$,~\dots, $\phi_{m-1} V$ les racines de la proposée.

Écrivons les $n$~permutations suivantes des racines:
\[
\begin{array}{*{6}{l}}
(V)       & \phi V,  & \phi_{1} V,  & \phi_{2} V,  & \dots, & \phi_{m-1} V,\\
(V')      & \phi V', & \phi_{1} V', & \phi_{2} V', & \dots, & \phi_{m-1} V',\\
(V'')     & \phi V'',& \phi_{1} V'',& \phi_{2} V'',& \dots, & \phi_{m-1} V'',\\
\dots,    & \dots,   & \dots,     & \dots,     & \dots, & \dots, \\
(V^{(n-1)})& \phi V^{(n-1)},& \phi_{1} V^{(n-1)},& \phi_{2} V^{(n-1)},& \dots, & \phi_{m-1} V^{(n-1)}:
\end{array}
\]
je dis que ce groupe de permutations jouit de la propriété énoncée.

En effet:

\primo Toute fonction $F$ des racines, invariable par les substitutions
de ce groupe, pourra être écrite ainsi: $F = \psi V$, et l'on aura
\[
\psi V = \psi V' = \psi V'' = \dots = \psi V^{(n-1)}.
\]
La valeur de $F$ pourra donc se déterminer rationnellement.

\secundo Réciproquement, si une fonction~$F$ est déterminable rationnellement,
et que l'on pose $F = \psi V$, on devra avoir
\[
\psi V = \psi V' = \psi V'' = \dots = \psi V^{(n-1)},
\]
puisque l'équation en~$V$ n'a pas de diviseur commensurable et que
$V$~satisfait à l'équation $F = \psi V$, $F$~étant une quantité rationnelle.
Donc la fonction~$F$ sera nécessairement invariable par les substitutions
du groupe écrit ci-dessus.

Ainsi, ce groupe jouit de la double propriété dont il s'agit dans
le théorème proposé. Le théorème est donc démontré.

Nous appellerons \emph{groupe de l'équation} le groupe en question.

\Par{Scolie~\upshape I.} Il est évident que, dans le groupe de permutations
dont il s'agit ici, la disposition des lettres n'est point à considérer,
mais seulement les substitutions de lettres par lesquelles on passe
d'une permutation à l'autre.

Ainsi l'on peut se donner arbitrairement une première permutation,
pourvu que les autres permutations s'en déduisent toujours
par les mêmes substitutions de lettres. Le nouveau groupe ainsi
formé jouira évidemment des mêmes propriétés que le premier,
%% -----File: 048.png---Folio 40-------
puisque, dans le théorème précédent, il ne s'agit que des substitutions
que l'on peut faire dans les fonctions.

\Par{Scolie~\upshape II.} Les substitutions sont indépendantes même du
nombre des racines.

\Subsection{PROPOSITION II.}

\begin{Theoreme}[\protect\footnotemark.]
Si l'on adjoint à une équation donnée la
racine~$r$ d'une équation auxiliaire irréductible, \primo il arrivera
de deux choses l'une: ou bien le groupe de l'équation ne sera
pas changé, ou bien il se partagera en $p$~groupes appartenant
chacun à l'équation proposée respectivement quand on lui adjoint
chacune des racines de l'équation auxiliaire; \secundo ces
groupes jouiront de la propriété remarquable, que l'on passera
de l'un à l'autre en opérant dans toutes les permutations du
premier une même substitution de lettres.
\end{Theoreme}

\footnotetext{Dans l'énoncé du théorème, après ces mots: \emph{la racine~$r$ d'une équation
  auxiliaire irréductible}, Galois avait mis d'abord ceux-ci: \emph{de degré $p$ premier},
  qu'il a effacés plus tard. De même, dans la démonstration, au lieu de $r$,~$r'$,
  $r''$,~\dots\ étant d'autres valeurs de~$r$, la rédaction primitive portait: $r$,~$r'$, $r''$,~\dots\
  étant les diverses valeurs de~$r$. Enfin on trouve à la marge du manuscrit la
  note suivante de l'auteur:

  «Il y a quelque chose à compléter dans cette démonstration. Je n'ai pas le
  temps.»

  Cette ligne a été jetée avec une grande rapidité sur le papier; circonstance
  qui, jointe aux mots: «Je n'ai pas le temps», me fait penser que Galois a relu
  son Mémoire pour le corriger avant d'aller sur le terrain. \Annot{(A. Ch.)}}

\primo Si, après l'adjonction de~$r$, l'équation en~$V$, dont il est
question plus haut, reste irréductible, il est clair que le groupe
de l'équation ne sera pas changé. Si, au contraire, elle se réduit,
alors l'équation en~$V$ se décomposera en $p$~facteurs, tous de même
degré et de la forme
\[
f(V, r) × f(V, r') × f(V, r'') × \dots,
\]
$r$, $r'$, $r''$,~\dots\ étant d'autres valeurs de~$r$. Ainsi, le groupe de
l'équation proposée se décomposera aussi en groupes chacun
d'un même nombre de permutations, puisque à chaque valeur
de~$V$ correspond une permutation. Ces groupes seront respectivement
%% -----File: 049.png---Folio 41-------
ceux de l'équation proposée, quand on lui adjoindra successivement
$r$,~$r'$, $r''$,~\dots.

\secundo Nous avons vu plus haut que toutes les valeurs de~$V$ étaient
des fonctions rationnelles les unes des autres. D'après cela, supposons
que, $V$~étant une racine de $f(V, r) = 0$, $F(V)$~en soit une
autre; il est clair que de même si $V'$ est une racine de $f(V, r') = 0$,
$F(V')$~en sera une autre; car on aura
\[
f \bigl[F(V), r\bigr] = \text{une fonction divisible par $f(V, r)$}.
\]

%[** TN: "lemme" italicized in the original here, below]
Donc (lemme~I)
\[
f \bigl[F(V'), r'\bigr] = \text{une fonction divisible par $f(V', r')$}.
\]

Cela posé, je dis que l'on obtient le groupe relatif à~$r'$ en opérant
partout dans le groupe relatif à~$r$ une même substitution de
lettres.

En effet, si l'on a, par exemple,
\[
\phi_{\mu} F(V) = \phi_{\nu} (V),
\]
on aura encore (lemme~I),
\[
\phi_{\mu} F(V') = \phi_{\nu} (V').
\]
Donc, pour passer de la permutation $\bigl[F(V)\bigr]$ à la permutation
$\bigl[F(V')\bigr]$, il faut faire la même substitution que pour passer de la
permutation~$(V)$ à la permutation~$(V')$.

Le théorème est donc démontré.


\Subsection{PROPOSITION III.}

\begin{Theoreme}
Si l'on adjoint à une équation \emph{toutes} les racines
d'une équation auxiliaire, les groupes dont il est question
dans le théorème~II jouiront, de plus, de cette propriété que
les substitutions sont les mêmes dans chaque groupe.
\end{Theoreme}

On trouvera la démonstration\footnotemark.

\footnotetext{Dans le manuscrit, l'énoncé du théorème qu'on vient de lire se trouve en
  marge et en remplace un autre que Galois avait écrit avec sa démonstration sous
  le même titre: \textit{Proposition~III}. Voici le texte primitif:
%[** TN: Set in-line in the original]
  \begin{Theoreme} Si l'équation
  en $r$~est de la forme $r^{p} = A$, et que les racines~$p^\text{ièmes}$ de l'unité se trouvent
  au nombre des quantités précédemment adjointes, les $p$~groupes dont il est
  question dans le Théorème~II jouiront, de plus, de cette \DPtypo{proprieté}{propriété} que les
  substitutions de lettres par lesquelles on passe d'une permutation à une
  autre dans chaque groupe soient les mêmes pour tous les groupes.\end{Theoreme}

  \noindent En effet,
  dans ce cas, il revient au même d'adjoindre à l'équation telle ou telle valeur
  de~$r$. Par conséquent, ses propriétés doivent être les mêmes après l'adjonction de
  telle ou telle valeur. Ainsi son groupe doit être le même quant aux substitutions
  (Proposition~I, scolie). Donc,~etc.

  Tout cela est effacé avec soin; le nouvel énoncé porte la date 1832 et montre,
  par la manière dont il est écrit, que l'auteur était extrêmement pressé, ce qui
  confirme l'assertion que j'ai avancée dans la Note précédente.
  \Annot{(A.~Ch.)}}
%% -----File: 050.png---Folio 42-------


\Subsection{PROPOSITION IV.}

\begin{Theoreme}
Si l'on adjoint à une équation la valeur \emph{numérique}
d'une certaine fonction de ses racines, le groupe de
l'équation s'abaissera de manière à n'avoir plus d'autres
permutations que celles par lesquelles cette fonction est invariable.
\end{Theoreme}

En effet, d'après la proposition~I, toute fonction connue doit
être invariable par les permutations du groupe de l'équation.


\Subsection{PROPOSITION V.}

\begin{Probleme}
Dans quel cas une équation est-elle soluble par
de simples radicaux?
\end{Probleme}

J'observerai d'abord que, pour résoudre une équation, il faut
successivement abaisser son groupe jusqu'à ne contenir plus
qu'une seule permutation. Car, quand une équation est résolue,
une fonction quelconque de ses racines est connue, même quand
elle n'est invariable par aucune permutation.

Cela posé, cherchons à quelle condition doit satisfaire le groupe
d'une équation, pour qu'il puisse s'abaisser ainsi par l'adjonction
de quantités radicales.

Suivons la marche des opérations possibles dans cette solution,
en considérant comme opérations distinctes l'extraction de chaque
racine de degré premier.
%% -----File: 051.png---Folio 43-------

Adjoignons à l'équation le premier radical extrait dans la solution.
Il pourra arriver deux cas: ou bien, par l'adjonction de ce
radical, le groupe des permutations de l'équation sera diminué; ou
bien, cette extraction de racine n'étant qu'une simple préparation,
le groupe restera le même.

Toujours sera-t-il qu'après un certain nombre \emph{fini} d'extractions
de racines, le groupe devra se trouver diminué, sans quoi l'équation
ne serait pas soluble.

Si, arrivé à ce point, il y avait plusieurs manières de diminuer le
groupe de l'équation proposée par une simple extraction de racine,
il faudrait, pour ce que nous allons dire, considérer seulement un
radical du degré le moins haut possible parmi tous les simples
radicaux, qui sont tels que la connaissance de chacun d'eux
diminue le groupe de l'équation.

Soit donc $p$ le nombre premier qui représente ce degré minimum,
en sorte que par une extraction de racine de degré~$p$, on
diminue le groupe de l'équation.

Nous pouvons toujours supposer, du moins pour ce qui est
relatif au groupe de l'équation, que, parmi les quantités adjointes
précédemment à l'équation, se trouve une racine~$p^{\text{ième}}$ de l'unité,~$\alpha$.
Car, comme cette expression s'obtient par des extractions de
racines de degré inférieur à~$p$, sa connaissance n'altérera en rien
le groupe de l'équation.

Par conséquent, d'après les théorèmes II~et~III, le groupe de
l'équation devra se décomposer en $p$ groupes jouissant les uns par
rapport aux autres de cette double propriété: \primo que l'on passe
de l'un à l'autre par une seule et même substitution; \secundo que tous
contiennent les mêmes substitutions.

Je dis réciproquement que, si le groupe de l'équation peut se
partager en $p$~groupes qui jouissent de cette double propriété, on
pourra, par une simple extraction de racine~$p^{\text{ième}}$, et par l'adjonction
de cette racine~$p^{\text{ième}}$, réduire le groupe de l'équation à l'un de
ces groupes partiels.

Prenons, en effet, une fonction des racines qui soit invariable
pour toutes les substitutions de l'un des groupes partiels, et varie
pour toute autre substitution. (Il suffit, pour cela, de choisir une
fonction symétrique des diverses valeurs que prend, par toutes les
%% -----File: 052.png---Folio 44-------
permutations de l'un des groupes partiels, une fonction qui n'est
invariable par aucune substitution.)

Soit $\theta$ cette fonction des racines.

Opérons sur la fonction~$\theta$ une des substitutions du groupe total
qui ne lui sont pas communes avec les groupes partiels. Soit $\theta_{1}$ le
résultat. Opérons sur la fonction~$\theta_{1}$ la même substitution, et
soit $\theta_{2}$ le résultat, et ainsi de suite.

Comme $p$ est un nombre premier, cette suite ne pourra s'arrêter
qu'au terme~$\theta_{p-1}$; ensuite l'on aura $\theta_{p} = \theta_{1}$, $\theta_{p+1}=\theta_{1}$, et
ainsi de suite.

Cela posé, il est clair que la fonction
\[
(\theta + \alpha\theta_{1}
        + \alpha^{2}\theta_{2} + \dots
        + \alpha^{p-1}\theta_{p-1})^{p}
\]
sera invariable par toutes les permutations du groupe total et, par
conséquent, sera actuellement connue.

Si l'on extrait la racine~$p^\text{ième}$ de cette fonction, et qu'on l'adjoigne
à l'équation, alors, par la proposition~IV, le groupe de
l'équation ne contiendra plus d'autres substitutions que celles des
groupes partiels.

Ainsi, pour que le groupe d'une équation puisse s'abaisser par
une simple extraction de racine, la condition ci-dessus est nécessaire
et suffisante.

Adjoignons à l'équation le radical en question; nous pourrons
raisonner maintenant sur le nouveau groupe comme sur le précédent,
et il faudra qu'il se décompose lui-même de la manière
indiquée, et ainsi de suite, jusqu'à un certain groupe qui ne contiendra
plus qu'une seule permutation.

\Par{Scolie.} Il est aisé d'observer cette marche dans la résolution
connue des équations générales du quatrième degré. En effet, ces
équations se résolvent au moyen d'une équation du troisième
degré, qui exige elle-même l'extraction d'une racine carrée. Dans
la suite naturelle des idées, c'est donc par cette racine carrée
qu'il faut commencer. Or, en adjoignant à l'équation du quatrième
degré cette racine carrée, le groupe de l'équation, qui contenait
en tout vingt-quatre substitutions, se décompose en deux qui n'en
contiennent que douze. En désignant par $a$,~$b$, $c$,~$d$ les racines,
%% -----File: 053.png---Folio 45-------
voici l'un de ces groupes:
\[
\begin{array}{*{3}{l}}
  abcd, & acdb, & adbc,  \\
  badc, & cabd, & dacb,  \\
  cdab, & dbac, & bcad,  \\
  dcba, & bdca, & cbda.
\end{array}
\]
Maintenant ce groupe se partage lui-même en trois groupes,
comme il est indiqué aux théorèmes II~et~III\@. Ainsi, par l'extraction
d'un seul radical du troisième degré, il reste simplement le
groupe
\[
\begin{array}{l}
  abcd,  \\
  badc,  \\
  cdab,  \\
  dcba:
\end{array}
\]
ce groupe se partage de nouveau en deux groupes:
\[
\begin{array}{ll}
  abcd, & cdab,  \\
  badc, & dcba.
\end{array}
\]
Ainsi, après une simple extraction de racine carrée, il restera
\[
\begin{array}{l}
  abcd,  \\
  badc;
\end{array}
\]
ce qui se résoudra enfin par une simple extraction de racine
carrée.

On obtient ainsi, soit la solution de Descartes, soit celle
d'Euler; car, bien qu'après la résolution de l'équation auxiliaire
%[** TN: [sic] extraye]
du troisième degré ce dernier extraye trois racines carrées, on
sait qu'il suffit de deux, puisque la troisième s'en déduit rationnellement.

Nous allons maintenant appliquer cette condition aux équations
irréductibles dont le degré est premier.
%% -----File: 054.png---Folio 46-------


\Section{}{Application aux équations irréductibles de degré premier.}

\Subsection{PROPOSITION VI.}

\begin{Lemme}
Une équation irréductible de degré premier ne
peut devenir réductible par l'adjonction d'un radical dont
l'indice serait autre que le degré même de l'équation.
\end{Lemme}

Car si $r$,~$r'$, $r''$,~\dots\ sont les diverses valeurs du radical, et
$Fx = 0$ l'équation proposée, il faudrait que $Fx$ se partageât en
facteurs
\[
f(x,r) × f(x,r') × \dots,
\]
tous de même degré, ce qui ne se peut, à moins que $f(x, r)$ ne
soit du premier degré en~$x$.

Ainsi, une équation irréductible de degré premier ne peut
devenir réductible, à moins que son groupe ne se réduise à une
seule permutation.


\Subsection{PROPOSITION VII.}

\begin{Probleme}
Quel est le groupe d'une équation irréductible
d'un degré premier~$n$, soluble par radicaux?
\end{Probleme}

D'après proposition précédente, le plus petit groupe possible,
avant celui qui n'a qu'une seule permutation, contiendra $n$~permutations.
Or, un groupe du permutations d'un nombre premier~$n$
de lettres ne peut se réduire à $n$~permutations, à moins que
l'une de ces permutations ne se déduise de l'autre par une substitution
circulaire de l'ordre~$n$. (\emph{Voir} le Mémoire de M.~Cauchy,
\Title{Journal de l'École Polytechnique}, XVII\ieme~cahier.) Ainsi, l'avant-dernier
groupe sera
\[
\Tag{(G)}
\left\{
\begin{array}{*{8}{l}}
x_{0},    &x_{1}, &x_{2}, &x_{3}, &\dots, &x_{n-3}, &x_{n-2}, &x_{n-1}, \\
x_{1},    &x_{2}, &x_{3}, &x_{4}, &\dots, &x_{n-2}, &x_{n-1}, &x_{0}, \\
x_{2},    &x_{3}, &\dots, &\dots, &\dots, &x_{n-1}, &x_{0}, &x_{1}, \\
\dots,  &\dots, &\dots, &\dots, &\dots, &\dots, &\dots, &\dots, \\
x_{n-1}, &x_{0}, &x_{1}, &\dots, &\dots, &x_{n-4}, &x_{n-3}, &x_{n-2},
\end{array}
\right.
\]
$x_{0}$, $x_{1}$, $x_{2}$,~\dots, $x_{n-1}$ étant les racines.
%% -----File: 055.png---Folio 47-------

Maintenant, le groupe qui précédera immédiatement celui-ci
dans l'ordre des décompositions devra se composer d'un certain
nombre de groupes ayant tous les mêmes substitutions que
celui-ci. Or, j'observe que ces substitutions peuvent s'exprimer
ainsi (faisons, en général, $x_{n} = x_{0}$, $x_{n+1} = x_{1}$,~\dots; il est clair
que chacune des substitutions du groupe~\Eq{(G)} s'obtient en mettant
partout à la place de $x_{k}$,~$x_{k+c}$, $c$~étant une constante).

Considérons l'un quelconque des groupes semblables au
groupe~\Eq{(G)}. D'après le théorème~II, il devra s'obtenir en opérant
partout dans ce groupe une même substitution; par exemple, en
mettant partout dans le groupe~\Eq{(G)}, à la place de $x_{k}$,~$x_{f(k)}$, $f$~étant
une certaine fonction.

Les substitutions de ces nouveaux groupes devant être les
mêmes que celles du groupe~\Eq{(G)}, on devra avoir
\[
f(x + c) = f(k) + C,
\]
$C$~étant indépendant de~$k$.

Donc
\begin{alignat*}{2}
&f(k + 2c) &&= f(k) + 2C, \\
\multispan{5}{\dotfill} \\
&f(k + mc) &&= f(k) + mC.
\end{alignat*}
Si $c = 1$, $k = 0$, on trouvera
\[
f(m) = am + b
\]
ou bien
\[
f(k) = ak + b,
\]
$a$~et~$b$ étant des constantes.

Donc, le groupe qui précède immédiatement le groupe~\Eq{(G)} ne
devra contenir que des substitutions telles que
\[
x_{k}, \quad x_{ak+b},
\]
et ne contiendra pas, par conséquent, d'autre substitution circulaire
que celle du groupe~\Eq{(G)}.
%% -----File: 056.png---Folio 48-------

On raisonnera sur ce groupe comme sur le précédent, et il s'ensuivra
que le premier groupe dans l'ordre des décompositions,
c'est-à-dire le groupe \emph{actuel} de l'équation, ne peut contenir que
des substitutions de la forme
\[
x_{k},\quad x_{ak+b}.
\]

Donc, \begin{Thm}si une équation irréductible de degré premier est
soluble par radicaux, le groupe de cette équation ne saurait
contenir que des substitutions de la forme
\[
x_{k}, \quad x_{ak+b},
\]
$a$~et~$b$ étant des constantes.\end{Thm}

Réciproquement, si cette condition a lieu, je dis que l'équation
sera soluble par radicaux. Considérons, en effet, les fonctions
\begin{alignat*}{5}
& (x_{0} + \alpha x_{1}     &&{}+ \alpha^{2} x_{2}
&&{}+ \dots + \alpha^{n-1} x_{n-1}       &&)^{n} = X_{1}     && ,
\\
& (x_{0} + \alpha x_{a}     &&{}+ \alpha^{2} x_{2a}
&&{}+ \dots + \alpha^{n-1} x_{(n-1)a}    &&)^{n} = X_{a}     && ,
\\
& (x_{0} + \alpha x_{a^{2}} &&{}+ \alpha^{2} x_{2a^{2}}
&&{}+ \dots + \alpha^{n-1} x_{(n-1)a^{2}}  &&)^{n} = X_{a^{2}} && ,
\\
\multispan{9}{\dotfill} & ,
\end{alignat*}
$\alpha$ étant une racine~$n^{\text{ième}}$ de l'unité, $a$~une racine primitive de~$n$.

Il est clair que toute fonction invariable par les substitutions
circulaires des quantités $X_{1}$, $X_{a}$, $X_{a^{2}}$,~\dots\ sera, dans ce cas, immédiatement
connue. Donc, on pourra trouver $X_{1}$, $X_{a}$, $X_{a^{2}}$,~\dots, par
la méthode de M.~Gauss pour les équations binômes. Donc,~etc.

Ainsi, pour qu'une équation irréductible de degré premier soit
soluble par radicaux, il \emph{faut} et il \emph{suffit} que toute fonction invariable
par les substitutions
\[
x_{k}, \quad  x_{ak+b}
\]
soit rationnellement connue.

Ainsi, la fonction
\[
(X_{1} - X) (X_{a} - X) (X_{a^{2}} - X) \dots
\]
devra, quel que soit~$X$, être connue.
%% -----File: 057.png---Folio 49-------

Il \emph{faut} donc et il \emph{suffit} que l'équation qui donne cette fonction
des racines admette, quel que soit~$X$, une valeur rationnelle.

Si l'équation proposée a tous ses coefficients rationnels, l'équation
auxiliaire qui donne cette fonction les aura tous aussi, et il
suffira de reconnaître si cette équation auxiliaire du degré
$1 · 2 · 3 \dots (n - 2)$ a ou non une racine rationnelle, ce que l'on
sait faire.

C'est là le moyen qu'il faudrait employer dans la pratique. Mais
nous allons présenter le théorème sous une autre forme.


\Subsection{PROPOSITION VIII.}

\begin{Theoreme}
Pour qu'une équation irréductible de degré
premier soit soluble par radicaux, il \emph{faut} et il \emph{suffit} que deux
quelconques des racines étant connues, les autres s'en déduisent
rationnellement.
\end{Theoreme}

Premièrement, il le faut, car la substitution
\[
x_{k}, \quad x_{ak+b}
\]
ne laissant jamais deux lettres à la même place, il est clair qu'en
adjoignant deux racines à l'équation, par la proposition~IV, son
groupe devra se réduire à une seule permutation.

En second lieu, cela suffit; car, dans ce cas, aucune substitution
du groupe ne laissera deux lettres aux mêmes places. Par conséquent,
le groupe contiendra tout au plus $n(n - 1)$~permutations.
Donc, il ne contiendra qu'une seule substitution circulaire (sans
quoi il y aurait au moins $n^{2}$~permutations). Donc, toute substitution
du groupe, $x_{k}$,~$x_{fk}$, devra satisfaire à la condition
\[
f(k + c) = f(k) + C.
\]
Donc, etc.

Le théorème est donc démontré.
%% -----File: 058.png---Folio 50-------


\Subsection{EXEMPLE DU THÉORÈME VII.}

Soit $n = 5$; le groupe sera le suivant:
%[** TN: Set in one column in the original; commented code below]
\iffalse
\[
\begin{array}{l}
  abcde \\
  bcdea \\
  cdeab \\
  deabc \\
  eabcd \\[2ex]
  acebd \\
  cebda \\
  ebdac \\
  bdace \\
  daceb \\[2ex]
  aedcb \\
  edcba \\
  dcbae \\
  cbaed \\
  baedc \\[2ex]
  adbec \\
  dbeca \\
  becad \\
  ecadb \\
  cadbe
\end{array}
\]
\fi
\[
\begin{array}{c@{\qquad}*{3}{|@{\qquad}c@{\qquad}}}
  abcde & acebd & aedcb & adbec \\
  bcdea & cebda & edcba & dbeca \\
  cdeab & ebdac & dcbae & becad \\
  deabc & bdace & cbaed & ecadb \\
  eabcd & daceb & baedc & cadbe
\end{array}
\]
%% -----File: 059.png---Folio 51-------


\Article{DES ÉQUATIONS PRIMITIVES}{}{QUI SONT SOLUBLES PAR RADICAUX\footnotemark. \\[6pt]
(Fragment.)}

\footnotetext{\emph{Voir} la Note~1 de la page~\pageref{page33}.}

Cherchons, en général, dans quel cas une équation primitive
est soluble par radicaux. Or, nous pouvons de suite établir un
caractère général fondé sur le degré même de ces équations. Ce
caractère est celui-ci: \begin{Thm}Pour qu'une équation primitive soit résoluble
par radicaux, il faut que son degré soit de la forme~$p^{\nu}$, $p$~étant
premier.\end{Thm} Et de là suivra immédiatement que, lorsqu'on aura
à résoudre par radicaux une équation irréductible dont le degré
admettrait des facteurs premiers inégaux, on ne pourra le faire
que par la méthode de décomposition due à M.~Gauss; sinon
l'équation sera insoluble.

Pour établir la propriété générale que nous venons d'énoncer
relativement aux équations primitives qu'on peut résoudre par
radicaux, nous pouvons supposer que l'équation que l'on veut
résoudre soit primitive, mais cesse de l'être par l'adjonction d'un
simple radical. En d'autres termes, nous pouvons supposer que,
$n$ étant premier, le groupe de l'équation se partage en $n$ groupes
irréductibles conjugués, mais non primitifs. Car, à moins que le
degré de l'équation soit premier, un pareil groupe se présentera
toujours dans la suite des décompositions.

Soit $N$ le degré de l'équation, et supposons qu'après une extraction
de racine de degré premier~$n$, elle devienne non primitive et
se partage en $Q$~équations primitives de degré~$P$, au moyen d'une
seule équation de degré~$Q$.

Si nous appelons $G$ le groupe de l'équation, ce groupe devra se
partager en $n$~groupes conjugués non primitifs, dans lesquels les
lettres se rangeront en systèmes composés de $P$~lettres conjointes
chacun. Voyons de combien de manières cela pourra se faire.

Soit $H$ l'un des groupes conjugués non primitifs. Il est aisé de
%% -----File: 060.png---Folio 52-------
voir que, dans ce groupe, deux lettres quelconques prises à volonté
feront partie d'un certain système de $P$~lettres conjointes,
et ne feront partie que d'un seul.

Car, en premier lieu, s'il y avait deux lettres qui ne pussent
faire partie d'un même système de $P$~lettres conjointes, le groupe~$G$,
qui est tel que l'une quelconque de ses substitutions transforme
les unes dans les autres toutes les substitutions du groupe~$H$,
serait non primitif, ce qui est contre l'hypothèse.

En second lieu, si deux lettres faisaient partie de plusieurs
systèmes différents, il s'ensuivrait que les groupes qui répondent
aux divers systèmes de $P$~lettres conjointes ne seraient pas primitifs,
ce qui est encore contre l'hypothèse.

Cela posé, soient
\[
\begin{array}{*{5}{l}}
a_{0}, & a_{1}, & a_{2}, & \dots, & a_{P-1}, \\
b_{0}, & b_{1}, & b_{2}, & \dots, & b_{P-1}, \\
c_{0}, & c_{1}, & c_{2}, & \dots, & c_{P-1}
\end{array}
\]
les $N$~lettres: supposons que chaque ligne horizontale représente
un système de lettres conjointes. Soient
\[
\DPtypo{a_{0}}{a_{0,0}},\quad a_{0,1}, \quad a_{0,2}, \ \dots, \quad a_{0,P-1}
\]
$P$~lettres conjointes toutes situées dans la première colonne verticale.
(Il est clair que nous pouvons faire qu'il en soit ainsi, en
intervertissant l'ordre des lignes horizontales.)

Soient, de même,
\[
a_{1,0}, \quad a_{1,1}, \quad a_{1,2}, \quad a_{1,3}, \ \dots, \quad a_{1,P-1}
\]
$P$~lettres conjointes toutes situées dans la seconde colonne verticale,
en sorte que
\[
a_{1,0}, \quad a_{1,1}, \quad a_{1,2}, \quad a_{1,3}, \ \dots, \quad a_{1,P-1}
\]
appartiennent respectivement aux mêmes lignes horizontales
que
\[
a_{0,0}, \quad a_{0,1}, \quad a_{0,2}, \quad a_{0,3}, \ \dots, \quad a_{0,P-1};
\]
soient, de même, les systèmes de lettres conjointes
\[
\begin{array}{*{6}{l}}
a_{2,0}, &a_{2,1}, &a_{2,2}, &a_{2,3}, & \dots, &a_{2,P-1}, \\
a_{3,0}, &a_{3,1}, &a_{3,2}, &a_{3,3}, & \dots, &a_{3,P-1}, \\
\dots,  & \dots,  & \dots, & \dots, & \dots, & \dotfill,
\end{array}
\]
%% -----File: 061.png---Folio 53-------
nous obtiendrons ainsi, en tout, $P^{2}$~lettres. Si le nombre total
des lettres n'est pas épuisé, on prendra un troisième indice, en
sorte que
\[
a_{m,n,0}, \quad a_{m,n,1}, \quad a_{m,n,2}, \quad a_{m,n,3},\ \dots, \quad a_{m,n,P-1}
\]
soit, en général, un système de lettres conjointes; et l'on parviendra
ainsi à cette conclusion, que $N = P^{\mu}$, $\mu$~étant un certain
nombre égal à celui des indices différents dont on aura besoin. La
forme générale des lettres sera
\[
a_{k_{1}, k_{2}, k_{3}, \dots, k_{\mu}},
\]
$k_{1}$,~$k_{2}$, $k_{3}$,~\dots, $k_{\mu}$ étant des indices qui peuvent prendre chacun
les $P$~valeurs $0$,~$1$, $2$, $3$,~\dots, $P - 1$.

On voit aussi, par la manière dont nous avons procédé, que,
dans le groupe~$H$, toutes les substitutions seront de la forme
\[
\bigl[a_{k_{1}, k_{2}, k_{3} \dots k_{\mu}}, \quad
      a_{\phi(k_{1}), \DPtypo{\psi(k)_{2}}{\psi(k_{2})}, \chi(k_{3}) \dots, \sigma(k_{\mu})}
\bigr],
\]
puisque chaque indice correspond à un système de lettres conjointes.

Si $P$ n'est pas un nombre premier, on raisonnera sur le groupe
de permutations de l'un quelconque des systèmes de lettres conjointes,
comme sur le groupe~$G$, en remplaçant chaque indice
par un certain nombre de nouveaux indices, et l'on trouvera
$P = R^{\alpha}$, et ainsi de suite; d'où enfin $N = p^{\nu}$, $p$~étant un nombre
premier.


\Section{}{Des équations primitives de degrée~$p^{2}$.}

Arrêtons-nous un moment pour traiter de suite les équations
primitives d'un degré~$p^{2}$, $p$~étant premier impair. (Le cas de
$p = 2$ a été examiné.) Si une équation du degré~$p^{2}$ est soluble
par radicaux, supposons-la d'abord telle, qu'elle devienne non
primitive par une extraction de radical.

Soit donc $G$ un groupe primitif de $p^{2}$~lettres qui se partage
en $n$~groupes non primitifs conjugués à~$H$.

Les lettres devront nécessairement, dans le groupe~$H$, se ranger
%% -----File: 062.png---Folio 54-------
ainsi:
\[
\begin{array}{*{6}{l}}
  a_{0,0}, & a_{0,1}, & a_{0,2}, & a_{0,3}, & \dots, & a_{0,p-1},\\
  a_{1,0}, & a_{1,1}, & a_{1,2}, & a_{1,3}, & \dots, & a_{1,p-1},\\
  a_{2,0}, & a_{2,1}, & a_{2,2}, & a_{2,3}, & \dots, & a_{2,p-1},\\
  \dots,  & \dots,  & \dots,  & \dots,  & \dots, & \dotfill, \\
  a_{p-1,0}, & a_{p-1,1}, & a_{p-1,2}, & a_{p-1,3}, & \dots, & \rlap{$a_{p-1,p-1},$}
\end{array}
\]
chaque ligne horizontale et chaque ligne verticale étant un système
de lettres conjointes.

Si l'on permute entre elles les lignes horizontales, le groupe
que l'on obtiendra, étant primitif et de degré premier, ne devra
contenir que des substitutions de la forme
\[
(a_{k_{1},k_{2}},\ a_{mk_{1}+n\DPtypo{}{,} k_{2}}),
\]
les indices étant pris relativement au module~$p$.

Il en sera de même pour les lignes qui ne pourront donner que
des substitutions de la forme
\[
(a_{k_{1},k_{2}},\ a_{k_{1}, qk_{2}+r}).
\]
Donc enfin toutes les substitutions du groupe~$H$ seront de la
forme
\[
(\DPtypo{a_{k_{2},k_{3}}}{a_{k_{1},k_{2}}},\ a_{m_{1} k_{1}+n_{1}, m_{2} k_{2}+n_{2}}).
\]
Si un groupe~$G$ se partage en $n$~groupes conjugués à celui que
nous venons de décrire, toutes les substitutions du groupe~$G$ devront
transformer les unes dans les autres les substitutions circulaires
du groupe~$H$, qui sont toutes écrites comme il suit:
\[
\Tag{(a)}
(a_{k_{1},k_{2}},\ \dots,\ a_{k_{1}+\alpha_{1}, k_{2}+\alpha_{2}},\ \dots).
\]
Supposons donc que l'une des substitutions du groupe~$G$ se forme
en remplaçant respectivement
\begin{align*}
k_{1} & \quad\text{par}\quad \phi_{1} (k_{1}, k_{2}),\\
k_{2} & \quad\text{par}\quad \phi_{2} (k_{1}, k_{2}).
\end{align*}
Si, dans les fonctions $\phi_{1}$,~$\phi_{2}$, on substitue pour $k_{1}$~et~$k_{2}$ les valeurs
$k_{1} + \alpha_{1}$, $k_{2} + \alpha_{2}$, il devra venir des résultats de la forme
\[
\phi_{1} + \beta_{1},\quad \phi_{2} + \beta_{2},
\]
%% -----File: 063.png---Folio 55-------
et de là il est aisé de conclure immédiatement que les substitutions
du groupe~$G$ doivent être toutes comprises dans la formule
\[
\Tag{(A)}
(a_{k_{1}, k_{2}},\ a_{m_{1} k_{1} + n_{1} k + \alpha_{1} m_{2} k_{1} + n_{2} k_{2} + \alpha_{2}}),
\]
Or nous savons, par le \no~\footnotemark,
que les substitutions du groupe~$G$
ne peuvent embrasser que $p^{2} - 1$ ou $p^{2} - p$ lettres. Ce n'est
point $p^{2} - p$, puisque, dans ce cas, le groupe~$G$ serait non primitif.
Si donc, dans le groupe~$G$, on ne considère que les permutations
où la lettre~$a_{0,0}$, par exemple, conserve toujours la même
place, on n'aura que des substitutions de l'ordre $p^{2} - 1$ entre les
$p^{2} - 1$ autres lettres.
\footnotetext{Ce Mémoire faisant suite à un travail de Galois que je ne possède pas, il
  m'est impossible d'indiquer le Mémoire cité ici et plus bas. \Annot{(A.~Ch.)}}

Mais rappelons-nous ici que c'est simplement pour la démonstration
que nous avons supposé que le groupe primitif~$G$ se partageât
en groupes conjugués non primitifs. Comme cette condition
n'est nullement nécessaire, les groupes seront souvent beaucoup
plus composés.

Il s'agit donc de reconnaître dans quel cas ces groupes pourront
admettre des substitutions où $p^{2} - p$ lettres seulement varieraient,
et cette recherche va nous retenir quelque temps.

Soit donc $G$ un groupe qui contienne quelque substitution de
l'ordre $p^{2} - p$; je dis d'abord que toutes les substitutions de ce
groupe seront linéaires, c'est-à-dire de la forme~\Eq{(A)}.

La chose est reconnue vraie pour les substitutions de l'ordre
$p^{2} - 1$; il suffit donc de la démontrer pour celles de l'ordre
$p^{2} - p$. Ne considérons donc qu'un groupe où les substitutions
seraient toutes $m$ de l'ordre~$p^{2}$ ou de l'ordre $p^{2} - p$. (\emph{Voyez} l'endroit
cité.)

Alors les $p$~lettres qui, dans une substitution de l'ordre $p^{2} - p$,
ne varieront pas, devront être des lettres conjointes.

Supposons que ces lettres conjointes soient
\[
a_{0,0},\quad a_{0,1},\quad a_{0,2},\quad \dots, \quad a_{0,p-1}.
\]
Nous pouvons déduire toutes les substitutions où ces $p$~lettres ne
changent pas de place, nous pouvons les déduire de substitutions
%% -----File: 064.png---Folio 56-------
de la forme
\[
(a_{k_{1},k_{2}},\ a_{k_{1},\phi k_{2}})
\]
et de substitutions de l'ordre $p^{2} - p$, dont la période serait de $p$~termes.
(\emph{Voyez} encore l'endroit cité.)

Les premiers doivent nécessairement, pour que le groupe
jouisse de la propriété voulue, se réduire à la forme
\[
(a_{k_{1},k_{2}},\ a_{k_{1},m k_{2}})
\]
d'après ce qu'on a vu pour les équations de degré~$p$.

Quant aux substitutions dont la période serait de $p$~termes,
comme elles sont conjuguées aux précédentes, nous pouvons supposer
un groupe qui les contienne sans contenir celles-ci: donc
elles devront transformer les substitutions circulaires~\Eq{(a)} les unes
dans les autres; donc elles seront aussi linéaires.

Nous sommes donc arrivés à cette conclusion, que le groupe
primitif de permutations de $p^{2}$~lettres doit ne contenir que des
substitutions de la forme~\Eq{(A)}.

Maintenant, prenons le groupe total que l'on obtient en opérant
sur l'expression
\[
a_{k_{1},k_{2}}
\]
toutes les substitutions linéaires possibles, et cherchons quels
sont les diviseurs de ce groupe qui peuvent jouir de la propriété
voulue pour la résolubilité des équations.

Quel est d'abord le nombre total des substitutions linéaires?
Premièrement, il est clair que toute transformation de la forme
\[
k_{1},\ k_{2},\quad m_{1}k_{1} + n_{1}k_{2} + \alpha_{1},\ m_{2}k_{1} + n_{2}k_{2} + \alpha_{2}
\]
ne sera pas pour cela une substitution; car il faut, dans une substitution,
qu'à chaque lettre de la première permutation il ne réponde
qu'une seule lettre de la seconde, et réciproquement.

Si donc on prend une lettre quelconque $a_{l_{1},l_{2}}$ de la seconde permutation,
et que l'on remonte à la lettre correspondante dans la
première, on devra trouver une lettre $a_{k_{1},k_{2}}$ où les indices $k_{1}$,~$k_{2}$
seront parfaitement déterminés. Il faut donc que, quels que soient
$l_{1}$~et~$l_{2}$, on ait, par les deux équations
\[
m_{1}k_{1} + n_{1}k_{2} + \alpha_{1} = l_{1}, \quad
m_{2}k_{1} + n_{2}k_{2} + \alpha_{2} = l_{2},
\]
%% -----File: 065.png---Folio 57-------
des valeurs de $k_{1}$~et~$k_{2}$ finies et déterminées. Ainsi la condition
pour qu'une pareille transformation soit réellement une substitution
est que $m_{1}n_{2} - m_{2}n_{1}$ ne soit ni nul ni divisible par le module~$p$,
ce qui est la même chose.

Je dis maintenant que, bien que ce groupe à substitutions
linéaires n'appartienne pas toujours, comme on le verra, à des
équations solubles par radicaux, il jouira toutefois de cette propriété,
que, si dans une quelconque de ses substitutions il y a $n$~lettres
de fixes, $n$~divisera le nombre des lettres. Et, en effet, quel
que soit le nombre des lettres qui restent fixes, on pourra exprimer
cette circonstance par des équations linéaires qui donneront
tous les indices de l'une des lettres fixes, au moyen d'un certain
nombre d'entre eux. Donnant à chacun de ces indices, restés arbitraires,
$p$~valeurs, on aura $p^{m}$~systèmes de valeurs, $m$~étant un certain
nombre. Dans le cas qui nous occupe, $m$~est nécessairement
$< 2$ et se trouve par conséquent être $0$ ou~$1$. Donc le nombre des
substitutions ne saurait être plus grand que
\[
p^{2}(p^{2} - 1)(p^{2} - p).
\]

Ne considérons maintenant que les substitutions linéaires où la
lettre~$a_{0,0}$ ne varie pas; si, dans ce cas, nous trouvons le nombre
total des permutations du groupe qui contient toutes les substitutions
linéaires possibles, il nous suffira de multiplier ce nombre
par~$p^{2}$.

Or, premièrement, en substituant $p$ à l'indice~$k_{2}$, toutes les
substitutions de la forme
\[
(a_{k_{1}, k_{2}},\ a_{m_{1}k_{1}, k_{2}})
\]
donneront en tout $p - 1$ substitutions. On en aura $p^{2} - p$ en
ajoutant au terme~$k_{2}$ le terme~$m_{2}k_{1}$, ainsi qu'il suit:
\[
\Tag{(m')}
(k_{1},\ k_{2},\ m_{1}k_{1},\ m_{2}k_{1} + k_{2}).
\]

D'un autre côté, il est aisé de trouver un groupe linéaire
de $p^{2} - 1$ permutations, tel que, dans chacune de ses substitutions,
toutes les lettres, à l'exception de~$a_{0,0}$, varient. Car, en
remplaçant le double indice $k_{1}$,~$k_{2}$ par l'indice simple $k_{1} + ik_{2}$,
%% -----File: 066.png---Folio 58-------
$i$~étant une racine primitive de
\[
x^{p^{2}-1}\! - 1 = 0\ (\mod p),
\]
il est clair que toute substitution de la forme
\[
[a_{k_{1}+k_{2} l},\ a_{(m_{1}+m_{2}l)(k_{1}+k_{2}l)}]
\]
sera une substitution linéaire; mais, dans ces substitutions, aucune
lettre ne reste à la même place, et elles sont au nombre
de~$p^{2} - 1$.

Nous avons donc un système de $p^{2} - 1$ permutations tel que,
dans chacune de ses substitutions, toutes les lettres varient, à
l'exception de~$a_{0,0}$. Combinant ces substitutions avec les $p^{2} - p$
dont il est parlé plus haut, nous aurons
\[
(p^{2} - 1)(p^{2} - p) \text{ substitutions}.
\]

Or, nous avons vu \textit{a~priori} que le nombre des substitutions
où $a_{0,0}$~reste fixe ne pouvait être plus grand que $(p^{2} - 1)(p^{2} - p)$.
Donc il est précisément égal à $(p^{2} - 1)(p^{2} - p)$, et le groupe
linéaire total aura en tout
\[
p^{2}(p^{2} - 1)(p^{2} - p) \text{ permutations}.
\]

Il reste à chercher les diviseurs de ce groupe, qui peuvent
jouir de la propriété d'être solubles par radicaux. Pour cela, nous
allons faire une transformation qui a pour but d'abaisser autant
que possible les équations générales de degré~$p^{2}$ dont le groupe
serait linéaire.

Premièrement, comme les substitutions circulaires d'un pareil
groupe sont telles, que toute autre substitution du groupe les
transforme les unes dans les autres, on pourra abaisser l'équation
d'un degré et considérer une équation de degré $p^{2} - 1$ dont le
groupe n'aurait que des substitutions de la forme
\[
(b_{k_{1}, k_{2}},\ b_{m_{1}k_{1}+n_{1}k_{2},\: m_{2}k_{1}+n_{2}k_{2}}),
\]
les $p^{2} - 1$ lettres étant
\[
\begin{array}{*{5}{l}}
         &b_{0,1}, &b_{0,2}, &b_{0,3}, &\dots,\\
b_{1,0}, &b_{1,1}, &b_{1,2}, &b_{1,3}, &\dots,\\
b_{2,0}, &b_{2,1}, &b_{2,2}, &b_{2,3}, &\dots,\\
\dots,  &\dots,  &\dots,  &\dots,  &\dots.
\end{array}
\]
%% -----File: 067.png---Folio 59-------

J'observe maintenant que ce groupe est non primitif, en sorte
que toutes les lettres où le rapport des deux indices est le même
sont des lettres conjointes. Si l'on remplace par une seule lettre
chaque système de lettres conjointes, on aura un groupe dont
toutes les substitutions seront de la forme
\[
\biggl(b_{\efrac{k_{1}}{k_{2}}},\
      b_{\efrac{m_{1} k_{1} + n_{1} k_{2}}{m_{2} k_{1} + n_{2} k_{2}}}\biggr),
\]
$\dfrac{k_{1}}{k_{2}}$ étant les nouveaux indices. En remplaçant ce rapport par un
seul indice~$k$, on voit que les $p + 1$~lettres seront
\[
b_{0}, \quad b_{1}, \quad b_{2}, \quad b_{3}, \quad \dots, \quad b_{p-1}, \quad b_{\efrac{1}{0}},
\]
et les substitutions seront de la forme
\[
\biggl(k,\ \frac{mk + n}{rk + s}\biggr).
\]

Cherchons combien de lettres, dans chacune de ces substitutions,
restent à la même place; il faut pour cela résoudre l'équation
\[
(rk + s)k - m(mk + n) = 0,
\]
qui aura deux, ou une, ou aucune racine, suivant que $(m - s)^{2} + 4nr$
sera résidu quadratique, nul ou non résidu quadratique. Suivant
ces trois cas, la substitution sera de l'ordre $p - 1$, ou~$p$, ou~$p + 1$.

On peut prendre pour type des deux premiers cas les substitutions
de la forme
\[
(k,\ mk + n),
\]
où la seule lettre~$b_{\efrac{1}{0}}$ ne varie pas, et de là on voit que le nombre
total des substitutions du groupe réduit est
\[
(p + 1)p(p - 1).
\]

C'est après avoir ainsi réduit ce groupe que nous allons le
traiter généralement. Nous chercherons d'abord dans quel cas un
diviseur de ce groupe, qui contiendrait des substitutions de
l'ordre~$p$, pourrait appartenir à une équation soluble par radicaux.

Dans ce cas, l'équation serait primitive et elle ne pourrait être
%% -----File: 068.png---Folio 60-------
soluble par radicaux, à moins que l'on n'eût $p + 1 = 2^{n}$, $n$~étant
un certain nombre.

Nous pouvons supposer que le groupe ne contienne que des
substitutions de l'ordre~$p$ et de l'ordre~$p + 1$. Toutes les substitutions
de l'ordre~$p + 1$ seront par conséquent semblables, et leur
période sera de deux termes.

Prenons donc l'expression
\[
\biggl(k,\ \frac{mk + n}{rk + s}\biggr),
\]
et voyons dans quel cas cette substitution peut avoir une période
de deux termes. Il faut pour cela que la substitution inverse se
confonde avec elle. La substitution inverse est
\[
\biggl(k,\ \frac{-sk + n}{rk - m}\biggr).
\]

Donc on doit avoir $m = -s$, et toutes les substitutions en
question seront
\[
\biggl(k,\ \frac{mk + n}{k - m}\biggr),
\]
ou encore
\[
k,\ m + \frac{N}{k - m},
\]
$N$~étant un certain nombre qui est le même pour toutes les substitutions,
puisque ces substitutions doivent être transformées les
unes dans les autres par toutes les substitutions de l'ordre~$p$,
$(k, k+m)$; or ces substitutions doivent, de plus, être conjuguées
les unes des autres. Si donc
\[
\biggl(k,\ m + \frac{N}{k - m}\biggr), \qquad
\biggl(k,\ n + \frac{N}{k - m}\biggr)
\]
sont deux pareilles substitutions, il faut que l'on ait
\[
n + \frac{N}{\dfrac{N}{k-m} + m - n} =
m + \frac{N}{\dfrac{N}{k-n} + n - m},
\]
savoir
\[
(m - n)^{2} = 2N.
\]

Donc la différence entre deux valeurs de~$m$ ne peut acquérir
%% -----File: 069.png---Folio 61-------
que deux valeurs différentes; donc $m$ ne peut avoir plus de trois
valeurs; donc enfin $p = 3$. Ainsi, c'est seulement dans ce cas
que le groupe réduit pourra contenir des substitutions de l'ordre~$p$.

Et, en effet, la réduite sera alors du quatrième degré, et, par
conséquent, soluble par radicaux.

Nous savons par là qu'en général, parmi les substitutions de notre
groupe réduit, il ne devra pas se trouver de substitutions de
l'ordre~$p$. Peut-il y en avoir de l'ordre~$p - 1$? C'est ce que je vais
rechercher\footnotemark.

\footnotetext{J'ai cherché inutilement dans les papiers de Galois la continuation de ce
  qu'on vient de lire. \Annot{(A.~Ch.)}}

\vfil
\begin{center}
  FIN.
\end{center}
%% -----File: 070.png---Folio 62-------


\BackMatter
\Contents % ** Prints: "TABLE DES MATIÈRES." and "Pages."

\ToCArt{\textsc{Introduction}}{1}


\ToCChap{I.}{ARTICLES PUBLIÉS PAR GALOIS.}

\ToCArt{Démonstration d'un théorème sur les fractions continues périodiques}{1}

\ToCArt{Notes sur quelques points d'Analyse}{9}

\ToCArt{Analyse d'un Mémoire sur la résolution algébrique des équations}{11}

\ToCArt{Note sur la résolution des équations numériques}{13}

\ToCArt{Sur la théorie des nombres}{15}


\ToCChap{II.}{{\OE}UVRES POSTHUMES.}

\ToCArt{Lettre à Auguste Chevalier}{25}

\ToCArt{Mémoire sur les conditions de résolubilité des équations par radicaux}{33}

\ToCArt{Des équations primitives qui sont solubles par radicaux (fragment)}{51}

\vfil
\begin{center}
FIN DE LA TABLE DES MATIÈRES.
\end{center}
%%%%%%%%%%%%%%%%%%%%%%%%% GUTENBERG LICENSE %%%%%%%%%%%%%%%%%%%%%%%%%%


\FlushRunningHeads
\PGLicense
\begin{PGtext}
End of the Project Gutenberg EBook of Oeuvres mathématiques d'Évariste Galois, by
Évariste Galois

*** END OF THIS PROJECT GUTENBERG EBOOK OEUVRES MATHÉMATIQUES ***

***** This file should be named 40213-pdf.pdf or 40213-pdf.zip *****
This and all associated files of various formats will be found in:
        http://www.gutenberg.org/4/0/2/1/40213/

Produced by Andrew D. Hwang, K. F. Greiner, Paul Murray
and the Online Distributed Proofreading Team at
http://www.pgdp.net


Updated editions will replace the previous one--the old editions
will be renamed.

Creating the works from public domain print editions means that no
one owns a United States copyright in these works, so the Foundation
(and you!) can copy and distribute it in the United States without
permission and without paying copyright royalties.  Special rules,
set forth in the General Terms of Use part of this license, apply to
copying and distributing Project Gutenberg-tm electronic works to
protect the PROJECT GUTENBERG-tm concept and trademark.  Project
Gutenberg is a registered trademark, and may not be used if you
charge for the eBooks, unless you receive specific permission.  If you
do not charge anything for copies of this eBook, complying with the
rules is very easy.  You may use this eBook for nearly any purpose
such as creation of derivative works, reports, performances and
research.  They may be modified and printed and given away--you may do
practically ANYTHING with public domain eBooks.  Redistribution is
subject to the trademark license, especially commercial
redistribution.



*** START: FULL LICENSE ***

THE FULL PROJECT GUTENBERG LICENSE
PLEASE READ THIS BEFORE YOU DISTRIBUTE OR USE THIS WORK

To protect the Project Gutenberg-tm mission of promoting the free
distribution of electronic works, by using or distributing this work
(or any other work associated in any way with the phrase "Project
Gutenberg"), you agree to comply with all the terms of the Full Project
Gutenberg-tm License available with this file or online at
  www.gutenberg.org/license.


Section 1.  General Terms of Use and Redistributing Project Gutenberg-tm
electronic works

1.A.  By reading or using any part of this Project Gutenberg-tm
electronic work, you indicate that you have read, understand, agree to
and accept all the terms of this license and intellectual property
(trademark/copyright) agreement.  If you do not agree to abide by all
the terms of this agreement, you must cease using and return or destroy
all copies of Project Gutenberg-tm electronic works in your possession.
If you paid a fee for obtaining a copy of or access to a Project
Gutenberg-tm electronic work and you do not agree to be bound by the
terms of this agreement, you may obtain a refund from the person or
entity to whom you paid the fee as set forth in paragraph 1.E.8.

1.B.  "Project Gutenberg" is a registered trademark.  It may only be
used on or associated in any way with an electronic work by people who
agree to be bound by the terms of this agreement.  There are a few
things that you can do with most Project Gutenberg-tm electronic works
even without complying with the full terms of this agreement.  See
paragraph 1.C below.  There are a lot of things you can do with Project
Gutenberg-tm electronic works if you follow the terms of this agreement
and help preserve free future access to Project Gutenberg-tm electronic
works.  See paragraph 1.E below.

1.C.  The Project Gutenberg Literary Archive Foundation ("the Foundation"
or PGLAF), owns a compilation copyright in the collection of Project
Gutenberg-tm electronic works.  Nearly all the individual works in the
collection are in the public domain in the United States.  If an
individual work is in the public domain in the United States and you are
located in the United States, we do not claim a right to prevent you from
copying, distributing, performing, displaying or creating derivative
works based on the work as long as all references to Project Gutenberg
are removed.  Of course, we hope that you will support the Project
Gutenberg-tm mission of promoting free access to electronic works by
freely sharing Project Gutenberg-tm works in compliance with the terms of
this agreement for keeping the Project Gutenberg-tm name associated with
the work.  You can easily comply with the terms of this agreement by
keeping this work in the same format with its attached full Project
Gutenberg-tm License when you share it without charge with others.

1.D.  The copyright laws of the place where you are located also govern
what you can do with this work.  Copyright laws in most countries are in
a constant state of change.  If you are outside the United States, check
the laws of your country in addition to the terms of this agreement
before downloading, copying, displaying, performing, distributing or
creating derivative works based on this work or any other Project
Gutenberg-tm work.  The Foundation makes no representations concerning
the copyright status of any work in any country outside the United
States.

1.E.  Unless you have removed all references to Project Gutenberg:

1.E.1.  The following sentence, with active links to, or other immediate
access to, the full Project Gutenberg-tm License must appear prominently
whenever any copy of a Project Gutenberg-tm work (any work on which the
phrase "Project Gutenberg" appears, or with which the phrase "Project
Gutenberg" is associated) is accessed, displayed, performed, viewed,
copied or distributed:

This eBook is for the use of anyone anywhere at no cost and with
almost no restrictions whatsoever.  You may copy it, give it away or
re-use it under the terms of the Project Gutenberg License included
with this eBook or online at www.gutenberg.org

1.E.2.  If an individual Project Gutenberg-tm electronic work is derived
from the public domain (does not contain a notice indicating that it is
posted with permission of the copyright holder), the work can be copied
and distributed to anyone in the United States without paying any fees
or charges.  If you are redistributing or providing access to a work
with the phrase "Project Gutenberg" associated with or appearing on the
work, you must comply either with the requirements of paragraphs 1.E.1
through 1.E.7 or obtain permission for the use of the work and the
Project Gutenberg-tm trademark as set forth in paragraphs 1.E.8 or
1.E.9.

1.E.3.  If an individual Project Gutenberg-tm electronic work is posted
with the permission of the copyright holder, your use and distribution
must comply with both paragraphs 1.E.1 through 1.E.7 and any additional
terms imposed by the copyright holder.  Additional terms will be linked
to the Project Gutenberg-tm License for all works posted with the
permission of the copyright holder found at the beginning of this work.

1.E.4.  Do not unlink or detach or remove the full Project Gutenberg-tm
License terms from this work, or any files containing a part of this
work or any other work associated with Project Gutenberg-tm.

1.E.5.  Do not copy, display, perform, distribute or redistribute this
electronic work, or any part of this electronic work, without
prominently displaying the sentence set forth in paragraph 1.E.1 with
active links or immediate access to the full terms of the Project
Gutenberg-tm License.

1.E.6.  You may convert to and distribute this work in any binary,
compressed, marked up, nonproprietary or proprietary form, including any
word processing or hypertext form.  However, if you provide access to or
distribute copies of a Project Gutenberg-tm work in a format other than
"Plain Vanilla ASCII" or other format used in the official version
posted on the official Project Gutenberg-tm web site (www.gutenberg.org),
you must, at no additional cost, fee or expense to the user, provide a
copy, a means of exporting a copy, or a means of obtaining a copy upon
request, of the work in its original "Plain Vanilla ASCII" or other
form.  Any alternate format must include the full Project Gutenberg-tm
License as specified in paragraph 1.E.1.

1.E.7.  Do not charge a fee for access to, viewing, displaying,
performing, copying or distributing any Project Gutenberg-tm works
unless you comply with paragraph 1.E.8 or 1.E.9.

1.E.8.  You may charge a reasonable fee for copies of or providing
access to or distributing Project Gutenberg-tm electronic works provided
that

- You pay a royalty fee of 20% of the gross profits you derive from
     the use of Project Gutenberg-tm works calculated using the method
     you already use to calculate your applicable taxes.  The fee is
     owed to the owner of the Project Gutenberg-tm trademark, but he
     has agreed to donate royalties under this paragraph to the
     Project Gutenberg Literary Archive Foundation.  Royalty payments
     must be paid within 60 days following each date on which you
     prepare (or are legally required to prepare) your periodic tax
     returns.  Royalty payments should be clearly marked as such and
     sent to the Project Gutenberg Literary Archive Foundation at the
     address specified in Section 4, "Information about donations to
     the Project Gutenberg Literary Archive Foundation."

- You provide a full refund of any money paid by a user who notifies
     you in writing (or by e-mail) within 30 days of receipt that s/he
     does not agree to the terms of the full Project Gutenberg-tm
     License.  You must require such a user to return or
     destroy all copies of the works possessed in a physical medium
     and discontinue all use of and all access to other copies of
     Project Gutenberg-tm works.

- You provide, in accordance with paragraph 1.F.3, a full refund of any
     money paid for a work or a replacement copy, if a defect in the
     electronic work is discovered and reported to you within 90 days
     of receipt of the work.

- You comply with all other terms of this agreement for free
     distribution of Project Gutenberg-tm works.

1.E.9.  If you wish to charge a fee or distribute a Project Gutenberg-tm
electronic work or group of works on different terms than are set
forth in this agreement, you must obtain permission in writing from
both the Project Gutenberg Literary Archive Foundation and Michael
Hart, the owner of the Project Gutenberg-tm trademark.  Contact the
Foundation as set forth in Section 3 below.

1.F.

1.F.1.  Project Gutenberg volunteers and employees expend considerable
effort to identify, do copyright research on, transcribe and proofread
public domain works in creating the Project Gutenberg-tm
collection.  Despite these efforts, Project Gutenberg-tm electronic
works, and the medium on which they may be stored, may contain
"Defects," such as, but not limited to, incomplete, inaccurate or
corrupt data, transcription errors, a copyright or other intellectual
property infringement, a defective or damaged disk or other medium, a
computer virus, or computer codes that damage or cannot be read by
your equipment.

1.F.2.  LIMITED WARRANTY, DISCLAIMER OF DAMAGES - Except for the "Right
of Replacement or Refund" described in paragraph 1.F.3, the Project
Gutenberg Literary Archive Foundation, the owner of the Project
Gutenberg-tm trademark, and any other party distributing a Project
Gutenberg-tm electronic work under this agreement, disclaim all
liability to you for damages, costs and expenses, including legal
fees.  YOU AGREE THAT YOU HAVE NO REMEDIES FOR NEGLIGENCE, STRICT
LIABILITY, BREACH OF WARRANTY OR BREACH OF CONTRACT EXCEPT THOSE
PROVIDED IN PARAGRAPH 1.F.3.  YOU AGREE THAT THE FOUNDATION, THE
TRADEMARK OWNER, AND ANY DISTRIBUTOR UNDER THIS AGREEMENT WILL NOT BE
LIABLE TO YOU FOR ACTUAL, DIRECT, INDIRECT, CONSEQUENTIAL, PUNITIVE OR
INCIDENTAL DAMAGES EVEN IF YOU GIVE NOTICE OF THE POSSIBILITY OF SUCH
DAMAGE.

1.F.3.  LIMITED RIGHT OF REPLACEMENT OR REFUND - If you discover a
defect in this electronic work within 90 days of receiving it, you can
receive a refund of the money (if any) you paid for it by sending a
written explanation to the person you received the work from.  If you
received the work on a physical medium, you must return the medium with
your written explanation.  The person or entity that provided you with
the defective work may elect to provide a replacement copy in lieu of a
refund.  If you received the work electronically, the person or entity
providing it to you may choose to give you a second opportunity to
receive the work electronically in lieu of a refund.  If the second copy
is also defective, you may demand a refund in writing without further
opportunities to fix the problem.

1.F.4.  Except for the limited right of replacement or refund set forth
in paragraph 1.F.3, this work is provided to you 'AS-IS', WITH NO OTHER
WARRANTIES OF ANY KIND, EXPRESS OR IMPLIED, INCLUDING BUT NOT LIMITED TO
WARRANTIES OF MERCHANTABILITY OR FITNESS FOR ANY PURPOSE.

1.F.5.  Some states do not allow disclaimers of certain implied
warranties or the exclusion or limitation of certain types of damages.
If any disclaimer or limitation set forth in this agreement violates the
law of the state applicable to this agreement, the agreement shall be
interpreted to make the maximum disclaimer or limitation permitted by
the applicable state law.  The invalidity or unenforceability of any
provision of this agreement shall not void the remaining provisions.

1.F.6.  INDEMNITY - You agree to indemnify and hold the Foundation, the
trademark owner, any agent or employee of the Foundation, anyone
providing copies of Project Gutenberg-tm electronic works in accordance
with this agreement, and any volunteers associated with the production,
promotion and distribution of Project Gutenberg-tm electronic works,
harmless from all liability, costs and expenses, including legal fees,
that arise directly or indirectly from any of the following which you do
or cause to occur: (a) distribution of this or any Project Gutenberg-tm
work, (b) alteration, modification, or additions or deletions to any
Project Gutenberg-tm work, and (c) any Defect you cause.


Section  2.  Information about the Mission of Project Gutenberg-tm

Project Gutenberg-tm is synonymous with the free distribution of
electronic works in formats readable by the widest variety of computers
including obsolete, old, middle-aged and new computers.  It exists
because of the efforts of hundreds of volunteers and donations from
people in all walks of life.

Volunteers and financial support to provide volunteers with the
assistance they need are critical to reaching Project Gutenberg-tm's
goals and ensuring that the Project Gutenberg-tm collection will
remain freely available for generations to come.  In 2001, the Project
Gutenberg Literary Archive Foundation was created to provide a secure
and permanent future for Project Gutenberg-tm and future generations.
To learn more about the Project Gutenberg Literary Archive Foundation
and how your efforts and donations can help, see Sections 3 and 4
and the Foundation information page at www.gutenberg.org


Section 3.  Information about the Project Gutenberg Literary Archive
Foundation

The Project Gutenberg Literary Archive Foundation is a non profit
501(c)(3) educational corporation organized under the laws of the
state of Mississippi and granted tax exempt status by the Internal
Revenue Service.  The Foundation's EIN or federal tax identification
number is 64-6221541.  Contributions to the Project Gutenberg
Literary Archive Foundation are tax deductible to the full extent
permitted by U.S. federal laws and your state's laws.

The Foundation's principal office is located at 4557 Melan Dr. S.
Fairbanks, AK, 99712., but its volunteers and employees are scattered
throughout numerous locations.  Its business office is located at 809
North 1500 West, Salt Lake City, UT 84116, (801) 596-1887.  Email
contact links and up to date contact information can be found at the
Foundation's web site and official page at www.gutenberg.org/contact

For additional contact information:
     Dr. Gregory B. Newby
     Chief Executive and Director
     gbnewby@pglaf.org

Section 4.  Information about Donations to the Project Gutenberg
Literary Archive Foundation

Project Gutenberg-tm depends upon and cannot survive without wide
spread public support and donations to carry out its mission of
increasing the number of public domain and licensed works that can be
freely distributed in machine readable form accessible by the widest
array of equipment including outdated equipment.  Many small donations
($1 to $5,000) are particularly important to maintaining tax exempt
status with the IRS.

The Foundation is committed to complying with the laws regulating
charities and charitable donations in all 50 states of the United
States.  Compliance requirements are not uniform and it takes a
considerable effort, much paperwork and many fees to meet and keep up
with these requirements.  We do not solicit donations in locations
where we have not received written confirmation of compliance.  To
SEND DONATIONS or determine the status of compliance for any
particular state visit www.gutenberg.org/donate

While we cannot and do not solicit contributions from states where we
have not met the solicitation requirements, we know of no prohibition
against accepting unsolicited donations from donors in such states who
approach us with offers to donate.

International donations are gratefully accepted, but we cannot make
any statements concerning tax treatment of donations received from
outside the United States.  U.S. laws alone swamp our small staff.

Please check the Project Gutenberg Web pages for current donation
methods and addresses.  Donations are accepted in a number of other
ways including checks, online payments and credit card donations.
To donate, please visit:  www.gutenberg.org/donate


Section 5.  General Information About Project Gutenberg-tm electronic
works.

Professor Michael S. Hart was the originator of the Project Gutenberg-tm
concept of a library of electronic works that could be freely shared
with anyone.  For forty years, he produced and distributed Project
Gutenberg-tm eBooks with only a loose network of volunteer support.

Project Gutenberg-tm eBooks are often created from several printed
editions, all of which are confirmed as Public Domain in the U.S.
unless a copyright notice is included.  Thus, we do not necessarily
keep eBooks in compliance with any particular paper edition.

Most people start at our Web site which has the main PG search facility:

     www.gutenberg.org

This Web site includes information about Project Gutenberg-tm,
including how to make donations to the Project Gutenberg Literary
Archive Foundation, how to help produce our new eBooks, and how to
subscribe to our email newsletter to hear about new eBooks.
\end{PGtext}

% %%%%%%%%%%%%%%%%%%%%%%%%%%%%%%%%%%%%%%%%%%%%%%%%%%%%%%%%%%%%%%%%%%%%%%% %
%                                                                         %
% End of the Project Gutenberg EBook of Oeuvres mathématiques d'Évariste Galois, by
% Évariste Galois                                                         %
%                                                                         %
% *** END OF THIS PROJECT GUTENBERG EBOOK OEUVRES MATHÉMATIQUES ***       %
%                                                                         %
% ***** This file should be named 40213-t.tex or 40213-t.zip *****        %
% This and all associated files of various formats will be found in:      %
%         http://www.gutenberg.org/4/0/2/1/40213/                         %
%                                                                         %
% %%%%%%%%%%%%%%%%%%%%%%%%%%%%%%%%%%%%%%%%%%%%%%%%%%%%%%%%%%%%%%%%%%%%%%% %

\end{document}
###
@ControlwordReplace = (
  ['\\OE', 'Oe'],
  ['\\end{Theoreme}', ''],
  ['\\end{Probleme}', ''],
  ['\\begin{Thm}', ''],
  ['\\end{Thm}', ''],
  ['\\end{Lemme}', ''],
  ['\\end{Definitions}', '']
  );

@ControlwordArguments = (
  ['\\ToCChap', 1, 1, '', ' ', 1, 1, '', ''],
  ['\\ToCArt', 1, 1, '', '.', 1, 0, '', ''],
  ['\\Chapter', 1, 1, '', ' ', 1, 1, '', ' '],
  ['\\Article', 1, 1, '', ' ', 1, 1, '', ' ', 1, 1, '', ' '],
  ['\\Section', 1, 1, '', ' ', 1, 1, '', ' '],
  ['\\Subsection', 1, 1, '', ''],
  ['\\Par', 1, 1, '', ' '],
  ['\\begin{Theoreme}', 0, 1, 'Théorème ', ''],
  ['\\begin{Probleme}', 0, 1, 'Problème ', ''],
  ['\\begin{Lemma}', 0, 1, 'Lemme ', ''],
  ['\\begin{Definitions}', 0, 1, 'Definitions ', ''],
  ['\\Signature', 1, 1, '', ' ', 1, 1, '', ''],
  ['\\Date', 1, 1, '', ' ', 1, 1, '', ''],
  ['\\Eq', 1, 1, '', ''],
  ['\\OEUVRES', 0, 0, '', ''],
  ['\\TB', 0, 0, '', ''],
  ['\\Label', 1, 0, '', ''],
  ['\\DPtypo', 1, 0, '', '', 1, 1, '', '']
  );
###
This is pdfTeX, Version 3.1415926-1.40.10 (TeX Live 2009/Debian) (format=pdflatex 2012.3.22)  11 JUL 2012 12:44
entering extended mode
 %&-line parsing enabled.
**40213-t.tex
(./40213-t.tex
LaTeX2e <2009/09/24>
Babel <v3.8l> and hyphenation patterns for english, usenglishmax, dumylang, noh
yphenation, pinyin, loaded.
(/usr/share/texmf-texlive/tex/latex/base/book.cls
Document Class: book 2007/10/19 v1.4h Standard LaTeX document class
(/usr/share/texmf-texlive/tex/latex/base/leqno.clo
File: leqno.clo 1998/08/17 v1.1c Standard LaTeX option (left equation numbers)
) (/usr/share/texmf-texlive/tex/latex/base/bk12.clo
File: bk12.clo 2007/10/19 v1.4h Standard LaTeX file (size option)
)
\c@part=\count79
\c@chapter=\count80
\c@section=\count81
\c@subsection=\count82
\c@subsubsection=\count83
\c@paragraph=\count84
\c@subparagraph=\count85
\c@figure=\count86
\c@table=\count87
\abovecaptionskip=\skip41
\belowcaptionskip=\skip42
\bibindent=\dimen102
) (/usr/share/texmf-texlive/tex/latex/base/inputenc.sty
Package: inputenc 2008/03/30 v1.1d Input encoding file
\inpenc@prehook=\toks14
\inpenc@posthook=\toks15
(/usr/share/texmf-texlive/tex/latex/base/latin1.def
File: latin1.def 2008/03/30 v1.1d Input encoding file
)) (/usr/share/texmf-texlive/tex/latex/base/fontenc.sty
Package: fontenc 2005/09/27 v1.99g Standard LaTeX package
(/usr/share/texmf-texlive/tex/latex/base/t1enc.def
File: t1enc.def 2005/09/27 v1.99g Standard LaTeX file
LaTeX Font Info:    Redeclaring font encoding T1 on input line 43.
)) (/var/lib/texmf/tex/generic/babel/babel.sty
Package: babel 2008/07/06 v3.8l The Babel package
(/usr/share/texmf-texlive/tex/generic/babel/frenchb.ldf
Language: frenchb 2009/03/16 v2.3d French support from the babel system
(/usr/share/texmf-texlive/tex/generic/babel/babel.def
File: babel.def 2008/07/06 v3.8l Babel common definitions
\babel@savecnt=\count88
\U@D=\dimen103
)

Package babel Warning: No hyphenation patterns were loaded for
(babel)                the language `French'
(babel)                I will use the patterns loaded for \language=0 instead.

\l@french = a dialect from \language0
Package babel Info: Making : an active character on input line 120.
Package babel Info: Making ; an active character on input line 121.
Package babel Info: Making ! an active character on input line 122.
Package babel Info: Making ? an active character on input line 123.
\FB@Mht=\dimen104
\std@mcc=\count89
\dec@mcc=\count90
\parindentFFN=\dimen105
*************************************
* Local config file frenchb.cfg used
*
(/usr/share/texmf-texlive/tex/generic/babel/frenchb.cfg))) (/usr/share/texmf-te
xlive/tex/latex/carlisle/scalefnt.sty) (/usr/share/texmf-texlive/tex/latex/grap
hics/keyval.sty
Package: keyval 1999/03/16 v1.13 key=value parser (DPC)
\KV@toks@=\toks16
) (/usr/share/texmf-texlive/tex/latex/base/ifthen.sty
Package: ifthen 2001/05/26 v1.1c Standard LaTeX ifthen package (DPC)
) (/usr/share/texmf-texlive/tex/latex/amsmath/amsmath.sty
Package: amsmath 2000/07/18 v2.13 AMS math features
\@mathmargin=\skip43
For additional information on amsmath, use the `?' option.
(/usr/share/texmf-texlive/tex/latex/amsmath/amstext.sty
Package: amstext 2000/06/29 v2.01
(/usr/share/texmf-texlive/tex/latex/amsmath/amsgen.sty
File: amsgen.sty 1999/11/30 v2.0
\@emptytoks=\toks17
\ex@=\dimen106
)) (/usr/share/texmf-texlive/tex/latex/amsmath/amsbsy.sty
Package: amsbsy 1999/11/29 v1.2d
\pmbraise@=\dimen107
) (/usr/share/texmf-texlive/tex/latex/amsmath/amsopn.sty
Package: amsopn 1999/12/14 v2.01 operator names
)
\inf@bad=\count91
LaTeX Info: Redefining \frac on input line 211.
\uproot@=\count92
\leftroot@=\count93
LaTeX Info: Redefining \overline on input line 307.
\classnum@=\count94
\DOTSCASE@=\count95
LaTeX Info: Redefining \ldots on input line 379.
LaTeX Info: Redefining \dots on input line 382.
LaTeX Info: Redefining \cdots on input line 467.
\Mathstrutbox@=\box26
\strutbox@=\box27
\big@size=\dimen108
LaTeX Font Info:    Redeclaring font encoding OML on input line 567.
LaTeX Font Info:    Redeclaring font encoding OMS on input line 568.
\macc@depth=\count96
\c@MaxMatrixCols=\count97
\dotsspace@=\muskip10
\c@parentequation=\count98
\dspbrk@lvl=\count99
\tag@help=\toks18
\row@=\count100
\column@=\count101
\maxfields@=\count102
\andhelp@=\toks19
\eqnshift@=\dimen109
\alignsep@=\dimen110
\tagshift@=\dimen111
\tagwidth@=\dimen112
\totwidth@=\dimen113
\lineht@=\dimen114
\@envbody=\toks20
\multlinegap=\skip44
\multlinetaggap=\skip45
\mathdisplay@stack=\toks21
LaTeX Info: Redefining \[ on input line 2666.
LaTeX Info: Redefining \] on input line 2667.
) (/usr/share/texmf-texlive/tex/latex/amsfonts/amssymb.sty
Package: amssymb 2009/06/22 v3.00
(/usr/share/texmf-texlive/tex/latex/amsfonts/amsfonts.sty
Package: amsfonts 2009/06/22 v3.00 Basic AMSFonts support
\symAMSa=\mathgroup4
\symAMSb=\mathgroup5
LaTeX Font Info:    Overwriting math alphabet `\mathfrak' in version `bold'
(Font)                  U/euf/m/n --> U/euf/b/n on input line 96.
)) (/usr/share/texmf-texlive/tex/latex/base/alltt.sty
Package: alltt 1997/06/16 v2.0g defines alltt environment
) (/usr/share/texmf-texlive/tex/latex/tools/indentfirst.sty
Package: indentfirst 1995/11/23 v1.03 Indent first paragraph (DPC)
) (/usr/share/texmf-texlive/tex/latex/footmisc/footmisc.sty
Package: footmisc 2009/09/15 v5.5a a miscellany of footnote facilities
\FN@temptoken=\toks22
\footnotemargin=\dimen115
\c@pp@next@reset=\count103
\c@@fnserial=\count104
Package footmisc Info: Declaring symbol style bringhurst on input line 855.
Package footmisc Info: Declaring symbol style chicago on input line 863.
Package footmisc Info: Declaring symbol style wiley on input line 872.
Package footmisc Info: Declaring symbol style lamport-robust on input line 883.

Package footmisc Info: Declaring symbol style lamport* on input line 903.
Package footmisc Info: Declaring symbol style lamport*-robust on input line 924
.
) (/usr/share/texmf-texlive/tex/latex/tools/calc.sty
Package: calc 2007/08/22 v4.3 Infix arithmetic (KKT,FJ)
\calc@Acount=\count105
\calc@Bcount=\count106
\calc@Adimen=\dimen116
\calc@Bdimen=\dimen117
\calc@Askip=\skip46
\calc@Bskip=\skip47
LaTeX Info: Redefining \setlength on input line 76.
LaTeX Info: Redefining \addtolength on input line 77.
\calc@Ccount=\count107
\calc@Cskip=\skip48
) (/usr/share/texmf-texlive/tex/latex/fancyhdr/fancyhdr.sty
\fancy@headwidth=\skip49
\f@ncyO@elh=\skip50
\f@ncyO@erh=\skip51
\f@ncyO@olh=\skip52
\f@ncyO@orh=\skip53
\f@ncyO@elf=\skip54
\f@ncyO@erf=\skip55
\f@ncyO@olf=\skip56
\f@ncyO@orf=\skip57
) (/usr/share/texmf-texlive/tex/latex/geometry/geometry.sty
Package: geometry 2008/12/21 v4.2 Page Geometry
(/usr/share/texmf-texlive/tex/generic/oberdiek/ifpdf.sty
Package: ifpdf 2009/04/10 v2.0 Provides the ifpdf switch (HO)
Package ifpdf Info: pdfTeX in pdf mode detected.
) (/usr/share/texmf-texlive/tex/generic/oberdiek/ifvtex.sty
Package: ifvtex 2008/11/04 v1.4 Switches for detecting VTeX and its modes (HO)
Package ifvtex Info: VTeX not detected.
)
\Gm@cnth=\count108
\Gm@cntv=\count109
\c@Gm@tempcnt=\count110
\Gm@bindingoffset=\dimen118
\Gm@wd@mp=\dimen119
\Gm@odd@mp=\dimen120
\Gm@even@mp=\dimen121
\Gm@dimlist=\toks23
(/usr/share/texmf-texlive/tex/xelatex/xetexconfig/geometry.cfg)) (/usr/share/te
xmf-texlive/tex/latex/hyperref/hyperref.sty
Package: hyperref 2009/10/09 v6.79a Hypertext links for LaTeX
(/usr/share/texmf-texlive/tex/generic/ifxetex/ifxetex.sty
Package: ifxetex 2009/01/23 v0.5 Provides ifxetex conditional
) (/usr/share/texmf-texlive/tex/latex/oberdiek/hycolor.sty
Package: hycolor 2009/10/02 v1.5 Code for color options of hyperref/bookmark (H
O)
(/usr/share/texmf-texlive/tex/latex/oberdiek/xcolor-patch.sty
Package: xcolor-patch 2009/10/02 xcolor patch
))
\@linkdim=\dimen122
\Hy@linkcounter=\count111
\Hy@pagecounter=\count112
(/usr/share/texmf-texlive/tex/latex/hyperref/pd1enc.def
File: pd1enc.def 2009/10/09 v6.79a Hyperref: PDFDocEncoding definition (HO)
) (/usr/share/texmf-texlive/tex/generic/oberdiek/etexcmds.sty
Package: etexcmds 2007/12/12 v1.2 Prefix for e-TeX command names (HO)
(/usr/share/texmf-texlive/tex/generic/oberdiek/infwarerr.sty
Package: infwarerr 2007/09/09 v1.2 Providing info/warning/message (HO)
)
Package etexcmds Info: Could not find \expanded.
(etexcmds)             That can mean that you are not using pdfTeX 1.50 or
(etexcmds)             that some package has redefined \expanded.
(etexcmds)             In the latter case, load this package earlier.
) (/etc/texmf/tex/latex/config/hyperref.cfg
File: hyperref.cfg 2002/06/06 v1.2 hyperref configuration of TeXLive
) (/usr/share/texmf-texlive/tex/latex/oberdiek/kvoptions.sty
Package: kvoptions 2009/08/13 v3.4 Keyval support for LaTeX options (HO)
(/usr/share/texmf-texlive/tex/generic/oberdiek/kvsetkeys.sty
Package: kvsetkeys 2009/07/30 v1.5 Key value parser with default handler suppor
t (HO)
))
Package hyperref Info: Option `hyperfootnotes' set `false' on input line 2864.
Package hyperref Info: Option `bookmarks' set `true' on input line 2864.
Package hyperref Info: Option `linktocpage' set `false' on input line 2864.
Package hyperref Info: Option `pdfdisplaydoctitle' set `true' on input line 286
4.
Package hyperref Info: Option `pdfpagelabels' set `true' on input line 2864.
Package hyperref Info: Option `bookmarksopen' set `true' on input line 2864.
Package hyperref Info: Option `colorlinks' set `true' on input line 2864.
Package hyperref Info: Hyper figures OFF on input line 2975.
Package hyperref Info: Link nesting OFF on input line 2980.
Package hyperref Info: Hyper index ON on input line 2983.
Package hyperref Info: Plain pages OFF on input line 2990.
Package hyperref Info: Backreferencing OFF on input line 2995.
Implicit mode ON; LaTeX internals redefined
Package hyperref Info: Bookmarks ON on input line 3191.
(/usr/share/texmf-texlive/tex/latex/ltxmisc/url.sty
\Urlmuskip=\muskip11
Package: url 2006/04/12  ver 3.3  Verb mode for urls, etc.
)
LaTeX Info: Redefining \url on input line 3428.
(/usr/share/texmf-texlive/tex/generic/oberdiek/bitset.sty
Package: bitset 2007/09/28 v1.0 Data type bit set (HO)
(/usr/share/texmf-texlive/tex/generic/oberdiek/intcalc.sty
Package: intcalc 2007/09/27 v1.1 Expandable integer calculations (HO)
) (/usr/share/texmf-texlive/tex/generic/oberdiek/bigintcalc.sty
Package: bigintcalc 2007/11/11 v1.1 Expandable big integer calculations (HO)
(/usr/share/texmf-texlive/tex/generic/oberdiek/pdftexcmds.sty
Package: pdftexcmds 2009/09/23 v0.6 LuaTeX support for pdfTeX utility functions
 (HO)
(/usr/share/texmf-texlive/tex/generic/oberdiek/ifluatex.sty
Package: ifluatex 2009/04/17 v1.2 Provides the ifluatex switch (HO)
Package ifluatex Info: LuaTeX not detected.
) (/usr/share/texmf-texlive/tex/generic/oberdiek/ltxcmds.sty
Package: ltxcmds 2009/08/05 v1.0 Some LaTeX kernel commands for general use (HO
)
)
Package pdftexcmds Info: LuaTeX not detected.
Package pdftexcmds Info: \pdf@primitive is available.
Package pdftexcmds Info: \pdf@ifprimitive is available.
)))
\Fld@menulength=\count113
\Field@Width=\dimen123
\Fld@charsize=\dimen124
\Field@toks=\toks24
Package hyperref Info: Hyper figures OFF on input line 4377.
Package hyperref Info: Link nesting OFF on input line 4382.
Package hyperref Info: Hyper index ON on input line 4385.
Package hyperref Info: backreferencing OFF on input line 4392.
Package hyperref Info: Link coloring ON on input line 4395.
Package hyperref Info: Link coloring with OCG OFF on input line 4402.
Package hyperref Info: PDF/A mode OFF on input line 4407.
(/usr/share/texmf-texlive/tex/generic/oberdiek/atbegshi.sty
Package: atbegshi 2008/07/31 v1.9 At begin shipout hook (HO)
)
\Hy@abspage=\count114
\c@Item=\count115
)
*hyperref using driver hpdftex*
(/usr/share/texmf-texlive/tex/latex/hyperref/hpdftex.def
File: hpdftex.def 2009/10/09 v6.79a Hyperref driver for pdfTeX
\Fld@listcount=\count116
)
\TmpLen=\skip58
\c@ToCArtNo=\count117
\c@ArtNo=\count118
(./40213-t.aux)
\openout1 = `40213-t.aux'.

LaTeX Font Info:    Checking defaults for OML/cmm/m/it on input line 513.
LaTeX Font Info:    ... okay on input line 513.
LaTeX Font Info:    Checking defaults for T1/cmr/m/n on input line 513.
LaTeX Font Info:    ... okay on input line 513.
LaTeX Font Info:    Checking defaults for OT1/cmr/m/n on input line 513.
LaTeX Font Info:    ... okay on input line 513.
LaTeX Font Info:    Checking defaults for OMS/cmsy/m/n on input line 513.
LaTeX Font Info:    ... okay on input line 513.
LaTeX Font Info:    Checking defaults for OMX/cmex/m/n on input line 513.
LaTeX Font Info:    ... okay on input line 513.
LaTeX Font Info:    Checking defaults for U/cmr/m/n on input line 513.
LaTeX Font Info:    ... okay on input line 513.
LaTeX Font Info:    Checking defaults for PD1/pdf/m/n on input line 513.
LaTeX Font Info:    ... okay on input line 513.
LaTeX Info: Redefining \degres on input line 513.
LaTeX Info: Redefining \dots on input line 513.
LaTeX Info: Redefining \up on input line 513.
*geometry auto-detecting driver*
*geometry detected driver: pdftex*
-------------------- Geometry parameters
paper: class default
landscape: --
twocolumn: --
twoside: true
asymmetric: --
h-parts: 9.03374pt, 379.4175pt, 9.03375pt
v-parts: 7.04944pt, 560.53635pt, 10.5742pt
hmarginratio: 1:1
vmarginratio: 2:3
lines: --
heightrounded: --
bindingoffset: 0.0pt
truedimen: --
includehead: true
includefoot: true
includemp: --
driver: pdftex
-------------------- Page layout dimensions and switches
\paperwidth  397.48499pt
\paperheight 578.15999pt
\textwidth  379.4175pt
\textheight 498.66255pt
\oddsidemargin  -63.23625pt
\evensidemargin -63.23624pt
\topmargin  -65.22055pt
\headheight 15.0pt
\headsep    19.8738pt
\footskip   30.0pt
\marginparwidth 98.0pt
\marginparsep   7.0pt
\columnsep  10.0pt
\skip\footins  10.8pt plus 4.0pt minus 2.0pt
\hoffset 0.0pt
\voffset 0.0pt
\mag 1000
\@twosidetrue \@mparswitchtrue
(1in=72.27pt, 1cm=28.45pt)
-----------------------
(/usr/share/texmf-texlive/tex/latex/graphics/color.sty
Package: color 2005/11/14 v1.0j Standard LaTeX Color (DPC)
(/etc/texmf/tex/latex/config/color.cfg
File: color.cfg 2007/01/18 v1.5 color configuration of teTeX/TeXLive
)
Package color Info: Driver file: pdftex.def on input line 130.
(/usr/share/texmf-texlive/tex/latex/pdftex-def/pdftex.def
File: pdftex.def 2009/08/25 v0.04m Graphics/color for pdfTeX
\Gread@gobject=\count119
(/usr/share/texmf-texlive/tex/context/base/supp-pdf.mkii
[Loading MPS to PDF converter (version 2006.09.02).]
\scratchcounter=\count120
\scratchdimen=\dimen125
\scratchbox=\box28
\nofMPsegments=\count121
\nofMParguments=\count122
\everyMPshowfont=\toks25
\MPscratchCnt=\count123
\MPscratchDim=\dimen126
\MPnumerator=\count124
\everyMPtoPDFconversion=\toks26
)))
Package hyperref Info: Link coloring ON on input line 513.
(/usr/share/texmf-texlive/tex/latex/hyperref/nameref.sty
Package: nameref 2007/05/29 v2.31 Cross-referencing by name of section
(/usr/share/texmf-texlive/tex/latex/oberdiek/refcount.sty
Package: refcount 2008/08/11 v3.1 Data extraction from references (HO)
)
\c@section@level=\count125
)
LaTeX Info: Redefining \ref on input line 513.
LaTeX Info: Redefining \pageref on input line 513.
(./40213-t.out) (./40213-t.out)
\@outlinefile=\write3
\openout3 = `40213-t.out'.

\AtBeginShipoutBox=\box29
LaTeX Font Info:    Try loading font information for T1+cmtt on input line 520.

(/usr/share/texmf-texlive/tex/latex/base/t1cmtt.fd
File: t1cmtt.fd 1999/05/25 v2.5h Standard LaTeX font definitions
)
LaTeX Font Info:    Try loading font information for U+msa on input line 545.
(/usr/share/texmf-texlive/tex/latex/amsfonts/umsa.fd
File: umsa.fd 2009/06/22 v3.00 AMS symbols A
)
LaTeX Font Info:    Try loading font information for U+msb on input line 545.
(/usr/share/texmf-texlive/tex/latex/amsfonts/umsb.fd
File: umsb.fd 2009/06/22 v3.00 AMS symbols B
) [1



{/var/lib/texmf/fonts/map/pdftex/updmap/pdftex.map}] [2

] [1

] [2]
Underfull \vbox (badness 1910) has occurred while \output is active []

[3


]
Underfull \vbox (badness 10000) has occurred while \output is active []

[4]
Underfull \vbox (badness 6792) has occurred while \output is active []

[5] [6] [1





] [2] [3]
Underfull \vbox (badness 10000) has occurred while \output is active []

[4]
Underfull \vbox (badness 10000) has occurred while \output is active []

[5] [6] [7] [8]
Overfull \hbox (4.09212pt too wide) detected at line 1140
\OML/cmm/m/it/10.95 x \OT1/cmr/m/n/10.95 = 3 + []\OML/cmm/m/it/10.95 x \OT1/cmr
/m/n/10.95 = 3 \OMS/cmsy/m/n/10.95 ^^@ []
 []

[9] [10] [11


] [12] [13


] [14] [15


] [16] [17


] [18] [19] [20] [21] [22] [23] [24] [25] [26


] [27] [28] [29] [30] [31] [32] [33]

LaTeX Font Warning: Font shape `T1/cmr/m/n' in size <6.49994> not available
(Font)              size <6> substituted on input line 2279.

[34


]
Overfull \hbox (0.79935pt too wide) in paragraph at lines 2333--2337
[]\T1/cmr/m/n/12 Mais quand les co-ef-fi-cients d'une équa-tion ne seront \T1/c
mr/m/it/12 pas tous \T1/cmr/m/n/12 numériques
 []

[35] [36] [37] [38] [39] [40] [41] [42] [43] [44]
Underfull \vbox (badness 1152) has occurred while \output is active []

[45] [46] [47] [48] [49] [50] [51


] [52] [53] [54] [55] [56] [57] [58] [59] [60] [61] [62





]
Overfull \hbox (3.21407pt too wide) in paragraph at lines 3673--3673
[]\T1/cmtt/m/n/9 End of the Project Gutenberg EBook of Oeuvres mathématiques d'
Évariste Galois, by[]
 []


Underfull \vbox (badness 10000) has occurred while \output is active []

[1




]
Underfull \vbox (badness 10000) has occurred while \output is active []

[2]
Underfull \vbox (badness 10000) has occurred while \output is active []

[3]
Underfull \vbox (badness 10000) has occurred while \output is active []

[4]
Underfull \vbox (badness 10000) has occurred while \output is active []

[5]
Underfull \vbox (badness 10000) has occurred while \output is active []

[6]
Underfull \vbox (badness 10000) has occurred while \output is active []

[7] [8] (./40213-t.aux)

 *File List*
    book.cls    2007/10/19 v1.4h Standard LaTeX document class
   leqno.clo    1998/08/17 v1.1c Standard LaTeX option (left equation numbers)
    bk12.clo    2007/10/19 v1.4h Standard LaTeX file (size option)
inputenc.sty    2008/03/30 v1.1d Input encoding file
  latin1.def    2008/03/30 v1.1d Input encoding file
 fontenc.sty
   t1enc.def    2005/09/27 v1.99g Standard LaTeX file
   babel.sty    2008/07/06 v3.8l The Babel package
 frenchb.ldf    2009/03/16 v2.3d French support from the babel system
 frenchb.cfg
scalefnt.sty
  keyval.sty    1999/03/16 v1.13 key=value parser (DPC)
  ifthen.sty    2001/05/26 v1.1c Standard LaTeX ifthen package (DPC)
 amsmath.sty    2000/07/18 v2.13 AMS math features
 amstext.sty    2000/06/29 v2.01
  amsgen.sty    1999/11/30 v2.0
  amsbsy.sty    1999/11/29 v1.2d
  amsopn.sty    1999/12/14 v2.01 operator names
 amssymb.sty    2009/06/22 v3.00
amsfonts.sty    2009/06/22 v3.00 Basic AMSFonts support
   alltt.sty    1997/06/16 v2.0g defines alltt environment
indentfirst.sty    1995/11/23 v1.03 Indent first paragraph (DPC)
footmisc.sty    2009/09/15 v5.5a a miscellany of footnote facilities
    calc.sty    2007/08/22 v4.3 Infix arithmetic (KKT,FJ)
fancyhdr.sty
geometry.sty    2008/12/21 v4.2 Page Geometry
   ifpdf.sty    2009/04/10 v2.0 Provides the ifpdf switch (HO)
  ifvtex.sty    2008/11/04 v1.4 Switches for detecting VTeX and its modes (HO)
geometry.cfg
hyperref.sty    2009/10/09 v6.79a Hypertext links for LaTeX
 ifxetex.sty    2009/01/23 v0.5 Provides ifxetex conditional
 hycolor.sty    2009/10/02 v1.5 Code for color options of hyperref/bookmark (HO
)
xcolor-patch.sty    2009/10/02 xcolor patch
  pd1enc.def    2009/10/09 v6.79a Hyperref: PDFDocEncoding definition (HO)
etexcmds.sty    2007/12/12 v1.2 Prefix for e-TeX command names (HO)
infwarerr.sty    2007/09/09 v1.2 Providing info/warning/message (HO)
hyperref.cfg    2002/06/06 v1.2 hyperref configuration of TeXLive
kvoptions.sty    2009/08/13 v3.4 Keyval support for LaTeX options (HO)
kvsetkeys.sty    2009/07/30 v1.5 Key value parser with default handler support
(HO)
     url.sty    2006/04/12  ver 3.3  Verb mode for urls, etc.
  bitset.sty    2007/09/28 v1.0 Data type bit set (HO)
 intcalc.sty    2007/09/27 v1.1 Expandable integer calculations (HO)
bigintcalc.sty    2007/11/11 v1.1 Expandable big integer calculations (HO)
pdftexcmds.sty    2009/09/23 v0.6 LuaTeX support for pdfTeX utility functions (
HO)
ifluatex.sty    2009/04/17 v1.2 Provides the ifluatex switch (HO)
 ltxcmds.sty    2009/08/05 v1.0 Some LaTeX kernel commands for general use (HO)

atbegshi.sty    2008/07/31 v1.9 At begin shipout hook (HO)
 hpdftex.def    2009/10/09 v6.79a Hyperref driver for pdfTeX
   color.sty    2005/11/14 v1.0j Standard LaTeX Color (DPC)
   color.cfg    2007/01/18 v1.5 color configuration of teTeX/TeXLive
  pdftex.def    2009/08/25 v0.04m Graphics/color for pdfTeX
supp-pdf.mkii
 nameref.sty    2007/05/29 v2.31 Cross-referencing by name of section
refcount.sty    2008/08/11 v3.1 Data extraction from references (HO)
 40213-t.out
 40213-t.out
  t1cmtt.fd    1999/05/25 v2.5h Standard LaTeX font definitions
    umsa.fd    2009/06/22 v3.00 AMS symbols A
    umsb.fd    2009/06/22 v3.00 AMS symbols B
 ***********


LaTeX Font Warning: Size substitutions with differences
(Font)              up to 0.49994pt have occurred.

 )
Here is how much of TeX's memory you used:
 6859 strings out of 495046
 93628 string characters out of 1181937
 181032 words of memory out of 3000000
 9733 multiletter control sequences out of 15000+50000
 31956 words of font info for 78 fonts, out of 3000000 for 9000
 28 hyphenation exceptions out of 8191
 41i,47n,61p,298b,504s stack positions out of 5000i,500n,10000p,200000b,50000s
{/usr/share/texmf/fonts/enc/dvips/cm-super/cm-super-t1.enc}</usr/share/texmf-
texlive/fonts/type1/public/amsfonts/cm/cmex10.pfb></usr/share/texmf-texlive/fon
ts/type1/public/amsfonts/cm/cmmi10.pfb></usr/share/texmf-texlive/fonts/type1/pu
blic/amsfonts/cm/cmmi5.pfb></usr/share/texmf-texlive/fonts/type1/public/amsfont
s/cm/cmmi7.pfb></usr/share/texmf-texlive/fonts/type1/public/amsfonts/cm/cmmi9.p
fb></usr/share/texmf-texlive/fonts/type1/public/amsfonts/cm/cmr10.pfb></usr/sha
re/texmf-texlive/fonts/type1/public/amsfonts/cm/cmr5.pfb></usr/share/texmf-texl
ive/fonts/type1/public/amsfonts/cm/cmr7.pfb></usr/share/texmf-texlive/fonts/typ
e1/public/amsfonts/cm/cmr9.pfb></usr/share/texmf-texlive/fonts/type1/public/ams
fonts/cm/cmsy10.pfb></usr/share/texmf-texlive/fonts/type1/public/amsfonts/cm/cm
sy5.pfb></usr/share/texmf-texlive/fonts/type1/public/amsfonts/cm/cmsy7.pfb></us
r/share/texmf-texlive/fonts/type1/public/amsfonts/cm/cmsy9.pfb></usr/share/texm
f/fonts/type1/public/cm-super/sfbx0800.pfb></usr/share/texmf/fonts/type1/public
/cm-super/sfbx1000.pfb></usr/share/texmf/fonts/type1/public/cm-super/sfbx1200.p
fb></usr/share/texmf/fonts/type1/public/cm-super/sfbx1728.pfb></usr/share/texmf
/fonts/type1/public/cm-super/sfbx2074.pfb></usr/share/texmf/fonts/type1/public/
cm-super/sfbx2488.pfb></usr/share/texmf/fonts/type1/public/cm-super/sfcc1000.pf
b></usr/share/texmf/fonts/type1/public/cm-super/sfcc1095.pfb></usr/share/texmf/
fonts/type1/public/cm-super/sfcc1200.pfb></usr/share/texmf/fonts/type1/public/c
m-super/sfcc1440.pfb></usr/share/texmf/fonts/type1/public/cm-super/sfrm0600.pfb
></usr/share/texmf/fonts/type1/public/cm-super/sfrm0700.pfb></usr/share/texmf/f
onts/type1/public/cm-super/sfrm0800.pfb></usr/share/texmf/fonts/type1/public/cm
-super/sfrm0900.pfb></usr/share/texmf/fonts/type1/public/cm-super/sfrm1000.pfb>
</usr/share/texmf/fonts/type1/public/cm-super/sfrm1095.pfb></usr/share/texmf/fo
nts/type1/public/cm-super/sfrm1200.pfb></usr/share/texmf/fonts/type1/public/cm-
super/sfrm1440.pfb></usr/share/texmf/fonts/type1/public/cm-super/sfti0700.pfb><
/usr/share/texmf/fonts/type1/public/cm-super/sfti1000.pfb></usr/share/texmf/fon
ts/type1/public/cm-super/sfti1200.pfb></usr/share/texmf/fonts/type1/public/cm-s
uper/sftt0900.pfb></usr/share/texmf/fonts/type1/public/cm-super/sftt1095.pfb>
Output written on 40213-t.pdf (78 pages, 727389 bytes).
PDF statistics:
 630 PDF objects out of 1000 (max. 8388607)
 159 named destinations out of 1000 (max. 500000)
 73 words of extra memory for PDF output out of 10000 (max. 10000000)

