\documentclass{article}
\setlength{\headheight}{22.50113pt}
\addtolength{\topmargin}{-10.50113pt}

\usepackage{preamble}
\usepackage{letterfont}
\usepackage{macros}

\fancyhead[L]{\bd{Josh Park \\ Prof. Shkredov}}
\fancyhead[C]{\bd{MA 45401-H01 \\ Galois Theory Honors}}
\fancyhead[R]{\bd{Spring 2025 \\ Homework 3}}

\begin{document}

% \begin{note}[Notation]
% In the following problems, I denote the principal ideal generated by \( f(x) \) by \( \pid{f(x)} \). \\
% In the context of a factor ring \( \fr{L}{K} \), I may use equivalence classes or coset representatives interchangeably. The meaning should be clear from context, but I will do my best to disambiguate as necessary.
% \end{note}

\setcounter{section}{3}
\setcounter{exercise}{1}
\begin{subexercise}
Show that \( t^3+t+1 \) is irreducible in \( \Ftw[t] \).
\end{subexercise}

\begin{solution}
Assume, for the sake of contradiction, that \( f(t) = t^3+t+1 \) is reducible over \( \Ftw[t] \).
Then, \( f(t) = g(t)h(t) \) for some \( g(t),h(t)\in \Ftw[t] \).
Without loss of generality, \( \deg g(t) = 2 \tand \deg h(t) = 1 \).
Since \( \deg h(t) = 1 \) over \( \Ftw[t] \), we have that either \( h(t) = t \) or \( h(t) = t+1 \).
However, notice that \( f(1) \neq 0\tand f(0)\neq 0 \).
Thus \( f(t) \) has no linear factors, contradicting that \( \deg h(t) = 1 \).
Therefore \( f(t) = t^3+t+1 \) must be irreducible over the field \( \Ftw[t] \).
\end{solution}

\begin{subexercise}
Consider the quotient ring \( L:= \fr{\Ftw[t]}{\pid{t^3+t+1}} \) and compute its size.
\end{subexercise}

\begin{solution}
Let \( f = t^3+t+1 \).

Then the factor ring \( \fr{\Ftw[t]}{\pid{f}} \) partitions elements of \( \Ftw[t] \) into the following equivalence classes: \begin{align*}
  [0],\ [1],\ [t] ,\ [t+1],\ [t\sq],\ [t\sq+1],\ [t\sq+t],\ [t\sq+t+1]
\end{align*}
Hence \( \order{L} = 8 \).
\end{solution}

\begin{subexercise}
Take \( g=t+1 \) and prove the set \( \{0,g,g^2,\ldots,g^7\} \) coincides with \( L \).
\end{subexercise}

\begin{solution}
Obviously this set has 8 elements, which agrees with our result in Exercise 3.1.2.
It remains to show that each element corresponds to a unique equivalence class from above (taken mod \( f \)).
\[\begin{array}{lll}
  0   \equiv 0       &\!\!\!\!\!\!\!\!\pmod f \imp 0 \in [0] \\
  g   \equiv t+1     &\!\!\!\!\!\!\!\!\pmod f \imp g \in [t+1] \\
  g^2 \equiv t^2+1   &\!\!\!\!\!\!\!\!\pmod f \imp g^2 \in [t^2+1] \\
  g^3 \equiv t^2     &\!\!\!\!\!\!\!\!\pmod f \imp g^3 \in [t^2] \\
  g^4 \equiv t^2+t+1 &\!\!\!\!\!\!\!\!\pmod f \imp g^4 \in [t^2+t+1] \\
  g^5 \equiv t       &\!\!\!\!\!\!\!\!\pmod f \imp g^5 \in [t] \\
  g^6 \equiv t^2+t   &\!\!\!\!\!\!\!\!\pmod f \imp g^6 \in [t^2+t] \\
  g^7 \equiv 1       &\!\!\!\!\!\!\!\!\pmod f \imp g^7 \in [1]
\end{array}\]
Thus there is a clear bijection between the set \( \{0,g,g\sq,\ldots,g^7\} \) and \( L \).
\end{solution}

% \stepcounter{exercise}
\begin{exercise}
Let \( K \) be a field and \( p, q \in K[t] \) be irreducible polynomials over \( K \), \( \pid p \neq \pid q \) (this is equivalent to the statement that \( p \) is coprime to \( q \)). Consider the field \( \bbF := K(t) \) and the polynomial \( g(x) = x^n + px + pq \in \bbF[x] \). Prove that \( g \) is irreducible over \( \bbF \).
\end{exercise}

\begin{solution}
From lecture, \( F[t] \) is a Euclidean domain for any field \( F \) and any Euclidean domain is also a unique factorization domain, so \( \bbF[x] \) is a UFD.
Next, it is easy to see that \( \gcd(g(x)) = \gcd(1, p, pq) = 1 \).
Notice that for the irreducible polynomial \( p \in \bbF \), we have that \( p \divs p \), \( p\divs pq \), \( p\ndivs 1 \) and obviously \( p\sq \ndivs pq \) (otherwise \( p\sq\divs pq \imp p\divs q \) contradicts that they are coprime).
Thus by Eisenstein's Criterion \( g \) is irreducible over \( \bbF \).
\end{solution}

\begin{exercise}
Prove that \( t^2 -7 \) is irreducible over \( \Q(\sqrt 5) \).
\end{exercise}

\begin{solution}
Let \( f(t) = t\sq - 7 \).
Assume for the sake of contradiction that \( f \) is reducible.
By definiton of reducible, \( f \) must equal the product of polynomials with strictly lower degree, so \( f=gh \imp \deg(g) = \deg(h) = 1 \).
This means \( g\tand h \) are linear factors, which implies that \( \exists x\in \Q(\sqrt 5) \) such that \( f(x) = 0 \).
Since \( x\in \Q(\sqrt 5) \imp x=a+b\sqrt 5 \) for \( a,b\in \Q \), notice \begin{align*}
  f(x) = 0 &\imp (a+b\sqrt{5})\sq - 7 = 0 \\
  &\imp a\sq + 2ab\sqrt 5 + 5b\sq - 7 = 0 \\
  &\imp a\sq + 5b\sq - 7 = - 2ab\sqrt 5 \\
  &\imp \frac{a\sq + 5b\sq - 7}{-2ab} = \sqrt 5 \imp \sqrt 5\in \Q
\end{align*}
which is obviously a contradiction.
Thus \( f(t) = t\sq - 7 \) is irreducible over \( \Q(\sqrt 5) \).
\end{solution}

\stepcounter{exercise}
\begin{subexercise}
Let \( \alpha=2^{1/6} \) and \( \veps_3^3=1 \), \( \veps_3\neq 1 \). Find the minimal polynomials of \( \alpha \) over \begin{align*}
  a)\ \Q, \quad b)\ \Q(\alpha), \quad c)\ \Q(\alpha\sq), \quad d)\ \Q(\alpha\veps_3).
\end{align*}
\end{subexercise}

\begin{solution}
\begin{enumerate}[label=\alph*)]
\item In \( \Q \), \begin{align*}
  \alpha = 2^{1/6} &\imp x = 2^{1/6}\\
  &\imp x^6 = 2 \\
  &\imp x^6 - 2 = 0.
\end{align*}
Let \( f(x) = x^6-2 \).
By Eisenstein (using \( p=2 \)), \( f \) is irreducible.
Thus \( \mu_\alpha^\Q (x) = x^6 - 2 \).

\item In \( \Q(\alpha)=\Q(2^{1/6}) \), \begin{align*}
  \alpha = 2^{1/6} &\imp x=2^{1/6} \\
  &\imp x - 2^{1/6} = 0.
\end{align*}
Let \( g(x) = x-2^{1/6} \).
Since \( \deg(x-2^{1/6}) = 1 \), it can not be decomposed into  polynomials of smaller degree and is therefore irreducible by definition.
Thus \( \mu_\alpha^{\Q(\alpha)} (x) = x - 2^{1/6} \).

\item In \( \Q(\alpha\sq)=\Q(2^{1/3}) \), \begin{align*}
  \alpha\sq = 2^{1/3} &\imp x\sq = 2^{1/3} \\
  &\imp x\sq - 2^{1/3} = 0.
\end{align*}
Let \( h(x) = x\sq-2^{1/3} \). Assuming \( h \) is reducible, it must decompose into linear factors.
However, notice \( h \) does not have any roots in \( \Q(2^{1/3}) \), since \( h \) only has 2 solutions by the Fundamental Theorem of Algebra, but \( \pm\alpha\not\in \Q(2^{1/3}) \).
Thus \( h \) can not be reduced into linear factors, whence \( \mu_\alpha^{\Q(\alpha^2)}(x) = x^2-2^{1/3} \).

\item Let \( \beta=\alpha\veps_3 \). In \( \Q(\beta) \), \begin{align*}
  \beta = \alpha\veps_3 &\imp \beta\cb = \alpha\cb \veps\cb \\
  &\imp \alpha\cb - \beta\cb = 0 \\
  &\imp x\cb - \beta\cb = 0 \\
  &\imp (x-\beta)(x\sq+\beta x + \beta\sq) = 0.
\end{align*}
Let \( p(x) = x-\beta \) and \( q(x) = x\sq+\beta x + \beta\sq \).
We know \( \alpha \) must satisfy at least one of these, but notice \( \alpha-\beta = 0 \imp \alpha = \beta \), which is obviously a contradiction since \( \beta=\alpha\veps_3 \) and \( \veps_3 \neq 1 \).
Thus \( \alpha \) satisfies \( q(x) \) and not \( p(x) \). Now, supposing \( q(x) \) is reducible, it must decompose into linear factors.
However, we can easily see that the only roots of \( q(x) \) are \( \pm \alpha \) but \( \pm\alpha\not\in \Q(\beta) \), otherwise \( \veps_3 \in \Q(\beta) \) which is obviously not true.
Therefore \( q(x) =  x\sq+\beta x + \beta\sq \) must be irreducible over \( \Q(\beta) = \Q(\alpha\veps_3) \), whence \( \mu_\alpha^{\Q(\alpha\veps_3)}(x) =  x\sq+\beta x + \beta\sq = x\sq+\alpha\veps_3x + \alpha\sq\veps_3\sq \).
\end{enumerate}
\end{solution}

\begin{subexercise}
In each case (a---d), find the conjugate elements of all roots of \( x^6-2 \).
\end{subexercise}

\begin{solution}
\begin{enumerate}[label=\alph*)]
  \item In \( \Q \), we have that \( \mu_\alpha^{\Q}(x) = x^6-2 \).
  By the Fundamental Theorem of Algebra, this equation has 6 solutions.
  Thus, the conjugates of \( \alpha \) over \( \Q \) are \( \{\alpha\veps_6^k \mid 0\leq k < 6 \} \), where \( \veps_6 = \exp(\frac{i\pi}{3}) = \frac{1}{2}+\frac{i\sqrt 3}{2} \).

  \item In \( \Q(\alpha) \), we have that \( \mu_\alpha^{\Q(\alpha)}(x) =  x - 2^{1/6} \).
  By FTA, this equation only has one solution.
  Thus the sole conjuage of \( \alpha \) over \( \Q(\alpha) \) is itself, \( \alpha \).

  \item In \( \Q(\alpha\sq) \), we have that \( \mu_\alpha^{\Q(\alpha\sq)}(x) = x^2-2^{1/3} \).
  By FTA, this equation has 2 solutions.
  It is trivial to see that the conjuates of \( \alpha \) over \( \Q(\alpha\sq) \) are \( \pm\alpha \).

  \item In \( \Q(\alpha\veps_3) \), we have that \( \mu_\alpha^{\Q(\alpha\veps_3)}(x) = x\sq+\alpha\veps_3x + \alpha\sq\veps\sq \).
  By FTA, this equation has 2 solutions.
  We know that one solution to this equation is \( x_1 = \alpha \), so by Vieta's formulae we have \( x_2 = -\alpha\veps_3 - \alpha = -\alpha(\veps_3+1) \).
  Notice that \( \veps_3^3 = 1 \imp \veps_3^3-1=0 \).
  We know \( \veps_3\neq 1 \), so we can factor out the linear term for which the solution is 1.
  That is, \( \veps_3^3-1 = (\veps_3-1)(\veps_3\sq+\veps_3+1) \).
  So our root of unity must satisfy \( \veps_3\sq+\veps_3+1 = 0 \imp \veps_3\sq=-(\veps_3+1) \).
  Thus we can substitute this into our previous equation to find \( x_2 = \alpha\veps_3\sq \), whence our algebraic conjugates of \( \alpha \) are \( \alpha, \alpha\veps_3\sq \).
\end{enumerate}
\end{solution}

\end{document}
