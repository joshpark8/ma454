\documentclass{article}
\setlength{\headheight}{22.50113pt}
\addtolength{\topmargin}{-10.50113pt}

\input{preamble}
\input{letterfont}
\input{macros}

\fancyhead[L]{\bd{Josh Park \\ Prof. Shkredov}}
\fancyhead[C]{\bd{MA 45401-H01 -- Galois Theory Honors \\ Homework 5 (Feb 28)}}
\fancyhead[R]{\bd{Spring 2025 \\ Page \thepage}}

\begin{document}

\setcounter{section}{5} % HW NUMBER
\begin{exercise}
Which of the following field extensions are normal? Justify your answers.
\end{exercise}
\begin{enumerate}
\item \( \Q(i):\Q \) % PROBLEM 5.1.1
\begin{solution}
Normal.
By theorem, we know that any finite extension \( L:K \) is normal \iff \( L \) is a \sfe~for some non-constant \( f\in K[t] \).
Hence, since \( \Q(i) \) is the splitting field for \( t\sq+1 \) over \( \Q \), the extension \( \Q(i):\Q \) is normal.
\end{solution}

\item \( \Q(2^{1/4}):\Q \) % PROBLEM 5.1.2
\begin{solution}
Not normal.
By definition, an extension \( L:K \) is normal if \( \forall \alpha\in L \), the minimum polynomial of \( \alpha \) over \( K \), \( \mu_\alpha^K(t) \), splits over \( L[t] \).
Obviously, \( \sqrt[4]2 \in \Q(2^{1/4}) \) by construction.
However, notice that for \( \alpha=\sqrt[4]2 \), \begin{align*}
  \mu_\alpha^\Q(t) &= t^4-2 \\
  &= (t\sq+\sqrt 2)(t\sq-\sqrt 2) \\
  &= (t+i\sqrt[4]2)(t-i\sqrt[4]2)(t+\sqrt[4]2)(t-\sqrt[4]2),
\end{align*}
but the linear factors \( (t+i\sqrt[4]2) \) and \( (t-i\sqrt[4]2) \) are not in \( \Q(2^{1/4})[t] \). Hence, the extension \( \Q(2^{1/4}):\Q \) is not normal by definition.
\end{solution}

\item \( \Q(2^{1/4},i):\Q \) % PROBLEM 5.1.3
\begin{solution}
Normal.
Consider the polynomial \( f(t) = (t^4-2)(t^2-1) \in \Q(2^{1/4},i)[t] \).
Then, \begin{align*}
  f(t) &= (t+i\sqrt[4]2)(t-i\sqrt[4]2)(t+\sqrt[4]2)(t-\sqrt[4]2)(t+i)(t-i),
\end{align*}
whence \( \Q(2^{1/4},i):\Q \) is a \sfe~for \( f \).
By applying the same theorem as in part 1, this extension is normal.
\end{solution}

\item \( \Q(2^{1/4},i,\sqrt 5):\Q \) % PROBLEM 5.1.4
\begin{solution}
Normal.
Consider the polynomial \( f(t) = (t^4-2)(t\sq-1)(t\sq-5) \in \Q(2^{1/4},i,\sqrt 5)[t] \).
Then, \begin{align*}
  f(t) &= (t+i\sqrt[4]2)(t-i\sqrt[4]2)(t+\sqrt[4]2)(t-\sqrt[4]2)(t+i)(t-i)(t-\sqrt 5)(t+\sqrt 5),
\end{align*}
whence \( \Q(2^{1/4},i,\sqrt 5):\Q \) is a \sfe~for \( f \).
By applying the same theorem as in part 1, this extension is normal.
\end{solution}

\item \( \Q(3^{1/3},i,\sqrt 3):\Q \) % PROBLEM 5.1.5
\begin{solution}
Normal.
Consider the polynomial \( f(t) = (t\sq-3)(t\cb-3) \).
Then, \begin{align*}
  f(t) &= (t+\sqrt 3)(t-\sqrt 3)(t-\cbrt 3)(t-\veps_3\cbrt 3)(t-\veps_3^2\cbrt 3),
\end{align*}
where \( \veps_3 = \exp\lt(\frac{2\pi}{3}i\rt) \).
Notice, \begin{align*}
  \veps_3 &= \cos\lt(\frac{2\pi}{3}\rt) + i\sin\lt(\frac{2\pi}{3}\rt)    & \veps_3^2 &= \cos\lt(\frac{4\pi}{3}\rt) + i\sin\lt(\frac{4\pi}{3}\rt) \\
          &= -\frac{1}{2}+i\frac{\sqrt 3}{2} & &= -\frac{1}{2}-i\frac{\sqrt 3}{2}\in \Q(3^{1/3},i,\sqrt 3) \\
          &= \frac{1}{2}\lt(-1+i\sqrt 3\rt)\in \Q(3^{1/3},i,\sqrt 3) & &= \frac{1}{2}\lt(-1-i\sqrt 3\rt) \in \Q(3^{1/3},i,\sqrt 3).
\end{align*}
Thus \( \Q(3^{1/3},i,\sqrt 3):\Q \) is a \sfe~for \( f \), whence must be normal by the same theorem as part 1.
\end{solution}
\end{enumerate}

\begin{exercise} % PROBLEM 5.2
  Let \( \psi:L \to M \) be a homomorphism, suppose that \( L \) is algebraically closed.
  Prove that \( \psi(L) \) is algebraically closed.
\end{exercise}
\begin{solution}
  Let \( g\in \psi(L)[t] \) be some irreducible polynomial over \( \psi(L) \).
  Then, we have some \( f \in L[t] \) such that \( g=\psi f \) with \( \deg g = \deg f \).
  Now, assume ad absurdum that \( g \) has a degree greater than 1.
  Then \( \deg f > 1 \).
  By algebraic closure of \( L \), any irreducible polynomials must be linear.
  Since \( \deg f \neq 1 \), \( f \) must be reducible and thus \( f = h\ell \) for some \( h,\ell\in L[t] \) such that \( \deg h \geq 1 \) and \( \deg \ell \geq 1 \).
  Since \( \psi \) must preserve operations, this implies that \( \exists \hat h,\hat \ell\in \psi(L) \) such that \( g=\hat h\hat \ell \), where \( \deg \hat h \geq 1 \) and \( \deg\hat\ell \geq 1 \).
  However, this contradicts the fact that \( g \) is irreducible, so our assumption that \( \deg g > 1 \) must be false and hence \( \deg g = 1 \).
  Therefore, \( \psi(L) \) is algebraically closed.
\end{solution}

\begin{exercise} % PROBLEM 5.3
  Let \( L:K \) be a field extension. Then \( \bar K \) is isomorphic to \( \bar L \).
  In addition, if \( K\ss L \sseq \bar L \), then \( \bar K = \bar L \).
\end{exercise}
\begin{solution}
  Let \( \vphi_1:K\to L \) and \( \vphi_2:L\to \bar L \) be the monomorphisms corresponding to the field extensions \( L:K \tand \bar L:L \), respectively.
  Then \( \bar L:K \) is the field extension relative to the composition \( \vphi_2\circ \vphi_1 \).
  By algebraic closure of \( \bar L \), it must be an algebraic closure for \( K \).
  By theorem from lecture, we know that any two algebraic closures for the same field must be isomorphic to one another.
  Thus, \( \bar L\iso \bar K \).

  Assume ad absurdum that we have some algebraic closure \( \bar K \) of \( K \) such that \( \order{\bar K} < \order{\bar L} \).
  By definition of algebraic closure, we know that \( \bar K \) and \( \bar L \) are both algebraic extensions of \( K \), so \( K\sseq \bar K \) and \( K\sseq \bar L \).
  Let \( \psi:K\hookrightarrow\bar L \) be the monomorphism corresponding to the extension \( \bar L:K \).
  By theorem, since \( \bar K:K \) is an algebraic extension, there exists an extension of \( \psi \) to another mono from \( \bar K \to \bar L \).
  Hence \( \bar L:\bar K \) is an algebraic extension with degree greater than 1, since \( \bar K \) is smaller than \( \bar L \).
  However, this contradicts the algebraic closure of \( \bar K \), since the only algebraic extension of an algebraically closed field is itself.
  Thus, \( \bar K = \bar L \).
\end{solution}

\begin{exercise} % PROBLEM 5.4
  Let \( K-L \) be a normal extension, \( K\sseq L\sseq \Kbar \).
  Then for any \( K \)-\homo~\( \tau:L\to \Kbar \) one has \( \tau(L) = L \).
\end{exercise}
\begin{solution}
  Suppose we have some \( K \)-\homo~\( \tau:L\to\bar K \), and let \( \ell\in L \).
  By definition of normal extension, \( K-L \) must be algebraic, whence \( \mu_\ell^K \) exists.
  Since \( \tau \) fixes all elements of \( K \) and \( \mu_\ell^K \) is a polynomial with coefficients in \( K \), we can see that \( \tau(\mu_\ell^K(\ell)) = \mu_\ell^K(\tau(\ell)) = 0 \).
  By theorem, the normality of \( L:K \) implies that all algebraic conjugates of \( \ell \) are in \( L \).
  Thus, we have that \( \tau(\ell)\in L \).
  Since \( \ell \) is an arbitrary element of \( L \), this implies that \( \tau(L) \sseq L \).
  By theorem, since \( L \) extends \( K \) and \( \tau:L\to L \) is a \( K \)-homomorphism, we have that \( \tau \) is an automorphism of \( L \).
  Thus \( \tau(L) = L \).
\end{solution}

\begin{exercise} % PROBLEM 5.5
  Put \( K=\Ftw(t) \) and consider \( L=K(t^{1/3}) \).
  Prove that the extension \( L:K \) is algebraic but not normal.
\end{exercise}

\begin{solution}
Obviously since \( K(t^{1/3}):K \) is a finite field extension, it is algebraic.
Suppose \( x\in \bar K \) solves the equation \( x^3-t=0 \).
Then, \( x = t^{1/3} \) \imp \( \lt(\frac{x}{t^{1/3}}\rt)\cb=1 \) \imp \( x=yt^{1/3} \) such that \( y^3 = 1 \).
Then we have \( y^3-1 = (y-1)(y^2+y+1) = 0 \), so either \( y=1 \) or \( y \) is a root of \( y^2+y+1=0 \).
Notice that \( y\sq+y+1 \) is irreducible over \( K \), since neither 0 nor 1 are roots.
If we suppose there was some \( z\in L\setminus K \) such that \( z^2+z+1=0 \), then that implies that there exists some \( f\in K[t] \) with \( f(t^{1/3}) = 0 \).
However, \( t \) is obviously transcendental over \( K \), forcing a contradiction.
Thus, the only solution to the cubic \( x\cb - t \) in \( L \) is \( x=1 \), whence the minimum polynomial for \( t^{1/3} \in L \) does not split over \( L \).
Therefore \( L:K \) does not meet the requirements to be a normal field extension.
\end{solution}
\end{document}
