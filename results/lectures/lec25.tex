\documentclass[a4paper]{article}

\input{preamble}
\input{letterfont}
\input{macros}

\begin{document}
\section{Solvability by radicals and Galois theory I}

\begin{ttheorem}
  Let \( K \) be a field with \( \chr K = 0 \).
  Then \( f\in K[t] \) is solvable by radicals \( \iff \) \( \Gal_K(f) \) is soluble.
\end{ttheorem}

\begin{tlemma}
  Let \( \chr K = 0 \) and \( R:K \) be a radical extension.
  Then there exists a tower \( K-R-N \) such that \( N:K \) is normal and radical.
\end{tlemma}

\begin{tdefinition}[Cyclic extension]
  Let \( L \) be the splitting field of some polynomial \( f \) over \( K \).
  If \( \Gal(L:K) \) is a cyclic group, then \( L:K \) is a \it{cyclic} extension.
\end{tdefinition}

\begin{tlemma}
  Let \( \chr K = 0 \) and let \( n \) be a positive integer such that \( t^n-1 \) splits over \( K \), and let \( L:K \) be the \sfe~for \( t^{n}-a \) for some \( a\in K \).
  Then \( \Gal(L:K) \) is abelian.
\end{tlemma}

\begin{ttheorem}
  Let \( \chr K = 0 \) and \( L:K \) be Galois.
  Suppose there exists some extension \( M:L \) such that \( M:K \) is normal.
  Then \( \Gal(L:K) \) is soluble.
\end{ttheorem}

\begin{tcorollary}
  Let \( \chr K = 0 \).
  Then \( f\in K[t] \) is SBR \( \imp \) \( \Gal_K(f) \) is soluble.
\end{tcorollary}
\end{document}