\documentclass[a4paper]{article}
\usepackage{asymptote}

\input{preamble.tex}
\input{letterfont.tex}
\input{macros.tex}

% \renewcommand{\arraystretch}{1.25} % space out table rows
% \setlength{\parindent}{0pt}
% \setlength{\parskip}{1em}
% \linespread{1} % 1.3 for one-and-half spacing, 1.6 for double spacing

\begin{document}
\section{Galois Groups I}
\begin{tdefinition}[Galois group of polynomial]
  Let \( L=K(\alpha_1,\ldots,\alpha_n) \) and let \( P(\alpha_1,\ldots,\alpha_n) \) where \( P\in K[\alpha_1,\ldots,\alpha_n] \) is an element of \( L \).
  Then we define \begin{align*}
    \Gal_K(f)=\left\{\sigma\in S_n\mid \forall P\in K[\alpha_1,\ldots,\alpha_n], \text{ if } P(\alpha_1,\ldots,\alpha_n)=0 \text{ then } P(\alpha_{\sigma(1)},\ldots,\alpha_{\sigma(n)})\right\}
  \end{align*}
\end{tdefinition}

\begin{tlemma}
  \( \Gal_K(f) \sgp S_n \)
\end{tlemma}

\begin{tlemma}
  If \( K_1:K \), then \( \Gal_{K_1}(f)\sgp \Gal_K(f) \).
\end{tlemma}

\begin{tdefinition}
  Let \( L:K \) be a field extension.
  Then \begin{align*}
    \Gal_K(L) = \Gal(L:K) = \lt\{ \vphi\in\Aut(L) : \vphi\text{ is a K-\homo} \rt\}
  \end{align*}
\end{tdefinition}

\begin{tlemma}
  Suppose that $ M:K $ is a normal extension.
  Then: \begin{enumerate}[label=(\alph*)]
    \item for any $ \sigma\in\Gal(M:K) $ and $ \alpha\in M $, we have $ \mu_{\sigma(\alpha)}^K=\mu_\alpha^K $;
    \item for any $ \alpha,\beta\in M $ with $ \mu_\alpha^K=\mu_\beta^K $, there exists $ \tau\in\Gal(M:K) $ having the property that $ \tau(\alpha)=\beta $.
  \end{enumerate}
\end{tlemma}
\end{document}