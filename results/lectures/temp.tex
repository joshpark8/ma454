\documentclass[a4paper]{article}

\input{preamble}
\input{letterfont}
\input{macros}

\begin{document}
\section{extra}
\begin{tproposition}
  Suppose that \( K \) and \( L \) are fields and that \( \vphi : K \to L \) is a \homo. With \( t \) and \( y \) denoting indeterminates, extend the \homo~\( \vphi \) to the mapping \( \psi: K[t] \to L[y] \) by defining \begin{align*}
    \psi(a_0 + a_1t+\cdots + a_n t^n) = \vphi(a_0) + \vphi(a_1)y+\cdots + \vphi(a_n)y^n.
  \end{align*}
  Then \( \psi:K[t]\to L[y] \) is an injective \homo.
  Also, when \( \vphi:K\to L \) is surjective, then \( \psi: K[t]\to L[y] \) is surjective and maps irreducible polynomials in \( K[t] \) to irreducible polynomials in \( L[y] \).
\end{tproposition}

\begin{tproposition}
  Suppose \( L: K \) is a field extension with \( K \sseq L \), and \( \alpha\in L \).  Then \( E_\alpha \) is a ring \homo.
\end{tproposition}

\begin{tproposition}
  Let \( L : K \) be a field extension with \( K \sseq L \), and suppose that \( \alpha\in L \) is algebraic over \( K \).
  Then \begin{align*}
    I = \ker(E_\alpha) = {f \in K[t] : f(\alpha) = 0}
  \end{align*}
  is a nonzero ideal of \( K[t] \), and there is a unique monic polynomial \( \mu_\alpha^K \in K[t] \) that generates \( I \).
\end{tproposition}

\begin{tproposition}
  Let \( L:K \) be a field extension with \( K\sseq L \).
  Let \( A\sseq L \) and \begin{align*}
    \mcC=\lt\{ C\sseq A : C \text{ is a finite set} \rt\}.
  \end{align*}
  Then \( K(A)=\cup_{C\in\mcC}K(C) \).
  Further, when \( \fdeg{K(C)}{K}<\infty \) for all \( C\in \mcC \), then \( K(A):K \) is an algebraic extension.
\end{tproposition}

\begin{tproposition}
  Let \( L : K \) be a field extension with \( K \sseq L \), and suppose that \( \alpha\in L \).
  Then \begin{align*}
    K[\alpha] = \lt\{ c_0 + c_1\alpha+\cdots+c_d \alpha^d : d\in \ZZ{\leq 0},\ c_0,\ldots,c_d\in K \rt\}
  \end{align*}
  and \begin{align*}
    K(\alpha) = \lt\{ f/g : f,g\in K[\alpha], g\neq 0 \rt\}.
  \end{align*}
\end{tproposition}

\begin{tproposition}
  Let \( L : K \) be a field extension with \( K \sseq L \), and suppose that \( \alpha\in L \).
  Then \( \alpha \) is algebraic over \( K \) if and only if \( \fdeg{K(\alpha)}{K}<\infty \).
\end{tproposition}

\subsection{Review of finite fields and tests for irreducibility}
\begin{tdefinition}[Characteristic]
  Let \( K \) be a field with additive identity \( 0_K \) and multiplicative identity \( 1_K \).
  When \( n\in \N \), we write \( n\cdot 1_K \) to denote \( 1_K+\ldots+ 1_K \) (as an \( n \)-fold sum).
  We define the \it{characteristic} of \( K \), denoted by \( \chr K \), to be the smallest positive integer \( m \) with the property that \( m\cdot 1_K = 0_K \);
  if no such integer \( m \) exists, we define the characteristic of K to be 0.
\end{tdefinition}

\begin{tproposition}
  Let \( K \) be a field with \( \chr{K} > 0 \). Then \( \chr K \) is equal to a prime number \( p \), and then for all \( x\in K \) one has \( p\cdot x=0 \).
\end{tproposition}

\begin{ttheorem}[Localisation principle]
  Let \( R \) be an integral domain, and let \( I \) be a prime ideal of \( R \).
  Define \( \vphi:R[X]\to (\qg{R}{I})[X] \) by putting \begin{align*}
    \vphi(a_0+a_1X+\cdots+a_nX^n) =\bar a_0+\bar a_1X+\cdots+\bar a_nX^n,
  \end{align*}
  where \( \bar a_j = a_j + I \).
  Then \( \varphi \) is a surjective \homo.
  Moreover, if \( f\in R[X] \) is primitive with leading coefficient not in \( I \), then \( f \) is irreducible in \( R[X] \) whenever \( \vphi(f) \) is irreducible in \( (\qg{R}{I})[X] \).
\end{ttheorem}

\bd{Note:}\quad Proposition 3.1 tells us that when \( f\in K[t] \) and \( \sigma\in \Gal(L:K) \), the mapping \( \sigma \) permutes the roots of \( f \) that lie in \( L \).

\begin{ttheorem}
  Suppose that \( L:K \) is an algebraic extension, and \( \sigma:L\to L \) is a \( K \)-\homo.
  Then \( \sigma \) is an automorphism of \( L \).
\end{ttheorem}

\begin{ttheorem}
  If \( L:K \) is a finite extension, then \( \order{\Gal(L:K)}\leq \fdeg{L}{K} \).
\end{ttheorem}

\begin{tcorollary}
  Suppose that \( L:F \) and \( L:F' \) are finite extensions with \( F\sseq L\ tand F'\sseq L \), and further that \( \psi:F\to F' \) is an isomorphism.
  Then there are at most \( \fdeg{L}{F} \) ways to extend \( \psi \) to a \homo~from \( L \) into \( L \).
\end{tcorollary}

\begin{tcorollary}
  Let \( L:K \) be a finite extension with \( K\sseq L \).
  Suppose that \( \llist{\alpha}{1}{n}\in L \) and put \( L=K(\llist{\alpha}{1}{n}) \).
  Let \( K_0 = K \), and for \( 1\leq i\leq n \), let \( K_i = K_{i-1}(\alpha_i) \).
  Then every automorphism \( \tau \in \Gal(L:K) \) corresponds to a sequence of \homo s \( \llist{\sigma}{1}{n} \), such that \( \sigma_0:K\to L \) is the inclusion map, one has \( \sigma_n=\tau \), and for \( 1\leq i\leq n \), the map \( \sigma_i : K_i\to L \) is a \homo~extending \( \sigma_{i-1}:K_{i-1}\to L \).
\end{tcorollary}

\section{Algebraic closures}
\subsection{The definition of an algebraic closure, and Zorn's Lemma}

\subsection{The existence of an algebraic closure}

\begin{tcorollary}
  When \( K \) is a field, the field \( \Kbar \) is a maximal algebraic extension of \( K \).
\end{tcorollary}

\subsection{Properties of algebraic closures}

\begin{tcorollary}
  Suppose that \( \Kbar \) is an algebraic closure of \( K \), and assume that \( K\sseq \Kbar \).
  Take \( \alpha\in \Kbar \) and suppose that \( \sigma:K\to \Kbar \) is a homomorphism.
  Then the number of distinct roots of \( \mu_\alpha^K \) in \( \Kbar \) is equal to the number of distinct roots of \( \sigma(\mu_\alpha^K) \) in \( \Kbar \).
\end{tcorollary}

\begin{tproposition}
  Suppose that \( L \) and \( M \) are fields such that \( L \) is algebraically closed, and \( \psi : L \to M \) is a homomorphism.
  Then \( \psi(L) \) is algebraically closed.
\end{tproposition}

\begin{tproposition}
  If \( L:K \) is an algebraic extension, then \( \Lbar \) is an algebraic closure of \( K \), and hence \( \Lbar\iso\Kbar \).
  If in addition \( K\sseq L\sseq \Lbar \), then we can take \( \Kbar = \Lbar \).
\end{tproposition}

\section{Splitting field extensions}
\begin{tdefinition}[Splitting field, \sfe]
  Suppose that \( L:K \) is a field extension relative to the embedding \( \vphi : K\to L \), and \( f\in K[t]\setminus K \). \begin{enumerate}[label=(\roman*)]
    \item We say that \it{\( f \) splits over \( L \)} if \( \vphi(f)=\lambda(t-\alpha_1)\cdots(t-\alpha_n) \), for some \( \lambda\in \vphi(K) \) and \( \llist{\alpha}{1}{n}\in L \).
    \item Suppose that \( f \) splits over \( L \), and let \( M \) be a field with \( \vphi(K)\sseq M\sseq L \).
      We say that \it{\( M:K \) is a splitting field extension for \( f \)} if \( M \) is the smallest subfield of \( L \) containing \( \vphi(K) \) over which \( f \) splits.
    \item More generally, suppose that \( S\sseq K[t]\setminus K \) has the property that every \( f\in S \) splits over \( L \).
      Let \( M \) be a field with \( \vphi(K)\sseq M\sseq L \).
      We say that \it{M:K is a \sfe~for \( S \)} if \( M \) is the smallest subfield of \( L \) containing \( \vphi(K) \) over which every polynomial \( f\in S \) splits.
  \end{enumerate}
\end{tdefinition}

\begin{tproposition}
  Suppose that \( L:K \) is a \sfe~for the polynomial \( f\in K[t]\setminus K \) with associated embedding \( \vphi:K\to L \).
  Let \( \llist{\alpha}{1}{n}\in L \) be the roots of \( \varphi(f) \).
  Then \( L=\vphi(K)(\llist{\alpha}{1}{n}) \).
\end{tproposition}

\begin{tproposition}
  Suppose that \( L:K \) is a \sfe~for the polynomial \( f\in K[t]\setminus K \).
  Then \( \fdeg{L}{K}\leq (\deg f)! \)
\end{tproposition}

\begin{tproposition}
  Given \( S\sseq K[t]\setminus K \), there exists a \sfe~\( L:K \) for \( S \), and \( L:K \) is an algebraic extension.
  More explicitly, suppose that \( \Kbar \) is an algebraic closure of \( K \), and that \( \Kbar:K \) is an extension relative to the embedding \( \vphi:\Kbar\to K \). Let
  \begin{align*}
    A=\lt\{ \alpha\in \Kbar : \alpha\text{ is a root of \( \vphi(f) \), for some } f\in S \rt\}.
  \end{align*}
  Put \( K'=\vphi(K) \).
  Then \( K'(A):K \) is a \sfe~for \( S \).
\end{tproposition}

\begin{ttheorem}
  Let \( f\in K[t]\setminus K \), and suppose that \( L:K \) and \( M:K \) are \sfe s for \( f \).
  Then \( L\iso M \), and thus \( \fdeg{L}{K}=\fdeg{M}{K} \).
\end{ttheorem}

\begin{ttheorem}
  Suppose that \( S\sseq K[t]\setminus K \), and suppose that \( L:K \) and \( M:K \) are \sfe s for \( S \).
  Then \( L\iso M\tand \fdeg{L}{K}=\fdeg{M}{K} \).
\end{ttheorem}

\section{Normal extensions and composita}
\subsection{Normal extensions and splitting field extensions}

\begin{tproposition}
  Suppose that \( L:M:K \) is a tower of field extensions and \( L:K \) is a normal extension.
  Then \( L:M \) is also a normal extension.
\end{tproposition}

\subsection{Normal closures}
\begin{ttheorem}
  Suppose that \( M:L:K \) is a tower of field extensions such that \( M:K \) is normal.
  Assume that \( K\sseq L\sseq M \).
  Then the following are equivalent: \begin{enumerate}[label=(\roman*)]
    \item the field extension \( L:K \) is normal;
    \item any \( K \)-\homo~of \( L \) into \( M \) is an automorphism of \( L \);
    \item whenever \( \sigma:M\to M \) is a \( K \)-automorphism, then \( \sigma(L)\sseq L \).
  \end{enumerate}
\end{ttheorem}

\subsection{Composita of field extensions}

\begin{tproposition}
  Suppose that \( E:K \) and \( F:K \) are finite extensions such that \( K,\ E\tand F \) are contained in a field \( L \).
  Then \( EF:K \) is a finite extension.
\end{tproposition}

\begin{ttheorem}
  Let \( E:K \) and \( F:K \) be finite extensions such that \( K,\ E\tand F \) are contained in a field \( L \). \begin{enumerate}[label=(\alph*)]
    \item When \( E:K \) is normal, then \( EF:F \) is normal.
    \item When \( E:K \) and \( F:K \) are both normal, then \( EF:K \) and \( E\cap F:K \) are normal.
  \end{enumerate}
\end{ttheorem}

\subsection{Normal closures (non-examinable)}

\section{Separability} % 7
% No bolded subsections appear in this section
\setcounter{tdefinition}{24}

\begin{ttheorem}
  Suppose that \( L:M:K \) is a tower of algebraic extensions.
  Then \( L:K \) is separable if and only if \( L:M \) and \( M:K \) are both separable.
\end{ttheorem}

\begin{ttheorem}
  Suppose tht \( E:K \) and \( F:K \) are finite extensions with \( E\sseq L \) and \( F\sseq L \), where \( L \) is a field.
  \begin{enumerate}[label=(\alph*)]
    \item When \( E:K \) is separable, then so too is \( EF:F \);
    \item When \( E:K \) and \( F:K \) are both separable, then so too are \( EF:K \) and \( E\cap F:K \).
  \end{enumerate}
\end{ttheorem}

\section{Inseparable polynomials, differentiation, and the Frobenius map}
\subsection{Inseparable polynomials and differentiation}

\subsection{The Frobenius map}





\section{The Primitive Element Theorem}

\section{Fixed fields and Galois extensions}

\begin{ttheorem}
  Suppose that \( L:K \) is an algebraic extension.
  Then \( L:K \) is Galois if and only if \( K=\Fix{L}{\Gal(L:K)} \).
\end{ttheorem}

\begin{ttheorem}
  Suppose that \( L \) is a field and \( G \) is a finite subgroup of \( \Aut(L) \), and put \( K=\Fix{L}{G} \).
  Then \( L:K \) is a finite Galois extension with \( \fdeg{L}{K}=\order{\Gal(L:K)} \), and furthermore \( G=\Gal(L:K) \).
\end{ttheorem}

\begin{ttheorem}
  Suppose that \( L:K \) is a finite extension.
  Then, if \( L:K \) is a Galois extension, one has \( \order{\Gal(L:K)}=\fdeg{L}{K} \) and \( K=\Fix{L}{\Gal(L:K)} \).
  If \( L:K \) is not Galois, meanwhile, one has \( \order{\Gal(L:K)}<\fdeg{L}{K} \) and \( K \) is a proper subfield of \( \Fix{L}{\Gal(L:K)} \).
\end{ttheorem}

\begin{tproposition}
  Suppose that \( L:K \) is a Galois extension, and further that \( L:M:K \) is a tower of field extensions.
  Then \( L:M \) is a Galois extension.
\end{tproposition}

\section{The main theorems of Galois theory}
\subsection{The Fundamental Theorem}
\begin{tdefinition}
  Suppose that \( L:K \) is a field extension.
  When \( G \) is a subgroup of \( \Aut(L) \), we write \( \phi(G) \) for \( \Fix{L}{G} \), and when \( L:M:K_0 \) is a tower of field extensions with \( K_0=\phi(\Gal(L:K)) \), we write \( \gamma(M) \) for \( \Gal(L:M) \).
\end{tdefinition}

\begin{ttheorem}[The Fundamental Theorem of Galois Theory]
  Suppose that \( L:K \) is a finite extension, let \( G=\Gal(L:K) \), and put \( K_0=\phi(G) \).
  Then one has the following: \begin{enumerate}[label=(\alph*)]
    \item the map \( \phi \) is a bijection from the set of subgroups of \( G \) onto the set of fields \( M \) intermediate between \( L \) and \( K_0 \), and \( \gamma \) is the inverse map;
    \item if \( H \sgp G \), then \( H\idl G \) if and only if \( \phi(H):K_0 \) is a normal extension;
    \item if \( H\idl G \), one has \( \Gal(\phi(H):K_0) \iso \qg{G}{H} \).
      In particular, if \( \sigma\in G \), one has \( \sigma|_{\phi(H)} \in \Gal(\phi(H):K_0) \), and the map \( \sigma\mapsto\sigma|_{\phi(H)} \) is a homomorphism of \( G \) onto \( \Gal(\phi(H):K_0) \) with kernel \( H \).
  \end{enumerate}
\end{ttheorem}

\begin{tdefinition}[Galois group of polynomial]
  When \( f\in K[t] \) and \( L:K \) is a \sfe~for \( f \), we define the \it{Galois group of the polynomial \( f \) over \( K \)} to be \( \Gal_K(f) = \Gal(L:K) \).
\end{tdefinition}

\subsection{Non-examinable: consequences for composita and intersections}

\section{Finite fields}

\section{Solvability and solubility}
\begin{tdefinition}[Soluble group]
  A finite group \( G \) is \it{soluble} if there is a series of groups \begin{align*}
    \lt\{ \nf{id} \rt\} = G_0 \sgp G_1 \sgp \cdots\sgp G_n = G,
  \end{align*}
  with the property that \( G_i \nsgp G_{i+1} \) and \( \qg{G_{i+1}}{G_i} \) is abelian (\( 0\leq i < n \)).
\end{tdefinition}

\begin{ttheorem}
  Let \( K \) be a field of characteristic 0.
  Then \( f\in K[t] \) is solvable by radicals if and only if \( \Gal_K(f) \) is soluble.
\end{ttheorem}

\begin{tlemma}
  Suppose \( \chr K = 0 \) and \( L:K \) is a radical extension.
  Then there exists an extension \( N:L \) such that \( N:K \) is normal and radical.
\end{tlemma}

\begin{tdefinition}[Cyclic extension]
  The extension \( L:K \) is \it{cyclic} if \( L:K \) is a Galois extension and \( \Gal(L:K) \) is a cyclic group.
\end{tdefinition}

\begin{tlemma}
  Suppose that \( \chr K = 0 \) and let \( p \) be a prime number.
  Also, let \( L:K \) be a \sfe~for \( t^p-1 \).
  Then \( \Gal(L:K) \) is cyclic, and hence \( L:K \) is a cyclic extension.
\end{tlemma}

\begin{tlemma}
  Let \( \chr K = 0 \) and suppose that \( n \) is an integer such that \( t^n-1 \) splits over \( K \).
  Let \( L:K \) be a \sfe~for \( t^n-a \), for some \( a\in K \).
  Then \( \Gal(L:K) \) is abelian.
\end{tlemma}

\begin{ttheorem}
  Let \( \chr K = 0 \) and suppose that \( L:K \) is Galois.
  Suppose that there is an extension \( M:L \) with the property that \( M:K \) is radical.
  Then \( \Gal(L:K) \) is soluble.
\end{ttheorem}

\begin{tcorollary}
  Suppose that \( \chr K = 0 \).
  Then \( \Gal_K(f) \) is soluble whenever \( f\in K[t] \) is soluble by radicals.
\end{tcorollary}

\begin{tcorollary}
  There exist quintic polynomials in \( \Q[t] \) with insoluble Galois groups, such as \( f(t) = t^5-4t+2 \), and which are not solvable by radicals.
\end{tcorollary}

\begin{tlemma}
  Let \( \chr K = 0 \), and suppose that \( L:K \) is a cyclic extension of degree \( n \).
  Suppose also that \( K \) contains a primitive \( n \)-th root of 1.
  Then there exists \( \theta \in K \) such that \( t^n -\theta \) is irreducible over \( K \), and \( L:K \) is a \sf~for \( t^n-\theta \).
  Further, if \( \beta \) is a root of \( t^n-\theta \) over \( L \), then \( L=K(\beta) \).
\end{tlemma}

\begin{ttheorem}
  Let \( \chr K = 0 \), and suppose that \( f\in K[t]\setminus K \).
  Then \( f \) is solvable by radicals whenever \( \Gal_K(f) \) is soluble.
\end{ttheorem}

\end{document}