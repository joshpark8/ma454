\documentclass[a4paper]{article}

\usepackage{preamble}
\usepackage{letterfont}
\usepackage{macros}

% \renewcommand{\arraystretch}{1.25} % space out table rows
% \setlength{\parindent}{0pt}
% \setlength{\parskip}{1em}
% \linespread{1} % 1.3 for one-and-half spacing, 1.6 for double spacing

\begin{document}
\section{Introduction II}
\begin{ttheorem}[Lagrange]
  Let \( \vphi=\vphi(\llist{x}{1}{n}) \) and \begin{align*}
    \orb(\vphi) = \lt\{ \vphi^\omega = \vphi(x_{\omega(1)},\ldots,x_{\omega(n)}) \mid\omega\in S_n \rt\}.
  \end{align*}
  Then \( \llist{y}{1}{k} \) are roots of some polynomial with degree \( \leq k \) whose coefficients depend on elementary symmetric polynomials \llist{\sigma}{1}{n} in a polynomial way.
\end{ttheorem}

\begin{ttheorem}[Lagrange]
  Let \( \vphi,\psi\in K[\llist{x}{1}{n}] \) and \( G_\vphi=\lt\{ \omega\in S_n \mid \vphi^\omega=\vphi \rt\} \sgp G_\psi \).
  Then \( \psi = R(\varphi) \) where \( R \) is a rational function whose coefficients are symmetric functions on \( \llist{x}{1}{n} \).
\end{ttheorem}

\begin{tdefinition}[Group action]
  Let \( G \) be a group and \( X \) be a set. The (left) group action of \( G \) on \( X \) is the map \( \cdot:G\times X\to X \) such that \begin{enumerate}
    \item \( e_G\cdot x = x, \quad \forall x\in X \)
    \item \( g\cdot (h\cdot x)=(g\cdot h)\cdot x,\quad \forall x\in X,\forall g,h\in G \)
  \end{enumerate}
\end{tdefinition}

\begin{tdefinition}[Orbit]
  Let \( G \) be a group, \( X \) be a set, and \( x\in X \).
  Then we define \ul{the orbit} of \( x \), \( G\cdot x = \text{orb}(x) \), as \( \{g\cdot x \ \mid \ g\in G\} \).
  Moreover, \( \text{orb}(x)\subseteq X \).
\end{tdefinition}

\begin{tdefinition}[Stabilizer]
  Let \( G \) be a group, \( X \) be a set, and \( x\in X \).
  Then we define \ul{the stabilizer} of \( x \), \( \text{stab}(x) \), as \( \{g\in G \ \mid \ g\cdot x = g\} \).
  Moreover, \( \text{stab}(x)\leq G \).
\end{tdefinition}

\begin{ttheorem}
  Let \( G \) be a finite group that acts on \( X \). Then for all \( x\in X \), \( \order{\orb(x)} \cdot \order{\stab(x)} = \order{G} \).
\end{ttheorem}

\begin{tdefinition}[Polynomial ring]
  Let \( R \) be a commutative ring. Then the ring of polynomials with coefficients in \( R \) is
  \begin{align*}
    R[t] = \left\{\sum_{i=0}^n c_it^i : n\in\mathbb{Z}_+, c_i\in R\right\}
  \end{align*}
\end{tdefinition}

\end{document}