\documentclass[a4paper]{article}
\usepackage{asymptote}

\input{preamble.tex}
\input{letterfont.tex}
\input{macros.tex}

\begin{document}
\section{Algebraic Closure II}
\begin{ttheorem}
  Let \( E \) be an algebraic extension of \( K \) with \( K\sseq E \), and let \( \Kbar \) be an algebraic closure of \( K \).

  Given a homomorphism \( \vphi:K\to \Kbar \), the map \( \vphi \) can be extended to a homomorphism from \( E \) into \( \Kbar \).
\end{ttheorem}

\begin{ttheorem}

  If \( L \) and \( M \) are both algebraic closures of \( K \), then \( L \iso M \).
\end{ttheorem}

\begin{tcorollary}
  Let \( L:K \) be an extension with \( K\sseq L \).
  Suppose that \( g\in L[t] \) is irreducible over \( L \), and that \( g\divs f \) in \( L[t] \), where \( f\in K[t]\setminus \lt\{ 0 \rt\} \).
  The \( g \) divides a factor of \( f \) that is irreducible over \( K \).

  Thus, there exists an irreducible \( h\in K[t] \) having the property that \( h\divs f \) in \( K[t] \), and \( g\divs h \) in \( L[t] \).
\end{tcorollary}

\begin{tdefinition}[Normal extension]
  The extension \( L:K \) is \tul{normal} if it is algebraic, and every irreducible polynomial \( f\in K[t] \) either splits over \( L \) or has no root in \( L \).
\end{tdefinition}

\begin{ttheorem}
  \( K(\alpha):K \) is normal \iff all conjugates of \( \alpha \) are contained in \( K(\alpha) \).
\end{ttheorem}

\begin{ttheorem}
  A finite extension \( L:K \) is normal \iff \( L \) is a \sfe~for some \( f\in K[t]\setminus K \).
\end{ttheorem}
\end{document}