\documentclass[a4paper]{article}
\usepackage{asymptote}

\input{preamble.tex}
\input{letterfont.tex}
\input{macros.tex}

\begin{document}
\section{Algebraic Closure I}
\begin{tdefinition}[Algebraically closed field, algebraic closure]
  Let \( M \) be a field. \begin{enumerate}[label=(\roman*)]
    \item We say that \( M \) is \tul{algebraically closed} if every non-constant polynomial \( f\in M[t] \) has a root in \( M \).
    \item We say that \( M \) is an algebraic closure of \( K \) if \( M:K \) is an algebraic field extension having the property that \( M \) is algebraically closed.
  \end{enumerate}
\end{tdefinition}

\begin{tlemma}
  Let \( M \) be a field.
  The following are equivalent: \begin{enumerate}[label=(\roman*)]
    \item The field \( M \) is algebraically closed;
    \item every non-constant polynomial \( f\in M[t] \) factors in \( M[t] \) as a product of linear factors;
    \item every irreducible polynomial in \( M[t] \) has degree 1;
    \item the only algebraic extension of \( M \) containing \( M \) is itself.
  \end{enumerate}
\end{tlemma}

\begin{tdefinition}[Chain]
  Suppose that \( X \) is a nonempty, partially ordered set with \( \leq \) denoting the partial ordering.
  A \tul{chain} \( C \) in \( X \) is a collection of elements \( \lt\{ a_i \rt\}_{i\in I} \) of \( X \) having the property that for every \( i,j\in I \), either \( a_i\leq a_j \tor a_j\leq a_i \).
\end{tdefinition}

\tbo{Zorn's Lemma:}\quad Suppose that \( X \) is a nonempty, partially ordered set with \( \leq \) the partial ordering.
Suppose that every non-empty chain \( C \) in \( X \) has an upper bound in \( X \).
Then \( X \) has at least one maximal element \( m \), meaning that if \( b\in X \) with \( m\leq b \), then \( b=m \).

\begin{tcorollary}
  Any proper ideal \( A \) of a commutative ring \( R \) is contained in a maximal ideal.
\end{tcorollary}

\begin{tlemma}
  Let \( K \) be a field. Then there exists an algebraic extension \( E:K \), with \( K\sseq E \), having the property that \( E \) contains a root of every irreducible \( f\in K[t] \), and hence also every \( g\in K[t]\setminus K \).
\end{tlemma}

\begin{ttheorem}[Existence of Algebraic Closures]
  Suppose that \( K \) is a field.
  Then there exists an algebraic extension \( \Kbar \) of \( K \) having the property that \( \Kbar \) is algebraically closed.
\end{ttheorem}

\begin{tdefinition}[Extension of field \homo, isomorphic field extensions]
  For \( i = 1 \tand 2 \), let \( L_i : K_i \) be a field extension relative to the embedding \( \vphi_i : K_i\to L_i \).
  Suppose that \( \sigma : K_1\to K_2 \) and \( \tau:L_1\to L_2 \) are isomorphisms.
  We say that \tul{\( \tau \) extends \( \sigma \)} if \( \tau\circ\vphi_1 = \vphi_2\circ\sigma \).
  In such circumstances, we say that \( L_1 : K_1 \tand L_2 : K_2 \) are \tul{isomorphic field extensions}.
  \begin{center}
    \begin{tikzcd}
      L_{1} \arrow[r, "\tau"]
      & L_{2} \\[1em]  % extra vertical space
      K_{1}
      \arrow[r, "\sigma"]
      \arrow[u, "\vphi_{1}"]
      \arrow[ru, dashed]   % dashed diagonal arrow
      & K_{2}
      \arrow[u, "\vphi_{2}"]
    \end{tikzcd}
  \end{center}
  When \( \sigma:K_1\to K_2 \) and \( \tau:L_1\to L_2 \) are \homo s (instead of isomorphisms), then \tul{\( \tau \) extends \( \sigma \) as a \homo~of fields} when the isomorphism \( \tau:L_1\to L_1' = \tau(L_1) \) extends the isomorphism \( \sigma:K_1\to K_1' = \sigma(K_1) \).
\end{tdefinition}

\begin{tdefinition}[\(K\)-\homo]
  Let \( L : K \) be a field extension relative to the embedding \( \varphi : K \to L \), and let \( M \) be a subfield of \( L \) containing \( \varphi(K) \).
  Then, when \( \sigma : M \to L \) is a \homo, we say that \( \sigma \) is a \tul{\(K\)-\homo} if \( \sigma \) leaves \( \varphi(K) \) pointwise fixed, which is to say that for all \( \alpha \in \varphi(K) \), one has \( \sigma(\alpha) = \alpha \).
\end{tdefinition}

\begin{tlemma}
  Suppose that \( L:K \) is a field extension with \( K\sseq L \), and that \( \tau:L\to L \) is a \( K \)-\homo.
  Suppose that \( f\in K[t] \) has the property that \( \deg f \geq 1 \), and additionally that \( \alpha\in L \).
  \begin{enumerate}[label=(\roman*)]
    \item if \( f(\alpha) = 0 \), one has \( f(\tau(\alpha)) = 0 \);
    \item if \( \tau \) is a \( K \)-automorphism of \( L \), then \( f(\alpha) = 0 \iff f(\tau(\alpha)) = 0 \).
  \end{enumerate}
\end{tlemma}

\begin{ttheorem}
  Let \( \sigma :K_1\to K_2 \) be a field isomorphism.
  Suppose that \( L_i \) is a field with \( K_i\sseq L_i\ (i=1,2) \).
  Suppose also that \( \alpha\in L_1 \) is algebraic over \( K_1 \), and that \( \beta\in L_2 \) is algebraic over \( K_2 \).
  Then we can extend \( \sigma \) to an isomorphism \( \tau:K_1(\alpha)\to K_2(\beta) \) in such a manner that \( \tau(\alpha) = \beta \) if and only if \( m_\beta(K_2) = \sigma(m_\alpha(K_1)) \).
  \begin{center}
    \begin{tikzcd}[column sep=2em,row sep=2em]
      K_{2}
        \arrow[r, "\varphi_{2}"]
        \arrow[d, "\sigma"]
      & K_{2}(\beta)
        \arrow[r, hookrightarrow, "\iota_{2}"]
        \arrow[d, "\tau"]
      & L_{2} \\[1em]
      K_{1}
        \arrow[r, "\varphi_{1}"]
      & K_{1}(\alpha)
        \arrow[r, hookrightarrow, "\iota_{1}"]
      & L_{1}
    \end{tikzcd}
  \end{center}
\end{ttheorem}

\tbo{Note:}\quad When \( \tau:K_1(\alpha)\to K_2(\beta) \) is a \homo, and \( \tau \) extends the \homo~\( \sigma:K_1\to K_2 \), then \( \tau \) is completely determined by \( \sigma \) and the value of \( \tau(\alpha) \).

\begin{tcorollary}
  Let \( L:M \) be a field extension with \( M\sseq L \). Suppose that \( \sigma:M\to L \) is a \homo, and \( \alpha\in L \) is algebraic over \( M \).
  Then the number of ways we can extend \( \sigma \) to a \homo~\( \tau:M(\alpha)\to L \) is equal to the number of distinct roots of \( \sigma(m_\alpha(M)) \) that lie in \( L \).
\end{tcorollary}

\end{document}