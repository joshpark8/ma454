\documentclass[a4paper]{article}

\input{preamble}
\input{letterfont}
\input{macros}

\begin{document}
\section{Soluble Groups II}
\begin{ttheorem}[Theorem - Definition]
  Let \( G \) be a group.
  Then the following are equivalent:
  \begin{enumerate}
    \item[0.] \( G \) is a (finite) soluble group;
    \item There exists some \( n\in \Z^{+} \tst G^{(n)} = \left\{ e \right\} \);
    \item There exists a normal series \begin{align*}
      \left\{ Id. \right\} &= G_n \sgp G_{n-1} \sgp \cdots \sgp G_1 \sgp G_0 = G
    \end{align*}
    such that \( G_j \pnsgp G \) and all quotients \( \qg{G_{j-1}}{G_j} \) are abelian;
    \item There exists a subnormal series such that quotients \( \qg{G_{j-1}}{G_j} \) are abelian.
  \end{enumerate}
\end{ttheorem}

\begin{tdefinition}[Derived group]
  Let \( G \) be a group.
  Then the \it{derivative of G} is \( G' = \cyc{[x,y] : x,y\in G} = [G,G] \) where \( [x,y] = xyx\inv y\inv \) is the \it{commutator} of \( x\tand y \), and \( (G')' = G'' \).
\end{tdefinition}

% \begin{texercise}
%   \( G'\nsgp G \), \( \qg{G}{G'} \) abelian
% \end{texercise}

% \begin{texercise}
%   \( G' \) is a minimal normal subgroup of \( G \) such that \( \qg{G}{G'} \) is abelian.
% \end{texercise}

\begin{tdefinition}
  The \it{derived series} of \( G \) is \( G^{(n)} = \left( G^{(n-1)} \right)' \) and \( \tgp = G^{(n)} \pnsgp G^{(n-1)} \pnsgp \cdots \pnsgp G' \pnsgp G \) (not to be confused with \( G_{n+1} = [G_n, G]\), the \it{lower central series}).
\end{tdefinition}

\begin{tlemma}
  Let \( \vphi:G\mapsto H \) be an epimorphism.
  Then \( \vphi(G') = H' \).
\end{tlemma}

\begin{tdefinition}[Composition series]
  Let \( G \) be a group.
  Then a \it{composition series} of \( G \) is a subnormal series of finite length
  \begin{align*}
    \left\{ Id. \right\} &= G_0 \pnsgp G_{1} \pnsgp \cdots \pnsgp G_{\ell-1} \pnsgp G_\ell = G
  \end{align*}
  such that \( \qg{G_j}{G_{j-1}} \) is a simple group for all \( j \).
\end{tdefinition}

\begin{ttheorem}[Jordan-H\"older]
  Any 2 composition series of some group \( G \) are equivalent up to permutation and isomorphism.
\end{ttheorem}

\begin{ttheorem}
  Let \( K \) be a field with \( \chr K \neq 2 \) and let \( f\in K[t] \) be a separable polynomial with splitting field \( L \).
  Then \( f=0 \) is solvable by \it{quadratic} radicals \iff~\( \fdeg{L}{K}=2^t \).
\end{ttheorem}

\end{document}