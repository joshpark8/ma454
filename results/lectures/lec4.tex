\documentclass[a4paper]{article}

\input{preamble}
\input{letterfont}
\input{macros}

\begin{document}
\section{Field Extensions II}
\begin{tdefinition}[Smallest subring/subfield]
  Let \( L:K \) with \( K\sseq L \).
  \begin{enumerate}[label=(\roman*)]
    \item When \( \alpha\in L \), we denote by \( K[\alpha] \) the \ul{smallest subring of \( L \) containing \( K \) and \( \alpha \)}, and by \( K(\alpha) \) the \ul{smallest subfield of \( L \) containing \( K \) and \( \alpha \)};
    \item More generally, when \( A\sseq L \), we denote by \( K[A] \) the \ul{smallest subring of \( L \) containing \( K \tand A \)}, and by \( K(A) \) the \ul{smallest subfield of \( L \) containing \( K \tand A \)}.
  \end{enumerate}
  Then \begin{align*}
    K[\alpha] &= \lt\{ \sum_{i=0}^{d}c_i\alpha^i : d\in \ZZ{\leq 0},\ c_0,\ldots,c_d\in K \rt\} \\
    K(\alpha) &= \lt\{ f/g : f,g\in K[\alpha], g\neq 0 \rt\}.
  \end{align*}
\end{tdefinition}

\begin{tdefinition}[Algebraic/transcendental element]
  Suppose that \( L: K \) is a field extension with \( K\sseq L \) and \( \alpha\in L \).
  \begin{enumerate}[label=(\roman*)]
    \item We say \ul{\( \alpha \) is algebraic over K} if \( \exists f_{\not\equiv 0} \in K[t] \) such that \( f(\alpha)=0 \).
    \item If \( \alpha \) is not algebraic over \( K \), then we say \ul{\( \alpha \) is transcendental over \( K \)}.
    \item When every element of \( L \) is algebraic over \( K \), we say that \ul{\( L \) is algebraic over \( K \)}.
  \end{enumerate}
\end{tdefinition}

\begin{tdefinition}[Evaluation map]
  Suppose that \( L: K \) is a field extension with \( K \sseq L \), and that \( \alpha\in L \).
  We define the \ul{evaluation map} \( E_\alpha : K[t] \to L \) by putting \( E_\alpha(f) = f(\alpha) \) for each \( f \in K[t] \).
\end{tdefinition}

\begin{tdefinition}[Minimal polynomial]
  Suppose that \( L : K \) is a field extension with \( K \sseq L \), and suppose that \( \alpha\in L \) is algebraic over \( K \).
  Then the minimal polynomial of \( \alpha \) over \( K \) is the unique monic polynomial \( \mak \) having the property that \( \ker(E_\alpha) = (\mak) \).
\end{tdefinition}

\begin{tlemma}
  \begin{enumerate}
    \item \( \mak \) is irreducible over \( K \);
    \item If \( f\in K[t] \) such that \( f(\alpha) = 0 \), then \( \mak\divs f \);
    \item If \( f\in K[t] \) such that \( f(\alpha) = 0 \) and \( f \) is irreducible over \( K \), then \( \exists k\in K \) such that \( f=k\mak \).
  \end{enumerate}
\end{tlemma}

\begin{ttheorem}
  Let \( L : K \) with \( K \sseq L \), and suppose that \( \alpha\in L \) is algebraic over \( K \).
  \begin{enumerate}[label=(\roman*)]
    \item \( K[\alpha] \) is a field, and \( K[\alpha] = K(\alpha) \);
    \item If \( n=\deg \mak \), then \( \lt\{ 1,\alpha,\alpha^2,\ldots,\alpha^{n-1} \rt\} \) is a basis for \( K(\alpha) \) over \( K \) (\imp \( \fdeg{K(\alpha)}{K}=\deg \mak \)).
  \end{enumerate}
\end{ttheorem}

\begin{ttheorem}[Rational Root Theorem]
  Let \( \frac{p}{q} \) be a root of \( f= a_0t^n+\cdots + a_{n-1} t^{n-1} + a_n \), for \( a_j\in\Z \), where \( p\tand q \) are coprime.
  Then \( p\divs a_n \) and \( q\divs a_0 \).
\end{ttheorem}

\bd{Note:}\quad If \( \alpha \) is transcendental over \( K \), then \( K(\alpha)\iso K(x) \) (where \( x \) is a formal variable).

\begin{tcorollary}
  Let \( L : K \) with \( K \sseq L \), and suppose that \( \alpha\in L \) is algebraic over \( K \).
  Then every element of \( K(\alpha) \) is algebraic over \( K \).
\end{tcorollary}

\begin{tcorollary}
  Let \( L:K \) with \( K\sseq L \).
  Then \( \fdeg{L}{K} < \infty \iff L=K(\llist{\alpha}{1}{n}) \) for \( \alpha_j\in L \).
\end{tcorollary}

\begin{ttheorem}
  Let \( L:K \) be a field extension, and define \begin{align*}
    L^{\nf{alg}}=\{\alpha\in L : \alpha \text{ is algebraic over } K\}.
  \end{align*}
  Then \( L^{\nf{alg}} \) is a subfield of \( L \).
\end{ttheorem}

\end{document}