\documentclass[a4paper]{article}

\input{preamble.tex}
\input{letterfont.tex}
\input{macros.tex}


\begin{document}
\section{Galois Fields I}
\begin{tdefinition}[Formal derivative]
  We define the \ul{derivative operator} \( \mcD:K[t]\to K[t] \) by \begin{align*}
    \mcD\lt(\sum_{k=0}^{n}a_kt^k\rt) = \sum_{k=1}^{n}ka_kt^{k-1}.
  \end{align*}
\end{tdefinition}

\begin{ttheorem}
  Let \( f\in K[t]\setminus K \), and let \( L:K \) be a \sfe~for \( f \).
  Assume that \( K\sseq L \).
  Then the following are equivalent: \begin{enumerate}[label=(\roman*)]
    \item The polynomial \( f \) has a repeated root over \( L \);
    \item There is some \( \alpha\in L \) for which \( f(\alpha)=0=(\mcD f)(\alpha) \);
    \item There is some \( g\in K[t] \) having the property that \( \deg g \geq 1 \) and \( g \) divides both \( f \) and \( \mcD f \).
  \end{enumerate}
\end{ttheorem}

\begin{tdefinition}[Inseparable]
  A polynomial \( f \in K[t] \) is \ul{inseparable over \( K \)} if \( f \) is not separable over \( K \), meaning that \( f \) has an irreducible factor \( g \in K[t] \) having the property that \( g \) has fewer than \( \deg g \) distinct roots in \( K \).
\end{tdefinition}

\begin{ttheorem}
  Suppose that \( f\in K[t] \) is irreducible over \( K \).
  Then \( f \) is inseparable over \( K \) if and only if \( \char K=p>0 \), and \( f \in K[t^p]\), which is to say that \( f=a_0+a_1t^p+\cdots+a_mt^{mp} \), for some \( \llist{a}{0}{m}\in K \).
\end{ttheorem}

\begin{tdefinition}[Frobenius map]
  Suppose that \( \char K = p > 0 \).
  The \ul{Frobenius map} \( \phi:K\to K \) is defined by \( \phi(\alpha)=\alpha^p \).
\end{tdefinition}

\begin{ttheorem}
  Suppose that \( \char K = p > 0 \), and put \( F=\lt\{ c\cdot 1_K : c\in \Z \rt\} \).
  Then \( F \) is a subfield (called the prime subfield) of \( K \), and \( F\iso \qg{\Z}{p\Z} \).
\end{ttheorem}

\begin{tdefinition}[Fixed field]
  Let \( L:K \) be a field extension.
  When \( G \) is a subgroup of \( \Aut(L) \), we define the fixed field of \( G \) to be \begin{align*}
    \Fix{L}{G} = \lt\{ \alpha\in L : \sigma(\alpha) = \alpha \text{ for all } \sigma\in G \rt\}.
  \end{align*}
\end{tdefinition}

\begin{ttheorem}
  Suppose that \( \char K = p>0 \), and let \( F \) be the prime subfield of \( K \).
  Let \( \phi:K\to K \) denote the Frobenius map.
  Then \( \phi \) is an injective homomorphism, and \( \Fix{\phi}{K} = F \).
\end{ttheorem}

\begin{tcorollary}
  Suppose that \( \char K = p > 0 \) and \( K \) is algebraic over its prime subfield.
  Then the Frobenius map is an automorphism of \( K \).
\end{tcorollary}

\begin{tcorollary}
  Suppose that \( \char K = p > 0 \) and \( K \) is algebraic over its prime subfield.
  Then all polynomials in \( K[t] \) are separable over \( K \).
\end{tcorollary}

\begin{tcorollary}[**]
  Suppose that \( \char K = 0 \).
  Then all polynomials in \( K[t] \) are separable over \( K \).
\end{tcorollary}

\begin{ttheorem}
  Suppose that \( \char K = p > 0 \). Let \begin{align*}
    f(t) = g(t^p) = a_0+a_1t^p+\cdots+a_{n-1}t^{(n-1)p}+t^{np}
  \end{align*}
  be a non-constant monic polynomial over \( K \).
  Then \( f(t) \) is irreducible in \( K[t] \) if and only if \( g(t) \) is irreducible in \( K[t] \) and not all the coefficients \( a_i \) are \( p \)-th powers in \( K \).
\end{ttheorem}

\end{document}