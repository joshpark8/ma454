\documentclass[a4paper]{article}

\input{preamble}
\input{letterfont}
\input{macros}

\begin{document}
\section{Fundamental Theorem of Galois Theory I}
\begin{ttheorem}[Fundamental Theorem of Galois Theory, Part 1]
  Let \( L:K \) be a Galois extension with \( G = \Gal(L:K) \).
  Define \( \mcI(K,L) \) and \( \mcS(G) \) as the set of all intermediate fields of \( L:K \) and the set of all subgroups of \( G \), respectively.
  For all \( P\in \mcI(K,L) \), we have \( P = L^{G_P} \) where \( G_P = \Aut_P(L)\)
  Then \begin{align*}
    \forall P\in \mcI(K,L), \quad &L^{G_P} = P,\\
    \forall H\in \mcS(G), \quad &G_{L^H} = H,
  \end{align*}
  Also, \( P_1\sseq P_2 \iff G_{P_1}\ogp G_{P_2} \) and \( H_1 \sgp H_2 \iff L^{H_1}\supeq L^{H_2} \). % by Theorem 1 of Lecture 19
\end{ttheorem}
\end{document}