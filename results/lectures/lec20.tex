\documentclass[a4paper]{article}

\input{preamble}
\input{letterfont}
\input{macros}

\begin{document}
\section{Fundamental Theorem of Galois Theory I}
\begin{tdefinition}[\( \mcI(K,L),\ \mcS(G) \)]
  Let \( L:K \) be a Galois extension with \( G = \Gal(L:K) \).
  Define \( \mcI(K,L) \) and \( \mcS(G) \) as the set of all intermediate fields of \( L:K \) and the set of all subgroups of \( G \), respectively.
\end{tdefinition}

\begin{ttheorem}[Fundamental Theorem of Galois Theory, Part 1]
  For all \( P\in \mcI(K,L) \), we have \( P = L^{G_P} \) where \( G_P = \Aut_P(L)\)
  Then \begin{align*}
    \forall P\in \mcI(K,L), \quad &L^{G_P} = P,\\
    \forall H\in \mcS(G), \quad &G_{L^H} = H,
  \end{align*}
  Also, \( P_1\sseq P_2 \iff G_{P_1}\ogp G_{P_2} \) and \( H_1 \sgp H_2 \iff L^{H_1}\supeq L^{H_2} \). % by Theorem 1 of Lecture 19
\end{ttheorem}
% Let \(L:K\) be any extension and let \(G \subseteq \Aut(L)\).
% Let \(\mathcal{I}(K,L)\) be the collection of all intermediate fields of \(L:K\) (i.e.\ all fields \(M\) with \(K \subseteq M \subseteq L\)),
% and let \(\mathcal{S}(G)\) be the family of all subgroups of \(G\).

% We have the sets:
% \[
%   \mathcal{I}(K,L)
%   \quad\text{and}\quad
%   \mathcal{S}(G).
% \]

% For any \(M \in \mathcal{I}(K,L)\), we consider \(\Gal(L:M) \subseteq \Aut(L)\).
% For any \(H \in \mathcal{S}(G)\), we consider \(\Fix{L}{H} = \{\,\alpha \in L : h(\alpha) = \alpha\ \forall h\in H\}\).

% \medskip

% \noindent
% \textbf{Galois Correspondence Claim.}\quad
% There is a one-to-one correspondence between
% \[
%   \mathcal{I}(K,L)
%   \quad\text{and}\quad
%   \mathcal{S}(G),
% \]
% where \(G = \Gal(L:K)\). In other words,
% \[
%   M \longleftrightarrow \Gal(L:M)
%   \quad\text{and}\quad
%   H \longleftrightarrow \Fix{L}{H}.
% \]

% \smallskip

% 1. If \(\widetilde{L}:K\) is a Galois extension (see Theorem 2 of the previous lecture), then \(L:K\) is also a Galois extension. By Corollary of Theorem 4, if \(K \subset M \subset L\) is a tower and \(L:K\) is Galois, then \(L:M\) is Galois as well.

% 2. Using the primitive element theorem, we know \(L = K(\alpha)\) for some \(\alpha \in L\).
%    Consider the orbit \(H \cdot \alpha\) and the polynomial
%    \[
%      \mu_{\alpha,K}(t) \;=\; \prod_{\beta \in H\cdot\alpha} (t-\beta).
%    \]
%    Since \(\alpha \in L\), we see \(\mu_{\alpha,K}(t) \in K[t]\). The orbit of \(\alpha\) under the action of \(H\) yields \(\deg(\mu_{\alpha,K}(t)) = |H\cdot \alpha|\).

% 3. We also know that
%    \[
%      K\bigl(\alpha\bigr)^H = \bigl\{\gamma \in K(\alpha) : h(\gamma)=\gamma \;\forall h\in H\bigr\}.
%    \]
%    Hence
%    \[
%      \bigl|\,\Gal\bigl(L:K\bigr)\bigr| \;=\; \bigl|H\cdot \alpha\bigr|.
%    \]

% \medskip

% \begin{center}
%   \textbf{Further Example: Cyclotomic Extensions}
% \end{center}

% \noindent
% For a prime \(p\), consider \( \zeta_p = e^{2\pi i/p} \). The extension \(\mathbb{Q}(\zeta_p) : \mathbb{Q}\) is Galois with
% \[
%   \Gal\bigl(\mathbb{Q}(\zeta_p) : \mathbb{Q}\bigr) \cong (\mathbb{Z}/p\mathbb{Z})^\times,
% \]
% where the non-trivial subgroup is generated by the complex conjugation or by Frobenius automorphisms.

% \smallskip

% \noindent
% \textbf{Subgroups and Fixed Fields.}\quad
% For example, with \(p=5\), we have \(\zeta_5 = e^{2\pi i/5}\). The only non-trivial subgroup of the Galois group corresponds to complex conjugation. Hence the fixed field of this subgroup is \(\mathbb{Q}(\zeta_5 + \zeta_5^{-1})\), etc.

% \medskip

% \noindent
% \textbf{Conclusion.}\quad
% This establishes a Galois correspondence between the intermediate fields \(\mathcal{I}(K,L)\) and the subgroups \(\mathcal{S}(\Gal(L:K))\).
% In particular, if \(M\) is an intermediate field, then \(\Gal(L:M)\) is the subgroup of \(\Gal(L:K)\) that fixes \(M\); and if \(H\) is a subgroup of \(\Gal(L:K)\), then \(\Fix{L}{H}\) is the intermediate field fixed by \(H\).

\end{document}