\documentclass[a4paper]{article}

\input{preamble.tex}
\input{letterfont.tex}
\input{macros.tex}

\begin{document}
\section{The Primitive Element Theorem}
\begin{tdefinition}[Simple extension]
  Suppose \( L:K \) is a field extension relative to the embedding \( \varphi:K\to L \).
  We say that \( L:K \) is a \ul{simple extension} if there is some \( \gamma\in L \) having the property that \( L=\vphi(K)(\gamma) \).
\end{tdefinition}

\begin{ttheorem}[The Primitive Element Theorem]
  If \( L:K \) be a finite, separable extension with \( K\sseq L \), then \( L:K \) is a simple extension.
\end{ttheorem}

\begin{tcorollary}
  Suppose that \( L:K \) is an algebraic, separable extension, and suppose that for every \( \alpha\in L \), the polynomial \( \mu_\alpha^K \) has degree at most \( n \) over \( K \).
  Then \( \fdeg{L}{K}\leq n \).
\end{tcorollary}

\bd{Fact:}\quad Let \( L:K \) be a normal extension and let \( \deg(\mak)\leq n \) for all \( \alpha\in L \).
Then \( \fdeg{L}{K}\leq n \).

\begin{tcorollary}
  If \( f\in K[t] \) is irreducible over \( K \), then \( \Gal_K(f) \) acts transitively on the roots of \( f \).
\end{tcorollary}

\end{document}