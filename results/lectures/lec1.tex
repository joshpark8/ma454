\documentclass[a4paper]{article}

\input{preamble}
\input{letterfont}
\input{macros}

\begin{document}
\section{Introduction I}
\begin{tdefinition}[Symmetric function]
  A function \( \vphi(\llist{x}{1}{n}) \) is called \it{symmetric} if \begin{align*}
    \varphi(x_1,\ldots,x_n) = \varphi(x_{\omega(1)},\ldots,x_{\omega(n)})
  \end{align*}
  for all \( \omega\in S_n \).
\end{tdefinition}

\begin{tdefinition}[Elementary symmetric polynomial]
  \begin{align*}
    \sigma_1 = \sigma_1(x_1,\ldots,x_n) &= x_1+\cdots+x_n \\
    \sigma_2 = \sigma_2(x_1,\ldots,x_n) &= x_1x_2+\cdots+x_1x_n+x_2x_3+\cdots+x_{n-1}x_n \\
    &\vdots \\
    \sigma_k = \sigma_k(x_1,\ldots,x_n)&=\sum\limits_{1\leq i_1<i_2<\cdots<i_k\leq n}x_{i_1}\cdots x_{i_k} \\
    &\vdots \\
    \sigma_n = \sigma_n(x_1,\ldots,x_n)&=\prod\limits_{i=1}^n x_i
  \end{align*}
\end{tdefinition}

\begin{ttheorem}
  For any symmetric function \( \psi(\llist{x}{1}{n}) \), there exists a unique polynomial \( P(\llist{t}{1}{n}) \) such that \( \psi(\llist{x}{1}{n}) = P(\llist{\sigma}{1}{n}) \).
\end{ttheorem}

\begin{tdefinition}[Vieta formulae]
Suppose \( f(t) = t^{n} + a_{n-1} t^{n-1} + \cdots + a_{0} \) has roots \( \llist{r}{1}{n} \).
Then,
  \begin{align*}
  r_1 + r_2 + \cdots + r_n &= -a_{n-1} \\
  \sum_{1 \le i < j \le n} r_i r_j &= a_{n-2} \\
  &\vdots \\
  \sum_{1 \le i_1 < i_2 < \dots < i_k \le n} r_{i_{1}}r_{i_{2}}\cdots r_{i_{k}} &= (-1)^k a_{n-k} \\
  &\vdots \\
  r_1 r_2 \cdots r_n &= (-1)^n a_0
  \end{align*}
\end{tdefinition}

\begin{tcorollary}
  The discriminant \( D \) of \( f\in R[x] \), where \( R \) is a ring and \( f= x^n+a_1x^{n-1}+\cdots + a_{n-1}x+a_n\), is a polynomial in \( \llist{a}{1}{n} \) and coefficients from \( R \) (i.e. \( D \in R[\llist{a}{1}{n}] \)).
\end{tcorollary}

\bd{Note:} \quad
Any cubic equation can be converted to a depressed cubic by \begin{align*}
  x\cb+Ax\sq+Bx+c = \left(x+\frac{A}{3}\right)\cb + p\left(x+\frac{A}{3}\right) + q.
\end{align*}

\begin{ttheorem}[Vieta's method]
Using the trigonometric identity \( \cos 3\vphi = 4\cos\cb \vphi-3\cos\vphi \), we can solve certain cubic equations.
For example, consider \( 4x\cb-3x=-\frac{1}{2} \).
Let \( x=\cos \vphi \).
Then \begin{align*}
  \cos 3\vphi = -\frac{1}{2} &\iff 3\vphi = \pm \frac{2\pi}{3}+2\pi k \quad \text{for } k\in \Z \\
  &\iff \vphi = \pm\frac{2\pi}{9}+2\pi k \\
  &\iff x \in \lt\{ \cos\frac{2\pi}{9},\cos\frac{4\pi}{9},\cos\frac{8\pi}{9} \rt\}.
\end{align*}
In general, we can use this method to solve \( 4x\cb-3x=a \imp x=\cos\vphi,\ \cos 3\vphi \) and \( \cos:\C\to\C \) is now a complex function.
For \( x\cb+px+q = 0 \), set \( x=ky \) such that \( \frac{k\cb}{pk} = \frac{-4}{3} \imp k = \pm\frac{\sqrt{-4p}}{3} \).
\end{ttheorem}

\begin{tdefinition}[Ferrari's resolvent]
  Let \( f(x) = x^4 + a x^2 + b x + c \), and assume \( b\sq-4ac\neq 0 \).
  Consider a parameter \( y \).
  Then \begin{align*}
    f(x) &= \left( x\sq + \frac{y}{2} \right)\sq + (a-y)x\sq + bx + c - \frac{y\sq}{4} \\
     \imp D &= b\sq - 4(a-y)\left( c-\frac{y\sq}{4} = 0 \right)
  \end{align*}
  and hence we obtain \it{Ferrari' resolvent}:
  \begin{align*}
    y\cb-ay\sq -4cy + 4ac - b\sq = 0.
  \end{align*}
  Solving the resolvent allows one to reduce solving \( f \) to solving a system of quadratics.
\end{tdefinition}
\end{document}