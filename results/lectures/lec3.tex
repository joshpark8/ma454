\documentclass[a4paper]{article}

\input{preamble}
\input{letterfont}
\input{macros}

\begin{document}
\section{Field Extensions I}
\begin{tdefinition}[Integral domain]
  Let \( R \) be a commutative ring. Then \( R \) is \it{an integral domain} if \( ab=0 \) implies that \( a=0 \) or \( b=0 \) for all \( a,b\in R \).
\end{tdefinition}

\begin{tdefinition}[Euclidean domain]
  Let \( R \) be an integral domain. Then \( R \) is a \it{Euclidean domain} if there exists some function \( f:R\setminus\{0\}\to \mathbb{Z}_{\geq 0} \) such that for all \( a,b_{\not\equiv 0} \in R \), there exist elements \( q,r\in R \) such that \( a=qb+r \) where \( r=0 \) or \( f(r)<f(b) \).
\end{tdefinition}

\begin{ttheorem}[B\'ezout's Identity]
  Let \( R \) be a Euclidean domain. For \( a,b\in R \), there exists \( \alpha,\beta\in R \) such that \( \gcd(a,b) = \alpha a+\beta b \)
\end{ttheorem}

\begin{tdefinition}[Irreducible]
  Let \( {F} \) be a field, and \( f\in F[t]\setminus F \).
  Then \( f \) is \it{irreducible} if \( \not\exists g,h\in F[t]\setminus F \) of strictly smaller degree such that \( f=gh \).
\end{tdefinition}

\begin{tdefinition}[Unique factorization domain]
  Let \( R \) be an integral domain.
  Then \( R \) is \it{a unique factorization domain (UFD)} if for irreducible \( p_i\in R \), any nonzero \( x\in R \) can be written uniquely (up to ordering) as \( x=p_1p_2\cdots p_k,\quad k\geq 1 \).
\end{tdefinition}

\bd{Fact:}\quad If \( R \) is an Euclidean domain, then \( R \) is a UFD (and PID)

\begin{tcorollary}
  Let \( f\in \F[t] \) be a monic polynomial with \( \deg f\geq 1 \).
  Then we can write \( f = f_1f_2\cdots f_k \) uniquely (up to ordering) for irreducible monic polynomials \( f_j \).
\end{tcorollary}

\begin{tdefinition}
  Let \( R \) be a UFD.
  When \( \llist{a}{0}{n}\in R \) are not all 0, we can generalize the \it{greatest common divisor} of \llist{a}{0}{n} (written gcd(\llist{a}{0}{n})) any element \( c\in R \) satisfying \begin{enumerate}[label=(\roman*)]
    \item \( c\divs a_i\ (0\leq i\leq n) \), and
    \item if \( d\divs a_i\ (0\leq i\leq n) \), then \( d\divs c \).
  \end{enumerate}
  When \( f=\sum_{j=0}^{d}a_jx^j\in R[x] \) is a non-zero polynomial, we define a \it{content} of \( f \) to be any \( \gcd (\llist{a}{0}{d}) \) and \( \gcd(f) = \gcd(\llist{a}{0}{d}) \).
  We say that \( f\in R[X] \) is \it{primitive} if \( f\neq 0 \) and the content of \( f \) is divisible only by units of \( R \).
\end{tdefinition}

\begin{tlemma}[Gauss]
  \( \gcd(fg) = \gcd f \cdot \gcd g\)
\end{tlemma}

\begin{tcorollary}
  \( f\in\Z[t] \) is irreducible \iff~\( f \) is irreducible over \( \Q[t] \)
\end{tcorollary}

\begin{tcorollary}
  If \( R \) is a UFD with field of fractions \( Q \) and \( f \in R[X] \) with \( \deg f > 0 \), then \( f \) is irreducible in \( R[X] \) \iff~\( f \) is irreducible in \( Q \).
\end{tcorollary}

\begin{ttheorem}[Eisenstein's Criterion]
  Let \( R \) be a UFD with field of fractions \( Q \) and let \( f=a_0+a_1X+\ldots +a_nX^n \in R[X] \) with \( \gcd(f) = 1 \).
  Suppose there exists an irreducible element \( p \in R \) such that
  \begin{enumerate}[label=(\roman*)]
    \item \( p\divs a_i \) for \( 0\leq i < n \),
    \item \( p\sq \ndivs a_0 \), and
    \item \( p\ndivs a_n \),
  \end{enumerate}
  then \( f \) is irreducible in \( R[X] \) (and hence also in \( Q[X] \)).
\end{ttheorem}

\begin{tdefinition}[Field extension]
  Let \( L \tand K \) be fields.
  Then \( L \) is an \it{extension} of \( K \) if there exists a \homo~\( \vphi: K \to L \).
  Then we write \( L:K \) or \( L/K \), \( \vphi(K)\iso K \) and identify \( \vphi(K) \) with \( K \).
\end{tdefinition}

\bd{Fact:}\quad Suppose that \( L \) is a field extension of \( K \) with associated embedding \( \vphi: K \to L \).
  Then \( L \) forms a vector space over \( K \), under the operations \begin{align*}
    \nf{(vector addition) } \psi: L\times L &\to L \quad \text{given by} \quad (v_1,v_2) \mapsto v_1 + v_2 \\
    \nf{(scalar multiplication) } \tau : K\times  L &\to L \quad \text{given by} \quad (k,v) \mapsto \vphi(k)v.
  \end{align*}

\begin{tdefinition}[Degree, finite extension]
  Let \( L: K \).
  Then the \it{degree} of \( L: K \) is \( \fdeg{L}{K}=\dim L \) over \( K \) as a vector space.
  We say that \( L : K \) is a \it{finite extension} if \( [L: K] <\infty \).
\end{tdefinition}

\begin{tdefinition}[Tower, intermediate field]
  We say that \( M : L : K \) is a \it{tower} of field extensions if \( M : L \tand L: K \) are field extensions, and in this case we say that \( L \) is an \it{intermediate field} (relative to the extension \( M : K \))
\end{tdefinition}

\begin{ttheorem}[The Tower Law]
  Suppose that \( M :L: K \) is a tower of field extensions.
  Then \( M : K \) is a field extension, and \( [M : K] = [M : L][L: K] \).
\end{ttheorem}

\begin{tcorollary}
  Suppose that \( L:K \) is a field extension for which \( [L: K] \) is a prime number.
  Then whenever \( L : M : K \) is a tower of field extensions with \( K \sseq M \sseq L \), one has either \( M= L \tor M= K \).
\end{tcorollary}
\end{document}