\documentclass[a4paper]{article}

\input{preamble}
\input{letterfont}
\input{macros}

\begin{document}
\section{Ruler and Compass Constructions}
\begin{tdefinition}[Constructible points/angles]
  Let \( P_0 = (0,0)\tand P_1 = (1,0) \), and let \( \mcS_n = (P_0,\ldots,P_n) \).
  Then \( P_{n+1} \) is a constructible point if it is the intersection of either \begin{enumerate}
    \item two lines containing points in \( \mcS_n \);
    \item two circles with centers in \( \mcS_n \);
    \item a circle and line with center and endpoints in \( \mcS_n \).
  \end{enumerate}
  Similarly, an angle \( \theta \) is constructible if for some \( a\in \R \), there exists some constructible point \( x \) such that \( x^2 = a^2 \).
\end{tdefinition}

\begin{tlemma}
  If \( n \)-gon constructible, then \( 2n \)-gon is constructible.
\end{tlemma}

\begin{tlemma}
  If \( a,b,c \) constructible (or polyquadratic), then \( a\pm b,\ \frac{ab}{c},\tand \sqrt{ab} \) constructible.
\end{tlemma}

\begin{tfact}
If \( m \)-gon and \( n \)-gon are constructible for coprime \( m,n \), then \( mn \)-gon is contructible.
\end{tfact}

\begin{tfact}
  If \( p\geq \) prime, then \( p^k \)-gon constructible for \( k\in \N \).
\end{tfact}

\begin{ttheorem}[Gauss]
  \begin{align*}
    \cos \frac{2\pi}{17} = \frac{-1 + \sqrt{17}+\sqrt{34-2\sqrt{17}}+2\sqrt{17+3\sqrt{17}-\sqrt{34-2\sqrt{17}}}}{16}
  \end{align*}
\end{ttheorem}

\begin{tcorollary}
  The 17-gon is constructible.
\end{tcorollary}

\begin{tcorollary}
  If \( a\in\R \) is constructible, then \( \fdeg{\Q(a)}{\Q}=2^n \) for some \( n\geq  \)
\end{tcorollary}

\begin{tcorollary}
  Given a cube \( C_1 \) with volume \( V_1 \), it is impossible to construct a cube \( C_2 \) with volume \( 2V_2 \) by ruler and compass.
  That is, the volume of a cube can not be duplicated by ruler and compass.
\end{tcorollary}

\begin{tcorollary}
  An arbitrary angle cannot be trisected by ruler and compass.
\end{tcorollary}

\begin{ttheorem}[Gauss-Wantzel]
  A regular \( n \)-gon is constructible \( \iff n=2^r \prod_{i=1}^s p_i \) for \( r\in\Z_{\geq 0} \) and Fermat primes \( p_i = 2^{\left( 2^k \right)}+1 \) for \( k\in \Z_{\geq 0} \).
\end{ttheorem}

\hl{TODO:} define \( V_n \)

\begin{tlemma}
  For all integers \( d \) and all \( d \)-periods \( \theta \), \( V_d = \Q(\theta) \).
\end{tlemma}


\end{document}