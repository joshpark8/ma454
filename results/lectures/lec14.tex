\documentclass[a4paper]{article}

\input{preamble.tex}
\input{letterfont.tex}
\input{macros.tex}

\begin{document}
\section{Separability}
\begin{tdefinition}[Separable]
  Let \( K \) be a field. \begin{enumerate}[label=(\roman*)]
    \item An irreducible polynomial \( f\in K[t] \) is \ul{separable over \( K \)} if it has no multiple roots, meaning that \( f=\lambda(t-\alpha_1)(t-\alpha_2)\cdots(t-\alpha_d) \), where \( \llist{\alpha}{1}{d}\in \Kbar \) are distinct.
    \item A non-zero polynomial \( f\in K[t] \) is \ul{separable over \( K \)} if its irreducible factors in \( K[t] \) are separable over \( K \).
    \item When \( L:K \) is a field extension, we say that \( \alpha \in L \) is \ul{separable over \( K \)} when \( \alpha \) is algebraic over \( K \) and \( \mu_\alpha^K \) is separable.
    \item An algebraic extension \( L:K \) is \ul{a separable extension} if every \( \alpha\in L \) is separable over \( K \).
  \end{enumerate}
\end{tdefinition}

\begin{tlemma}
  Suppose that \( L:M:K \) is a tower of algebraic field extensions.
  Assume that \( K\sseq M\sseq L\sseq \Kbar \), and suppose that \( f\in K[t]\setminus K \) satisfies the property that \( f \) is separable over \( K \).
  If \( g\in M[t]\setminus M \) has the property that \( g\divs f \), then \( g \) is separable over \( M \).
  Thus, if \( \alpha\in L \) is separable over \( K \) then \( \alpha \) is separable over \( M \), and if \( L:K \) is separable then so is \( L:M \).
\end{tlemma}

\begin{tlemma}
Suppose that \( L:M \) is an algebraic field extension.
  Let \( \alpha\in L \) and \( \sigma:M\to\Mbar \) be a \homo.
  Then \( \sigma(m_\alpha(M)) \) is separable over \( \sigma(M) \) if and only if \( m_\alpha(M) \) is separable over \( M \).
\end{tlemma}

\begin{ttheorem}
  Let \( L:K \) be a finite extension with \( K\sseq L\sseq \Kbar \), whence \( L=K(\llist{\alpha}{1}{n}) \) for some \( \llist{\alpha}{1}{n}\in L \).
  Put \( K_0=K \), and for \( 1\leq i\leq n \), set \( K_i=K_{i-1}(\alpha_i) \).
  Finally, let \( \sigma_0:K\to\Kbar \) be the inclusion map. \begin{enumerate}[label=(\roman*)]
    \item If \( \alpha_i \) is separable over \( K_{i-1} \) for \( 1\leq i\leq n \), then there are \( \fdeg{L}{K} \) ways to extend \( \sigma_0 \) to a \homo~\( \tau:L\to \Kbar \).
    \item If \( \alpha_i \) is not separable over \( K_{i-1} \) for some \( i \) with \( 1\leq i\leq n \), then there are fewer than \( \fdeg{L}{K} \) ways to extend \( \sigma_0 \) to a \homo~\( \tau:L\to \Kbar \).
  \end{enumerate}
\end{ttheorem}

\begin{ttheorem}
  Let \( L:K \) be a finite extension with \( L=K(\llist{\alpha}{1}{n}) \).
  Set \( K_0=K \), and for \( 1\leq i\leq n \), inductively define \( K_i \) by putting \( K_i=K_{i-1}(\alpha_i) \).
  Then the following are equivalent: \begin{enumerate}[label=(\roman*)]
    \item the element \( \alpha_i \) is separable over \( K_{i-1} \) for \( 1\leq i\leq n \);
    \item the element \( \alpha_i \) is separable over \( K \) for \( 1\leq i\leq n \);
    \item the extension \( L:K \) is separable.
  \end{enumerate}
\end{ttheorem}

\begin{tcorollary}
  Suppose that \( L:K \) is a finite extension.
  If \( L:K \) is a separable extension, then the number of \( K \)-\homo~\( \sigma:L\to\Kbar \) is \( \fdeg{L}{K} \), and otherwise the number is smaller than \( \fdeg{L}{K} \).
\end{tcorollary}

\begin{tcorollary}
  Suppose that \( f\in K[t]\setminus K \) and that \( L:K \) is a \sfe~for \( f \).
  Then \( L:K \) is a separable extension if and only if \( f \) is separable over \( K \).
  More generally, suppose that \( L:K \) is a \sfe~for \( S\sseq K[t]\setminus K \).
  Then \( L:K \) is a separable extension if and only if each \( f\in S \) is separable over \( K \).
\end{tcorollary}
\end{document}