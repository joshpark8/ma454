\documentclass[a4paper]{article}

\input{preamble}
\input{letterfont}
\input{macros}

\begin{document}
\section{Final remarks II}
\begin{tdefinition}[Resolvent invariant]
Let \( G\sgp S_n \) and \( P\in K[\llist{x}{1}{n}] \).
Then \( P \) is \it{resolvent invariant} for \( G \) if \( P^{g} = P \iff g\in G \).
\end{tdefinition}

\begin{tlemma}
  Let \( P \) be resolvent invariant for \( G \).
  Then \begin{enumerate}
    \item \( P^a = P^b \iff ab\inv \in G \) (obvious: \( P^a = P^b \iff P^{ab\inv} = P \))
    \item \( P^a \) is resolvent invariant for \( a\inv G a \)
  \end{enumerate}
\end{tlemma}

\begin{tcorollary}
  Let \( S_n = \sqcup_j a_j G \).
  Then \( P \) is resolvent invariant for \( G \iff P^{a_j} \) are distinct.
\end{tcorollary}

\begin{tdefinition}[Resolvent]
  Let \( P \) be a resolvent polynomial for \( G\sgp S_n \) and \( S_n = \sqcup_{j=1}^{s} a_j G \).
  Then \begin{align*}
    R_G(z) = R_G(z,\llist{x}{1}{n}) = (z-P^{a_1})\cdots(z-P^{a_s})
  \end{align*}
  is a \it{resolvent} for \( G \) (depends on \( P \)).
\end{tdefinition}

\begin{tlemma}
  Let \( G\sgp S_n \), \( f\in K[t] \) be a separable polynomial.
  If \( \Gal_K(f) \sgp G \) (and its conjugation), then \( \exists \jmath\in K \) such that \( R_{G,f}(\jmath) = 0 \)
\end{tlemma}

\begin{tlemma}
  Let \( \order{K}=\infty \) and \( f\in K[t] \) be a separable polynomial.
  Then \( \exists \llist{c}{1}{n}\in K \) such that for all \( k \), \begin{align*}
    h_k(\llist{x}{1}{k}) = c_1x_1+\cdots+c_kx_k
  \end{align*}
  has the property \begin{align*}
    h_k^a(\llist{\alpha}{1}{k}) = h_k^b(\llist{\alpha}{1}{k}) \iff x_i^{a} = x_i^{b} \mathrm{ for } i = 1,\ldots,k,
  \end{align*}
  where \( a,b\in S_n \) are any permutations.
\end{tlemma}

\begin{ttheorem}
  Let \( \order{K}=\infty \), \( f\in K[t] \) be a separable polynomial, and \( G\sgp S_n \).
  Then there exists a resultant \( R_{G,f} (z) \) with no multiple roots.
\end{ttheorem}

\begin{ttheorem}
  Let \( \order{K} = \infty \) and \( f\in K[t] \) be irreducible and separable with \( \deg f = 4 \).
  Then \begin{enumerate}
    \item \( \sqrt D \not\in K\) and \( R_{V_4}^{(f)} \) has no roots in \( K \imp G\iso S_4 \) or \( G\iso Z_4 \)
    \item \( \sqrt D \in K\) and \( R_{V_4}^{(f)} \) has no roots in \( K \imp G\iso A_4 \)
    \item \( \sqrt D \in K\) and \( R_{V_4}^{(f)} \) has a roots in \( K \imp G\iso V_4 \)
    \item \( \sqrt D \not\in K\) and \( R_{V_4}^{(f)} \) has no roots in \( K \imp G\iso S_4 \) or \( G\iso D_4 \)
  \end{enumerate}
  **Exercise**, the point is to show that computing each \( R^{(f)}_{V_4,D_4,Z_4,A_4} \) is not necessary
\end{ttheorem}
\end{document}