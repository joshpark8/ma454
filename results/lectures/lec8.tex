\documentclass[a4paper]{article}

\input{preamble}
\input{letterfont}
\input{macros}

\begin{document}
\section{Splitting Fields, Abel-Ruffini}
\begin{tdefinition}[Splitting field]
  Let \( L:K \) with embedding \( \vphi:K\to L \) and \( f\in K[t]\setminus K \).
  We say \ul{\( f \) splits over \( L \)} if \( \vphi(f) = c\prod\limits_{j=1}^n (x-\alpha_j) \) for \( \alpha_j \in L \) and \( c\in \vphi(K) \).
  We say that \( M:K \) is a \ul{\sfe} for \( f \) if \( f \) splits over \( L \), \( \vphi(K)\sseq M\sseq L \), and \( M \) is the smallest subfield of \( L \) containing \( \vphi(K) \) over which \( f \) splits.
\end{tdefinition}

\begin{tlemma}
  Let \( L:K \) be a \sfe~for \( f\in K[t] \) relative to the embedding \( \vphi:K\to L \), and let \( \alpha_j\in L \) be roots of \( \vphi(f) \).
  Then \( L=\vphi(K)(\llist{\alpha}{1}{n}) \).
\end{tlemma}

\begin{tlemma}
  Let \( L:K \) be a \sfe~for \( f\in K[t]\setminus K \).
  Then \( \fdeg{L}{K}\leq (\deg f)! \).
\end{tlemma}

\begin{tlemma}
  Let \( L:K \tand M:K \) be  \sfe s for \( f\in K[t]\setminus K \).
  Then \( L\iso M \) (in particular, \( \fdeg{L}{K}=\fdeg{M}{K} \)).
\end{tlemma}

\begin{tdefinition}[Radical, radical extension, solvability by radicals]
  Let \( L:K \) and \( \beta \in L \).
  We say that \( \beta \) is \ul{radical} over \( K \) when \( \beta^n \in K \) for some \( n \in \N \) (so \( \beta = \alpha^{1/n} \) for some \( \alpha \in K \) and some \( n \in \N \)).
  We say that \( L:K \) is \ul{an extension by radicals} when there is a tower of field extensions \( L = L_r : L_{r-1} : \cdots : L_0 = K \) such that \( L_i = L_{i-1}(\beta_i) \) with \( \beta_i \) radical over \( L_{i-1} \) (for \( 1 \leq i \leq r \)).
  We say \( f \in K[t] \) is \ul{solvable by radicals} if there is a radical extension of \( K \) over which \( f \) splits.
\end{tdefinition}

\begin{ttheorem}[Abel-Ruffini]
  Let \( K=\C(\llist{a}{1}{n}) \) where \llist{a}{1}{n} are formal variables.
  Let \( f(x) = x^n+a_1x^{n-1}+\cdots+a_n \in K[x] \) be the generic polynomial of degree \( n\geq 5 \) over \( K \).
  Then \( f(x) \) is not solvable by radicals.
\end{ttheorem}

\end{document}