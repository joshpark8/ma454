\documentclass[a4paper]{article}

\input{preamble}
\input{letterfont}
\input{macros}

\begin{document}
\section{Composita}
\begin{tremark}
  Let \( A,B \) be sets. Then \( A\cap B \) can be expressed using only the operation \( \sseq \).
  Notice \( A\cap B \sseq A,B \) and \( A\cap B \) is the maximal set with this property: \begin{align*}
    \forall C \nf{ such that } C\sseq A,B \qimp C\sseq A\cap B.
  \end{align*}
  Let \( H_1,H_2\sgp G \).
  Then \( H_1\cap H_2 \sgp G \) is the \it{maximal} subgroup contained in both \( H_1 \tand H_2 \).
  Hence by the Galois correspondence we have \( L^{H_1\cap H_2} \) is the \it{minimal} subfield of \( L \) containing both \( L^{H_1} \) and \( L^{H_2} \).
\end{tremark}
\begin{tdefinition}[Compositum]
  Let \( K_1 \) and \( K_2 \) be fields contained in some field \( L \).
  The \it{compositum} of \( K_1 \) and \( K_2 \) in \( L \) (or the \it{composite field}), denoted by \( K_1K_2 \), is the smallest subfield of \( L \) containing both \( K_1 \) and \( K_2 \).
\end{tdefinition}

\begin{tlemma}
  Let \( K,E,F \sseq L \).
  Then \begin{enumerate}
    \item \( E:K,\ F:K \) finite \( \imp \) \( EF:K \) finite;
    \item \( E:K,\ F:K \) normal \( \imp \) \( E\cap F:K \) normal;
    \item \( E:K,\ F:K \) finite and \( E:K \) normal \( \imp \) \( EF:F \) normal;
    \item \( E:K,\ F:K \) finite and normal \( \imp \) \( EF:K,\ E\cap F:K \) normal;
    \item \( E:K,\ F:K \) normal \( \imp \) \( EF:E\cap F \) normal.
  \end{enumerate}
\end{tlemma}
\end{document}