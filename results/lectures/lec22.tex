\documentclass[a4paper]{article}

\input{preamble}
\input{letterfont}
\input{macros}

\begin{document}
\section{Composita and further comments on FTGT}
\begin{tdefinition}[Compositum]
  Let \( K_1 \) and \( K_2 \) be fields contained in some field \( L \).
  The \ul{compositum} of \( K_1 \) and \( K_2 \) in \( L \) (or the \ul{composite field}), denoted by \( K_1K_2 \), is the smallest subfield of \( L \) containing both \( K_1 \) and \( K_2 \).
\end{tdefinition}

\begin{tlemma}
  Let \( K,E,F \sseq L \).
  Then \begin{enumerate}
    \item If \( E:K\tand F:K \) both finite, then \( \fdeg{EF}{K}<\infty \);
    \item If \( E:K\tand F:K \) both normal, then \( E\cap F:K \) normal;
    \item If \( E:K\tand F:K \) both finite and \( E:K \) normal, then \( EF:F \) normal;
    \item If \( E:K\tand F:K \) both finite and normal, then \( EF:K\tand E\cap F:K \) both normal;
    \item If \( E:K\tand F:K \) both normal, then \( EF:E\cap F \) is normal.
  \end{enumerate}
\end{tlemma}

\begin{tdefinition}[Subnormal series]
  Suppose \( \chr K = 0 \), \( \sqrt[\infty]{1}\subset K \), and \( K-L \) is a radical Galois extension.
  That is, we have that \begin{align*}
    K &= K_0 - K_1 - K_2 - \cdots - K_m = L,
    \intertext{for \( K_j=K_{j-1}(r_j),r_j^{n_j}\in K_{j-1} \)}
    \Gal_KL = G &= G_0 \ogp G_1 \ogp G_2 \ogp \cdots \ogp G_m = \left\{ Id. \right\}
  \end{align*}
  By assumption \( \sqrt[\infty]{1}\subset K \imp K_j:K_{j-1} \) is a normal extension, so \( G_j\nsgp G_{j-1} \) and we have \begin{align*}
    \Gal_KL = G &= G_0 \pnogp G_1 \pnogp G_2 \pnogp \cdots \pnogp G_m = \left\{ Id. \right\}.
  \end{align*}
  This is called a \ul{subnormal series}.
\end{tdefinition}

\begin{tdefinition}[Soluble group]
  A group \( G \) is \ul{soluble} if there exists a finite series of subgroups \begin{align*}
    \left\{ Id. \right\} = G &= G_0 \sgp G_1 \sgp G_2 \sgp \cdots \sgp G_m = G
  \end{align*}
\end{tdefinition}

\end{document}