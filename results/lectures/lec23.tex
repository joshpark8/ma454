\documentclass[a4paper]{article}

\input{preamble}
\input{letterfont}
\input{macros}

\begin{document}
\section{Soluble Groups I}

\begin{tdefinition}[Soluble group]
  A group \( G \) is \it{soluble} if there exists a finite series of subgroups \begin{align*}
    \left\{ Id. \right\} &= G_n \sgp G_{n-1} \sgp \cdots \sgp G_0 = G
  \end{align*}
  such that \begin{enumerate}
    \item \( G_j \pnsgp G_{j-1} \quad \forall 1\le j\le n \) and
    \item \( \qg{G_{j-1}}{G_j} \) is cyclic \( \forall 1\le j\le n  \).
  \end{enumerate}
\end{tdefinition}

\begin{texercise}
  The Heisenberg group \( \begin{pmatrix}
    1 & a &b \\ 0 &1 &c \\ 0 & 0 & 1
  \end{pmatrix} \) is soluble.
\end{texercise}

\begin{tdefinition}[Simple group]
  A group \( G \) is \it{simple} if \( G \) has no non-triival normal subgroups.
\end{tdefinition}

\begin{tlemma}
  For \( n\geq 5 \) the group \( A_n \) is simple (and hence not soluble).
\end{tlemma}

\begin{texercise}
  \( V_4 \) is the only non-trivial subgroup of \( A_4 \).
\end{texercise}

\begin{tlemma}
  Let \( G \) be a group with \( G\nsgp G \) and \( A\sgp G \).
  Then
  \begin{enumerate}
    \item \( (A\cap H)\nsgp A \) and \( \qg{A}{(A\cap H)}\iso \qg{(HA)}{H} \)
    \item if \( H\sseq A \) and \( A\nsgp G \), then \( H\nsgp A \), \( (\qg{A}{H})\nsgp (\qg{G}{H}) \) and \( \qg{(\qg{G}{H})}{(\qg{A}{H})}\iso \qg{G}{A} \).
  \end{enumerate}
\end{tlemma}

\begin{ttheorem}
  \begin{enumerate}
    \item If \( G \) is a soluble group with \( A\sgp G \), then \( A \) is soluble.
    \item Let \( H\nsgp G \). Then \( G \) is soluble \iff~\( H\tand \qg{G}{H} \) are soluble.
  \end{enumerate}
\end{ttheorem}

\begin{tcorollary}
  \( S_n \) is not soluble for \( n\geq 5 \).
\end{tcorollary}

\end{document}