\documentclass{article}
\setlength{\headheight}{22.50113pt}
\addtolength{\topmargin}{-10.50113pt}

\input{preamble}
\input{letterfont}
\input{macros}

\fancyhead[L]{\bd{Josh Park \\ Prof. Ilya Shkredov}}
\fancyhead[C]{\bd{MA 45401-H01 -- Galois Theory Honors \\ Exam \thesection~(Prof. Trevor Wooley, 2024)}}
\fancyhead[R]{\bd{Spring 2025 \\ Page \thepage}}

\begin{document}
\setcounter{section}{2}
\bd{Problem 1.} True or False \hfill [3+3+3+3+3+3=18 points]
\begin{enumerate}[label=(\alph*)]
  \item Let \( f \in \Z[t] \) be a polynomial, every root of which has multiplicity 2024. Then \( f \) is not separable over \( \Q \).\vfill
  \item If \( L : K \) is an algebraic extension of fields with \( K \subseteq L \), then the algebraic closure \( \overline{L} \) of \( L \) is isomorphic to the algebraic closure \( \overline{K} \) of \( K \).\vfill
  \item Every algebraic extension of \( \Q \) is separable.\vfill
  \item Suppose that \( K \) and \( L \) are fields with \( K \subseteq L \), and \( L \) is algebraically closed. Then the field extension \( L : K \) is normal.\vfill
  \item Suppose that \( L : M \) and \( M : K \) are field extensions with \( L : K \) normal. Then \( L : M \) is a normal field extension.\vfill
  \item Let \( f \in \Z[x] \) be a polynomial having prime degree \( p \), and let \( \theta \) be any root of \( f \) in a splitting field extension for \( f \) over \( \Q \). Then \( [\Q(\theta) : \Q] = p \).\vspace{0.16\textheight}
\end{enumerate}
\pagebreak

\bd{Problem 2.} \hfill [3+3+3+3=12 points]
\begin{enumerate}[label=(\alph*)]
  \item Define what it means for a field extension \( L : K \) to be a splitting field extension.
    \vfill
  \item Define what it means for a field extension \( L : K \) to be normal.
  \vspace{0.5\textheight}
  \pagebreak
  \item Let \( L : K \) be a field extension. Define what it means for an element \( \alpha \in L \) to be separable over \( K \).
  \vfill
  \item Define what it means for a field extension \( L : K \) to be separable.
  \vspace{0.5\textheight}
\end{enumerate}
\pagebreak

\bd{Problem 3.} \hfill [8+8+8=24 points] \\
This question concerns the polynomial \( f(t) = t^4 - (t+1)^2 \in \Q[t] \).
\begin{enumerate}[label=(\alph*)]
  \item Find a splitting field extension \( L : \Q \) for \( f \), justifying your answer.
    \pagebreak
  \item Determine the degree of your splitting field extension \( L : \Q \), justifying your answer.
    \pagebreak
  \item Determine the subgroup of \( S_4 \) to which \( \mathrm{Gal}(L : \Q) \) is isomorphic.
\end{enumerate}
\pagebreak

\bd{Problem 4.} \hfill [14 points] \\
Suppose that \( L : K \) is a splitting field extension for the polynomial \( f \in K[t] \setminus K \). Prove that \( [L : K] \) divides \( (\deg f)! \).
\pagebreak

\bd{Problem 5.} \hfill [7+7=14 points]
\begin{enumerate}[label=(\alph*)]
  \item Suppose that \( M \) is an algebraically closed field. Show that all polynomials in \( M[t] \) are separable.
    \pagebreak
  \item Suppose that \( p \) is a prime number and \( t \) is an indeterminate, and let \( L = \F_p(t) \), where \( \F_p \) denotes the algebraic closure of \( \F_p \). Are all polynomials in \( L[X] \) separable? Justify your answer.
\end{enumerate}
\pagebreak

\bd{Problem 6.} \hfill [8+8=16 points] \\
Throughout, let \( f \) denote the polynomial \( t^5 - 9t - 3 \in \Q[t] \), let \( L \) be a splitting field for \( f \) over \( \Q \), and let \( M \) be a field with \( \Q \subsetneq M \subsetneq L \) (that is, a field strictly intermediate between \( \Q \) and \( L \)).
\begin{enumerate}[label=(\alph*)]
  \item Show that, for any \( \sigma \in \mathrm{Gal}(L : \Q) \), and for any \( \alpha \in M \), the polynomial \( \sigma(m_\alpha(\Q)) \) is monic and irreducible over \( \Q \). Here \( m_\alpha(\Q) \) denotes the minimal polynomial of \( \alpha \) over \( \Q \).
    \pagebreak
  \item Suppose that \( M : \Q \) is normal and that \( f \) factors as a product of monic irreducibles \( f_1, \dots, f_r \) (of positive degree) over \( M[t] \). Show that \( \deg(f_i) = \deg(f_1) \) for each \( i \).
    \pagebreak
  \item Show that if \( M : \Q \) is normal, then \( f \) remains irreducible over \( M \).
\end{enumerate}
\end{document}

