\documentclass{article}
\setlength{\headheight}{22.50113pt}
\addtolength{\topmargin}{-10.50113pt}

\input{preamble}
\input{letterfont}
\input{macros}

\fancyhead[L]{\bd{Josh Park \\ Prof. Shkredov}}
\fancyhead[C]{\bd{MA 45401-H01 -- Galois Theory Honors \\ Homework \thesection~(Mar 14)}}
\fancyhead[R]{\bd{Spring 2025 \\ Page \thepage}}

\begin{document}
\setcounter{section}{7} % HW NUMBER
\begin{exercise}
Let \( K = \Q \), \( M = \Q(2^{1/3}) \) and \( L = \Q(2^{1/3}, \sqrt 3, i) \).
Prove that \( L:K \) and \( L:M \) are normal but \( M:K \) is not normal.
\end{exercise}
\begin{solution}
We know that a field extension \( F_1:F_2 \) is normal iff it is a \sfe~for some \( f\in F_2[t] \).
Consider the polynomial \( f(t) = (t^2-3)(t^2+1) \).
Then, \begin{align*}
  f(t) &= (t+\sqrt 3)(t-\sqrt 3)(t+i)(t-i),
\end{align*}
whence \( L:M \) is a \sfe~for \( f \).

Next, consider \( g(t) = (t^2-3)(t^2+1)(t^3-2) \).
Then, \begin{align*}
  f(t) &= (t+\sqrt 3)(t-\sqrt 3)(t+i)(t-i)(t-\cbrt 2)(t-\veps_3\cbrt 2)(t-\veps_3^2\cbrt 2),
\end{align*}
where \( \veps_3 = \exp\lt(\frac{2\pi}{3}i\rt) \).
Notice, \begin{align*}
  \veps_3 &= \cos\lt(\frac{2\pi}{3}\rt) + i\sin\lt(\frac{2\pi}{3}\rt)    & \veps_3^2 &= \cos\lt(\frac{4\pi}{3}\rt) + i\sin\lt(\frac{4\pi}{3}\rt) \\
          &= -\frac{1}{2}+i\frac{\sqrt 3}{2} & &= -\frac{1}{2}-i\frac{\sqrt 3}{2} \\
          &= \frac{1}{2}\lt(-1+i\sqrt 3\rt)\in \Q(2^{1/3},i,\sqrt 3) & &= \frac{1}{2}\lt(-1-i\sqrt 3\rt) \in \Q(2^{1/3},i,\sqrt 3).
\end{align*}
Thus \( L:K \) is a \sfe~for \( f \), hence it is normal.

By definition, an extension \( M:K \) is normal if \( \forall \alpha\in M \), the minimum polynomial of \( \alpha \) over \( K \), \( \mu_\alpha^K(t) \), splits over \( M[t] \).
Obviously, \( \sqrt[3]2 \in \Q(2^{1/3}) \) by construction.
However, notice that for \( \alpha=\sqrt[3]2 \), \begin{align*}
  \mu_\alpha^K(t) &= t^3-2 \\
  &= (t-\cbrt 2)(t-\veps_3\cbrt 2)(t-\veps_3^2\cbrt 2),
\end{align*}
where \( \veps_3 = \exp\lt(\frac{2\pi}{3}i\rt) \).
However, we just showed that \( \veps_3 \) and \( \veps_3^2 \) are complex numbers and thus the linear factors \( (t-\veps_3\cbrt 2)\tand (t-\veps_3^2\cbrt 2) \) do not lie in \( M[t] \).
Thus \( M:K \) is not a normal extension by definiton.
\end{solution}

\stepcounter{exercise}
\begin{subexercise}
  Let \( K-L \) be algebraic, \( \alpha\in L \) and \( \sigma:K\to \bar K \) be a homomorphism.
  Prove that \( \mu_\alpha^K \) is separable over \( K \) iff \( \sigma(\mu_\alpha^K) \) is separable over \( \sigma(K) \).
\end{subexercise}
\begin{solution}
  Since we have a homomorphism from \( K\to \bar K \), we know that the extension \( \Kbar:K \) exists.
  Moreover, it is obviously algebraic by definition of \( \Kbar \).
  Thus there exists some isomorphism \( \bar\sigma:\Kbar\to\Kbar \) extending \( \sigma \), and we note that \( \bar\sigma|_{K} = \sigma \).
  Since \( K-L \) is algebraic we know that \( \mu_\alpha^K \) exists.
  Further, since all coefficients of \( \mu_\alpha^K \) are in \( K \) and \( K \sseq \Kbar \), we can say \( \mu_\alpha^K(t)\in \Kbar[t] \).
  By definition of algebraic closure, observe that we can split \( \mu_\alpha^K \) over \( \Kbar[t] \) in the following form: \begin{align*}
    \mu_\alpha^K(t) = \prod_{i=1}^{d}(t-\alpha_i)^{r_i},\quad r\in \N
  \end{align*}
  Since \( \bar\sigma|_{K} = \sigma \), we have that \( \bar\sigma(\mu_\alpha^K) = \sigma(\mu_\alpha^K) \) and \( \bar\sigma(K) = \sigma(K) \).
  We know homomorphisms preserve operations, whence
  \begin{align*}
    \bar\sigma\left(\mu_\alpha^K(t)\right) = \prod_{i=1}^{d}(t-\bar\sigma(\alpha_i))^{r_i} = \prod_{i=1}^{d}(t-\sigma(\alpha_i))^{r_i}.
  \end{align*}
  Furthermore, any field homomorphism must be injective, so each \( \bar\sigma(\alpha_i) \) is necessarily distinct.
  Hence \( \mu_\alpha^K \) has multiple roots \iff \( \bar\sigma(\mu_\alpha^K)=\sigma(\mu_\alpha^K) \) has multiple roots.
  Moreover by irreducibility of \( \mu_\alpha^K \) over \( K \), we have that \( \bar\sigma(\mu_\alpha^K) = \sigma(\mu_\alpha^K) \) is irreducible over the image of \( K \).
  Thus \( \mu_\alpha^K \) is separable over \( K \) \iff \( \sigma(\mu_\alpha^K) \) is separable over \( \sigma(K) \).
\end{solution}

\begin{subexercise} \label{ex:sfe1}
  Let \( L:K \) be a splitting field for \( f\in K[t] \).
  Prove that if \( f \) is separable, then \( L:K \) is separable.
\end{subexercise}
\begin{solution}
  We are given that \( L:K \) is a \sfe~for \( f \), and by theorem we know \( L=K(\llist{\alpha}{1}{n}) \) where \( \alpha_j\in L \) is a root of \( f \) for \( 1\leq j\leq n \).
  Then for each \( j \) the minimum polynomial of \( \alpha_j \) must divide \( f \), and thus \( \mu_{\alpha_j}^K \) is separable over \( K \) by separability of \( f \) and the definition of separable.
  Then \( \alpha_j \) is separable over \( K \) for each \( j \) and hence \( L:K \) is separable by theorem.
\end{solution}

\begin{exercise}
  Let \( L:K \) be a \sfe~for a polynomial \( f\in K[t] \).
  Then \( L:K \) is separable iff \( f \) is separable over \( K \).
\end{exercise}
\begin{solution}
  We saw in \ref{ex:sfe1} that separability of \( f \) implies separability of \( L:K \).
  Hence it is enough to show that the separability of \( L:K \) implies the separability of \( f \).
  Similarly to the previous problem, we have that \( L=K(\llist{\alpha}{1}{n}) \) where \( \alpha_j\in L \) is a root of \( f \) for \( 1\leq j\leq n \).
  By theorem, the separability of \( L:K \) implies that each \( \alpha_j \) is separable over \( K \).
  Thus by definition of separability of \( \alpha_j \), we have that \( \mu_{\alpha_j}^K \) is separable.
  Then since \( \alpha_j \) is a root of \( f \), we know \( \mu_{\alpha_j}^K \divs f \) for all \( j \).
  Assume ad absurdum that \( f \) is not separable.
  Then upon splitting over \( L \), there must be some linear factor \( (t-\alpha_k) \) raised to the power of at least 2.
  By uniqueness of \( \mu_{\alpha_k}^K \) this tells us that \( \mu_{\alpha_k}^K \) must also have a repeated root, contradicting the separability of \( \mu_{\alpha_k}^K \).
  Hence \( f \) must be separable over \( K \).
\end{solution}

\begin{exercise}
  Let \( K-M-L \) be an algebraic extension.
  Prove that \( K-L \) is separable iff \( K-M \) and \( M-L \) are separable.
\end{exercise}
\begin{solution}
  (\imp) Suppose \( K-L \) is separable.
  Then \( \alpha \) is separable (i.e. algebraic and \( \mu_\alpha^K \) separable) over \( K \) for all \( \alpha\in L \).
  Since \( M\sseq L \), we have that \( \beta \) is separable over \( K \) for all \( \beta\in M \), whence \( K-M \) is separable.
  It remains to show that \( M-L \) is separable.
  Suppose \( \gamma\in M \).
  Since \( \gamma\in L \), we have that \( \mu_\gamma^K \) is separable.
  Consider \( \mu_\gamma^M \).
  We have by lemma that \( \mu_\gamma^M\divs \mu_\gamma^K \) in \( M[t] \).
  Since \( \mu_\gamma^K \) splits into distinct linear factors, this means \( \mu_\gamma^M \) must have distinct roots as well.
  So \( \mu_\gamma^M \) is separable and thus \( \gamma \) is separable for all \( \gamma\in L \).
  Thus by definition \( L:M \) is separable.

  (\pmi) Assume that both \( K-M \) and \( M-L \) are separable.
  We wish to show that \( L \) is separable over \( K \).
  Let \( \alpha\in L \). By separability of \( L:M \), we have that \( \mu_{\alpha}^{M}(t) \) is separable.
  Since \( \alpha \) is algebraic over \( K \), its minimal polynomial over \( K \), \( \mu_{\alpha}^{K}(t)\in K[t] \), exists.
  Moreover, because \( K\subset M \), we can view \( \mu_{\alpha}^{K}(t) \) as a polynomial in \( M[t] \).
  Since \( \mu_\alpha^K \) and \( \mu_\alpha^M \) share a root, we have that \( \mu_{\alpha}^{M}(t)\mid \mu_{\alpha}^{K}(t) \) in \( M[t] \).
  That is, there exists some \( h(t)\in M[t] \) such that \( \mu_{\alpha}^{K}(t)=\mu_{\alpha}^{M}(t) h(t) \).

  Now, assume ad absurdum that \( \mu_{\alpha}^{K}(t) \) is not separable.
  Then in its factorization over an algebraic closure some linear factor appears with multiplicity \( \geq \) 2.
  That is, there exists some \( \gamma \) such that \( (t-\gamma)^n \) divides \( \mu_{\alpha}^{K}(t) \) with \( n\geq 2 \).
  We know \( \mu_{\alpha}^{M}(t) \) has distinct roots, so \( (t-\gamma) \) must be a factor of \( h(t) \) with multiplicity \( \geq 1 \).
  Notice that \begin{align}
    D(\mu_\alpha^K(t)) &= D(\mu_{\alpha}^{M}(t)) h(t)+\mu_{\alpha}^{M}(t) D(h(t)), \\
    \intertext{and if we let \( t=\gamma \),}
    D(\mu_\alpha^K(\gamma)) &= D(\mu_{\alpha}^{M}(\gamma)) h(\gamma)+\mu_{\alpha}^{M}(\gamma) D(h(\gamma)).
  \end{align}
  Since \( \gamma \) is a repeated root of \( \mu_\alpha^K \), we have that \( D(\mu_\alpha^K(\gamma))=0 \).
  Also since \( \mu_\alpha^K(\gamma) = \mu_{\alpha}^{M}(\gamma) h(\gamma) = 0 \), either \( \mu_{\alpha}^{M}(\gamma) = 0 \) or \( h(\gamma) = 0 \) must be true.
  \begin{subproof}[Case 1 (\( \mu_{\alpha}^{M}(\gamma) = 0 \)).]
    In this case, equation (2) simplifies to \( 0 = D(\mu_{\alpha}^{M}(\gamma)) h(\gamma) \).
    If \( h(\gamma) \neq 0 \) then \( \gamma \) must be a repeated root of \( \mu_\alpha^M \), contradicting its separability.
  \end{subproof}
  \begin{subproof}[Case 2 (\( h(\gamma) = 0 \)).]
    In this case, equation (2) simplifies to \( 0 = \mu_{\alpha}^{M}(\gamma) D(h(\gamma)) \).
    We know \( \mu_{\alpha}^{M}(\gamma) \neq 0 \) otherwise we return to case 1 and reach a contradiction.
    Thus \( D(h(\gamma)) = 0 \) must be true, whence \( \gamma \) is a repeated root of \( h(t) \).
  \end{subproof}
  Thus a repeated root in \( \mu_\alpha^K \) forces one of its factors with coefficients in \( M \) to have a repeated root and thus be inseparable.
  But then \( \gamma \) would become inseparable since the minimum polynomial of \( \gamma \) over \( M \) must divide \( h \), which contradicts the fact that \( L:M \) is separable.
  Thus \( \mu_\alpha^K \) must be separable over \( L \) for arbitrary \( \alpha \), whence \( L:K \) is separable.
\end{solution}

\end{document}

