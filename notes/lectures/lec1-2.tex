\documentclass[a4paper]{article}
\usepackage{asymptote}

\input{preamble.tex}
\input{letterfont.tex}
\input{macros.tex}

% \renewcommand{\arraystretch}{1.25} % space out table rows
% \setlength{\parindent}{0pt}
% \setlength{\parskip}{1em}
% \linespread{1} % 1.3 for one-and-half spacing, 1.6 for double spacing

\rhead{}

\begin{document}
\subsection*{Lecture 1} % tue jan 14
\section{Introduction}
\subsection{Polynomials}
Since ancient times, people have been interested in \tit{polynomial equations}:
\begin{align}
  x^n + a_{1}x^{n-1} + \cdots + a_{n-1}x + a_n = 0\quad (n\geq 1),\label{eq:poly}
\end{align}
where the coefficients $ a_i $ are in, say, $ \bbR $.
It was \'Evariste Galois (1811-1832) who characterized \eqref{eq:poly} that are \tit{solvable by radicals}, transforming elementary algebra to higher algebra.

The case $ n=1 $ is a trivial.
If $ n=2 $, we have get the general quadratic equation:
\begin{align*}
  ax^2+bx+c = 0\quad (a\neq 0).
\end{align*}
We can make the substitution $ x=y-\frac{b}{2a} $, which gives us
\begin{align*}
  y^2 = \frac{b^2-4ac}{4a^2} := \frac{D}{4a} \qiff y=\pm\frac{\sqrt D}{2a},
\end{align*}
hence $ x_{1,2} = \frac{-b\pm \sqrt D}{2a}$.
Here, $ D $ is the \tit{discriminant} of the polynomial $ f(x) = ax^2+bx+c $.

One can check that $ D = (x_1-x_2)\sq\cdot a\sq $.
More generally, if we have \begin{align*}
  f(x) = a_0x^n+a_{1}x^{n-1}+\cdots+a_{n-1}x+a_n,
\end{align*}
then the discriminant is given by \begin{align*}
  D = D(f) = \prod_{i<j}(x_i-x_j)^2\cdot a_0^{2n-2},
\end{align*}
where $ x_1, \ldots, x_n $ are the complex roots of $ f(x) = 0 $.

\begin{example}
  Consider the cubic equation \begin{align*}
    f(x) = ax^3+bx^2+cx+d
  \end{align*} where $ x_1, x_2, x_3 $ are solutions of $ f $. Then, \begin{align*}
    D = (x_1-x_2)^2\cdot(x_1-x_3)^2\cdot(x_2-x_3)^2 \cdot a^4
  \end{align*}
\end{example}

Why do we square? Consider the discriminant as a polynomial in $ x_1,\ldots,x_n $:
\begin{align*}
  D(x_1,\cdots,x_n) = \prod_{i<j}(x_i-x_j)^2.
\end{align*}
Then, $ D(x_1,\cdots,x_n) $ is a \tit{symmetric} polynomial, e.g. \begin{align*}
  D(x_1,x_2) = (x_1-x_2)^2 = D(x_2,x_1) = b^2-4ac.
\end{align*}

\begin{definition}[Elementary symmetric polynomials in \( x_1,\ldots, x_n \)]
  \begin{align*}
    \sigma_1 &= \sigma_1(x_1,\ldots, x_n)= x_1+\ldots+x_n \\
    \sigma_2 &= \sigma_2(x_1,\ldots, x_n)= x_1x_2+x_1x_3+\cdots+x_1x_n+x_2x_3+\cdots+x_{n-1}x_n \\
    \vdots \\
    \sigma_k &= \sigma_n(x_1,\ldots, x_n)= \sum_{i_1<\ldots<i_ k} x_{i_1}\cdots x_{i_k} \quad \nf {(\# of terms is \( \binom{n}{k} \))} \\
    \vdots \\
    \sigma_n &= \sigma_n(x_1,\ldots, x_n)= \prod_{i=1}^{n} x_i
  \end{align*}
\end{definition}

If we consider the group $ S_n $ of all permutations on $ n $ symbols, then $ \forall k, \forall w\in S_n $, \begin{align*}
  \sigma_k(x_1,\ldots,x_n) = \sigma_k(x_{w(1)},\ldots,x_{w(n)}).
\end{align*}
More generally,
\begin{definition}[Symmetric function]
  Let \( \phi(x_1, \ldots, x_n) \) be a function. Then \( \phi \) is \tit{symmetric} if \( \forall \) permutations \( \omega\in S_n \), \( \phi(x_1,\ldots, x_n) = \phi(x_{\omega(1)},\ldots,x_{\omega(n)})\).
\end{definition}

\begin{theorem} For \( \forall \) symmetric function \( \phi \) \( \exists ! \) polynomial \( P(t_1, \ldots,t_n) \) such that \( \phi(x_1,\ldots,x_n) =P(\sigma_1,\ldots,\sigma_n)\).

Moreover, if \( \phi \) is a polynomial with coefficients in a ring \( R \) (\( \phi\in R[x] \)) then \( P\in R[x] \).
\end{theorem}

\begin{example}
  Let $ n=2 $, $ \phi(x_1,x_2) = x_1\sq+x_2\sq = (x_1\sq+x_2\sq)\sq - 2x_1x_2 $.
\end{example}

\begin{theorem}[Vieta Formula]
\begin{align*}
  x^n+a_1x^{n-1}+\ldots+a_n &= (x-x_1)\cdots(x-x_n) \\
  &= x^n-\sigma_1(x_1,\ldots, x_n)x^{n-1}+\sigma_2(x_1,\ldots, x_n)x^{n-2}+\cdots\\
  &\quad+(-1)^n\sigma_n(x_1,\ldots, x_n)
\end{align*}
\end{theorem}

\begin{corollary}
  If $ f\in R[x] $ where $ R $ is a ring and $ f(x) = x^n + a_{1}x^{n-1} + \cdots + a_{n-1}x + a_n $, then $ D(f)\in R[a_1,\ldots,a_n] $. That is, the discriminant is a polynomial in $ a_1,\ldots,a_n $ with coefficients from $ R $.
\end{corollary}

\subsection{Cubic polynomials}
If \( ax^3+bx^2+cx+d = 0 \), then one solution is
\begin{align*}
  x = \sqrt[3]{-\frac{1}{2} \left(\frac{2b^3 - 9abc + 27a^2d}{27a^3}\right) + \sqrt{\left(\frac{1}{2} \left(\frac{2b^3 - 9abc + 27a^2d}{27a^3}\right)\right)^2 + \left(\frac{3ac - b^2}{9a^2}\right)^3}} \\
  \quad + \sqrt[3]{-\frac{1}{2} \left(\frac{2b^3 - 9abc + 27a^2d}{27a^3}\right) - \sqrt{\left(\frac{1}{2} \left(\frac{2b^3 - 9abc + 27a^2d}{27a^3}\right)\right)^2 + \left(\frac{3ac - b^2}{9a^2}\right)^3}},
\end{align*}
and the other two have similar expressions.
Obviously this is not practical.
Suppose we modify our polynomial:
\begin{align*}
  x^3+Ax^2+Bx+C &= \lt(\underbrace{x+\frac{A}{3}}_{y}\rt)^3+p\lt(x+\frac{A}{3}\rt) + q
\end{align*}
for some $ p,q $, so we can simply consider the equation $ x^3+px+q = 0 $

\begin{align*}
  \lt(\underbrace{a+b}_x\rt)^3 = 3ab(a+b) + a^3+b^3 \\
  x^3-3abx-a^3-b^3 = 0,\ \ x_1= a_b
\end{align*}

\begin{align*}
  x_1+x_2+x_3 = 0 &\imp x_2+x_3 = -a-b \\
  x_1x_2+x_1x_3+x_2x_3 = a^3+b^3 &\imp x_2x_3 = \frac{a^3+b^3}{x_1} = \frac{a^3+b^3}{a+b} = a^2-ab+b^2
\end{align*}

\begin{theorem}[Inverse Vieta Theorem]

\end{theorem}

\begin{example}[Root of unity]
  \( \veps \)
\end{example}

\begin{example}
  What about \( x^3+px+q=0 \)?
\end{example}

\subsection{Quadric Method}
Let \( f(x) = x^4+ax^2+bx+c = 0 \). \begin{enumerate}
  \item If \( b=0, \) it is simply a quadratic equation.
  \item If \( x^4-g^2(x) = 0 \imp x^2 = g(x), x^2 = -g(x) \)
\end{enumerate}
\begin{align*}
  f(x) &= \lt(x^2+\frac{y}{2}\rt)^2 + (a-y)x^2+bx+c - \lt(\frac{y^2}{4}\rt) \\
  D &= b^2 - 4(a-y)(c-\frac{y^2}{4}) = 0
\end{align*}
\begin{definition}[Ferrari's Resolvent]
  \( y^3-ay^2-4cy+4ac-b^2 = 0 \)
\end{definition}
\begin{align*}
  g(x) &= Ax+B \\
  0 = f(x) &= \lt(x^2+\frac{y}{2}\rt)^2 - (Ax+B)^2 \\
           &= \lt(x^2+\frac{y}{2}-Ax-B\rt)\lt(x^2+\frac{y}{2}+Ax+B\rt) \\
\end{align*}
\begin{align*}
  x_1+x_2 = A;\ x_1x_2 = \frac{y}{2}-B  \\
  x_3+x_4 = -A;\ x_3x_4 = \frac{y}{2}+B
\end{align*}
\begin{align*}
  x_1x_2+x_3x_4 &= y_1 \\
  x_1x_3+x_2x_4 &= y_2 \\
  x_1x_4+x_2x_3 &= y_3 \\
  x_1+x_2+x_3+x_4 &= 0
\end{align*}
Suppose we have some quadric equation \( f(x) = x^4+ax^2+bx+c \). Then we have unknown roots \( x_1, x_2, x_3, \) and \( x_4 \).
\begin{claim}
  \( y_1, y_2, y_3 \) are roots of a cubic equation
\end{claim}
\( y_1+y_2+y_3 = \sigma_2(x_1,x_2,x_3,x_4) = a\) \\
\( \sigma_2(y_1,y_2,y_3) = \phi(x_1,x_2,x_3,x_4) = x_1x_2 + x_3x_4 \)

\begin{example}
  Consider the polynomial \( \phi(x_1,x_2,x_3,x_4) = x_1+x_2-x_3-x_4 \)

  \begin{align*}
    \lt\{
      \begin{array}{l}
      z_1=(x_1+x_2-x_3-x_4)^2\\
      z_2=(x_1-x_2+x_3-x_4)^2\\
      z_3=(x_1-x_2-x_3+x_4)^2
      \end{array}
      \rt.
  \end{align*}
\end{example}
\end{document}