\documentclass{article}

\input{preamble}
\input{letterfont}
\input{macros}

\fancyhead[L]{\bd{Josh Park \\ Prof. Ilya Shkredov}}
\fancyhead[C]{\bd{MA 45401-H01 -- Galois Theory Honors \\ Homework \thesection~(Apr 25)}}
\fancyhead[R]{\bd{Spring 2025 \\ Page \thepage}}

\begin{document}
\setcounter{section}{11} % HW NUMBER
\begin{exercise} \label{qs:11-1} % PROBLEM 11.1
Let \( G = \qg{\Z}{p^n\Z} \), where \( p \) is a prime number.
Construct a subnormal series \( G_j \) of subgroups of \( G \) such that \( \order{\qg{G_{j-1}}{G_j}}=p \).
\end{exercise}
\begin{solution}
Obviously \( G \) is a \( p \)-group by definition.
We know all \( p \)-groups are soluble, so it has a finite subnormal series for which \( G_{j-1}/G_j \) is cyclic by definition of soluble.
Additionally, since \( G \) is abelian every subgroup is normal.
For \( 0 \le j \le n \) set
\begin{align*}
  G_j \coloneqq \qg{p^{j} \Z}{p^{n} \Z}
  \quad\subseteq\quad
  G = \qg{\Z}{p^n\Z}.
\end{align*}
Because \( p^{j} \Z \supseteq p^{n} \Z \), the quotient \( p^{j}\Z / p^{n}\Z \) is a subgroup of \( G \).
The orders of the successive terms is thus
\begin{align*}
  \order{G_j} = \order{\frac{p^{j} \Z}{p^{n} \Z}} = p^{n-j}, \qquad 0 \le j \le n,
\end{align*}
and the orders of the quotients of consecutive terms is
\begin{align*}
  \order{\frac{G_{j-1}}{G_j}} = \frac{\order{G_{j-1}}}{\order{G_j}} = \frac{p^{\,n-(j-1)}}{p^{\,n-j}} = p .
\end{align*}
Thus every quotient \( \qg{G_{j-1}}{G_j} \) is cyclic of order \( p \).
\end{solution}

\stepcounter{exercise}
\begin{subexercise}
  Let \( G \) be a group.
  Prove that \( G' \) is a normal subgroup of \( G \) such that \( \qg{G}{G'} \) is abelian.
\end{subexercise}
\begin{solution}
By definition, we have that \( G' = [G,G] = \left\{ [x,y] : x,y\in G \right\} \).
Let \( g\in G \) be arbitrary and consider any generator \( [x,y]\in G' \).
Then \begin{align*}
  g[a,b]g\inv &= ga\inv b\inv ab g\inv \\
  &= \left(ga\inv g\inv\right)\left(gb\inv g\inv\right)\left(gag\inv\right)\left(gbg\inv\right) \\
  &= \left(gag\inv\right)\inv\left(gbg\inv\right)\inv \left(gag\inv\right)\left(gbg\inv\right) \\
  &= [gag\inv, gbg\inv]\in G'.
\end{align*}
Thus \( gG' g\inv \sseq G \) and hence \( G' \nsgp G \).

Now, consider \( a G', b G' \in \qg{G}{G'} \).
Then \begin{align*}
  a G' b G' = ab G' \qand bG'aG' = baG'.
\end{align*}
Notice that \( ba[a,b] = baa\inv b\inv ab = ab \).
But \( [a,b]\in \ker \left( \qg{G}{G'} \right) \) and thus \( ba[a,b]G' = baG' = ab G' \).
\end{solution}

\begin{subexercise}
  Prove that if \( N \) is any normal subgroup of \( G \) such that \( \qg{G}{N} \) is abelian, then \( G'\sgp N \).
\end{subexercise}
\begin{solution}
  Suppose \( N \nsgp G \) such that \( \qg{G}{N} \) is abelian.
  Obviously \( \vphi:G\mapsto \qg{G}{N} \) is a well defined epimorphism.
  We know for any epimorphism from a group to a subgroup, the image of the derived group is exactly the derived subgroup.
  Thus \( \vphi(G') = \left( \qg{G}{N} \right)' \).
  By hypothesis \( \qg{G}{N} \) is abelian so \( \left( \qg{G}{N} \right) = \tgp = \vphi(G') \).
  Thus \( G' \sseq \ker\vphi = N \) and hence \( G'\sgp N \).
\end{solution}

\begin{exercise}
  Let \( \F \) be a field and \begin{align*}
    H := \left\{ \begin{pmatrix}
      1 & a & b \\
      0 & 1 & c \\
      0 & 0 & 1
    \end{pmatrix} : a,b,c\in \F \right\}
  \end{align*}
  be the Heisenberg group.
  Prove that \( H \) is soluble.
\end{exercise}
\begin{solution}
Suppose we let \begin{align*}
X(a)=\begin{pmatrix}1&a&0\\0&1&0\\0&0&1\end{pmatrix},\quad
Y(b)=\begin{pmatrix}1&0&b\\0&1&0\\0&0&1\end{pmatrix},\quad
Z(c)=\begin{pmatrix}1&0&0\\0&1&c\\0&0&1\end{pmatrix}
\;\nf{ for }(a,b,c\in \F).
\end{align*}
Note that \( [X(a),Z(c)] = Y(ac) \).
So the derived subgroup of \( Y(b) \) is
\begin{align*}
  H' = \cyc{Y(b):b\in \F} = \left\{ \begin{pmatrix}1&0&b\\0&1&0\\0&0&1\end{pmatrix}: b\in \F
\right\} = Z(H).
\end{align*}
By definition of center, \( H' \) is abelian.
For any \( A,B\in H(F) \),
\begin{align*}
  ABH' = BAH'
\end{align*}
because the factor \( AB(BA)^{-1}=[A,B] \in H' \).
Thus \( \qg{H}{H'} \) is abelian.

Next, we have that \( H'' = \left\{ I_3 \right\} = \tgp \).
Hence the derived series terminates and \( H \) is soluble by definition.
\end{solution}

\begin{exercise}
  Prove that \( A_n \), \( n\geq 3 \) is generated by 3-cycles. \label{eleven-four}
\end{exercise}
\begin{solution}
We know that \( A_n \) is generated by the set \( S \) of products of transpositions.
Let \( S=\tgp\sqcup \mcB\sqcup \mcT \) where \( \mcB \) is the set of products of transpositions, and \( \mcT \) is the set of 3-cycles.
The case for \( n=3 \) is trivial, as \( A_3 = \left\{ \tgp, (1\,2\,3), (1\,3\,2) \right\} \).
Assume \( n\geq 4 \).
Since every element of \( A_n \) can be decomposed as a product of an even number of transpositions by definition, we can pair up transpositions and it is enough to show that each pair can be written as 3-cycles.
Consider some \( \sigma\tau\in\mcB \) such that \( \sigma\neq\tau \).
The only 2 cases are if \( \sigma\tand \tau \) overlap, or if they don't.
Suppose they overlap.
Then without loss of generality \( \sigma = (a\,b) \) and \( \tau = (a\,c) \) for some \( a<b<c \) so obviously \( \sigma\tau = (a\,b\,c) \).
Now, suppose they are disjoint.
Then without loss of generality \( \sigma = (a\,b) \) and \( \tau = (c\,d) \) for distinct \( a,b,c,d \).
Notice that \( (a\,b)(c\,d) = (a\,c\,b)(a\,c\,d) \).
Hence \( A_n \) can be generated by 3-cycles for \( n\geq 4 \).
\end{solution}

\begin{exercise}
  Let \( G \) be a group.
  Find \( G' \) for \\
  \( a)\ \ G = S_3 \quad b)\ \ G = A_4 \quad c)\ \ G = S_4 \) (use the previous question).
\end{exercise}
\begin{solution}
\begin{enumerate}[label=(\alph*)]
  \item Consider the epimorphism \( \vphi:S_3\to \Ztw \), where \( \sigma\mapsto 0 \) if \( \sigma \) is even, and \( \sigma\mapsto 1 \) if \( \sigma \) is odd. \label{eleven-five-a}
  Then we have \( \ker\vphi = A_3 \).
  Since any commutator is necessarily even, we have \( S_3'\sseq A_3 \).
  Notice that any 3-cycle \( (a\,b\,c) \) can be written as a commutator \( [(a\,c),(a\,b)] \), hence \( A_3 \sseq S_3'\).
  Thus \( S_3' = A_3 \).

  \item By exercise \ref{eleven-four}, we know \( A_4 \) can be generated by 3-cycles.
   A 3-cycle \( \gamma \) satisfies the relation \( \gamma\cb = \id \), so for any 3-cycles \( \alpha,\wh\alpha\in A_4 \), we have \( [\alpha,\wh\alpha] = \alpha\sq\wh\alpha\sq\alpha\wh\alpha \).
  If we define \( \alpha = (\alpha_1\, \alpha_2\, \alpha_3) \) and \( \wh\alpha = (\alpha_1\, \alpha_3\, \alpha_4) \), and we can quickly check that \( [\alpha,\wh\alpha] = (\alpha_1\,\alpha_4)(\alpha_2\,\alpha_3) \).
  Conjugation of this permutation yields \( (\alpha_1\,\alpha_2)(\alpha_3\,\alpha_4) \) and \( (\alpha_1\,\alpha_3)(\alpha_2\,\alpha_4) \).
  These elements together form the group \( \Ztw\edp\Ztw \).
  Then \( \order{A_4/(\Ztw\edp\Ztw)} = 3 \) so it must be isomorphic to \( \Zth \) and is hence abelian.
  Thus \( A_4' = \Ztw\edp\Ztw \).

  \item Using a similar mapping as in part \ref{eleven-five-a}, we can see that \( \vphi:S_4\to \Ztw \) has \( \ker\vphi = A_4 \).
  Then by the same logic as before, the kernel bounds the derived group below since the image of the map is abelian.
  Thus \( S_4'\sseq A_4 \).
  Recall that \( A_4 \) can be generated by 3 cycles.
  Then we notice that \( (1\,2\,3) = [(1\,3),(1\,2)] \) as before, hence \( A_4\sseq S_4' \).
  Thus \( S_4'=A_4 \).

\end{enumerate}
\end{solution}
\end{document}