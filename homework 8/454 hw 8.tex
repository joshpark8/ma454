\documentclass{article}
\setlength{\headheight}{22.50113pt}
\addtolength{\topmargin}{-10.50113pt}

\input{preamble}
\input{letterfont}
\input{macros}

\fancyhead[L]{\bd{Josh Park \\ Prof. Ilya Shkredov}}
\fancyhead[C]{\bd{MA 45401-H01 -- Galois Theory Honors \\ Homework \thesection~(Apr 4)}}
\fancyhead[R]{\bd{Spring 2025 \\ Page \thepage}}

\begin{document}
\setcounter{section}{8} % HW NUMBER
\begin{exercise} % PROBLEM 8.1
  Let \( K\sseq L \) be a \sfe~for some \( f\in K[t]\setminus K \).
  Then the following are equivalent: \begin{enumerate}[label=(\roman*)]
    \item \( f \) has a repeated root over \( L \);
    \item \( \exists \alpha\in L \) s.t. \( 0=f(\alpha)=(\mcD f)(\alpha) \);
    \item \( \exists g\in K[t] \), \( \deg g \geq 1 \) s.t. \( g \) divides both \( f \) and \( \mcD f \).
  \end{enumerate}
\end{exercise}
\begin{solution}
  % Let \( f = \prod_{i=0}^{d}(t-\alpha_i)^{r_i} \) where \( \llist{\alpha}{1}{d} \) are roots of \( f \) and \( r_i\in \N \) for all \( i \).

  ((i)\( \imp \)(ii)) Suppose \( f \in K[t]\setminus K \) has a repeated root in \( L \).
  That is, \( f = \prod_{i=0}^{d}(t-\alpha_i)^{r_i} \) where \( \llist{\alpha}{0}{d} \in L \) are roots of \( f \), \( r_j = n \geq 2 \) for some \( j \), and without loss of generality we can say \( j=0 \).
  Then \( f=gh \) over \( L \) where \( g,h\in L[t]\setminus L \) of strictly smaller degree such that \( g=(t-\alpha_0)^n \) and \( h=\prod_{i=1}^{d}(t-\alpha_i)^{r_i} \), whence
  \begin{align*}
  \mcD f &= \mcD(g)h + g\mcD(h)\\
         &= n(t-\alpha_0)^{n-1}h + (t-\alpha_0)^{n}h' \\
         &= (t-\alpha_0)[n(t-\alpha_0)^{n-2}h + (t-\alpha_0)^{n-1}h'].
  \end{align*}
  Thus \( f(\alpha_0) = \mcD f(\alpha_0) = 0 \).

  ((i)\( \pmi \)(ii)) Suppose \( f \in K[t]\setminus K \) does \it{not} have repeated a root in \( L \).
  That is, \( f = \prod_{i=0}^{d}(t-\alpha_i) \) where \( \llist{\alpha}{0}{d} \in L \) are distinct roots of \( f \).
  Let \( R_f = \{\llist{\alpha}{0}{d}\} \) be the set of all roots of \( f \).
  Then it is easy to see that \begin{align*}
    \mcD f(t) = \sum_{i=1}^{d}\left(\prod_{j\neq i} (t-\alpha_j)\right) \qimp \mcD f(\alpha_k) = \prod_{j\neq k} (\alpha_k-\alpha_j) \neq 0, \quad\forall \alpha_k\in R_f
  \end{align*}
  since \( \alpha_j\neq \alpha_k \) for all \( j\neq k \), so \( \not\exists \alpha\in L \) such that \( 0=f(\alpha)=(\mcD f)(\alpha) \).

  ((ii)\( \imp \)(iii)) Suppose \(\exists\alpha\in L \tst \mcD f(\alpha)=f(\alpha)=0 \) for some \( f\in K[t]\setminus K \).
  By definition of formal derivative, we know \( \mcD f \in K[t] \).
  Moreover we are given that \( L \) is a \sfe~for \( f \), so \( L:K \) must be finite and hence algebraic.
  Thus \( \exists\mak\in K[t] \), and by theorem we have that \( \mak\divs f \) and \( \mak \divs \mcD f \).

  ((iii)\( \imp \)(ii)) Suppose \( \exists g\in K[t] \) with \( \deg g \geq 1 \) such that \( g \) divides both \( f \) and \( \mcD f \).
  We know that \( f = \prod_{i=0}^{d}(t-\alpha_i)^{r_i} \) where \( \llist{\alpha}{0}{d} \in L \) are roots of \( f \) and \( r_i \in \N \) for all \( i \).
  Thus for \( g \) to divide \( f \) it must be divisible by some factor \( (t-\alpha_j) \) of \( f \) for some \( j \).
  It follows that \( \mcD f \) must also be divisible by \( (t-\alpha_j) \), whence \( \alpha_j \) is a root of both \( \mcD f \tand f \).

  Thus we have that (i) \iff~(ii) \iff~(iii).
\end{solution}

\begin{exercise} % PROBLEM 8.2
Let \( K \) be a field, \( \char K = p > 0 \) and \( f\in K[t^p] \) is an irreducible polynomial over \( K \).
Prove that \( f \) is inseparable.
\end{exercise}
\begin{solution}
  Suppose \( f = \sum_{i=0}^{d} a_i t^{ip} \in K[t^p] \).
  It follows that \( \mcD f = \sum_{i=1}^{d} a_i pi t^{i(p-1)} = p\sum_{i=1}^{d} a_i i t^{i(p-1)} = 0 \).

\end{solution}

\begin{exercise} % PROBLEM 8.3
Let \( K \) be a field, \( \char K = p > 0 \) and \( f\in K[t^p] \) is an irreducible polynomial over \( K \).
Prove that if there is \( g\in K[t] \) and a non-negative \( n \) such that \( f(t)=g\left(t^{p^n}\right) \) and \( g \) is an irreducible and separable polynomial.
\end{exercise}
\begin{solution}

\end{solution}

\begin{exercise} % PROBLEM 8.4
  Prove that \( \prod_{\alpha\in \Fq^*} \alpha = -1 \)
\end{exercise}
\begin{solution}

\end{solution}

\stepcounter{exercise}
\begin{subexercise} % PROBLEM 8.5.1
  Let \( \alpha\in\Fq \) and \( \alpha=\beta-\beta^p \) for some \( \beta\in\Fq \).
  Prove that \( \Tr \alpha = 0 \).
\end{subexercise}
\begin{solution}

\end{solution}

\begin{subexercise} % PROBLEM 8.5.2
Let \( \alpha\in\Fq \) and \( \alpha=\gamma^{1-p} \) for some nonzero \( \gamma\in\Fq \).
Prove that \( \Norm \alpha = 1 \).
\end{subexercise}
\begin{solution}

\end{solution}

\begin{subexercise} % PROBLEM 8.5.3
  Let \( \alpha\in\Fp \sseq \FF{p^n} \).
  Prove that \( \Tr \alpha = n\alpha \).
\end{subexercise}
\begin{solution}

\end{solution}

\begin{subexercise} % PROBLEM 8.5.4
  Let \( \alpha\in\Fp \sseq \FF{p^n} \).
  Prove that \( \Norm \alpha = \alpha^n \).
\end{subexercise}
\begin{solution}

\end{solution}
\end{document}

