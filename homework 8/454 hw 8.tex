\documentclass{article}
\setlength{\headheight}{22.50113pt}
\addtolength{\topmargin}{-10.50113pt}

\input{preamble}
\input{letterfont}
\input{macros}

\fancyhead[L]{\bd{Josh Park \\ Prof. Shkredov}}
\fancyhead[C]{\bd{MA 45401-H01 -- Galois Theory Honors \\ Homework \thesection~(Apr 4)}}
\fancyhead[R]{\bd{Spring 2025 \\ Page \thepage}}

\begin{document}
\setcounter{section}{8} % HW NUMBER
\begin{exercise}
  Let \( K\sseq L \) be a \sfe~for some \( f\in K[t]\setminus K \).
  Then the following are equivalent: \begin{enumerate}[label=(\roman*)]
    \item \( f \) has a repeated root over \( L \);
    \item \( \exists \alpha\in L \) s.t. \( 0=f(\alpha)=(\mcD f)(\alpha) \);
    \item \( \exists g\in K[t] \), \( \deg g \geq 1 \) s.t. \( g \) divides both \( f \) and \( \mcD f \).
  \end{enumerate}
\end{exercise}
\begin{solution}

\end{solution}
\pagebreak

\begin{exercise}
Let \( K \) be a field, \( \char K = p > 0 \) and \( f\in K[t^p] \) is an irreducible polynomial over \( K \).
Prove that \( f \) is inseparable.
\end{exercise}
\begin{solution}

\end{solution}
\pagebreak

\begin{exercise}
Let \( K \) be a field, \( \char K = p > 0 \) and \( f\in K[t^p] \) is an irreducible polynomial over \( K \).
Prove that there is \( g\in K[t] \) and a non-negative \( n \) such that \( f(t)=g\left(t^{p^n}\right) \) and \( g \) is  an irreducible and separable polynomial.
\end{exercise}
\begin{solution}

\end{solution}
\pagebreak

\begin{exercise}
  Prove that \( \prod_{\alpha\in \Fq^*} \alpha = -1 \)
\end{exercise}
\begin{solution}

\end{solution}
\pagebreak

\stepcounter{exercise}
\begin{subexercise}
  Let \( \alpha\in\Fq \) and \( \alpha=\beta-\beta^p \) for some \( \beta\in\Fq \).
  Prove that \( \Tr \alpha = 0 \).
\end{subexercise}
\begin{solution}

\end{solution}
\pagebreak

\begin{subexercise}
Let \( \alpha\in\Fq \) and \( \alpha=\gamma^{1-p} \) for some nonzero \( \gamma\in\Fq \).
Prove that \( \Norm \alpha = 1 \).
\end{subexercise}
\begin{solution}

\end{solution}
\pagebreak

\begin{subexercise}
  Let \( \alpha\in\Fp \sseq \FF{p^n} \).
  Prove that \( \Tr \alpha = n\alpha \).
\end{subexercise}
\begin{solution}

\end{solution}
\pagebreak

\begin{subexercise}
  Let \( \alpha\in\Fp \sseq \FF{p^n} \).
  Prove that \( \Norm \alpha = \alpha^n \).
\end{subexercise}
\begin{solution}

\end{solution}
\end{document}

