\documentclass{article}

\input{preamble}
\input{letterfont}
\input{macros}

\fancyhead[L]{\bd{Josh Park \\ Prof. Ilya Shkredov}}
\fancyhead[C]{\bd{MA 45401-H01 -- Galois Theory Honors \\ Homework \thesection~(Apr 11)}}
\fancyhead[R]{\bd{Spring 2025 \\ Page \thepage}}

\begin{document}
\setcounter{section}{9} % HW NUMBER
\stepcounter{exercise}
\begin{subexercise} \label{qs:1a} % PROBLEM 9.1a
Let \( L \) be the splitting field of the polynomial \( t^{13} - 1 \). Find all subgroups of \( \Gal_{\Q}(L) \).
\end{subexercise}
\begin{solution}
Let \( f=t^{13}-1 \).
Since 13 is prime, we have that \( f = (t-1)g \) where \( g = t^{12}+\cdots+t+1 = \Phi_{13} \) is the 13th cyclotomic polynomial.
We have that \( L \) is the splitting field of \( f \), so it is also the splitting field for \( \Phi_{13} \).
We know \( \Gal_K(f) = \Gal_K(L) \) and by theorem \( \Gal_\Q(\Phi_{13}) = \Z^*_{13} = \Z_{12} \), so \( \Gal_\Q(L) \iso \Z_{12} \).
From group theory, the only subgroups of a cyclic group \( \Z_n \) are the unique subgroups \( \cyc{d} \) generated by the divisors \( d \) of \( n \), which have order \( n/d \).
Thus the subgroups of \( \Gal_\Q(L) \) are \( \Gal_\Q(L) = \Z_{12},\ \Z_6,\ \Z_4,\ \Z_3,\ \Z_2, \) and the trivial group \tgp.
\end{solution}

\begin{subexercise} \label{qs:1b} % PROBLEM 9.1b
How many intermediate subfields are there in the extension \( L : \Q \)?
\end{subexercise}
\begin{solution}
By the Fundamental Theorem of Galois Theory, there is a bijection between the set of intermediate fields of a field extension \( L:K \) and the set of subgroups of the Galois group of that same extension.
By Exercise \ref{qs:1a} we have that \( \Gal_{\Q}(L) \) has 6 subgroups, so by the Fundamental Theorem of Galois Theory have that there are 6 intermediate fields in the extension \( L:\Q \), including \( L \) and \( \Q \).
\end{solution}

\begin{exercise} \label{qs:2} % PROBLEM 9.2
Draw the lattice of subfields and corresponding lattice of subgroups of \( \Gal_{\mathbb{F}_3}(\mathbb{F}_{3^8}) \). Find orders of all subgroups of \( \Gal_{\mathbb{F}_3}(\mathbb{F}_{3^8}) \).
\end{exercise}
\begin{solution}
Since \( p=3 \) is prime and \( q=3^8 \) is of the form \( p^n \) for some \( n\in \N \), we know that \( \Gal(\FF{3} : \FF{3^8}) \iso \Z/8\Z \iso \Zei \).
Then by the same reasoning as in Exercise \ref{qs:1a}, we have that the subgroups of \( \Gal(\FF{3} : \FF{3^8}) \) (up to isomorphism) are \( \Zei,\ \Zfo,\ \Ztw, \tand \tgp \).
The orders of these subgroups are obviously 8, 4, 2, and 1 respectively.
\img{ma454 hw9 q2.png}{0.5}
\end{solution}

\begin{exercise} \label{qs:3} % PROBLEM 9.3
Prove Artin's theorem: let \( [L : K] < \infty \), \( G := \Gal_K(L) \). Then \( [L : L^G] \) is a Galois extension.
\end{exercise}
\begin{solution}
We are given that \( L:K \) is a finite extension.
By theorem, we have that \( \order{\Gal(L:K)} \leq \fdeg{L}{K} \), so \( G \) is a finite group.
% Further, by definition \( L^G \) is the field of elements of \( L \) fixed under all \( K \)-automorphisms of \( L \), so obviously \( K\sseq L^G \).
% Hence we have a tower \( K-L^G-L \) where \( L:K \) is a finite extension, so \( L:L^G \) must also be a finite extension.
So we know \( \Gal_K(L)\sgp \Aut(L) \) and \( \order{G} < \infty \), and by theorem we have that \( L:L^G \) is a finite Galois extension.
\end{solution}

\begin{exercise} \label{qs:4} % PROBLEM 9.4
Let \( L : K \) be a finite Galois extension, \( G := \Gal_K(L) \). For any \( \alpha \in L \) define
\begin{align*}
\operatorname{Tr}(\alpha) = \sum_{g \in G} g(\alpha) \qand \operatorname{Norm}(\alpha) = \prod_{g \in G} g(\alpha).
\end{align*}
Prove that for an arbitrary \( \alpha \in L \) one has \( \operatorname{Tr}(\alpha), \operatorname{Norm}(\alpha) \in K \).
\end{exercise}
\begin{solution}
Since \( L:K \) is a Galois extension, we have that \( L^G = K \).
Then \( k\in K \) iff \( h(k) = k \) for all \( h\in G \).
Notice that
\begin{align*}
  h(\Tr{\alpha}) = h\left( \sum_{g\in G}g(\alpha) \right) = \sum_{g\in G} h(g(\alpha)) = \sum_{i\in G} i(\alpha) = \Tr\alpha,
\end{align*}
and
\begin{align*}
  h(\Norm{\alpha}) = h\left( \prod_{g\in G}g(\alpha) \right) = \prod_{g\in G} h(g(\alpha)) = \prod_{i\in G} i(\alpha) = \Norm\alpha.
\end{align*}
Thus \( \Tr \alpha,\Norm\alpha\in K \) for all \( \alpha\in L \).
\end{solution}

\stepcounter{exercise}
\begin{subexercise} \label{qs:5a} % PROBLEM 9.5a
Find all of the subfields of \( \Q(2^{1/3}, e^{2\pi i / 3}) \).
\end{subexercise}
\begin{solution}
We can write \( \Q(2^{1/3}, e^{2\pi i / 3}) = \Q(2^{1/3}, \veps_3) \) where \( \veps_3 = \exp(2\pi i / 3) \).
Then, subfields of \( \Q(2^{1/3}, \veps_3) \) are \( \Q(2^{1/3}, \veps_3),\ \Q(\veps_3),\ \Q(2^{1/3}),\ \Q(2^{1/3}\veps_3),\ \Q(2^{1/3}\veps_3^2),\tand \Q \).
\end{solution}

\begin{subexercise} \label{qs:fiveb} % PROBLEM 9.5b
Draw the lattice of subfields and corresponding lattice of subgroups of \\
\( \Gal_\Q(\Q(2^{1/3}, e^{2\pi i / 3})) \).
\end{subexercise}
\begin{solution}
The field \( \Q(2^{1/3}, \veps_3) \) is a splitting field for \( t^3-2 \) over \( \Q \), so it is separable since any irreducible polynomial over a field of characteristic 0 is separable.
Also, we note that the roots of \( t^3-2 \) are \( \thrt{2}, \veps_3\thrt{2}, \tand \veps_3^2\thrt{2} \).
Consider the permutation \( \sigma: \thrt{2}\mapsto \veps_3\thrt{2} \) such that \( \sigma \) fixes \( \veps_3 \), and let \( \tau \) be complex conjugation.
Notice that \( \sigma^3 = \tau^2 = \id \) and \( \tau\sigma\tau(\thrt{2}) = \tau\sigma(\thrt{2}) = \tau(\veps_3\thrt{2}) = \veps_3^2\thrt{2} = \sigma\inv(\thrt{2}) \).
These are the defining characterisitics of the group \( D_3 \), so \( \Gal_\Q(\Q(2^{1/3}, e^{2\pi i / 3})) \iso D_3 = \cyc{r,f : r^3 = f^2 = \id, frf = r\inv}\).
The only subgroups of \( D_3 \) are \( D_3,\cyc{r}, \cyc{f}, \cyc{rf}, \cyc{r^2f}, \tand\tgp \).
\img{ma454 hw9 q5b.png}{0.8}
\end{solution}
\end{document}
