\documentclass{article}

\input{preamble}
\input{macros}
\input{letterfont}

\begin{document}
\section{Field extensions and algebraic elements}
% \subsection{Field extensions}
  \begin{proposition}
    Suppose that $L$ is a field extension of $K$ with associated embedding $\vphi: K \to L$.
    Then $L$ forms a vector space over $K$, under the operations
    \begin{align*}
      \nf{(vector addition) } \psi: L\times L &\to L \quad \text{given by} \quad (v_1,v_2) \mapsto v_1 + v_2 \\
      \nf{(scalar multiplication) } \tau : K\times  L &\to L \quad \text{given by} \quad (k,v) \mapsto \vphi(k)v.
    \end{align*}
  \end{proposition}

  \begin{theorem}[The Tower Law]
    Suppose that $M :L: K$ is a tower of field extensions.
    Then $M : K$ is a field extension, and
    \begin{align*}
      [M : K] = [M : L][L: K].
    \end{align*}
  \end{theorem}

  \begin{corollary}
    Suppose that $L:K$ is a field extension for which $[L: K]$ is a prime number.
    Then whenever $L : M : K$ is a tower of field extensions with $K \sseq M \sseq L$, one has either $M= L \tor M= K$.
  \end{corollary}

% \subsection{Algebraic elements}
  \begin{proposition}
    Suppose that $K$ and $L$ are fields and that $\vphi : K \to L$ is a \homo. With $t$ and $y$ denoting indeterminates, extend the \homo~$\vphi$ to the mapping $\psi: K[t] \to L[y]$ by defining
    \begin{align*}
      \psi(a_0 + a_1t+\cdots + a_n t^n) = \vphi(a_0) + \vphi(a_1)y+\cdots + \vphi(a_n)y^n.
    \end{align*}
    Then $\psi:K[t]\to L[y]$ is an injective \homo. Also, when $\vphi:K\to L$ is surjective, then $\psi$ is surjective and maps irreducible polynomials in $K[t]$ to irreducible polynomials in $L[y]$.
  \end{proposition}

  \begin{proposition}
    Suppose $L: K$ is a field extension with $K\sseq L$, and $\alpha\in L$. Then the evaluation map
    \begin{align*}
      E_\alpha: K[t] \to L,\quad f\mapsto f(\alpha)
    \end{align*}
    is a ring \homo.
  \end{proposition}

  \begin{proposition}
    Let $L: K$ be a field extension with $K\sseq L$, and suppose that $\alpha\in L$ is algebraic over $K$.
    Then
    \begin{align*}
      I = \ker(E_\alpha) = \{ f \in K[t] : f(\alpha) = 0 \}
    \end{align*}
    is a non-zero ideal of $K[t]$, and there is a unique monic polynomial $m_\alpha(K) \in K[t]$ that generates $I$.
  \end{proposition}

  \begin{theorem}
    Suppose that $L:K$ is a field extension, and that $\alpha\in L$ is algebraic over $K$.
    Let $g$ be the minimal polynomial $m_\alpha(K)$ of $\alpha$ over $K$.
    Then $g$ is irreducible over $K$, and $\qg{K[t]}{(g)}$ is a field.
  \end{theorem}

  \begin{theorem}
    Let $K$ be a field, and suppose that $f\in K[t]$ is irreducible.
    Then there exists a field extension $L:K$, with associated embedding $\vphi:K[t]\to L[y]$, having the property that $L$ contains a root of $\vphi(f)$.
  \end{theorem}

  \begin{proposition}
    Let $L:K$ be a field extension with $K\sseq L$.
    Let $A\sseq L$ and
    \begin{align*}
      \mcC=\{ C\sseq A : C \text{ is a finite set} \}.
    \end{align*}
    Then $K(A)=\cup_{C\in\mcC}K(C)$. Further, when $[K(C):K]<\infty$ for all $C\in \mcC$, then $K(A):K$ is an algebraic extension.
  \end{proposition}

  \begin{proposition}
    Let $L:K$ be a field extension with $K\sseq L$, and suppose that $\alpha\in L$.
    Then
    \begin{align*}
      K[\alpha] = \{ c_0 + c_1\alpha+\cdots+c_d \alpha^d : d\in \ZZ{\geq 0},\ c_0,\ldots,c_d\in K \},
    \end{align*}
    and
    \begin{align*}
      K(\alpha) = \{ f/g : f,g\in K[\alpha],\ g\neq 0 \}.
    \end{align*}
  \end{proposition}

  \begin{theorem}
    Let $L:K$ be a field extension with $K\sseq L$, and suppose that $\alpha\in L$ is algebraic over $K$.
    \begin{enumerate}[label=(\roman*)]
      \item The ring $K[\alpha]$ is a field, and $K[\alpha]=K(\alpha)$;
      \item Let $n=\deg m_\alpha(K)$. Then $\{1,\alpha,\alpha^2,\ldots,\alpha^{n-1}\}$ is a basis for $K(\alpha)$ over $K$, and hence $[K(\alpha):K]=\deg m_\alpha(K)$.
    \end{enumerate}
  \end{theorem}

  \begin{proposition}
    Let $L:K$ be a field extension with $K\sseq L$, and suppose that $\alpha\in L$.
    Then $\alpha$ is algebraic over $K$ if and only if $[K(\alpha):K]<\infty$.
  \end{proposition}

  \begin{proposition}
    Suppose that $L:K$ is a field extension with $K\sseq L$, and suppose that $\alpha\in L$ is algebraic over $K$.
    Then every element of $K(\alpha)$ is algebraic over $K$.
  \end{proposition}

  \begin{theorem}
    Let $L:K$ be a field extension with $K\sseq L$. Then the following are equivalent:
    \begin{enumerate}[label=(\roman*)]
      \item $[L:K]<\infty$;
      \item The extension $L:K$ is algebraic, and there exist $\llist{\alpha}{1}{n}\in L$ having the property that $L=K(\llist{\alpha}{1}{n})$.
    \end{enumerate}
  \end{theorem}

  \begin{proposition}
    Let $L:K$ be a field extension, and define
    \begin{align*}
      L^{\nf{alg}}=\{\alpha\in L : \alpha \text{ is algebraic over } K\}.
    \end{align*}
    Then $L^{\nf{alg}}$ is a subfield of $L$.
  \end{proposition}

\section{Review of finite fields and tests for irreducibility}
  \begin{proposition}
    Let $K$ be a field with $\char{K} > 0$. Then $\char K$ is equal to a prime number $p$, and then for all $x\in K$ one has $p\cdot x=0$.
  \end{proposition}

  \begin{theorem}
    Suppose that $\char K = p > 0$, and put $F=\{ c\cdot 1_K : c\in \Z \}$. Then $F$ is a subfield (called the prime subfield) of $K$, and $F\iso \qg{\Z}{p\Z}$.
  \end{theorem}

  \begin{theorem}
    Let $K$ be a field, and denote by $K^\times$ the abelian multiplicative group $K\setminus\{ 0 \}$. Then every finite subgroup $G$ of $K^\times$ is cyclic. In particular, if $K$ is a finite field then $K^\times$ is cyclic.
  \end{theorem}

  \begin{theorem}[Gauss' Lemma]
    Suppose that $R$ is a UFD with field of fractions $Q$. Suppose that $f$ is a primitive element of $R[X]$ with $\deg f > 0$. Then $f$ is irreducible in $R[X]$ if and only if $f$ is irreducible in $Q[X]$.
  \end{theorem}

  \begin{theorem}[Eisenstein's Criterion]
    Suppose that $R$ is a UFD, and that $f=a_0+a_1X+\ldots+a_nX^n \in R[X]$ is primitive. Then provided that there is an irreducible element $p$ of $R$ having the property that
    \begin{enumerate}[label=(\roman*)]
      \item $p\divs a_i$ for $0\leq i<n$,
      \item $p^2 \ndivs a_0$, and
      \item $p \ndivs a_n$,
    \end{enumerate}
    then $f$ is irreducible in $R[X]$, and hence also in $Q[X]$, where $Q$ is the field of fractions of $R$.
  \end{theorem}

  \begin{theorem}[Localizatoin principle]
    Let $R$ be an integral domain, and let $I$ be a prime ideal of $R$. Define $\vphi:R[X]\to (R/I)[X]$ by
    \begin{align*}
      \vphi(a_0+a_1X+\cdots+a_nX^n) =\bar a_0+\bar a_1X+\cdots+\bar a_nX^n,
    \end{align*}
    where $\bar a_j = a_j + I$. Then $\vphi$ is a surjective \homo. Moreover, if $f\in R[X]$ is primitive with leading coefficient not in $I$, then $f$ is irreducible in $R[X]$ whenever $\vphi(f)$ is irreducible in $(R/I)[X]$.
  \end{theorem}

\section{Extending field \homo s and the Galois group of an extension}
  \begin{proposition}
    Suppose that $L:K$ is a field extension with $K\sseq L$, and that $\tau:L\to L$ is a $K$-\homo. Suppose that $f\in K[t]$ has the property that $\deg f\geq 1$, and additionally that $\alpha\in L$. Then
    \begin{enumerate}[label=(\roman*)]
      \item if $f(\alpha)=0$, one has $f(\tau(\alpha))=0$;
      \item when $\tau$ is a $K$-automorphism of $L$, one has that $f(\alpha)=0$ if and only if $f(\tau(\alpha))=0$.
    \end{enumerate}
  \end{proposition}

  \begin{theorem}
    Let $\sigma:K_1\to K_2$ be a field isomorphism. Suppose that $L_i$ is a field with $K_i\sseq L_i\ (i=1,2)$. Suppose also that $\alpha\in L_1$ is algebraic over $K_1$, and that $\beta\in L_2$ is algebraic over $K_2$. Then we can extend $\sigma$ to an isomorphism $\tau:K_1(\alpha)\to K_2(\beta)$ in such a manner that $\tau(\alpha)=\beta$ if and only if $m_\beta(K_2)=\sigma(m_\alpha(K_1))$.
  \end{theorem}

  \begin{corollary}
    Let $L:M$ be a field extension with $M\sseq L$. Suppose that $\sigma:M\to L$ is a \homo, and $\alpha\in L$ is algebraic over $M$. Then the number of ways we can extend $\sigma$ to a \homo~$\tau:M(\alpha)\to L$ is equal to the number of distinct roots of $\sigma(m_\alpha(M))$ that lie in $L$.
  \end{corollary}

  \begin{theorem}
    Suppose that $L:K$ is an algebraic extension, and $\sigma:L\to L$ is a $K$-\homo. Then $\sigma$ is an automorphism of $L$.
  \end{theorem}

  \begin{theorem}
    If $L:K$ is a finite extension, then $\order{\Gal(L:K)}\leq [L:K]$.
  \end{theorem}

  \begin{corollary}
    Suppose that $L:F$ and $L:F'$ are finite extensions with $F\sseq L\tand F'\sseq L$, and further that $\psi:F\to F'$ is an isomorphism. Then there are at most $[L:F]$ ways to extend $\psi$ to a \homo~from $L$ into $L$.
  \end{corollary}

  \begin{corollary}
    Let $L:K$ be a finite extension with $K\sseq L$. Suppose that $\llist{\alpha}{1}{n}\in L$ and put $L=K(\llist{\alpha}{1}{n})$. Let $K_0=K$, and for $1\leq i\leq n$, let $K_i=K_{i-1}(\alpha_i)$. Then every automorphism $\tau\in\Gal(L:K)$ corresponds to a sequence of \homo s $\llist{\sigma}{1}{n}$, having the property that $\sigma_0:K\to L$ is the inclusion map, one has $\sigma_n=\tau$, and for $1\leq i\leq n$, the map $\sigma_i:K_i\to L$ is a \homo~extending $\sigma_{i-1}:K_{i-1}\to L$.
  \end{corollary}

\section{Algebraic closures}
  \begin{lemma}
    Let $M$ be a field. The following are equivalent:
    \begin{enumerate}[label=(\roman*)]
      \item The field $M$ is algebraically closed;
      \item every non-constant polynomial $f\in M[t]$ factors in $M[t]$ as a product of linear factors;
      \item every irreducible polynomial in $M[t]$ has degree 1;
      \item the only algebraic extension of $M$ containing $M$ is itself.
    \end{enumerate}
  \end{lemma}

  \begin{proposition}
    Any proper ideal $A$ of a commutative ring $R$ is contained in a maximal ideal.
  \end{proposition}

  \begin{lemma}
    Let $K$ be a field. Then there exists an algebraic extension $E:K$, with $K\sseq E$, having the property that $E$ contains a root of every irreducible $f\in K[t]$, and hence also every $g\in K[t]\setminus\{ 0 \}$.
  \end{lemma}

  \begin{theorem}
    Suppose that $K$ is a field. Then there exists an algebraic extension $\Kbar$ of $K$ having the property that $\Kbar$ is algebraically closed.
  \end{theorem}

  \begin{corollary}
    When $K$ is a field, the field $\Kbar$ is a maximal algebraic extension of $K$.
  \end{corollary}

  \begin{theorem}
    Let $E$ be an algebraic extension of $K$ with $K\sseq E$, and let $\Kbar$ be an algebraic closure of $K$. Given a homomorphism $\vphi:K\to \Kbar$, the map $\vphi$ can be extended to a homomorphism from $E$ into $\Kbar$.
  \end{theorem}

  \begin{corollary}
    Suppose that $\Kbar$ is an algebraic closure of $K$, and assume that $K\sseq \Kbar$. Take $\alpha\in \Kbar$ and suppose that $\sigma:K\to \Kbar$ is a homomorphism. Then the number of distinct roots of $m_\alpha(K)$ in $\Kbar$ is equal to the number of distinct roots of $\sigma(m_\alpha(K))$ in $\Kbar$.
  \end{corollary}

  \begin{proposition}
    Suppose that $L$ and $M$ are fields having the property that $L$ is algebraically closed, and $\psi : L \to M $ is a homomorphism. Then $\psi(L)$ is algebraically closed.
  \end{proposition}

  \begin{proposition}
    If $L$ and $M$ are both algebraic closures of $K$, then $L \iso M$.
  \end{proposition}

  \begin{proposition}
    If $L:K$ is an algebraic extension, then $\Lbar$ is an algebraic closure of $K$, and hence $\Lbar\iso\Kbar$. If in addition $K\sseq L\sseq \Lbar$, then we can take $\Kbar = \Lbar$.
  \end{proposition}

  \begin{proposition}
    Let $L:K$ be an extension with $K\sseq L$. Suppose that $g\in L[t]$ is irreducible over $L$, and that $g\divs f$ in $L[t]$, where $f\in K[t]\setminus \{ 0 \}$. The $g$ divides a factor of $f$ that is irreducible over $K$. Thus, there exists an irreducible $h\in K[t]$ having the property that $h\divs f$ in $K[t]$, and $g\divs h$ in $L[t]$.
  \end{proposition}

\section{Splitting field extensions}
  \begin{proposition}
    Suppose that $L:K$ is a \sfe~for the polynomial $f\in K[t]\setminus K$ with associated embedding $\vphi:K\to L$. Let $\llist{\alpha}{1}{n}\in L$ be the roots of $\vphi(f)$. Then $L=\vphi(K)(\llist{\alpha}{1}{n})$.
  \end{proposition}

  \begin{proposition}
    Suppose that $L:K$ is a \sfe~for the polynomial $f\in K[t]\setminus K$. Then $[L:K]\leq (\deg f)!$.
  \end{proposition}

  \begin{proposition}
    Given $S\sseq K[t]\setminus K$, there exists a \sfe~$L:K$ for $S$, and $L:K$ is an algebraic extension. More explicitly, suppose that $\Kbar$ is an algebraic closure of $K$, and that $\Kbar:K$ is an extension relative to the embedding $\vphi:\Kbar\to K$. Let
    \[
      A=\{ \alpha\in \Kbar : \alpha\text{ is a root of } \vphi(f) \text{ for some } f\in S \}.
    \]
    Put $K'=\vphi(K)$. Then $K'(A):K$ is a \sfe~for $S$.
  \end{proposition}

  \begin{theorem}
    Let $f\in K[t]\setminus K$, and suppose that $L:K$ and $M:K$ are \sfe s for $f$. Then $L\iso M$, and thus $[L:K]=[M:K]$.
  \end{theorem}

  \begin{theorem}
    Suppose that $S\sseq K[t]\setminus K$, and suppose that $L:K$ and $M:K$ are \sfe s for $S$. Then $L\iso M\tand [L:K]=[M:K]$.
  \end{theorem}

\section{Normal extensions and composita}
% \subsection{Normal extensions and splitting field extensions}
  \begin{proposition}
    Suppose that $L:K$ is a normal extension with $K\sseq L\sseq \Kbar$. Then for any $K$-\homo~$\tau:L\to \Kbar$, we have $\tau(L)=L$.
  \end{proposition}

  \begin{proposition}
    An extension $L:K$ is a finite, normal extension if and only if it is a \sfe~for some $f\in K[t]\setminus K$. More generally, an extension $L:K$ is normal if and only if it is a \sfe~for some $S\sseq K[t]\setminus K$.
  \end{proposition}

  \begin{proposition}
    Suppose that $L:M:K$ is a tower of field extensions and $L:K$ is a normal extension. Then $L:M$ is also a normal extension.
  \end{proposition}

% \subsection{Normal closures}
  \begin{theorem}
    Suppose that $M:L:K$ is a tower of field extensions having the property that $M:K$ is normal. Assume that $K\sseq L\sseq M$. Then the following are equivalent:
    \begin{enumerate}[label=(\roman*)]
      \item The field extension $L:K$ is normal;
      \item Any $K$-\homo~of $L$ into $M$ is an automorphism of $L$;
      \item Whenever $\sigma:M\to M$ is a $K$-automorphism, then $\sigma(L)\sseq L$.
    \end{enumerate}
  \end{theorem}

  \begin{proposition}
    Suppose that $M:K$ is a normal extension. Then:
    \begin{enumerate}[label=(\alph*)]
      \item For any $\sigma\in\Gal(M:K)$ and $\alpha\in M$, we have $m_{\sigma(\alpha)}(K)=m_\alpha(K)$;
      \item For any $\alpha,\beta\in M$ with $m_\alpha(K)=m_\beta(K)$, there exists $\tau\in\Gal(M:K)$ having the property that $\tau(\alpha)=\beta$.
    \end{enumerate}
  \end{proposition}

% \subsection{Composita of field extensions}
  \begin{proposition}
    Suppose that $E:K$ and $F:K$ are finite extensions having the property that $K,\ E\tand F$ are contained in a field $L$. Then $EF:K$ is a finite extension.
  \end{proposition}

  \begin{theorem}
    Let $E:K$ and $F:K$ be finite extensions having the property that $K,\ E\tand F$ are contained in a field $L$.
    \begin{enumerate}[label=(\alph*)]
      \item When $E:K$ is normal, then $EF:F$ is normal.
      \item When $E:K$ and $F:K$ are both normal, then $EF:K$ and $E\cap F:K$ are normal.
    \end{enumerate}
  \end{theorem}

\section{Separability}
  \begin{proposition}
    Suppose that $L:M:K$ is a tower of algebraic field extensions. Assume that $K\sseq M\sseq L\sseq \Kbar$, and suppose that $f\in K[t]\setminus K$ satisfies the property that $f$ is separable over $K$. If $g\in M[t]\setminus M$ has the property that $g\divs f$, then $g$ is separable over $M$. Thus, if $\alpha\in L$ is separable over $K$ then $\alpha$ is separable over $M$, and if $L:K$ is separable then so is $L:M$.
  \end{proposition}

  \begin{proposition}
    Suppose that $L:M$ is an algebraic field extension. Let $\alpha\in L$ and $\sigma:M\to\Mbar$ be a \homo. Then $\sigma(m_\alpha(M))$ is separable over $\sigma(M)$ if and only if $m_\alpha(M)$ is separable over $M$.
  \end{proposition}

  \begin{theorem}
    Let $L:K$ be a finite extension with $K\sseq L\sseq \Kbar$, whence $L=K(\llist{\alpha}{1}{n})$ for some $\llist{\alpha}{1}{n}\in L$. Put $K_0=K$, and for $1\leq i\leq n$, set $K_i=K_{i-1}(\alpha_i)$. Finally, let $\sigma_0:K\to\Kbar$ be the inclusion map.
    \begin{enumerate}[label=(\roman*)]
      \item If $\alpha_i$ is separable over $K_{i-1}$ for $1\leq i\leq n$, then there are $[L:K]$ ways to extend $\sigma_0$ to a \homo~$\tau:L\to\Kbar$.
      \item If $\alpha_i$ is not separable over $K_{i-1}$ for some $i$ with $1\leq i\leq n$, then there are fewer than $[L:K]$ ways to extend $\sigma_0$ to a \homo~$\tau:L\to\Kbar$.
    \end{enumerate}
  \end{theorem}

  \begin{theorem}
    Let $L:K$ be a finite extension with $L=K(\llist{\alpha}{1}{n})$. Set $K_0=K$, and for $1\leq i\leq n$, inductively define $K_i=K_{i-1}(\alpha_i)$. Then the following are equivalent:
    \begin{enumerate}[label=(\roman*)]
      \item The element $\alpha_i$ is separable over $K_{i-1}$ for $1\leq i\leq n$;
      \item The element $\alpha_i$ is separable over $K$ for $1\leq i\leq n$;
      \item The extension $L:K$ is separable.
    \end{enumerate}
  \end{theorem}

  \begin{corollary}
    Suppose that $L:K$ is a finite extension. If $L:K$ is a separable extension, then the number of $K$-\homo~$\sigma:L\to\Kbar$ is $[L:K]$, and otherwise the number is smaller than $[L:K]$.
  \end{corollary}

  \begin{corollary}
    Suppose that $f\in K[t]\setminus K$ and that $L:K$ is a \sfe~for $f$. Then $L:K$ is a separable extension if and only if $f$ is separable over $K$. More generally, suppose that $L:K$ is a \sfe~for $S\sseq K[t]\setminus K$. Then $L:K$ is a separable extension if and only if each $f\in S$ is separable over $K$.
  \end{corollary}

  \begin{theorem}
    Suppose that $L:M:K$ is a tower of algebraic extensions. Then $L:K$ is separable if and only if both $L:M$ and $M:K$ are separable.
  \end{theorem}

  \begin{theorem}
    Suppose that $E:K$ and $F:K$ are finite extensions with $E\sseq L$ and $F\sseq L$, where $L$ is a field.
    \begin{enumerate}[label=(\alph*)]
      \item When $E:K$ is separable, then so too is $EF:F$;
      \item When $E:K$ and $F:K$ are both separable, then so too are $EF:K$ and $E\cap F:K$.
    \end{enumerate}
  \end{theorem}

\section{Inseparable polynomials, differentiation, and the Frobenius map}
% \subsection{Inseparable polynomials and differentiation}
  \begin{theorem}
    Let $f\in K[t]\setminus K$, and let $L:K$ be a \sfe~for $f$. Assume that $K\sseq L$. Then the following are equivalent:
    \begin{enumerate}[label=(\roman*)]
      \item The polynomial $f$ has a repeated root over $L$;
      \item There is some $\alpha\in L$ for which $f(\alpha)=0=(\mcD f)(\alpha)$;
      \item There is some $g\in K[t]$ having the property that $\deg g\geq 1$ and $g$ divides both $f$ and $\mcD f$.
    \end{enumerate}
  \end{theorem}

  \begin{theorem}
    Suppose that $f\in K[t]$ is irreducible over $K$. Then $f$ is inseparable over $K$ if and only if $\char K=p>0$, and $f \in K[t^p]$, which is to say that $f=a_0+a_1t^p+\cdots+a_mt^{mp}$ for some $\llist{a}{0}{m}\in K$.
  \end{theorem}

  \begin{corollary}
    Suppose that $\char K = 0$. Then all polynomials in $K[t]$ are separable over $K$.
  \end{corollary}

% \subsection{The Frobenius map}
  \begin{theorem}
    Suppose that $\char K = p>0$, and let $F$ be the prime subfield of $K$. Let $\phi:K\to K$ denote the Frobenius map. Then $\phi$ is an injective homomorphism, and $\Fix_\phi(K) = F$.
  \end{theorem}

  \begin{corollary}
    Suppose that $\char K = p > 0$ and $K$ is algebraic over its prime subfield. Then the Frobenius map is an automorphism of $K$.
  \end{corollary}

  \begin{corollary}
    Suppose that $\char K = p > 0$ and $K$ is algebraic over its prime subfield. Then all polynomials in $K[t]$ are separable over $K$.
  \end{corollary}

  \begin{theorem}
    Suppose that $\char K = p > 0$. Let
    \[
      f(t)=g(t^p)=a_0+a_1t^p+\cdots+a_{n-1}t^{(n-1)p}+t^{np}
    \]
    be a non-constant monic polynomial over $K$. Then $f(t)$ is irreducible in $K[t]$ if and only if $g(t)$ is irreducible in $K[t]$ and not all the coefficients $a_i$ are $p$-th powers in $K$.
  \end{theorem}

\section{The Primitive Element Theorem}
  \begin{theorem}[The Primitive Element Theorem]
    Let $L:K$ be a finite, separable extension with $K\sseq L$. Then $L:K$ is a simple extension.
  \end{theorem}

  \begin{corollary}
    Suppose that $L:K$ is an algebraic, separable extension, and suppose that for every $\alpha\in L$, the polynomial $m_\alpha(K)$ has degree at most $n$ over $K$. Then $[L:K]\leq n$.
  \end{corollary}

\section{Fixed fields and Galois extensions}
  \begin{proposition}
    Let $K,\ M \tand L$ be fields with $K\sseq L$ and $M\sseq L$. Suppose that $G\tand H$ are subgroups of $\Aut(L)$. Then one has the following:
    \begin{enumerate}[label=(\alph*)]
      \item if $K\sseq M$, then $\Gal(L:K) \geqslant \Gal(L:M)$;
      \item if $G\sgp H$, then $\Fix_L(G)\supseteq \Fix_L(H)$;
      \item one has $K\sseq \Fix_L(\Gal(L:K))$;
      \item one has $G\sgp\Gal(L:\Fix_L(G))$;
      \item one has $\Gal(L:K) = \Gal(L:\Fix_L(\Gal(L:K)))$;
      \item one has $\Fix_L(G) = \Fix_L(\Gal(L:\Fix_L(G)))$.
    \end{enumerate}
  \end{proposition}

  \begin{theorem}
    Suppose that $L:K$ is an algebraic extension. Then $L:K$ is Galois if and only if $K=\Fix_L(\Gal(L:K))$.
  \end{theorem}

  \begin{theorem}
    Suppose that $L$ is a field and $G$ is a finite subgroup of $\Aut(L)$, and put $K=\Fix_L(G)$. Then $L:K$ is a finite Galois extension with $[L:K]=\order{\Gal(L:K)}$, and furthermore $G=\Gal(L:K)$.
  \end{theorem}

  \begin{theorem}
    Suppose that $L:K$ is a finite extension. Then, if $L:K$ is a Galois extension, one has $\order{\Gal(L:K)}=[L:K]$ and $K=\Fix_L(\Gal(L:K))$. If $L:K$ is not Galois, meanwhile, one has $\order{\Gal(L:K)}<[L:K]$ and $K$ is a proper subfield of $\Fix_L(\Gal(L:K))$.
  \end{theorem}

  \begin{proposition}
    Suppose that $L:K$ is a Galois extension, and further that $L:M:K$ is a tower of field extensions. Then $L:M$ is a Galois extension.
  \end{proposition}

\section{The main theorems of Galois theory}
% \subsection{The Fundamental Theorem}
  \begin{theorem}[The Fundamental Theorem of Galois Theory]
    Suppose that $L:K$ is a finite extension, let $G=\Gal(L:K)$, and put $K_0=\phi(G)$. Then one has the following:
    \begin{enumerate}[label=(\alph*)]
      \item the map $\phi$ is a bijection from the set of subgroups of $G$ onto the set of fields $M$ intermediate between $L$ and $K_0$, and $\gamma$ is the inverse map;
      \item if $H\sgp G$, then $H\idl G$ if and only if $\phi(H):K_0$ is a normal extension;
      \item if $H\idl G$, one has $\Gal(\phi(H):K_0)\iso \qg{G}{H}$. In particular, if $\sigma\in G$, one has $\sigma|_{\phi(H)}\in \Gal(\phi(H):K_0)$, and the map $\sigma\mapsto\sigma|_{\phi(H)}$ is a homomorphism of $G$ onto $\Gal(\phi(H):K_0)$ with kernel $H$.
    \end{enumerate}
  \end{theorem}

  \begin{definition}[Galois group of polynomial]
    When $f\in K[t]$ and $L:K$ is a \sfe~for $f$, we define the \tun{Galois group of the polynomial $f$ over $K$} to be $ \Gal_K(f) = \Gal(L:K)$.
  \end{definition}

\section{Finite fields}
  \begin{theorem}
    Let $p$ be a prime, and let $q=p^n$ for some $n\in \N$. Then:
    \begin{enumerate}[label=(\alph*)]
      \item There exists a field $\F_q$ of order $q$, and this field is unique up to isomorphism.
      \item All elements of $\F_q$ satisfy the equation $t^q=t$, and hence $\F_q:\F_p$ is a \sfe~for $t^q-t$.
      \item There is a unique copy of $\F_q$ inside any algebraically closed field containing $\F_p$.
    \end{enumerate}
  \end{theorem}

  \begin{theorem}
    Let $p$ be a prime, and suppose that $q=p^n$ for some natural number $n$. Then:
    \begin{enumerate}[label=(\alph*)]
      \item the field extension $\F_q:\F_p$ is Galois with $\Gal(\F_q:\F_p)\iso \qg{\Z}{n\Z}$;
      \item The field $\F_q$ contains a subfield of order $p^d$ if and only if $d\divs n$. When $d\divs n$, moreover, there is a unique subfield of $\F_q$ of order $p^d$.
    \end{enumerate}
  \end{theorem}

\section{Solvability by radicals: polynomials of degree 2, 3 and 4}
  \begin{theorem}
    Let $K$ be a field of characteristic 0. Then $f\in K[t]$ is solvable by radicals if and only if $\Gal_K(f)$ is soluble.
  \end{theorem}

  \begin{lemma}
    Suppose $\char K = 0$ and $L:K$ is a radical extension. Then there exists an extension $N:L$ such that $N:K$ is normal and radical.
  \end{lemma}

  \begin{lemma}
    Suppose that $\char K = 0$ and let $p$ be a prime number. Also, let $L:K$ be a \sfe~for $t^p-1$. Then $\Gal(L:K)$ is cyclic, and hence $L:K$ is a cyclic extension.
  \end{lemma}

  \begin{lemma}
    Let $\char K = 0$ and suppose that $n$ is an integer such that $t^n-1$ splits over $K$. Let $L:K$ be a \sfe~for $t^n-a$, for some $a\in K$. Then $\Gal(L:K)$ is abelian.
  \end{lemma}

  \begin{theorem}
    Let $\char K = 0$ and suppose that $L:K$ is Galois. Suppose that there is an extension $M:L$ with the property that $M:K$ is radical. Then $\Gal(L:K)$ is soluble.
  \end{theorem}

  \begin{corollary}
    Suppose that $\char K = 0$. Then $\Gal_K(f)$ is soluble whenever $f\in K[t]$ is soluble by radicals.
  \end{corollary}

  \begin{corollary}
    There exist quintic polynomials in $\Q[t]$ with insoluble Galois groups, such as $f(t) = t^5-4t+2$, and which are not solvable by radicals.
  \end{corollary}

  \begin{lemma}
    Let $\char K = 0$, and suppose that $L:K$ is a cyclic extension of degree $n$. Suppose also that $K$ contains a primitive $n$-th root of 1. Then there exists $\theta \in K$ having the property that $t^n -\theta$ is irreducible over $K$, and $L:K$ is a \sfe~for $t^n-\theta$. Further, if $\beta$ is a root of $t^n-\theta$ over $L$, then $L=K(\beta)$.
  \end{lemma}

  \begin{theorem}
    Let $\char K = 0$, and suppose that $f\in K[t]\setminus K$. Then $f$ is solvable by radicals whenever $\Gal_K(f)$ is soluble.
  \end{theorem}

\end{document}
