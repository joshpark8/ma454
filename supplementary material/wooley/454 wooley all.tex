\documentclass{article}

\input{preamble}
\input{macros}
\input{letterfont}

\title{MA 45401-H01: Galois Theory Honors\\Definitions and Results}
\author{Based on notes by Prof. Trevor Wooley\\Transcribed by Josh Park}
\date{}

\fancyhead[L]{\bd{Josh Park \\ Prof. Trevor Wooley}}
\fancyhead[C]{\bd{MA 45401-H01 -- Galois Theory Honors\\Definitions and Results}}
\fancyhead[R]{\bd{Spring 2025\\Page \thepage}}


\begin{document}
\maketitle
\tableofcontents

\section{Field extensions and algebraic elements}
\subsection{Field extensions}
  \begin{tdefinition}[Field extension]
    When \( K \tand L \) are fields, we say that \( L \) is an \ul{extension} of \( K \) if there is a \homo~\( \vphi: K \to L \).
    We then talk about the \ul{field extension}
  \( (\vphi,K,L) \).
  \end{tdefinition}

  \begin{tdefinition}[Degree, finite extension]
    Suppose that \( L: K \) is a field extension.
    We define the \ul{degree} of \( L: K \) to be the dimension of \( L \) as a vector space over K.
    We use the notation \( [L : K] \) to denote the degree of \( L : K \).
    Further, we say that \( L : K \) is a \ul{finite extension} if \( [L: K] <\infty \).
  \end{tdefinition}

  \begin{tdefinition}[Tower, intermediate field]
    We say that \( M : L : K \) is a \ul{tower} of field extensions if \( M : L \tand L: K \) are field extensions, and in this case we say that \( L \) is an \ul{intermediate field} (relative to the extension \( M : K \))
  \end{tdefinition}

  \begin{tproposition}
    Suppose that \( L \) is a field extension of \( K \) with associated embedding \( \vphi: K \to L \).
    Then \( L \) forms a vector space over \( K \), under the operations \begin{align*}
      \nf{(vector addition) } \psi: L\times L &\to L \quad \text{given by} \quad (v_1,v_2) \mapsto v_1 + v_2 \\
      \nf{(scalar multiplication) } \tau : K\times  L &\to L \quad \text{given by} \quad (k,v) \mapsto \vphi(k)v.
    \end{align*}
  \end{tproposition}

  \begin{ttheorem}[The Tower Law]
    Suppose that \( M :L: K \) is a tower of field extensions.
    Then \( M : K \) is a field extension, and \( [M : K] = [M : L][L: K] \).
  \end{ttheorem}

  \begin{tcorollary}
    Suppose that \( L:K \) is a field extension for which \( [L: K] \) is a prime number.
    Then whenever \( L : M : K \) is a tower of field extensions with \( K \sseq M \sseq L \), one has either \( M= L \tor M= K \).
  \end{tcorollary}

\subsection{Algebraic elements}
  \begin{tproposition}
    Suppose that \( K \) and \( L \) are fields and that \( \vphi : K \to L \) is a \homo.

    With \( t \) and \( y \) denoting indeterminates, extend the \homo~\( \vphi \) to the mapping \( \psi: K[t] \to L[y] \) by defining \begin{align*}
      \psi(a_0 + a_1t+\cdots + a_n t^n) = \vphi(a_0) + \vphi(a_1)y+\cdots + \vphi(a_n)y^n.
    \end{align*}
    Then \( \psi:K[t]\to L[y] \) is an injective \homo.

    Also, when \( \vphi:K\to L \) is surjective, then \( \psi: K[t]\to L[y] \) is surjective and maps irreducible polynomials in \( K[t] \) to irreducible polynomials in \( L[y] \).
  \end{tproposition}

  \begin{tdefinition}[Algebraic/transcendental element]
    Suppose that \( L: K \) is a field extension with associated embedding \( \vphi \).
    Suppose also that \( \alpha\in L \).
\begin{enumerate}[label=(\roman*)]
      \item We say that \( \alpha \) is \ul{algebraic} over K when \( \alpha \) is the root of \( \vphi(f) \) for some non-zero polynomial \( f \in K[t] \).
      \item If \( \alpha \) is not algebraic over \( K \), then we say \( \alpha \) is \ul{transcendental} over \( K \).
      \item When every element of \( L \) is algebraic over \( K \), we say that the field \( L \) is algebraic over \( K \).
    \end{enumerate}
  \end{tdefinition}

  \begin{tdefinition}[Evaluation map]
    Suppose that \( L: K \) is a field extension with \( K \sseq L \), and that \( \alpha\in L \).
    We define the \ul{evaluation map} \( E_\alpha : K[t] \to L \) by putting \( E_\alpha(f) = f(\alpha) \) for each \( f \in K[t] \).
  \end{tdefinition}

  \begin{tproposition}
    Suppose \( L: K \) is a field extension with \( K \sseq L \), and \( \alpha\in L \).
     Then \( E_\alpha \) is a ring \homo.
  \end{tproposition}

  \begin{tproposition}
    Let \( L : K \) be a field extension with \( K \sseq L \), and suppose that \( \alpha\in L \) is algebraic over \( K \).
    Then \begin{align*}
      I = ker(E_\alpha) = \{f \in K[t] : f(\alpha) = 0\}
    \end{align*}
    is a nonzero ideal of \( K[t] \), and there is a unique monic polynomial \( m_\alpha(K) \in K[t] \) that generates \( I \).
  \end{tproposition}

  \begin{tdefinition}[Minimal polynomial]
    Suppose that \( L : K \) is a field extension with \( K \sseq L \), and suppose that \( \alpha\in L \) is algebraic over \( K \).
    Then the minimal polynomial of \( \alpha \) over \( K \) is the unique monic polynomial \( m_\alpha(K) \) having the property that \( \ker(E_\alpha) = (m_\alpha(K)) \).
  \end{tdefinition}

  \begin{ttheorem}
    Suppose that \( L : K \) is a field extension, and that \( \alpha\in L \) is algebraic over \( K \).
    Let \( g \) be the minimal polynomial \( m_\alpha(K) \) of \( \alpha \) over \( K \).
    Then \( g \) is irreducible over \( K \), and \( \qg{K[t]}{(g)} \) is a field.
  \end{ttheorem}

  \begin{ttheorem}
    Let \( K \) be a field, and suppose that \( f\in K[t] \) is irreducible.
    Then there exists a field extension \( L:K \), with associated embedding \( \vphi:K[t]\to L[y] \), having the property that \( L \) contains a root of \( \vphi(f) \).
  \end{ttheorem}

  \begin{tdefinition}[Smallest subring/subfield]
    Let \( L:K \) be a field extension with \( K\sseq L \).
    \begin{enumerate}[label=(\roman*)]
      \item When \( \alpha\in L \), we denote by \( K[\alpha] \) the \ul{smallest subring of \( L \) containing \( K \) and \( \alpha \)}, and by \( K(\alpha) \) the \ul{smallest subfield of \( L \) containing \( K \) and \( \alpha \)};
      \item More generally, when \( A\sseq L \), we denote by \( K[A] \) the \ul{smallest subring of \( L \) containing \( K \tand A \)}, and by \( K(A) \) the \ul{smallest subfield of \( L \) containing \( K \tand A \)}.
    \end{enumerate}
  \end{tdefinition}

  \begin{tproposition}
    Let \( L:K \) be a field extension with \( K\sseq L \).
    Let \( A\sseq L \) and \begin{align*}
      \mcC=\lt\{ C\sseq A : C \text{ is a finite set} \rt\}.
    \end{align*}
    Then \( K(A)=\cup_{C\in\mcC}K(C) \).
    Further, when \( \fdeg{K(C)}{K}<\infty \) for all \( C\in \mcC \), then \( K(A):K \) is an algebraic extension.
  \end{tproposition}

  \begin{tproposition}
    Let \( L : K \) be a field extension with \( K \sseq L \), and suppose that \( \alpha\in L \).
    Then \begin{align*}
      K[\alpha] = \lt\{ c_0 + c_1\alpha+\cdots+c_d \alpha^d : d\in \ZZ{\leq 0},\ c_0,\ldots,c_d\in K \rt\}
    \end{align*}
    and \begin{align*}
      K(\alpha) = \lt\{ f/g : f,g\in K[\alpha], g\neq 0 \rt\}.
    \end{align*}
  \end{tproposition}

  \begin{ttheorem}
    Let \( L : K \) be a field extension with \( K \sseq L \), and suppose that \( \alpha\in L \) is algebraic over \( K \).
    \begin{enumerate}[label=(\roman*)]
      \item The ring \( K[\alpha] \) is a field, and \( K[\alpha] = K(\alpha) \);
      \item Let \( n=\deg m_\alpha(K) \).
        Then \( \lt\{ 1,\alpha,\alpha^2,\ldots,\alpha^{n-1} \rt\} \) is a basis for \( K(\alpha) \) over \( K \), and hence \( \fdeg{K(\alpha)}{K}=\deg m_\alpha(K) \).
    \end{enumerate}
  \end{ttheorem}

  \begin{tproposition}
    Let \( L : K \) be a field extension with \( K \sseq L \), and suppose that \( \alpha\in L \).
    Then \( \alpha \) is algebraic over \( K \) if and only if \( \fdeg{K(\alpha)}{K}<\infty \).
  \end{tproposition}

  \begin{tproposition}
    Suppose that \( L : K \) is a field extension with \( K \sseq L \), and suppose that \( \alpha\in L \) is algebraic over \( K \).
    Then every element of \( K(\alpha) \) is algebraic over \( K \).
  \end{tproposition}

  \begin{ttheorem}
    Let \( L:K \) be a field extension with \( K\sseq L \).
    Then the following are equivalent:
    \begin{enumerate}[label=(\roman*)]
      \item one has \( \fdeg{L}{K}<\infty \);
      \item the extension \( L:K \) is algebraic, and there exist \( \llist{\alpha}{1}{n}\in L \) having the property that \( L=K(\llist{\alpha}{1}{n}) \).
    \end{enumerate}
  \end{ttheorem}

  \begin{tproposition}
    Let \( L:K \) be a field extension, and define \begin{align*}
      L^{\nf{alg}}=\{\alpha\in L : \alpha \text{ is algebraic over } K\}.
    \end{align*}
    Then \( L^{\nf{alg}} \) is a subfield of \( L \).
  \end{tproposition}

\subsection{Review of finite fields and tests for irreducibility}
  \begin{tdefinition}[Characteristic]
    Let \( K \) be a field with additive identity \( 0_K \) and multiplicative identity \( 1_K \).
    When \( n\in \N \), we write \( n\cdot 1_K \) to denote \( 1_K+\ldots+ 1_K \) (as an \( n \)-fold sum).
    We define the \ul{characteristic} of \( K \), denoted by \( \chr K \), to be the smallest positive integer \( m \) with the property that \( m\cdot 1_K = 0_K \);
    if no such integer \( m \) exists, we define the characteristic of K to be 0.
  \end{tdefinition}

  \begin{tproposition}
    Let \( K \) be a field with \( \chr{K} > 0 \).
    Then \( \chr K \) is equal to a prime number \( p \), and then for all \( x\in K \) one has \( p\cdot x=0 \).
  \end{tproposition}

  \begin{ttheorem}
    Suppose that \( \chr K = p > 0 \), and put \( F=\lt\{ c\cdot 1_K : c\in \Z \rt\} \).
    Then \( F \) is a subfield (called the prime subfield) of \( K \), and \( F\iso \qg{\Z}{p\Z} \).
  \end{ttheorem}

  \begin{ttheorem}
    Let \( K \) be a field, and denote by \( K^\times \) the abelian multiplicative group \( K\setminus \lt\{ 0 \rt\} \).
    Then every finite subgroup \( G \) of \( K^\times \) is cyclic.
    In particular, if \( K \) is a finite field then \( K^\times \) is cyclic.
  \end{ttheorem}

  \begin{tdefinition}[Highest common factor, content, primitive]
    Let \( R \) be a UFD.
    When \( \llist{a}{0}{n}\in R \) are not all 0, we define as a \ul{highest common factor} of \llist{a}{0}{n} (written hcf(\llist{a}{0}{n})) any element \( c\in R \) satisfying \begin{enumerate}[label=(\roman*)]
      \item \( c\divs a_i\ (0\leq i\leq n) \), and
      \item whenever \( d\divs a_i\ (0\leq i\leq n) \), then \( d\divs c \).
    \end{enumerate}
    When \( f=a_0+a_1X+\ldots +a_nX^n \) is a non-zero polynomial in \( R[X] \), we define a \ul{content} of \( f \) to be any hcf(\llist{a}{0}{n}).
    We say that \( f\in R[X] \) is \ul{primitive} if \( f\neq 0 \) and the content of \( f \) is divisible only by units of \( R \).
  \end{tdefinition}

  \begin{ttheorem}[Gauss' Lemma]
    Suppose that \( R \) is a UFD with field of fractions \( Q \).
    Suppose that \( f \) is a primitive element of \( R[X] \) with \( \deg f > 0 \).
    Then \( f \) is irreducible in \( R[X] \) if and only if \( f \) is irreducible in \( Q \).
  \end{ttheorem}

  \begin{ttheorem}[Eisenstein's Criterion]
    Suppose that \( R \) is a UFD, and that \( f=a_0+a_1X+\ldots +a_nX^n \in R[X] \) is primitive.
    Then provided that there is an irreducible element \( p \) of \( R \) having the property that \begin{enumerate}[label=(\roman*)]
      \item \( p\divs a_i \) for \( 0\leq i < n \),
      \item \( p\sq \ndivs a_0 \), and
      \item \( p\ndivs a_n \),
    \end{enumerate}
    then \( f \) is irreducible in \( R[X] \), and hence also in \( Q[X] \), where \( Q \) is the field of fractions of \( R \).
  \end{ttheorem}

  \begin{ttheorem}[Localisation principle]
    Let \( R \) be an integral domain, and let \( I \) be a prime ideal of \( R \).
    Define \( \vphi:R[X]\to (\qg{R}{I})[X] \) by putting \begin{align*}
      \vphi(a_0+a_1X+\cdots+a_nX^n) =\bar a_0+\bar a_1X+\cdots+\bar a_nX^n,
    \end{align*}
    where \( \bar a_j = a_j + I \).
    Then \( \varphi \) is a surjective \homo.
    Moreover, if \( f\in R[X] \) is primitive with leading coefficient not in \( I \), then \( f \) is irreducible in \( R[X] \) whenever \( \vphi(f) \) is irreducible in \( (\qg{R}{I})[X] \).
  \end{ttheorem}

% \section{Ruler and compass constructions}
%   \subsection{Constructible points and constructible real numbers}
%   \subsection{Conditions for constructibility, and the classical problems}
%   \subsection{A bit of non-examinable fun: cord and nail constructions}

\setcounter{tdefinition}{15}
\setcounter{section}{2}
\section{Extending field \homo s and the Galois group of an extension}
  \begin{tdefinition}[Extension of field \homo, isomorphic field extensions]
    For \( i = 1 \tand 2 \), let \( L_i : K_i \) be a field extension relative to the embedding \( \vphi_i : K_i\to L_i \).
    Suppose that \( \sigma : K_1\to K_2 \) and \( \tau:L_1\to L_2 \) are isomorphisms.
    We say that \ul{\( \tau \) extends \( \sigma \)} if \( \tau\circ\vphi_1 = \vphi_2\circ\sigma \).
    In such circumstances, we say that \( L_1 : K_1 \tand L_2 : K_2 \) are \ul{isomorphic field extensions}.

    When \( \sigma:K_1\to K_2 \) and \( \tau:L_1\to L_2 \) are \homo s (instead of isomorphisms), then \ul{\( \tau \) extends \( \sigma \) as a} \ul{\homo~of fields} when the isomorphism \( \tau:L_1\to L_1' = \tau(L_1) \) extends the isomorphism \( \sigma:K_1\to K_1' = \sigma(K_1) \).
  \end{tdefinition}

  \begin{tdefinition}[\( F \)-\homo]
    Let \(L : K\) be a field extension relative to the embedding \(\varphi : K \to L\), and let \(M\) be a subfield of \(L\) containing \(\varphi(K)\).
    Then, when \(\sigma : M \to L\) is a \homo , we say that \(\sigma\) is a \ul{\(K\)-\homo} if \(\sigma\) leaves \(\varphi(K)\) pointwise fixed, which is to say that for all \(\alpha \in \varphi(K)\), one has \( \sigma(\alpha) = \alpha \).
  \end{tdefinition}

  \begin{tproposition}
    Suppose that \( L:K \) is a field extension with \( K\sseq L \), and that \( \tau:L\to L \) is a \( K \)-\homo .
    Suppose that \( f\in K[t] \) has the property that \( \deg f \geq 1 \), and additionally that \( \alpha\in L \).
    Then \begin{enumerate}[label=(\roman*)]
      \item if \( f(\alpha) = 0 \), one has \( f(\tau(\alpha)) = 0 \);
      \item when \( \tau \) is a \( K \)-automorphism of \( L \), one has that \( f(\alpha) = 0 \) if and only if \( f(\tau(\alpha)) = 0 \).
    \end{enumerate}
  \end{tproposition}

  \begin{ttheorem}
    Let \( \sigma :K_1\to K_2 \) be a field isomorphism.
    Suppose that \( L_i \) is a field with \( K_i\sseq L_i\ (i=1,2) \).
    Suppose also that \( \alpha\in L_1 \) is algebraic over \( K_1 \), and that \( \beta\in L_2 \) is algebraic over \( K_2 \).
    Then we can extend \( \sigma \) to an isomorphism \( \tau:K_1(\alpha)\to K_2(\beta) \) in such a manner that \( \tau(\alpha) = \beta \) if and only if \( m_\beta(K_2) = \sigma(m_\alpha(K_1)) \).
  \end{ttheorem}

  \bd{Note:}\quad When \( \tau:K_1(\alpha)\to K_2(\beta) \) is a \homo, and \( \tau \) extends the \homo~\( \sigma:K_1\to K_2 \), then \( \tau \) is completely determined by \( \sigma \) and the value of \( \tau(\alpha) \).

  \begin{tcorollary}
    Let \( L:M \) be a field extension with \( M\sseq L \).
    Suppose that \( \sigma:M\to L \) is a \homo, and \( \alpha\in L \) is algebraic over \( M \).
    Then the number of ways we can extend \( \sigma \) to a \homo~\( \tau:M(\alpha)\to L \) is equal to the number of distinct roots of \( \sigma(m_\alpha(M)) \) that lie in \( L \).
  \end{tcorollary}

  \begin{tdefinition}[Galois group of extension]
    Suppose that \( L:K \) is a field extension.
    With \( \Aut(L) \) denoting the automorphism group of \( L \), we set \begin{align*}
      \Gal(L:K) = \lt\{ \sigma\in \Aut(L) : \sigma \text{ is a \( K \)-\homo} \rt\}
    \end{align*}
    and we call \( \Gal(L:K) \) the \ul{Galois group} of \( L:K \).
  \end{tdefinition}

  \bd{Note:}\quad Proposition 3.1 tells us that when \( f\in K[t] \) and \( \sigma\in \Gal(L:K) \), the mapping \( \sigma \) permutes the roots of \( f \) that lie in \( L \).

  \begin{ttheorem}
    Suppose that \( L:K \) is an algebraic extension, and \( \sigma:L\to L \) is a \( K \)-\homo.
    Then \( \sigma \) is an automorphism of \( L \).
  \end{ttheorem}

  \begin{ttheorem}
    If \( L:K \) is a finite extension, then \( \order{\Gal(L:K)}\leq \fdeg{L}{K} \).
  \end{ttheorem}

  \begin{tcorollary}
    Suppose that \( L:F \) and \( L:F' \) are finite extensions with \( F\sseq L\ tand F'\sseq L \), and further that \( \psi:F\to F' \) is an isomorphism.
    Then there are at most \( \fdeg{L}{F} \) ways to extend \( \psi \) to a \homo~from \( L \) into \( L \).
  \end{tcorollary}

  \begin{tcorollary}
    Let \( L:K \) be a finite extension with \( K\sseq L \).
    Suppose that \( \llist{\alpha}{1}{n}\in L \) and put \( L=K(\llist{\alpha}{1}{n}) \).
    Let \( K_0 = K \), and for \( 1\leq i\leq n \), let \( K_i = K_{i-1}(\alpha_i) \).
    Then every automorphism \( \tau \in \Gal(L:K) \) corresponds to a sequence of \homo s \( \llist{\sigma}{1}{n} \), having the property that \( \sigma_0:K\to L \) is the inclusion map, one has \( \sigma_n=\tau \), and for \( 1\leq i\leq n \), the map \( \sigma_i : K_i\to L \) is a \homo~extending \( \sigma_{i-1}:K_{i-1}\to L \).
  \end{tcorollary}

\section{Algebraic closures}
\subsection{The definition of an algebraic closure, and Zorn's Lemma}
  \begin{tdefinition}[Algebraically closed field, algebraic closure]
    Let \( M \) be a field.
    \begin{enumerate}[label=(\roman*)]
      \item We say that \( M \) is \ul{algebraically closed} if every non-constant polynomial \( f\in M[t] \) has a root in \( M \).
      \item We say that \( M \) is an algebraic closure of \( K \) if \( M:K \) is an algebraic field extension having the property that \( M \) is algebraically closed.
    \end{enumerate}
  \end{tdefinition}

  \begin{tlemma}
    Let \( M \) be a field.
    The following are equivalent: \begin{enumerate}[label=(\roman*)]
      \item The field \( M \) is algebraically closed;
      \item every non-constant polynomial \( f\in M[t] \) factors in \( M[t] \) as a product of linear factors;
      \item every irreducible polynomial in \( M[t] \) has degree 1;
      \item the only algebraic extension of \( M \) containing \( M \) is itself.
    \end{enumerate}
  \end{tlemma}

  \begin{tdefinition}[Chain]
    Suppose that \( X \) is a nonempty, partially ordered set with \( \leq \) denoting the partial ordering.
    A \ul{chain} \( C \) in \( X \) is a collection of elements \( \lt\{ a_i \rt\}_{i\in I} \) of \( X \) having the property that for every \( i,j\in I \), either \( a_i\leq a_j \tor a_j\leq a_i \).
  \end{tdefinition}

  \bd{Zorn's Lemma:}\quad Suppose that \( X \) is a nonempty, partially ordered set with \( \leq \) the partial ordering.
  Suppose that every non-empty chain \( C \) in \( X \) has an upper bound in \( X \).
  Then \( X \) has at least one maximal element \( m \), meaning that if \( b\in X \) with \( m\leq b \), then \( b=m \).

  \begin{tproposition}
    Any proper ideal \( A \) of a commutative ring \( R \) is contained in a maximal ideal.
  \end{tproposition}

\subsection{The existence of an algebraic closure}
  \begin{tlemma}
    Let \( K \) be a field.
    Then there exists an algebraic extension \( E:K \), with \( K\sseq E \), having the property that \( E \) contains a root of every irreducible \( f\in K[t] \), and hence also every \( g\in K[t]\setminus K \).
  \end{tlemma}

  \begin{ttheorem}
    Suppose that \( K \) is a field.
    Then there exists an algebraic extension \( \Kbar \) of \( K \) having the property that \( \Kbar \) is algebraically closed.
  \end{ttheorem}

  \begin{tcorollary}
    When \( K \) is a field, the field \( \Kbar \) is a maximal algebraic extension of \( K \).
  \end{tcorollary}

\subsection{Properties of algebraic closures}
  \begin{ttheorem}
    Let \( E \) be an algebraic extension of \( K \) with \( K\sseq E \), and let \( \Kbar \) be an algebraic closure of \( K \).
    Given a homomorphism \( \vphi:K\to \Kbar \), the map \( \vphi \) can be extended to a homomorphism from \( E \) into \( \Kbar \).
  \end{ttheorem}

  \begin{tcorollary}
    Suppose that \( \Kbar \) is an algebraic closure of \( K \), and assume that \( K\sseq \Kbar \).
    Take \( \alpha\in \Kbar \) and suppose that \( \sigma:K\to \Kbar \) is a homomorphism.
    Then the number of distinct roots of \( m_\alpha(K) \) in \( \Kbar \) is equal to the number of distinct roots of \( \sigma(m_\alpha(K)) \) in \( \Kbar \).
  \end{tcorollary}

  \begin{tproposition}
    Suppose that \( L \) and \( M \) are fields having the property that \( L \) is algebraically closed, and \( \psi : L \to M \) is a homomorphism.
    Then \( \psi(L) \) is algebraically closed.
  \end{tproposition}

  \begin{tproposition}
    If \( L \) and \( M \) are both algebraic closures of \( K \), then \( L \iso M \).
  \end{tproposition}

  \begin{tproposition}
    If \( L:K \) is an algebraic extension, then \( \Lbar \) is an algebraic closure of \( K \), and hence \( \Lbar\iso\Kbar \).
    If in addition \( K\sseq L\sseq \Lbar \), then we can take \( \Kbar = \Lbar \).
  \end{tproposition}

  \begin{tproposition}
    Let \( L:K \) be an extension with \( K\sseq L \).
    Suppose that \( g\in L[t] \) is irreducible over \( L \), and that \( g\divs f \) in \( L[t] \), where \( f\in K[t]\setminus \lt\{ 0 \rt\} \).
    The \( g \) divides a factor of \( f \) that is irreducible over \( K \).
    Thus, there exists an irreducible \( h\in K[t] \) having the property that \( h\divs f \) in \( K[t] \), and \( g\divs h \) in \( L[t] \).
  \end{tproposition}

\section{Splitting field extensions}
  \begin{tdefinition}[Splitting field, \sfe]
    Suppose that \( L:K \) is a field extension relative to the embedding \( \vphi : K\to L \), and \( f\in K[t]\setminus K \).
    \begin{enumerate}[label=(\roman*)]
      \item We say that \ul{\( f \) splits over \( L \)} if \( \vphi(f)=\lambda(t-\alpha_1)\cdots(t-\alpha_n) \), for some \( \lambda\in \vphi(K) \) and \( \llist{\alpha}{1}{n}\in L \).
      \item Suppose that \( f \) splits over \( L \), and let \( M \) be a field with \( \vphi(K)\sseq M\sseq L \).
        We say that \ul{\( M:K \) is a splitting} \ul{field extension for \( f \)} if \( M \) is the smallest subfield of \( L \) containing \( \vphi(K) \) over which \( f \) splits.
      \item More generally, suppose that \( S\sseq K[t]\setminus K \) has the property that every \( f\in S \) splits over \( L \).
        Let \( M \) be a field with \( \vphi(K)\sseq M\sseq L \).
        We say that \ul{M:K is a \sfe~for \( S \)} if \( M \) is the smallest subfield of \( L \) containing \( \vphi(K) \) over which every polynomial \( f\in S \) splits.
    \end{enumerate}
  \end{tdefinition}

  \begin{tproposition}
    Suppose that \( L:K \) is a \sfe~for the polynomial \( f\in K[t]\setminus K \) with associated embedding \( \vphi:K\to L \).
    Let \( \llist{\alpha}{1}{n}\in L \) be the roots of \( \varphi(f) \).
    Then \( L=\vphi(K)(\llist{\alpha}{1}{n}) \).
  \end{tproposition}

  \begin{tproposition}
    Suppose that \( L:K \) is a \sfe~for the polynomial \( f\in K[t]\setminus K \).
    Then \( \fdeg{L}{K}\leq (\deg f)! \)
  \end{tproposition}

  \begin{tproposition}
    Given \( S\sseq K[t]\setminus K \), there exists a \sfe~\( L:K \) for \( S \), and \( L:K \) is an algebraic extension.
    More explicitly, suppose that \( \Kbar \) is an algebraic closure of \( K \), and that \( \Kbar:K \) is an extension relative to the embedding \( \vphi:\Kbar\to K \).
    Let
    \begin{align*}
      A=\lt\{ \alpha\in \Kbar : \alpha\text{ is a root of \( \vphi(f) \), for some } f\in S \rt\}.
    \end{align*}
    Put \( K'=\vphi(K) \).
    Then \( K'(A):K \) is a \sfe~for \( S \).
  \end{tproposition}

  \begin{ttheorem}
    Let \( f\in K[t]\setminus K \), and suppose that \( L:K \) and \( M:K \) are \sfe s for \( f \).
    Then \( L\iso M \), and thus \( \fdeg{L}{K}=\fdeg{M}{K} \).
  \end{ttheorem}

  \begin{ttheorem}
    Suppose that \( S\sseq K[t]\setminus K \), and suppose that \( L:K \) and \( M:K \) are \sfe s for \( S \).
    Then \( L\iso M\tand \fdeg{L}{K}=\fdeg{M}{K} \).
  \end{ttheorem}

\section{Normal extensions and composita}
\subsection{Normal extensions and splitting field extensions}
  \begin{tdefinition}[Normal extension]
    The extension \( L:K \) is \ul{normal} if it is algebraic, and every irreducible polynomial \( f\in K[t] \) either splits over \( L \) or has no root in \( L \).
  \end{tdefinition}

  \begin{tproposition}
    Suppose that \( L:K \) is a normal extension with \( K\sseq L\sseq \Kbar \).
    Then for any \( K \)-\homo~\( \tau:L\to \Kbar \), we have \( \tau(L) = L \).
  \end{tproposition}

  \begin{tproposition}
    An extension \( L:K \) is a finite, normal extension if and only if it is a \sfe~for some \( f\in K[t]\setminus K \).
    More generally, an extension \( L:K \) is normal if and only if it is a \sfe~for some \( S\sseq K[t]\setminus K \).
  \end{tproposition}

  \begin{tproposition}
    Suppose that \( L:M:K \) is a tower of field extensions and \( L:K \) is a normal extension.
    Then \( L:M \) is also a normal extension.
  \end{tproposition}

\subsection{Normal closures}
  \begin{ttheorem}
    Suppose that \( M:L:K \) is a tower of field extensions having the property that \( M:K \) is normal.
    Assume that \( K\sseq L\sseq M \).
    Then the following are equivalent: \begin{enumerate}[label=(\roman*)]
      \item the field extension \( L:K \) is normal;
      \item any \( K \)-\homo~of \( L \) into \( M \) is an automorphism of \( L \);
      \item whenever \( \sigma:M\to M \) is a \( K \)-automorphism, then \( \sigma(L)\sseq L \).
    \end{enumerate}
  \end{ttheorem}

  \begin{tproposition}
    Suppose that \( M:K \) is a normal extension.
    Then: \begin{enumerate}[label=(\alph*)]
      \item for any \( \sigma\in\Gal(M:K) \) and \( \alpha\in M \), we have \( m_{\sigma(\alpha)}(K)=m_\alpha(K) \);
      \item for any \( \alpha,\beta\in M \) with \( m_\alpha(K)=m_\beta(K) \), there exists \( \tau\in\Gal(M:K) \) having the property that \( \tau(\alpha)=\beta \).
    \end{enumerate}
  \end{tproposition}

\subsection{Composita of field extensions}
  \begin{tdefinition}[Compositum]
    Let \( K_1 \) and \( K_2 \) be fields contained in some field \( L \).
    The \ul{compositum} of \( K_1 \) and \( K_2 \) in \( L \), denoted by \( K_1K_2 \), is the smallest subfield of \( L \) containing both \( K_1 \) and \( K_2 \).
  \end{tdefinition}

  \begin{tproposition}
    Suppose that \( E:K \) and \( F:K \) are finite extensions having the property that \( K,\ E\tand F \) are contained in a field \( L \).
    Then \( EF:K \) is a finite extension.
  \end{tproposition}

  \begin{ttheorem}
    Let \( E:K \) and \( F:K \) be finite extensions having the property that \( K,\ E\tand F \) are contained in a field \( L \).
    \begin{enumerate}[label=(\alph*)]
      \item When \( E:K \) is normal, then \( EF:F \) is normal.
      \item When \( E:K \) and \( F:K \) are both normal, then \( EF:K \) and \( E\cap F:K \) are normal.
    \end{enumerate}
  \end{ttheorem}

\subsection{Normal closures (non-examinable)}

\section{Separability} % 7
% No bolded subsections appear in this section
\setcounter{tdefinition}{24}
  \begin{tdefinition}[Separable]
    Let \( K \) be a field.
    \begin{enumerate}[label=(\roman*)]
      \item An irreducible polynomial \( f\in K[t] \) is \ul{separable over \( K \)} if it has no multiple roots, meaning that \( f=\lambda(t-\alpha_1)(t-\alpha_2)\cdots(t-\alpha_d) \), where \( \llist{\alpha}{1}{d}\in \Kbar \) are distinct.
      \item A non-zero polynomial \( f\in K[t] \) is \ul{separable over \( K \)} if its irreducible factors in \( K[t] \) are separable over \( K \).
      \item When \( L:K \) is a field extension, we say that \( \alpha \in L \) is \ul{separable over \( K \)} when \( \alpha \) is algebraic over \( K \) and \( m_\alpha(K) \) is separable.
      \item An algebraic extension \( L:K \) is \ul{a separable extension} if every \( \alpha\in L \) is separable over \( K \).
    \end{enumerate}
  \end{tdefinition}

  \begin{tproposition}
    Suppose that \( L:M:K \) is a tower of algebraic field extensions.
    Assume that \( K\sseq M\sseq L\sseq \Kbar \), and suppose that \( f\in K[t]\setminus K \) satisfies the property that \( f \) is separable over \( K \).
    If \( g\in M[t]\setminus M \) has the property that \( g\divs f \), then \( g \) is separable over \( M \).
    Thus, if \( \alpha\in L \) is separable over \( K \) then \( \alpha \) is separable over \( M \), and if \( L:K \) is separable then so is \( L:M \).
  \end{tproposition}

  \begin{tproposition}
    Suppose that \( L:M \) is an algebraic field extension.
    Let \( \alpha\in L \) and \( \sigma:M\to\Mbar \) be a \homo.
    Then \( \sigma(m_\alpha(M)) \) is separable over \( \sigma(M) \) if and only if \( m_\alpha(M) \) is separable over \( M \).
  \end{tproposition}

  \begin{ttheorem}
    Let \( L:K \) be a finite extension with \( K\sseq L\sseq \Kbar \), whence \( L=K(\llist{\alpha}{1}{n}) \) for some \( \llist{\alpha}{1}{n}\in L \).
    Put \( K_0=K \), and for \( 1\leq i\leq n \), set \( K_i=K_{i-1}(\alpha_i) \).
    Finally, let \( \sigma_0:K\to\Kbar \) be the inclusion map.
    \begin{enumerate}[label=(\roman*)]
      \item If \( \alpha_i \) is separable over \( K_{i-1} \) for \( 1\leq i\leq n \), then there are \( \fdeg{L}{K} \) ways to extend \( \sigma_0 \) to a \homo~\( \tau:L\to \Kbar \).
      \item If \( \alpha_i \) is not separable over \( K_{i-1} \) for some \( i \) with \( 1\leq i\leq n \), then there are fewer than \( \fdeg{L}{K} \) ways to extend \( \sigma_0 \) to a \homo~\( \tau:L\to \Kbar \).
    \end{enumerate}
  \end{ttheorem}

  \begin{ttheorem}
    Let \( L:K \) be a finite extension with \( L=K(\llist{\alpha}{1}{n}) \).
    Set \( K_0=K \), and for \( 1\leq i\leq n \), inductively define \( K_i \) by putting \( K_i=K_{i-1}(\alpha_i) \).
    Then the following are equivalent: \begin{enumerate}[label=(\roman*)]
      \item the element \( \alpha_i \) is separable over \( K_{i-1} \) for \( 1\leq i\leq n \);
      \item the element \( \alpha_i \) is separable over \( K \) for \( 1\leq i\leq n \);
      \item the extension \( L:K \) is separable.
    \end{enumerate}
  \end{ttheorem}

  \begin{tcorollary}
    Suppose that \( L:K \) is a finite extension.
    If \( L:K \) is a separable extension, then the number of \( K \)-\homo~\( \sigma:L\to\Kbar \) is \( \fdeg{L}{K} \), and otherwise the number is smaller than \( \fdeg{L}{K} \).
  \end{tcorollary}

  \begin{tcorollary}
    Suppose that \( f\in K[t]\setminus K \) and that \( L:K \) is a \sfe~ for \( f \).
    Then \( L:K \) is a separable extension if and only if \( f \) is separable over \( K \).
    More generally, suppose that \( L:K \) is a \sfe~ for \( S\sseq K[t]\setminus K \).
    Then \( L:K \) is a separable extension if and only if each \( f\in S \) is separable over \( K \).
  \end{tcorollary}

  \begin{ttheorem}
    Suppose that \( L:M:K \) is a tower of algebraic extensions.
    Then \( L:K \) is separable if and only if \( L:M \) and \( M:K \) are both separable.
  \end{ttheorem}

  \begin{ttheorem}
    Suppose tht \( E:K \) and \( F:K \) are finite extensions with \( E\sseq L \) and \( F\sseq L \), where \( L \) is a field.
    \begin{enumerate}[label=(\alph*)]
      \item When \( E:K \) is separable, then so too is \( EF:F \);
      \item When \( E:K \) and \( F:K \) are both separable, then so too are \( EF:K \) and \( E\cap F:K \).
    \end{enumerate}
  \end{ttheorem}

\section{Inseparable polynomials, differentiation, and the Frobenius map}
\subsection{Inseparable polynomials and differentiation}
  \begin{tdefinition}[Inseparable]
    A polynomial \( f \in K[t] \) is \ul{inseparable over \( K \)} if \( f \) is not separable over \( K \), meaning that \( f \) has an irreducible factor \( g \in K[t] \) having the property that \( g \) has fewer than \( \deg g \) distinct roots in \( K \).
  \end{tdefinition}

  \begin{tdefinition}[Formal derivative]
    We define the \ul{derivative operator} \( \mcD:K[t]\to K[t] \) by \begin{align*}
      \mcD\lt(\sum_{k=0}^{n}a_kt^k\rt) = \sum_{k=1}^{n}ka_kt^{k-1}.
    \end{align*}
  \end{tdefinition}

  \begin{ttheorem}
    Let \( f\in K[t]\setminus K \), and let \( L:K \) be a \sfe~for \( f \).
    Assume that \( K\sseq L \).
    Then the following are equivalent: \begin{enumerate}[label=(\roman*)]
      \item The polynomial \( f \) has a repeated root over \( L \);
      \item There is some \( \alpha\in L \) for which \( f(\alpha)=0=(\mcD f)(\alpha) \);
      \item There is some \( g\in K[t] \) having the property that \( \deg g \geq 1 \) and \( g \) divides both \( f \) and \( \mcD f \).
    \end{enumerate}
  \end{ttheorem}

  \begin{ttheorem}
    Suppose that \( f\in K[t] \) is irreducible over \( K \).
    Then \( f \) is inseparable over \( K \) if and only if \( \chr K=p>0 \), and \( f \in K[t^p] \), which is to say that \( f=a_0+a_1t^p+\cdots+a_mt^{mp} \), for some \( \llist{a}{0}{m}\in K \).
  \end{ttheorem}

  \begin{tcorollary}
    Suppose that \( \chr K = 0 \).
    Then all polynomials in \( K[t] \) are separable over \( K \).
  \end{tcorollary}

\subsection{The Frobenius map}
  \begin{tdefinition}[Frobenius map]
    Suppose that \( \chr K = p > 0 \).
    The \ul{Frobenius map} \( \phi:K\to K \) is defined by \( \phi(\alpha)=\alpha^p \).
  \end{tdefinition}

  \bd{Note: } \( \Fix{\phi}{K}=\lt\{ \alpha\in K:\phi(\alpha)=\alpha \rt\} \).

  \begin{ttheorem}
    Suppose that \( \chr K = p>0 \), and let \( F \) be the prime subfield of \( K \).
    Let \( \phi:K\to K \) denote the Frobenius map.
    Then \( \phi \) is an injective homomorphism, and \( \Fix{\phi}{K} = F \).
  \end{ttheorem}

  \begin{tcorollary}
    Suppose that \( \chr K = p > 0 \) and \( K \) is algebraic over its prime subfield.
    Then the Frobenius map is an automorphism of \( K \).
  \end{tcorollary}

  \begin{tcorollary}
    Suppose that \( \chr K = p > 0 \) and \( K \) is algebraic over its prime subfield.
    Then all polynomials in \( K[t] \) are separable over \( K \).
  \end{tcorollary}

  \begin{ttheorem}
    Suppose that \( \chr K = p > 0 \).
    Let \begin{align*}
      f(t) = g(t^p) = a_0+a_1t^p+\cdots+a_{n-1}t^{(n-1)p}+t^{np}
    \end{align*}
    be a non-constant monic polynomial over \( K \).
    Then \( f(t) \) is irreducible in \( K[t] \) if and only if \( g(t) \) is irreducible in \( K[t] \) and not all the coefficients \( a_i \) are \( p \)-th powers in \( K \).
  \end{ttheorem}

\section{The Primitive Element Theorem}
  \begin{tdefinition}[Simple extension]
    Suppose \( L:K \) is a field extension relative to the embedding \( \varphi:K\to L \).
    We say that \( L:K \) is a \ul{simple extension} if there is some \( \gamma\in L \) having the property that \( L=\vphi(K)(\gamma) \).
  \end{tdefinition}

  \begin{ttheorem}[The Primitive Element Theorem]
    Let \( L:K \) be a finite, separable extension with \( K\sseq L \).
    Then \( L:K \) is a simple extension.
  \end{ttheorem}

  \begin{tcorollary}
    Suppose that \( L:K \) is an algebraic, separable extension, and suppose that for every \( \alpha\in L \), the polynomial \( m_\alpha(K) \) has degree at most \( n \) over \( K \).
    Then \( \fdeg{L}{K}\leq n \).
  \end{tcorollary}

\section{Fixed fields and Galois extensions}
  \begin{tdefinition}[Fixed field]
    Let \( L:K \) be a field extension.
    When \( G \) is a subgroup of \( \Aut(L) \), we define the fixed field of \( G \) to be \begin{align*}
      \Fix{L}{G} = \lt\{ \alpha\in L : \sigma(\alpha) = \alpha \text{ for all } \sigma\in G \rt\}.
    \end{align*}
  \end{tdefinition}

  \begin{tproposition}
    Let \( K,\ M \tand L \) be fields with \( K\sseq L \) and \( M\sseq L \).
    Suppose that \( G\tand H \) are subgroups of \( \Aut(L) \).
    Then one has the following: \begin{enumerate}[label=(\alph*)]
      \item if \( K\sseq M \), then \( \Gal(L:K) \geqslant \Gal(L:M) \);
      \item if \( G\sgp H \), then \( \Fix{L}{G}\supeq \Fix{L}{H} \);
      \item one has \( K\sseq \Fix{L}{\Gal(L:K)} \);
      \item one has \( G\sgp\Gal(L:\Fix{L}{G}) \);
      \item one has \( \Gal(L:K) = \Gal(L:\Fix{L}{\Gal(L:K)}) \);
      \item one has \( \Fix{L}{G} = \Fix{L}{\Gal(L:\Fix{L}{G})} \).
    \end{enumerate}
  \end{tproposition}

  \begin{tdefinition}[Galois extension]
    When \( L:K \) is a field extension, we say that \( L:K \) is a \ul{Galois extension} if it is an extension that is normal and separable.
  \end{tdefinition}

  \begin{ttheorem}
    Suppose that \( L:K \) is an algebraic extension.
    Then \( L:K \) is Galois if and only if \( K=\Fix{L}{\Gal(L:K)} \).
  \end{ttheorem}

  \begin{ttheorem}
    Suppose that \( L \) is a field and \( G \) is a finite subgroup of \( \Aut(L) \), and put \( K=\Fix{L}{G} \).
    Then \( L:K \) is a finite Galois extension with \( \fdeg{L}{K}=\order{\Gal(L:K)} \), and furthermore \( G=\Gal(L:K) \).
  \end{ttheorem}

  \begin{ttheorem}
    Suppose that \( L:K \) is a finite extension.
    Then, if \( L:K \) is a Galois extension, one has \( \order{\Gal(L:K)}=\fdeg{L}{K} \) and \( K=\Fix{L}{\Gal(L:K)}\).
    If \( L:K \) is not Galois, meanwhile, one has \( \order{\Gal(L:K)}<\fdeg{L}{K} \) and \( K \) is a proper subfield of \( \Fix{L}{\Gal(L:K)}\).
  \end{ttheorem}

  \begin{tproposition}
    Suppose that \( L:K \) is a Galois extension, and further that \( L:M:K \) is a tower of field extensions.
    Then \( L:M \) is a Galois extension.
  \end{tproposition}

\section{The main theorems of Galois theory}
\subsection{The Fundamental Theorem}
  \begin{tdefinition}
    Suppose that \( L:K \) is a field extension.
    When \( G \) is a subgroup of \( \Aut(L) \), we write \( \phi(G) \) for \( \Fix{L}{G} \), and when \( L:M:K_0 \) is a tower of field extensions with \( K_0=\phi(\Gal(L:K)) \), we write \( \gamma(M) \) for \( \Gal(L:M) \).
  \end{tdefinition}

  \begin{ttheorem}[The Fundamental Theorem of Galois Theory]
    Suppose that \( L:K \) is a finite extension, let \( G=\Gal(L:K) \), and put \( K_0=\phi(G) \).
    Then one has the following: \begin{enumerate}[label=(\alph*)]
      \item the map \( \phi \) is a bijection from the set of subgroups of \( G \) onto the set of fields \( M \) intermediate between \( L \) and \( K_0 \), and \( \gamma \) is the inverse map;
      \item if \( H \sgp G \), then \( H\idl G \) if and only if \( \phi(H):K_0 \) is a normal extension;
      \item if \( H\idl G \), one has \( \Gal(\phi(H):K_0) \iso \qg{G}{H} \).
        In particular, if \( \sigma\in G \), one has \( \sigma|_{\phi(H)} \in \Gal(\phi(H):K_0) \), and the map \( \sigma\mapsto\sigma|_{\phi(H)} \) is a homomorphism of \( G \) onto \( \Gal(\phi(H):K_0) \) with kernel \( H \).
    \end{enumerate}
  \end{ttheorem}

  \begin{tdefinition}[Galois group of polynomial]
    When \( f\in K[t] \) and \( L:K \) is a \sfe~for \( f \), we define the \ul{Galois group of the polynomial \( f \) over \( K \)} to be \( \Gal_K(f) = \Gal(L:K) \).
  \end{tdefinition}

\subsection{Non-examinable: consequences for composita and intersections}

\section{Finite fields}
  \begin{ttheorem}
    Let \( p \) be a prime, and let \( q=p^n \) for some \( n\in \N \).
    Then: \begin{enumerate}[label=(\alph*)]
      \item There exists a field \( \F_q \) of order \( q \), and this field is unique up to isomorphism.
      \item All elements of \( \F_q \) satisfy the equation \( t^q=t \), and hence \( \F_q:\F_p \) is a \sfe~for \( t^q-t \).
      \item There is a unique copy of \( \F_q \) inside any algebraically closed field containing \( \F_p \).
    \end{enumerate}
  \end{ttheorem}

  \begin{ttheorem}
    Let \( p \) be a prime, and suppose that \( q=p^n \) for some natural number \( n \).
    Then: \begin{enumerate}[label=(\alph*)]
      \item the field extension \( \F_q:\F_p \) is Galois with \( \Gal(\F_q:\F_p)\iso \qg{\Z}{n\Z} \);
      \item The field \( \F_q \) contains a subfield of order \( p^d \) if and only if \( d\divs n \).
        When \( d\divs n \), moreover, there is a unique subfield of \( \F_q \) of order \( p^d \).
    \end{enumerate}
  \end{ttheorem}

\section{Solvability by radicals: polynomials of degree 2, 3 and 4}
\subsection{Finding roots of quadratic, cubic, and quartic polynomials}
  \begin{tdefinition}[Radical element/extension]
    Suppose that \( L:K \) is a field extension, and \( \beta\in L \).
    We say that \( \beta \) is \ul{radical} over \( K \) when \( \beta^n\in K \) for some \( n\in \N \) (so \( \beta=\alpha^{1/n} \) for some \( \alpha\in K \) and some \( n\in\N \)).
    We say that \( L:K \) is \ul{an extension by radicals} when there is a tower of field extensions \( L=L_r:L_{r-1}:\cdots:L_0=K \) such that \( L_i=L_{i-1}(\beta_i) \) with \( \beta_i \) radical over \( L_{i-1} \) (\( 1\leq i\leq r \)).
    We say \( f\in K[t] \) is \ul{solvable by radicals} if there is a radical extension of \( K \) over which \( f \) splits.
  \end{tdefinition}

\section{Solvability and solubility}
  \begin{tdefinition}[Soluble group]
    A finite group \( G \) is \ul{soluble} if there is a series of groups \begin{align*}
      \lt\{ \nf{id} \rt\} = G_0 \sgp G_1 \sgp \cdots\sgp G_n = G,
    \end{align*}
    with the property that \( G_i \nsgp G_{i+1} \) and \( \qg{G_{i+1}}{G_i} \) is abelian (\( 0\leq i < n \)).
  \end{tdefinition}

  \begin{ttheorem}
    Let \( K \) be a field of characteristic 0.
    Then \( f\in K[t] \) is solvable by radicals if and only if \( \Gal_K(f) \) is soluble.
  \end{ttheorem}

  \begin{tlemma}
    Suppose \( \chr K = 0 \) and \( L:K \) is a radical extension.
    Then there exists an extension \( N:L \) such that \( N:K \) is normal and radical.
  \end{tlemma}

  \begin{tdefinition}[Cyclic extension]
    The extension \( L:K \) is \ul{cyclic} if \( L:K \) is a Galois extension and \( \Gal(L:K) \) is a cyclic group.
  \end{tdefinition}

  \begin{tlemma}
    Suppose that \( \chr K = 0 \) and let \( p \) be a prime number.
    Also, let \( L:K \) be a \sfe~for \( t^p-1 \).
    Then \( \Gal(L:K) \) is cyclic, and hence \( L:K \) is a cyclic extension.
  \end{tlemma}

  \begin{tlemma}
    Let \( \chr K = 0 \) and suppose that \( n \) is an integer such that \( t^n-1 \) splits over \( K \).
    Let \( L:K \) be a \sfe~for \( t^n-a \), for some \( a\in K \).
    Then \( \Gal(L:K) \) is abelian.
  \end{tlemma}

  \begin{ttheorem}
    Let \( \chr K = 0 \) and suppose that \( L:K \) is Galois.
    Suppose that there is an extension \( M:L \) with the property that \( M:K \) is radical.
    Then \( \Gal(L:K) \) is soluble.
  \end{ttheorem}

  \begin{tcorollary}
    Suppose that \( \chr K = 0 \).
    Then \( \Gal_K(f) \) is soluble whenever \( f\in K[t] \) is soluble by radicals.
  \end{tcorollary}

  \begin{tcorollary}
    There exist quintic polynomials in \( \Q[t] \) with insoluble Galois groups, such as \( f(t) = t^5-4t+2 \), and which are not solvable by radicals.
  \end{tcorollary}

  \begin{tlemma}
    Let \( \chr K = 0 \), and suppose that \( L:K \) is a cyclic extension of degree \( n \).
    Suppose also that \( K \) contains a primitive \( n \)-th root of 1.
    Then there exists \( \theta \in K \) having the property that \( t^n -\theta \) is irreducible over \( K \), and \( L:K \) is a \sf~for \( t^n-\theta \).
    Further, if \( \beta \) is a root of \( t^n-\theta \) over \( L \), then \( L=K(\beta) \).
  \end{tlemma}

  \begin{ttheorem}
    Let \( \chr K = 0 \), and suppose that \( f\in K[t]\setminus K \).
    Then \( f \) is solvable by radicals whenever \( \Gal_K(f) \) is soluble.
  \end{ttheorem}
\end{document}
