\documentclass{article}

\input{preamble}
\input{macros}
\input{letterfont}

\begin{document}
\section{Field extensions and algebraic elements}
\subsection{Field extensions}
  \begin{tdefinition}[Field extension]
    When \( K \tand L \) are fields, we say that \( L \) is an \ul{extension} of \( K \) if
  there is a \homo~\( \vphi: K \to L \). We then talk about the \ul{field extension}
  \( (\vphi,K,L) \).
  \end{tdefinition}

  \begin{tdefinition}[Degree, finite extension]
    Suppose that \( L:K \) is a field extension.
    We define the \ul{degree} of \( L:K \) to be the dimension of \( L \) as a vector space over \( K \).
    We use the notation \( [L : K] \) to denote the degree of \( L : K \).
    Further, we say that \( L : K \) is a \ul{finite extension} if \( [L: K] <\infty \).
  \end{tdefinition}

  \begin{tdefinition}[Tower, intermediate field]
    We say that \( M : L : K \) is a \ul{tower} of field extensions if \( M : L \tand L: K \) are field extensions, and in this case we say that \( L \) is an \ul{intermediate field} (relative to the extension \( M : K \))
  \end{tdefinition}

\subsection{Algebraic elements}
  \begin{tdefinition}[Algebraic/transcendental element]
    Suppose that \( L: K \) is a field extension with associated embedding \( \vphi \).
    Suppose also that \( \alpha\in L \). \begin{enumerate}[label=(\roman*)]
      \item We say that \( \alpha \) is \ul{algebraic} over \( K \) when \( \alpha \) is the root of \( \vphi(f) \) for some non-zero polynomial \( f \in K[t] \).
      \item If \( \alpha \) is not algebraic over \( K \), then we say \( \alpha \) is \ul{transcendental} over \( K \).
      \item When every element of \( L \) is algebraic over \( K \), we say that the field \( L \) is algebraic over \( K \).
    \end{enumerate}
  \end{tdefinition}

  \begin{tdefinition}[Evaluation map]
    Suppose that \( L: K \) is a field extension with \( K \sseq L \), and that \( \alpha\in L \).
    We define the \ul{evaluation map} \( E_\alpha : K[t] \to L \) by putting \( E_\alpha(f) = f(\alpha) \) for each \( f \in K[t] \).
  \end{tdefinition}

  \begin{tdefinition}[Minimal polynomial]
    Suppose that \( L : K \) is a field extension with \( K \sseq L \), and suppose that \( \alpha\in L \) is algebraic over \( K \).
    Then the minimal polynomial of \( \alpha \) over \( K \) is the unique monic polynomial \( m_\alpha(K) \) in \( K[t] \) having the property that \( \ker(E_\alpha) = (m_\alpha(K)) \).
  \end{tdefinition}

  \begin{tdefinition}[Smallest subring/subfield]
    Let \( L:K \) be a field extension with \( K\sseq L \). \begin{enumerate}[label=(\roman*)]
      \item When \( \alpha\in L \), we denote by \( K[\alpha] \) the \ul{smallest subring of \( L \) containing \( K \) and \( \alpha \)}, and by \( K(\alpha) \) the \ul{smallest subfield of \( L \) containing \( K \) and \( \alpha \)};
      \item More generally, when \( A\sseq L \), we denote by \( K[A] \) the \ul{smallest subring of \( L \) containing \( K \tand A \)}, and by \( K(A) \) the \ul{smallest subfield of \( L \) containing \( K \tand A \)}.
    \end{enumerate}
  \end{tdefinition}

\section{Review of finite fields and tests for irreducibility}
  \begin{tdefinition}[Characteristic]
    Let \( K \) be a field with additive identity \( 0_K \) and multiplicative identity \( 1_K \).
    When \( n\in \N \), we write \( n\cdot 1_K \) to denote \( 1_K+\ldots+ 1_K \) (as an \( n \)-fold sum).
    We define the \ul{characteristic} of \( K \), denoted by \( \chr K \), to be the smallest positive integer \( m \) with the property that \( m\cdot 1_K = 0_K \);
    if no such integer \( m \) exists, we define the characteristic of \( K \) to be 0.
  \end{tdefinition}

  \begin{tdefinition}[Highest common factor, content, primitive]
    Let \( R \) be a UFD. When \( \llist{a}{0}{n}\in R \) are not all 0, we define as a \ul{highest common factor} of \( \llist{a}{0}{n} \) (written hcf(\( \llist{a}{0}{n} \))) any element \( c\in R \) satisfying \begin{enumerate}[label=(\roman*)]
      \item \( c\divs a_i\ (0\leq i\leq n) \), and
      \item whenever \( d\divs a_i\ (0\leq i\leq n) \), then \( d\divs c \).
    \end{enumerate}
    When \( f=a_0+a_1X+\ldots +a_nX^n \) is a non-zero polynomial in \( R[X] \), we define a \ul{content} of \( f \) to be any hcf(\( \llist{a}{0}{n} \)).
    We say that \( f\in R[X] \) is \ul{primitive} if \( f\neq 0 \) and the content of \( f \) is divisible only by units of \( R \).
  \end{tdefinition}

\section{Extending field \homo s and the Galois group of an extension}
\setcounter{tdefinition}{15}
\setcounter{section}{2}
  \section{Extending field \homo s and the Galois group of an extension}
  \begin{tdefinition}[Extension of field \homo, isomorphic field extensions]
    For \( i = 1 \tand 2 \), let \( L_i : K_i \) be a field extension relative to the embedding \( \vphi_i : K_i\to L_i \).
    Suppose that \( \sigma :K_1\to K_2 \) and \( \tau:L_1\to L_2 \) are isomorphisms.
    We say that \ul{\( \tau \) extends \( \sigma \)} if \( \tau\circ\vphi_1 = \vphi_2\circ\sigma \).
    In such circumstances, we say that \( L_1 : K_1 \tand L_2 : K_2 \) are \ul{isomorphic field extensions}.

    When \( \sigma:K_1\to K_2 \) and \( \tau:L_1\to L_2 \) are \homo s (instead of isomorphisms), then \ul{\( \tau \) extends \( \sigma \) as a} \ul{\homo~of fields} when the isomorphism \( \tau:L_1\to L_1' = \tau(L_1) \) extends the isomorphism \( \sigma:K_1\to K_1' = \sigma(K_1) \).
  \end{tdefinition}

  \begin{tdefinition}[\( F \)-\homo]
    Let \(L : K\) be a field extension relative to the embedding \(\vphi : K \to L\), and let \(M\) be a subfield of \(L\) containing \(\vphi(K)\).
    Then, when \(\sigma : M \to L\) is a \homo, we say that \(\sigma\) is a \ul{\(K\)-\homo} if \(\sigma\) leaves \(\vphi(K)\) pointwise fixed, which is to say that for all \(\alpha \in \vphi(K)\), one has \( \sigma(\alpha) = \alpha \).
  \end{tdefinition}

\section{Algebraic closures}
\subsection{The definition of an algebraic closure, and Zorn's Lemma}
  \begin{tdefinition}[Algebraically closed field, algebraic closure]
    Let \( M \) be a field. \begin{enumerate}[label=(\roman*)]
      \item We say that \( M \) is \ul{algebraically closed} if every non-constant polynomial \( f\in M[t] \) has a root in \( M \).
      \item We say that \( M \) is an algebraic closure of \( K \) if \( M:K \) is an algebraic field extension having the property that \( M \) is algebraically closed.
    \end{enumerate}
  \end{tdefinition}

  \begin{tdefinition}[Chain]
    Suppose that \( X \) is a nonempty, partially ordered set with \( \leq \) denoting the partial ordering.
    A \ul{chain} \( C \) in \( X \) is a collection of elements \( \lt\{ a_i \rt\}_{i\in I} \) of \( X \) having the property that for every \( i,j\in I \), either \( a_i\leq a_j \tor a_j\leq a_i \).
  \end{tdefinition}

\subsection{The existence of an algebraic closure}
  \begin{tdefinition}[Algebraic closure of \( K \)]
    When \( K \) is a field, an algebraic extension \( \Kbar:K \) that is algebraically closed is called an \ul{algebraic closure} of \( K \).
  \end{tdefinition}

\section{Splitting field extensions}
  \begin{tdefinition}[Splitting field, \sfe]
    Suppose that \( L:K \) is a field extension relative to the embedding \( \vphi : K\to L \), and \( f\in K[t]\setminus K \). \begin{enumerate}[label=(\roman*)]
      \item We say that \ul{\( f \) splits over \( L \)} if \( \vphi(f)=\lambda(t-\alpha_1)\cdots(t-\alpha_n) \), for some \( \lambda\in \vphi(K) \) and \( \llist{\alpha}{1}{n}\in L \).
      \item Suppose that \( f \) splits over \( L \), and let \( M \) be a field with \( \vphi(K)\sseq M\sseq L \). We say that \ul{\( M:K \) is a splitting} \ul{field extension for \( f \)} if \( M \) is the smallest subfield of \( L \) containing \( \vphi(K) \) over which \( f \) splits.
      \item More generally, suppose that \( S\sseq K[t]\setminus K \) has the property that every \( f\in S \) splits over \( L \). Let \( M \) be a field with \( \vphi(K)\sseq M\sseq L \). We say that \ul{\( M:K \) is a \sfe~for \( S \)} if \( M \) is the smallest subfield of \( L \) containing \( \vphi(K) \) over which every polynomial \( f\in S \) splits.
    \end{enumerate}
  \end{tdefinition}

\section{Normal extensions and composita}
\subsection{Normal extensions}
  \begin{tdefinition}[Normal extension]
    The extension \( L:K \) is \ul{normal} if it is algebraic, and every irreducible polynomial \( f\in K[t] \) either splits over \( L \) or has no root in \( L \).
  \end{tdefinition}

\subsection{Composita of field extensions}
  \begin{tdefinition}[Compositum]
    Let \( K_1 \) and \( K_2 \) be fields contained in some field \( L \).
    The \ul{compositum} of \( K_1 \) and \( K_2 \) in \( L \), denoted by \( K_1K_2 \), is the smallest subfield of \( L \) containing both \( K_1 \) and \( K_2 \).
  \end{tdefinition}

\section{Separability}
  \begin{tdefinition}[Separable]
    Let \( K \) be a field. \begin{enumerate}[label=(\roman*)]
      \item An irreducible polynomial \( f\in K[t] \) is \ul{separable over \( K \)} if it has no multiple roots, meaning that \( f=\lambda(t-\alpha_1)(t-\alpha_2)\cdots(t-\alpha_d) \), where \( \llist{\alpha}{1}{d}\in \Kbar \) are distinct.
      \item A non-zero polynomial \( f\in K[t] \) is \ul{separable over \( K \)} if its irreducible factors in \( K[t] \) are separable over \( K \).
      \item When \( L:K \) is a field extension, we say that \( \alpha \in L \) is \ul{separable over \( K \)} when \( \alpha \) is algebraic over \( K \) and \( m_\alpha(K) \) is separable.
      \item An algebraic extension \( L:K \) is \ul{a separable extension} if every \( \alpha\in L \) is separable over \( K \).
    \end{enumerate}
  \end{tdefinition}

\section{Inseparable polynomials, differentiation, and the Frobenius map}
\subsection{Inseparable polynomials and differentiation}
  \begin{tdefinition}[Inseparable]
    A polynomial \( f \in K[t] \) is \ul{inseparable over \( K \)} if \( f \) is not separable over \( K \), meaning that \( f \) has an irreducible factor \( g \in K[t] \) having the property that \( g \) has fewer than \( \deg g \) distinct roots in \( K \).
  \end{tdefinition}

  \begin{tdefinition}[Formal derivative]
    We define the \ul{derivative operator} \( \mcD:K[t]\to K[t] \) by \begin{align*}
      \mcD\lt(\sum_{k=0}^{n}a_kt^k\rt) = \sum_{k=1}^{n}ka_kt^{k-1}.
    \end{align*}
  \end{tdefinition}

\subsection{The Frobenius map}
  \begin{tdefinition}[Frobenius map]
    Suppose that \( \chr K = p > 0 \).
    The \ul{Frobenius map} \( \phi:K\to K \) is defined by \( \phi(\alpha)=\alpha^p \).
  \end{tdefinition}

\section{The Primitive Element Theorem}
  \begin{tdefinition}[Simple extension]
    Suppose \( L:K \) is a field extension relative to the embedding \( \varphi:K\to L \).
    We say that \( L:K \) is a \ul{simple extension} if there is some \( \gamma\in L \) having the property that \( L=\vphi(K)(\gamma) \).
  \end{tdefinition}

\section{Fixed fields and Galois extensions}
  \begin{tdefinition}[Fixed field]
    Let \( L:K \) be a field extension.
    When \( G \) is a subgroup of \( \Aut(L) \), we define the fixed field of \( G \) to be \begin{align*}
      \Fix_L(G) = \lt\{ \alpha\in L : \sigma(\alpha) = \alpha \text{ for all } \sigma\in G \rt\}.
    \end{align*}
  \end{tdefinition}

  \begin{tdefinition}[Galois extension]
    When \( L:K \) is a field extension, we say that \( L:K \) is a \ul{Galois extension} if it is an extension that is normal and separable.
  \end{tdefinition}

\section{The main theorems of Galois theory}
\subsection{The Fundamental Theorem}
  \begin{tdefinition}
    Suppose that \( L:K \) is a field extension.
    When \( G \) is a subgroup of \( \Aut(L) \), we write \( \phi(G) \) for \( \Fix_L(G) \), and when \( L:M:K_0 \) is a tower of field extensions with \( K_0=\phi(\Gal(L:K)) \), we write \( \gamma(M) \) for \( \Gal(L:M) \).
  \end{tdefinition}

  \begin{tdefinition}[Galois group of polynomial]
    When \( f\in K[t] \) and \( L:K \) is a \sfe~for \( f \), we define the \ul{Galois group of the polynomial \( f \) over \( K \)} to be \( \Gal_K(f) = \Gal(L:K) \).
  \end{tdefinition}

\section{Solvability by radicals: polynomials of degree 2, 3 and 4}
  \begin{tdefinition}[Radical element/extension]
    Suppose that \( L:K \) is a field extension, and \( \beta\in L \).
    We say that \( \beta \) is \ul{radical} over \( K \) when \( \beta^n\in K \) for some \( n\in \N \) (so \( \beta=\alpha^{1/n} \) for some \( \alpha\in K \) and some \( n\in\N \)).
    We say that \( L:K \) is \ul{an extension by radicals} when there is a tower of field extensions \( L=L_r:L_{r-1}:\cdots:L_0=K \) such that \( L_i=L_{i-1}(\beta_i) \) with \( \beta_i \) radical over \( L_{i-1} \).
    We say \( f\in K[t] \) is \ul{solvable by radicals} if there is a radical extension of \( K \) over which \( f \) splits.
  \end{tdefinition}

\section{Solvability and solubility}
  \begin{tdefinition}[Soluble group]
    A finite group \( G \) is \ul{soluble} if there is a series of groups \begin{align*}
      \lt\{ \nf{id} \rt\} = G_0 \sgp G_1 \sgp \cdots\sgp G_n = G,
    \end{align*}
    with the property that \( G_i \nsgp G_{i+1} \) and \( \qg{G_{i+1}}{G_i} \) is abelian (\( 0\leq i < n \)).
  \end{tdefinition}

  \begin{tdefinition}[Cyclic extension]
    The extension \( L:K \) is \ul{cyclic} if \( L:K \) is a Galois extension and \( \Gal(L:K) \) is a cyclic group.
  \end{tdefinition}

\end{document}
